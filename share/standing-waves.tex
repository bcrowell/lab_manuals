\lab{Standing Waves}\label{lab:standing-waves}

\apparatus
\equip{string}
\equip{weights (in lab benches)}
\equip{pulley}
\equip{vibrator}
\equip{paperclips}
\equip{metersticks}
\equip{butcher paper}
\equip{scissors}
\equip{weight holders}
\equip{digital balances}

\begin{goals}

\item[] Observe the resonant modes of vibration of a string.

\item[] Find how the speed of waves on a string depends on the
tension in the string.
\end{goals}

\introduction

The Greek philosopher Pythagoras is said to have been the
first to observe that two plucked strings sounded good
together when their lengths were in the proportion of two
small integers.  (This is assuming the strings are of the
same material and under the same tension.)  For instance, he
thought a pleasant combination of notes was produced when
one string was twice the length of the other, but that the
combination was unpleasant when the ratio was, say, 1.4 to 1
(like the notes B and F).  Although different combinations
of notes are used in different cultures and different styles
of music, there is at least some scientific justification
for Pythagoras' statement.  We now know that a plucked
string does not just vibrate at a single frequency but
simultaneously at a whole series of frequencies $f_1$,
$2f_1$, $3f_1$,...  These frequencies are called the
harmonics.  If one string is twice the length of the other,
then its lowest harmonic is at half the frequency of the
other string's, and its harmonics coincide with the
odd-numbered harmonics of the other string.  If the ratio is
1.4 to 1, however, then there is essentially no regular
relationship between the two sets of frequencies, and many
of the harmonics lie close enough in frequency to produce unpleasant beats.

\fig{vw-sta-setup}

\setup

The apparatus allows you to excite vibrations at a fixed
frequency of $f$ (twice the frequency of the alternating
current from the wall that runs the vibrator). Since the point of the lab
is to determine a proportionality, any constant factor, such as $f$, can
be discarded.

The tension in the string
can be controlled by varying the weight.

You may find it helpful to put a strip of white butcher paper
behind the black string for better visual contrast.

It's important to get the vibrator set up properly along the
same line as the string, not at an angle.

If there's a loud buzzing, try moving the vibrator so that it doesn't
touch the bracket holding it.

\observations

Observe as many patterns (``modes'') of vibration as you can. Each mode
can be labeled by $N$, the number of humps or half-wavelengths.  You will
probably not be able to observe the fundamental ($N=1$)
because it would require too much weight.  In each
case, you will want to fine-tune the weight to get as close
as possible to the middle of the resonance, where the
amplitude of vibration is at a maximum.  When you're close
to the peak of a resonance, an easy way to tell whether to
add or remove weight is by gently pressing down or lifting
up on the weights with your finger to see whether the
amplitude increases or decreases.

For large values of $N$, you may find that you need to use a
paperclip instead of the weight holder, in order to make the mass
sufficiently small. Keep in mind, however, that you won't really
improve the quality of your data very much by taking data for very
high values of $N$, since the 1-gram precision with which you can
locate these resonances results in a poor relative precision
compared to a small weight.

\prelab

\prelabquestion  How is the tension in the string, $T$, related to the
mass of the hanging weight?

\prelabquestion The figure below shows the $N=1$, 2, and 3 patterns
of vibration. Suppose the length of the string is one meter. In each case,
find the wavelength.

\fig{vw-sta-prelab}

\prelabquestion Generalize your numerical results from P3 to give a general
equation for $\lambda$ in terms of $N$ and $L$, the length of the string. Check its units, and check
that it recovers the special cases done numerically in P3.

\prelabquestion  How can the velocity of the waves be determined if you
know the frequency, $f$, the length of the string, $L$, and
the number of humps, $N$?

\selfcheck

Do your analysis in lab.

\analysis

Use the graphing technique given in appendix \ref{appendix:powerlaws} to see if you can
find a power-law relationship between the velocity of the
waves in the string and the tension in the string.  Note that you can omit
constant factors without affecting the exponent. (Do not
just try to find the correct power law in the textbook,
because besides observing the phenomenon of resonance, the
point of the lab is to prove experimentally what the
power-law relationship is, and to test whether this is
always a good approximation in real life.)
