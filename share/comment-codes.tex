















% This file is generated by 
%    /home/bcrowell/Documents/teaching/lab/comment_codes/comment_codes.rb
% Don't edit it directly.






















\thispagestyle{empty}
%
\pagestyle{empty}  
\begin{multicols}{2}[\myappendix{comment-codes}{Comment Codes for Lab Writeups}]
\textbf{A. General}\\
a1. Don't write numbers without units. (25\% off)\\
a2. If something is wrong, cross it out. Don't make me guess which version to grade.\\
a3. Your writeup is too long. The length limit is 3 pages, not including raw data.\\
a4. If your writeup includes printouts, staple them in sideways with a single staple.\\
a5. See appendix 1 for the format of lab writeups.\\
a6. Don't state speculation as a firm conclusion.\\
a7. Leave more space for me to write comments.\\
a8. Cut unnecessary words. Use active voice. Write in a simple, direct style.\\
a9. Don't write walls of text. Use paragraph breaks.\\
a10. Cut any sentence that doesn't carry information.\\
a11. This paragraph needs a topic sentence.\\
a12. Express this as an equation.\\
a13. Don't present details unless you've already made it clear why we would care. Don't write slavishly in chronological order.\\
a14. The first sentence of any piece of writing must make an implicit promise that the remainder will interest the reader.\\
\textbf{B. Raw data}\\
b1. Don't mix raw data with calculations. (25\% off)\\
b2.  Write raw data in pen, directly in the notebook.\\
b3. This isn't raw data. This is a summary or copy.\\
\textbf{C. Procedure}\\
c1. Don't repeat the lab manual.\\
c2. Don't write anything about your procedure unless it's something truly original that you think I would be interested in knowing about, or I wouldn't be able to understand your writeup without it.\\
\textbf{D. Abstract -- see appendix 1}\\
d1. Your abstract is too long.\\
d2. Don't recap raw data in your abstract.\\
d3. Don't describe calculations in your abstract.\\
d4. The only numbers that should be in your abstract are important final results that support your conclusion or that constitute the purpose of the lab.\\
d5. Your abstract needs to include numerical results that support your conclusions.\\
d6. Give error bars in your abstract.\\
d7. Where is your abstract?\\
d8. Your abstract is for results. This isn't a result of your experiment.\\
d9. This isn't important enough to go in your abstract.\\
d10. What was the point of the lab, and why would anyone care?\\
d11. Don't just give results. Interpret them.\\
d12. We knew this before you did the lab.\\
d13. This lab was a quantitative test. Restating it qualitatively isn't interesting.\\
d14. This lab is a comparison of theory and experiment. Did they agree, or not?\\
d15. Your results don't support your conclusions. Write about what really happened, not what you wanted to happen.\\
d16. One observation can never prove a general rule.\\
\textbf{E. Error analysis -- see appendices 2 and 3}\\
e1. A standard deviation only measures error if it comes from numbers that were supposed to be the same, e.g., repeated measurements of the same thing.\\
e2. In propagation of errors, don't do both high and low. See appendix 3.\\
e3. In propagation of errors, only change one variable at a time. See appendix 3.\\
e4. Don't round severely when calculating Q's. Your Q's are just measuring your rounding errors.\\
e5. A Q is the amount by which the output of the calculation changes, not its inputs.\\
e6. A Q is a change in the result, not the result itself.\\
e7. Use your error bars in forming your conclusions. Otherwise what was the point of calculating them?\\
e8. Give a probabilistic interpretation, as in the examples at the end of appendix 2.\\
e9. You're interpreting this probability incorrectly. It's the probability that your results would have differed this much from the hypothesis, if the hypothesis were true.\\
e10. \% errors are useless. Teachers have you do them if you don't know about real error analysis.\\
e11. If random errors are included in your propagation of errors, listing them here verbally is pointless.\\
e12. Don't speculate about systematic errors without investigating them. Estimate their possible size. Would they produce an effect in the right direction?\\
\textbf{G. Graphing -- see appendix 4}\\
g1. Label the axes to show what variables are being graphed and what their units are, e.g., x (km).\\
g2. Your graph should be bigger.\\
g3. If graphing by hand, do it on graph paper.\\
g4. Choose an appropriate scale for your graph, so that the data are not squished down. Don't just accept the default from the software if it's wrong. See app. 4 for how to do this using Libre Office.\\
g5. ``Dot to dot'' style is wrong in a scientific graph.\\
g6. The independent variable (the one you control directly) goes on the x axis, and  the dependent variable on the y. Or: cause on x, effect on y.\\
g7. On a scientific graph, use dots to show data, a line or curve for theory or a fit to the data.\\
g8. ``Trend line'' is scientifically illiterate. It's called a line of best fit.\\
\textbf{S. Calculations and sig figs}\\
s1. \emph{Think} about the sizes of numbers and whether they make sense. This number doesn't make sense.\\
s2. Where did this number come from?\\
s3. This number has too many sig figs (e.g., more than the number of sig figs in the raw data).\\
s4. Don't round off severely for sig figs at intermediate steps. Rounding errors can accumulate.\\
s5. You're wasting your time by writing down many non-significant figures.\\
s6. Your result has too many sig figs. The error bars show that you don't have this much precision.\\
s7. The Calculations and Reasoning section usually just consists of the calculations you've already written.You don't need to write a separate narrative.\\
s8. Put your calculator in scientific notation mode.\\
\end{multicols}
