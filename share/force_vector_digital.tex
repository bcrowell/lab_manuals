\lab{Vector Addition of Forces}\label{lab:force_vector_digital}

\apparatus
\equip{circular table with angle scales}
\equip{spirit level}
\equipn{force sensors}{4/group}
\equipn{clamps for mounting sensors on table}{4/group}
\equip{LabPro interface}
\equip{string}
\equip{small ring}
\equip{500 g hooked weight  (in lab benches in 415)}
\equip{right-angle clamp}
\equip{30-cm steel rod}

\goal{Test whether the vector sum of the forces acting on an}
object at rest is equal to zero.

\introduction

Modern physics claims that when a bridge, an earthquake
fault, or an oak tree doesn't move, it is because the forces
acting on it, which combine according to vector addition,
add up to zero.  Although this may seem like a reasonable
statement, it was far from obvious to premodern scientists. 
Aristotle, for instance, said that it was the nature of each
of the four elements, earth, fire, water and air, to return
to its natural location.  Rain would fall from the sky
because it was trying to return to its natural location in
the lakes and oceans, and once it got to its natural
location it would stop moving because that was its nature.

When a modern scientist considers a book resting on a table,
she says that it holds still because the force of gravity
pulling the book down is exactly canceled by the normal
force of the table pushing up on the book.  Aristotle would
have denied that this was possible, because he believed that
at any one moment an object could have only one of two
mutually exclusive types of motion: natural motion (the
tendency of the book to fall to the ground, and resume its
natural place), and forced motion (the ability of another
object, such as the table, to move the book).  According to
his theory, there could be nothing like the addition of
forces, because the object being acted on was only capable
of ``following orders''  from one source at a time.  The
incorrect Aristotelian point of view has great intuitive
appeal, and beginning physics students tend to make
Aristotelian statements such as, ``The table's force
overcomes the force of gravity,'' as if the forces were
having a contest, in which the victor annihilated the loser.

\fig{me-vec-sensors}

\observations

The apparatus consists of a small, lightweight metal ring
held taut by four strings, each pulling in a different direction
in the horizontal plane. Each string is connected to a force sensor
that can be read out on a computer, giving the magnitude of the
force vector.


Set the force sensors on their 10 N scales using the switch on each
sensor.
Plug the sensors into channels 1 through 4 on the LabPro interface.
When you start up the Logger Pro 3 software, it will automatically detect
the four sensors and start displaying their readings. 


Possible computer hassles:
\begin{itemize}
\item[] Normally the computer automatically detects the sensor and sets itself up
properly. If this doesn't happen, you need to do the setup manually under
Experiment:Set up sensors:Show all interfaces.
\item[] If the readings are displayed in a small font, you can make them bigger
by clicking on the surrounding box and using the little black tabs to resize it.
\end{itemize}

Push and pull on the sensors, and verify that they can measure forces.
What do you notice about the signs of the forces?
You will notice that when you don't apply any force to a sensor, its reading
is small, but not exactly zero.
With the sensors placed horizontally, use the software to zero them all, so that
this systematic error is eliminated. If you now rotate a sensor so that its hook
is pointing down, you should see that the force reading changes by a certain amount,
which is the weight of the hook, about $0.21$ N.

Try pushing gently sideways on the force sensor. You will see that it doesn't
sense the force.
Each sensor can only measure the component of the force along its
own axis.

After taking data with the circular table, you'll do a complete calibration of the
sensors. They drift over time, so a calibration done right now wouldn't be useful.
However, sometimes we have sensors that are off by a very large
amount such as 50\%, and in that case it's better to find out about that now, because
it will make it look like your data don't even make sense. As a quick check, try
pairs of sensors hooked together and pulling against each other, and make sure that
they seem to at least approximately obey Newton's third law.

Level the table with the spirit level. The clamps are for mounting the sensors around
the circumference of the table, but our clamps and our sensors are slightly incompatible;
they position the ring and the strings so that they touch the table. To get around this, put
nuts between the table and the top of the clamp. Also, flip the ring so that its
cross-bar is on top.

\fig{me-vec-align}

Avoid a symmetric arrangement of
the strings (e.g., don't space them all 90 degrees apart), and
don't make any forces collinear with each other. 
The ring is an extended object, so in order to treat it
mathematically as a pointlike object you should make sure
that all the strings are lined up with the center of the
ring, as shown in the figure.

The angles are a large source of error, so reduce it by positioning every string so it is exactly 
on one of the lines of the angle scale.

Getting the ring in equilibrium in the center is like pitching a tent; at the end, you need
to make small adjustments to the tension. There are three techniques you can use for
adjustment:

\begin{enumerate}
\item Tie the strings to approximately the right length.
\item Tighten or loosen the screws that hold the force sensors in the clamps.
\item Move the clamp.
\end{enumerate}

First get an approximate
equilibrium in which all four forces are roughly 3-7 N. (Since the angles are the main source
of random error, you may as well position the strings directly over the marks on the angle scale,
e.g., at 37.0\degunit, rather than at some angle like 37.4\degunit where you have to estimate the tenths.)
Then use one of the other methods to
perform a final fine adjustment to get a real equilibrium,
without exceeding the maximum force.
Although the sensors are labeled ``10 N,'' some of them actually max out at 9 or 9.5 N.
The sensors will not give any
error or warning if this limit is exceeded. They will simply max out at some
reading, and this reading will be incorrect. When increasing the tension in a string, keep an eye
on the reading on the screen; if it stops increasing, you've exceeded that sensor's limit.
Once you think you have a good equilibrium, check that when you tug gently on each string, the
reading actually goes up.

Take all the data you will need in order to carry out the vector addition of the four forces.
To decrease the random errors in the force measurements, you can tell the software to average
the data over a period of time. Highlight part of the graph, then go to Analyze:Statistics.

As a cross-check in case confusion arises later about how your angles were defined, which
force was which, etc., slide a piece of paper underneath the strings and trace them onto it.

After this, you will do some calibrations to eliminate systematic errors. These calibrations
unfortunately require cutting the strings and removing the sensors from the clamps,
which means that if something is wrong with your
data, you can't easily put the setup back together. Therefore you should do a graphical
addition of your forces \emph{before} cutting the strings, using the uncorrected data,
and make sure that this comes out fairly well. Don't use this graphical addition in forming
your final conclusions for the lab, since it's uncalibrated.

Also, before taking the sensors out of the clamps,
mark the clamps 1, 2, 3, and 4 with masking tape. Leave the clamps in position when you remove the
sensors.

Your readings will include two sources of systematic error, which you need
to get rid of: (1) We have found that putting the
sensors under strain for a while causes their readings to drift, and (2)
I've also found that the sensors differ from each other quite a bit in their
calibration, with four freshly zeroed sensors giving readings
that covered a range of about 5\% when used to measure a 500 g weight.
To eliminate error 1, unhook the strings and immediately record the readings (which
will no longer be zero if the sensors have drifted), and subtract these
numbers from your observed weights. To eliminate error 2, set up the 30-cm steel bar
horizontally and mount all four sensors on it vertically, with their hooks pointing down.
Zero the sensors by clicking on the blue zero icon, so that error 1 is eliminated and the weights of the hooks are tared out.
Hang the 500 g weight from each sensor, and use these readings to
rescale each sensor's reading from the circular table setup by the proper amount.


\prelab

m4_ifdef([:__sn:],[:%
%%%%%%%%%% 221
\prelabquestion For use in the analysis of this lab, write a python function \verb@x_comp(magnitude,angle)@ that
takes the magnitude and direction of a vector as its inputs and calculates the vector's $x$ component.
(You will also need a \verb@y_comp@ function for the actual analysis.)
Turn in a printout showing both your program and the results of a few tests demonstrating correct results
using simple inputs for which you know the answer.
To get the trig functions to work, (1) remember to put \verb@import math@ at the top, (2) remember to do \verb@math.sin@
and \verb@math.cos@ rather than just \verb@sin@ and \verb@cos@, and (3) include a conversion from degrees to radians, since
your raw data will all be in degrees, but the python functions are defined in radians.
:],[:%
%%%%%%%%%%%% 205
\prelabquestion  Describe a typical scale that you might use for drawing
force vectors on a piece of paper, e.g., how long might you
choose to make a 1-N force?  Assume that your forces are
from 5 to 10 newtons, and pick a scale that results in a drawing that
is as big as it can be (for maximum precision) while still fitting on
a piece of paper.

\prelabquestion Thinking ahead to your analytic addition of the forces,
which of the following would be the most convenient form in which to have
your angle measurements? --- (1) the angles of the four ``pie slices,''
(2) the angle of each string measured clockwise around the table,
or (3) the angle of each string measured counterclockwise around the table.
:])

\analysis

m4_ifdef([:__sn:],[:%
%%%%%%%%%% 221
Calculate the magnitude of the vector sum of the forces on the
ring. Automate this procedure by writing the analysis as a Python program.
The reason for doing it as a Python program is that otherwise the propagation of errors
is extremely tedious, because you have to redo the vector addition by hand eight times,
once for each of the eight pieces of raw data (four angles and four sensor readings).
Turn in a copy of your python code and its output with your writeup. 
:],[:%
%%%%%%%%%% 205
Calculate the magnitude of vector sum of the forces on the
ring analytically, using the corrected data.

Propagation of error for this lab could in principle be very time-consuming,
since you would have to redo the vector addition eight times for the eight
pieces of raw data (four angles and four magnitudes). 
To avoid this, take the following shortcuts: (1) Whichever is larger,
$F_{total,x}$ or $F_{total,y}$, just do error analysis on that component.
(2) The random errors in the result are dominated by the errors in the angles,
so don't bother propagating the error from the magnitudes.
:])

Are your results consistent with theory, taking into account
the random errors involved? Perform a statistical test like the
one in the example on p.~\pageref{eg:fine-structure} (``How significant?'').
