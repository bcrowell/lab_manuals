\lab{Conservation of Momentum}\label{lab:momentum}

\apparatus
\equip{computer with Logger Pro software}
\equip{track}
\equip{2 carts with magnets and mounting brackets for force sensors}
\equip{1-kg weight}
\equip{500 g slotted weight  (in lab benches in 415)}
\equip{masking tape}
\equip{2 force sensors with rubber corks}

The format of this lab is informal. You can write your answers in the blanks in the lab
manual for parts A-G, and you don't need to write up anything at the end. I'll just discuss
the physics verbally with your group as a whole.

\section*{Qualitative Observations}
Level the track.

First you're going to observe some collisions between two carts and see how
 conservation of momentum plays out. If you really wanted to take numerical data, it
 would be a hassle, because momentum depends on mass and velocity, and there would be
 four different velocity numbers you'd have to measure: cart 1 before the collision, cart 1
 after the collision, cart 2 before, and cart 2 after. To avoid all this complication, the
 first part of the lab will use only visual observations.

Try gently pressing the two carts together on the track. As they come close to each
 other, you'll feel them repelling each other! That's because they have magnets built
 into the ends. The magnets act like perfect springs. For instance, if you hold one
 cart firmly in place and let the other one roll at it, the incoming cart will bounce back
 at almost exactly the same speed. It's like a perfect superball. 

\labpart{Equal masses, target at rest, elastic collision}
Roll one cart toward the other. The target cart is initially at rest. Don't hold the target cart in place. Conservation of momentum reads like this,

\newcommand{\momentumeqn}[2]{%
        #1 $\times$ \_\_\_\_\_\_\_\_ + #2 $\times$ \_\_\_\_\_\_\_\_ \\
        \hfill =?
        #1 $\times$ \_\_\_\_\_\_\_\_ + #2 $\times$ \_\_\_\_\_\_\_\_
 \qquad ,}

\momentumeqn{$M$}{$M$}

where the two blanks on the left stand for the two carts' velocities before the collision, and the two
 blanks on the right are for their velocities after the collision. All conservation laws work like this: the
 total amount of something remains the same. You don't have any real numbers, but just from eyeballing the
 collision, what seems to have happened? Let's just arbitrarily say that the mass of a cart is one unit, so
 that wherever it says ``M x'' in the equation, you're just multiplying by one. You also don't have any numerical values
 for the velocities, but suppose we say that the initial velocity of the incoming cart is one unit. Does it
 look like conservation of momentum was satisfied?

\labpart{Mirror symmetry}
Now reenact the collision from part A, but do everything as a mirror image. The roles of the target cart
 and incoming cart are reversed, and the direction of motion is also reversed.

\momentumeqn{$M$}{$M$}

What happens now? Note that mathematically, we use positive and negative signs to indicate the direction
 of a velocity in one dimension.

\labpart{ An explosion}
Now start with the carts held together, with their magnets repelling. As soon as you release them, they'll break
 contact and fly apart due to the repulsion of the magnets.

\momentumeqn{$M$}{$M$}

Does momentum appear to have been conserved?

\labpart{Head-on collision}
Now try a collision in which the two carts head towards each other at equal speeds (meaning that one cart's
 initial velocity is positive, while the other's is negative).

\momentumeqn{$M$}{$M$}

\labpart{Sticking}
Arrange a collision in which the carts will stick together rather than rebounding. You can do this by
 letting the velcro ends hit each other instead of the magnet ends. Make a collision in which the target
 is initially stationary.

\momentumeqn{$M$}{$M$}

The collision is no longer perfectly springy. Did it seem to matter, or was conservation of momentum still valid?

\labpart{Hitting the end of the track}
One end of the track has magnets in it. Take one cart off the track entirely, and let the other cart roll
 all the way to the end of the track, where it will experience a repulsion from the fixed magnets built into the
 track. Was momentum conserved? Discuss this with your instructor.

\labpart{Unequal masses}
Now put a one kilogram mass on one of the carts, but leave the other cart the way it was. Attach the mass to
 it securely using masking tape. A bare cart has a mass of half a kilogram, so you've now tripled the mass of
 one cart. In terms of our silly (but convenient) mass units, we now have masses of one unit and three units for
 the two carts. Make the triple-mass cart hit the initially stationary one-mass-unit cart.

\momentumeqn{$3M$}{$M$}

These velocities are harder to estimate by eye, but if you estimate numbers roughly, does it seem possible
 that momentum was conserved?

\section*{Quantitative Observations}
Now we're going to explore the reasons why momentum always seems to be conserved. Parts H and I will be
demonstrated by the instructor for the whole class at once.

Attach the force sensors to the carts, and put on the rubber stoppers. Make sure that the rubber stoppers are
 positioned sufficiently far out from the body of the cart so that they will not rub against the
 edge of the cart. Put the switch on the sensor in the +10 N position. Plug the sensors into the CH1 and CH2
 ports on the interface box, and plug the interface box into the Windows computer. Start up the Logger Pro software (version 3), and do
File$>$Open$>$Probes and Sensors$>$Force Sensors$>$Dual Range Forrce$>$2-10 N Dual Range.
Go to Experiment$>$Data Collection, and set Sampling rate to 250 samples/s.

Tell the computer to zero the sensors.
Try collecting data and pushing and pulling on the rubber stopper. You should get a graph showing
 how the force went up and down over time. The sensor uses negative numbers (bottom half of the graph)
 for forces that squish the sensor, and positive numbers (top half) for forces that stretch it. Try
 both sensors, and make sure you understand what the red and blue traces on the graph are showing you.


{\textbf H.} Put the extra 1-kilogram weight on one of the carts. Put it on the track by itself,
 without the other cart. Try accelerating it from rest with a gentle, steady force from your finger. You'll want to set the collection
time to a longer period than the default. Position the track so that you can walk all the way along its
length (not diagonally across the bench). 

What does the graph on the computer look like?

{\textbf I.} Now repeat H, but use a more rapid acceleration to bring the cart up to the same momentum. Sketch a comparison of
 the graphs from parts H and I.

Discuss with your instructor how this relates to momentum.

{\textbf J.} You are now going to reenact collision A, but don't do it yet! You'll let the carts' rubber corks bump
 into each other, and record the forces on the sensors. The carts will have equal mass, and both forces will
 be recorded simultaneously. Before you do it, predict what you think the graphs will look like, and show
 your sketch to your instructor.

Switch both sensors to the +50 N position, and open the corresponding file on the computer.

Zero the sensors, then check the calibration by balancing a 500 g slotted weight on top, taking data,
zooming in, and putting the mouse cursor on the graph. You will probably find that the absolute calibration
of the sensor is very poor when it's used on the 50 N scale; keep this in mind when interpreting your results
from the collision.

Now try it. To zoom in on the relevant part of the graph, use the mouse to draw a box, and then click on the
magnifying glass icon. You will notice by eye that the motion after the collision is a tiny bit different than it was
 with the magnets, but it's still pretty similar. Looking at the graphs, how do you explain the fact
 that one cart lost exactly as much momentum as the other one gained? Discuss this with your instructor before going on.

{\textbf K.} Now imagine -- but don't do it yet -- that you are going to reenact part G, where
you used unequal masses. Sketch your
 prediction for the two graphs, and show your sketch to your instructor before you go on.





Now try it, and discuss the results with your instructor.
