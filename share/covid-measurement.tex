\addtocounter{chapter}{-1}
\renewcommand\thechapter{c1.1a}
\lab{Measuring Motion}\label{lab:covid-measurement}

\section*{About this lab}

\covid\ 
It is intended to be used during the first week in Physics 205, 210, or 221.
It assumes no previous knowledge of physics.

\introduction

In his 1638 book \emph{Discourses on Two New Sciences}, Galileo described a series
of experiments in which he rolled balls down inclined planes, starting from rest.
He used a crude water clock to determine how the time $t$ for the motion depended
on the distance $x$. Two hypotheses that he tested were that $t \propto x$ or
that $t \propto \sqrt{x}$. Both of these are of the form
\begin{equation*}
  t \propto x^p
\end{equation*}
for some exponent $p$. There were competing scientific theories that predicted
$p=1$ and $p=1/2$. A proportionality like this can always be interpreted as a statement
about ratios,
\begin{equation*}
  \left(\frac{t_1}{t_2}\right) = \left(\frac{x_1}{x_2}\right)^p.
\end{equation*}


\observations

The idea of this lab is to test these hypotheses using materials from
around the house. You don't have to build a wooden ramp or use a ball as Galileo did.
Possibilities include releasing a Hot Wheels car on a sloping track, or going outside
and riding a skateboard down a sloping driveway. The photo shows what I came up when
when I tested the lab myself. I found a piece of an old bunk bed in the garage and
leaned it against a couch. I marked two points on one of the wood slats with pieces
of white tape, so that $x_1/x_2=2$. (Only the ratio $x_1/x_2$ matters here, so it
wasn't necessary for me to scrounge up a tape measure and measure $x_1$ and $x_2$ in
units of meters or feet.) I released a penny and let it slide from the two different
heights, timing it with my cheapo digital watch's stopwatch function.

\fig{co-measurement-couch}

For timing, you may find it convenient to use a stopwatch app on a smartphone. If
all you have is a computer with an internet connection, then you can google and find web pages
with stopwatch applications. If you do have a smartphone and it's a relatively recent, fancy
model, then it probably has the ability to take slow-motion video. These seem to slow down
their video by 8 times, but you should verify that against a clock. When I've tried doing
this kind of thing with a webcam, it didn't work very well, because there was a great deal
of motion blurring, and cheap computer video uses progressive scanning, so that the entire
picture is not actually updated all at once for each frame.

As a general rule in science, set yourself up for success by choosing experimental conditions
in which you can measure a \emph{big} thing. It's hard to measure a small thing with good
precision. That's why, in my version of this lab, I went to the trouble of finding the nice
long piece of wood. I could have done the lab on a tabletop using a book, for example, but
the results would have been lousy, because it would have been impossible to measure the
short times with good precision.

\analysis

We are testing two hypotheses, each of which is stated as an equation. For any real-world
experiment in which we measure continuously varying quantities, we are guaranteed that the two sides of an equation will
never actually equal one another. Therefore in order to test the hypothesis expressed by the
equation, we need to estimate the uncertainties inherent in our measuring technique. In this
lab, it's easy to measure the distances very precisely, but the short time intervals are
hard to measure. That means that we need to get a numerical estimate of the range of uncertainty
in the ratio $t_1/t_2$. This type of thing is discussed in appendices 2 and 3 in the
normal lab manual. Appendix 2 describes how to find the uncertainty in your raw data. In this
lab, that will involve taking multiple measurements of each time. Appendix 3 describes how
to find the uncertainty in the result of a calculation, based on the uncertainty of each piece
of raw data that went into the calculation. The result is a quantitative estimate of the expected
uncertainty --- the ``error bars'' --- so you'll have something like $t_1/t_2=\_\_\_\_\_\_ \pm \_\_\_\_\_\_$.

Finally,
you will want to compare your result with the prediction of each hypothesis, and do a statistical test to see if the
error bars are roughly big enough to account for the difference between the hypothesis and the actual results. This
is described in appendix 2.
The final result is a probability. For example, my own result for $t_1/t_2$ was off compared to one of
the hypotheses by about 2 times my estimated uncertainty. The probability of being off by at least this much, if the
hypothesis is true, is about 5\%. Since 5\% is a fairly big probability, my results were statistically consistent
with the hypothesis.


\prelab

\prelabquestion  
Suppose that $x_1/x_2=2$, as in the example I did with the penny. Predict $t_1/t_2$ in
the cases $p=1$ and $p=1/2$. Interpret these numbers. In each case, is the object
being described as speeding up?



