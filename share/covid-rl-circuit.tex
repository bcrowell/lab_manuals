\renewcommand\thechapter{c2.13b}
\lab{LR circuits}\label{lab:lr-circuit}

\section*{About this lab}

\covid\ 
It is intended to be used around the 13th week in Physics 222.

\apparatus
\equip{home-made inductor}
\equip{banana plug cables}
\equip{alligator clips}
\equip{oscilloscope}

\begin{goals}

\item[] Observe the exponential curve of an LR circuit.

\item[] Determine the inductance of the inductor that you made.
\end{goals}

\section*{Preparation}

Use the result of problem 13-10 in Fields and Circuits to estimate the
inductance of the inductor $L$ that you made. For a coil consisting of a couple of
hundred turns of wire wound around a toilet paper roll, this result should be
on the order of a millihenry.

We expect this inductance value to be too high, because the equation was based on the magnetic
field inside an infinitely long solenoid, which is an overestimate for a short solenoid
like yours. The purpose of this lab is to refine the estimate by determining the
inductance empirically. The resulting inductance value will be handy to know for
future labs.

In lab \ref{lab:covid-oscilloscope}, you measured the internal resistance $R$
of your scope's calibration output. In this lab we will use that $R$ in series
with your $L$. Use your estimated inductance to predict a rough value
the time constant of this series RL circuit. You will need to know this in
order to have some idea of how to set up your oscilloscope.

\observations

The circuit is the same as in labs \ref{lab:covid-oscilloscope}, part E, and
lab \ref{lab:rc-circuits}, part D. The only difference is that where in lab
\ref{lab:covid-oscilloscope} we used an external resistor, and in \ref{lab:rc-circuits}
a capacitor, in this lab we will use an external inductor.

\analysis

Extract the actual inductance from the observed RL time constant and compare with
your original estimate.