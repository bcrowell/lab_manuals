\addtocounter{chapter}{-1}
\renewcommand\thechapter{c1.4b}
\lab{The capstan}\label{lab:covid-capstan}

\section*{About this lab}

\covid\ 
It is intended to be used during the fourth week in Physics 221.
It assumes knowledge of Newton's laws and models of friction. For Physics
205/210, use lab \ref{lab:covid-mu-k}.

\introduction

The photo shows an arrangement called a capstan. The cord is wound around the cylinder
with $n=5$ turns. Although the top of the cord is only held in place by a piece of
scotch tape, it can sustain a rather large force from the hand. This principle
is used extensively in applications such as rock climbing and sailing.

\fig{co-capstan}

The cord winds around the capstan, so that the physics is inherently two-dimensional.
However, we can correctly infer some properties of the full mathematical solution through
the heuristic of speaking as though the cord was straight, referring to fake
directions of ``left'' and ``right.'' We then have three forces acting on the cord:
a leftward force $-T_1$ from the tape, a rightward force $T_2$ from the hand, and
a leftward frictional force $F$. In our standard model of friction, a frictional force
is proportional to the normal force pressing the two surfaces together, and in this
example the normal force exists because of $T_1$ and $T_2$. It therefore follows that
if the list of numbers $(T_1,T_2,F)$ represent a correct solution of the full two-dimensional
problem, then scaling all these numbers by the same factor $\alpha$, to give
$(\alpha T_1,\alpha T_2,\alpha F)$, also gives a solution. This means that the value
of $T_2$ is proportional to the value of $T_1$. That is, the capstan acts as an \emph{amplifier}
of the force applied by the tape, increasing its effect by a factor $A$. In the case of static friction, this amplification is
a maximum value, while for kinetic friction it is a single value.

Because the loops of cord can be broken down into smaller parts, each of which
provides an amplification factor, the over-all amplification factor grows exponentially
with the amount of cord. For example, if $A$ is the amplification factor provided by
one loop, then $n$ loops should gave $A^n$.

These conclusions can be found without going into the full details of the analysis,
which require a knowledge of vectors and the application of some calculus. This
analysis is sketched in homework problem 8-19 in \emph{Mechanics} and 5-d8 in \emph{Problems in Introductory Physics}.
The only additional information provided by that analysis is the ability to find the
base of the exponent in terms of the coefficient of friction $\mu$. The result is
that for loops of cord forming a total wrapping angle $\theta$, $A=e^{\mu\theta}$.

Of course, all of this is just a model, and that model is itself based on another model --- our
simplified standard textbook model of friction. In this lab, you will explore whether this
model holds.

\observations

The experimental technique is to set up an experiment like the one shown in the photo,
but with the cylinder's axis oriented horizontally and weights hung from the two ends
of the rope. Because both ends hang straight down, the number of turns will always be
a half-integer. The most convenient weights available may be coins. If you have a
jar full of pennies, then you can just use the penny as your unit of force. If you
need to use a mixture of coins, you can look up their masses online. If you have an
accurate postal scale, you can use it to measure large weights, and you can use
any convenient objects at all as your weights. However, if $T_1$ is small (like 1 penny),
don't try to use a postal scale to measure it, because the precision will be poor.
You will need to rig up some kind of holder, such as an envelope or a sock, and
include the weight of the holder in your analysis.

A. Measure $A$ for static friction, by adding weight to $T_2$ until it slips.
Repeat this for as many different values as possible of the number of turns of cord.
Make sure to use a stout piece of cord so that the cord will not break when $T_2$ is
large.

B. Find a convenient setup, with some number of turns, that is easy to work with.
Intentionally make $T_2/T_1$ larger than $A$, so that the cord slips, and use 
timing techniques as described in lab \ref{lab:covid-measurement} to measure the
time and distance traveled.

\analysis

Test whether the theoretical exponential dependence of $A$ on the number of turns
actually holds from your data in part A. A nice way to see this visually is
to plot the logarithm of $A$ versus the number of turns.

Using your data from part B, estimate the coefficient of kinetic friction
$\mu_k$ for your cord on your cylinder. This will require applying Newton's
laws to the weights.


