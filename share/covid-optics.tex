\addtocounter{chapter}{-1}
\renewcommand\thechapter{c3.8a}
\lab{Optics}\label{lab:optics}

\section*{About this lab}

\covid\ 
It is intended to be used around the 8th week of Physics 206, 211, or 223.
It assumes knowledge of images, refraction, and lenses.
Each student will need an unknown converging lens. In April 2020, I mailed
the necessary lenses to my Physics 223 students.

\section{Theory}

The theory for this lab is in Crowell, \emph{Modern physics},
\url{lightandmatter.com/mod}, ch.~11, or \emph{Light and matter}, ch.~30-31. This treatment uses
the European sign convention, in which distances and focal lengths
are positive by definition, and signs are added externally to
the image location equation based on analysis of ray diagrams.

\section{Measurement of the focal length}

Go outside and use the lens to focus the light from the sun
to a spot on the ground, as when little kids burn ants. (If there
are clouds or you need to do this at night, then use a source of
light that is as far away as possible.) You may find at first that
the spot of light on the ground has a weird, irregular shape, but
if you carefully adjust the orientation of the lens to be perpendicular
to the light, you should be able to make a small, round spot. This spot
is a disk, and it's the image of the sun's disk.

Figure out whether this is a real image or a virtual image. Draw
the appropriate ray diagram. Use the method described in the book
to determine the image location equation with the appropriate signs.
Take whatever data you will need in order to estimate the focal
length $f$ of the lens.

\section{Image of a distant object}

Look through the lens at a distant object such as a tree. Hold
the lens at a distance from your eye that is greater than the focal
length you have just determined. (If your lens has a long focal length
like 50 cm, then your arm may not be long enough, and you may have to
enlist a helper or rig up a stand for the lens. Just make sure the lens
doesn't fall on the ground and break, and don't scratch or dirty its
optical surfaces.) Determine whether the image is inverted or uninverted.

\section{Image of a nearby object}

Now use the lens like a magnifying glass, to look at an object whose
distance from the lens is less than the lens's focal length.
Again, determine whether the image is inverted or uninverted.

\section{Image location}

Find a compact, bright light source such as a candle, cigarette lighter,
or the flashlight LED on a cell phone. You need a compact light source,
so that it is a good approximation to treat it as a point source at a definite
distance $d_o$ from the lens. We are going to use the lens to make a real
image of this object and project it onto a surface such as a piece of paper
or a wall. We want $d_o$ big enough so that the image is real, but not
so large compared to the focal length that this is effectively an object
at infinity (which you've already done). If the object is not very bright,
then you will need to do this in a dark environment, so that the image is
bright enough to see.

Measure $d_o$ and $d_i$.

\section{Analysis}

Determine the lens's focal length, with propagation of errors.

Draw ray diagrams for parts 4 and 5, and use the ray diagrams to explain
why the images were inverted or uninverted. Use the method demonstrated
in \emph{Modern physics} problem 11-8, but you will probably find it
easiest if you pick one point to be on the optical axis, and always make
one ray on axis.

For part 6, compare your observed $d_i$ with theory. Do a
propagation of errors, and interpret your result using a statistical
test as in the example in appendix 2 of the lab manual.



