\lab{Resonances of Sound}\label{lab:resonances-of-sound}

Note to the lab technician: the computers in 415 with working sound are
Lassie, Buck, Mudge, Dug, and Ribsy. Use Lassie (bench 1) and Mudge (bench 4).

\apparatus
\equipn{wave generator (Pasco PI-8127, in lab benches)}{1/group}
\equipn{speaker (Thornton)}{1/group}
\equipn{100 mL graduated cylinder}{1/group}
\equip{Linux computers with digital oscilloscope software installed (see note above)}
\equipn{flexible whistling tube}{1}
\equipn{tuning fork marked with frequency, mounted on a wooden box}{1}
\equipn{aluminum rod, 3/4-inch dia, about 1 m long}{2}
\equipn{wood block}{1/group}
\equipn{thermometer}{1}

\begin{goals}

\item[] Find the resonant frequencies of the air inside a
cylinder by two methods.

\item[] Measure the speeds of sound in air and in aluminum.
\end{goals}

\introduction

In the womb, your first sensory experiences were of your
mother's voice, and soon after birth you learned to
distinguish the particular sounds of your parents' voices
from those of strangers.  The human ear-brain system is
amazingly sophisticated in its ability to classify vowels
and consonants, recognize people's voices, and analyze
musical sound.  Until the 19th-century investigations of
Helmholtz, the whole process was completely mysterious.  How
could we so easily tell a cello from a violin playing the
same note?  A radio station in Chicago has a weekly contest
in which jazz fanatics are asked to identify instrumentalists
simply by their distinctly individual timbres --- how is this possible?

Helmholtz found (using incredibly primitive nonelectronic
equipment) that part of the answer lay in the relative
strengths of the overtones.  The psychological sensation of
pitch is related to frequency, e.g., 440 Hz is the note
``A.''  But a saxophonist playing the note ``A'' is actually
producing a rich spectrum of frequencies, including 440 Hz,
880 Hz, 1320 Hz, and many other multiples of the lowest
frequency, known as the fundamental.  The ear-brain system
perceives all these overtones as a single sound because they
are all multiples of the fundamental frequency.  (The
Javanese orchestra called the gamelan sounds strange to
westerners partly because the various gongs and cymbals have
overtones that are not integer multiples of the fundamental.)

One of the things that would make ``A'' on a clarinet sound
different from ``A'' on a saxophone is that the 880 Hz
overtone would be quite strong for the saxophone, but almost
entirely missing for the clarinet.  Although Helmholtz
thought the relative strengths of the overtones was the
whole story when it came to musical timbre, actually it is
more complex than that, which is why electronic synthesizers
still do not sound as good as acoustic instruments.  The
timbre depends not just on the general strength of the
overtones but on the details of how they first build up (the
attack) and how the various overtones fade in and out
slightly as the note continues.

Why do different instruments have different sound spectra,
and why, for instance, does a saxophone have an overtone
that the clarinet lacks?  Many musical instruments can be
analyzed physically as tubes that have either two open ends,
two closed ends, or one open end and one closed end.  The
overtones correspond to specific resonances of the air
column inside the tube.  A complete treatment of the subject
is given in your textbook, but the basic principle is that
the resonant standing waves in the tube must have an
antinode (point of maximum vibration) 
at any closed end of the tube, and a node (point of zero
vibration) at any open end.

\mysubsubsection{Using the Wave Generator}

The wave generator works like the amplifier of your stereo,
but instead of playing a CD, it produces a sine wave whose
frequency and amplitude you can control.  By connecting it
to a speaker, you can convert its electrical currents to
sound waves, making a pure tone.  The frequency of the sine
wave corresponds to musical pitch, and the amplitude
corresponds to loudness.

\setup

Plug the speaker into the wave generator.  The banana plugs
go in the two holes on the right marked GND and LO $\Omega$. The side of
the banana plug converter with the tiny tab marked GND should be the one that
goes into the GND output.  Set the frequency to
something audible.  Turn the amplitude knob up until you hear a sound.

The wave generator and the speaker are not really designed
to work together, so if you leave the volume up very high
for a long time, it is possible to blow the speaker or
damage the wave generator.  Also, the sine waves are
annoying when played continuously at loud volumes!

\section{Preliminary Observations}

\observations

This lab has three parts, A, B, and C.  It is not really
possible for more than one group to do part A in the same
room, both because their sounds interfere with one another
and because the noise becomes annoying for everyone.  Your
instructor will probably have three groups working on part A
at one time, one group in the main room, one in the small
side room, and one in the physics stockroom.  Meanwhile, the
other groups will be doing parts B and C.

m4_ifdef([:__sn:],[:%
\emph{Odd-numbered groups will do part A at the first meeting and parts B and C at the second.
Even groups will do it the other way around.}
:],[::])

\labpart{ Direct Measurement of Resonances by Listening}

Set up the graduated cylinder so its mouth is about 3 mm from the center of the speaker grille.
Find as many frequencies as possible
at which the cylinder resonates.  When you sweep through those frequencies, the
sound becomes louder.  To make sure you're really hearing
a resonance of the cylinder, make sure to repeat each
observation with the cylinder removed, and make sure the
resonance goes away.
  For each resonance, take
several measurements of its frequency --- if you are
careful, you can pin it down to within $\pm 10$ Hz or so.  You
can probably speed up your search significantly by
calculating approximately where you expect the resonances to
be, then looking for them.

\labpart{ Electronic Measurement of Resonances of an Air Column}

The resonances of the air column in a cylinder can also be
excited by a stream of air flowing over an opening, as with
a flute.  In this part of the lab, you will excite
resonances of a long, flexible plastic tube by grabbing it
at one end and swinging it in a circle.  The frequency of
the sound will be determined electronically. Note that your
analysis for these resonances will be somewhat different,
since the tube is open at both ends, and it therefore has
different patterns of resonances from the graduated
cylinder, which was only open at one end.

To measure the frequency, you will use a computer to analyze
the sound. The Linux computers are the ones with the right hardware and software.
As a warmup before attempting the actual
measurements with the whistling tube, try the following. 
First, start up the program if nobody else has already done
so.  It is called ``scope,'' and you can run it by double-clicking
on its icon on the desktop. 
In real time, the program will monitor the sound coming into the
microphone, and display a graph of loudness versus frequency.
Try whistling. The frequency at which you whistled should show
up as a prominent peak. 

Next you need to verify that you can actually measure a known frequency and reproduce its value.
Besides letting you practice using the software,
this is important because I've seen some cheap computer sound input chips that produce frequencies that are off by large amounts, like 10\%.
Put the microphone near the wooden box that the tuning fork is mounted on and hit the tuning fork gently with the rubber mallet.
To get an accurate frequency measurement, you need to zoom in on the peak.
To do this, click on the peak to get an extreme close-up. To zoom back out, click on the graph again.
(The Zoom In button doesn't let you get close enough, so don't use it for this purpose.)
When you get
the graph you want to see, you can freeze it by clicking on the
Freeze/Go button.

Once you have done these warmups, you are ready to analyze
the sound from the whistling tube. You only need
to analyze data from one frequency, although if you're not sure
which mode you produced, it may be helpful to observe the
pattern of the frequencies. (If you guess wrong about which
mode it was, you'll find out, because the value you extract for
the speed of sound will be way off.)

\labpart{ The Speed of Sound in Aluminum}

The speed of sound in a solid is much faster than its
speed in air.  In this part of the lab, you will extract the
speed of sound in aluminum from a measurement of the lowest
resonant frequency of a solid aluminum rod.  You will use
the computer for an electronic measurement of the frequency, as in part B.

Grab the rod with two fingers exactly in the middle, hold it vertically, and
tap it on the lab bench.  You will hear two different notes
sounding simultaneously. A quick look at their frequencies shows that they are
not in a 2:1 ratio as we would expect based on our experiences with symmetric
wave patterns. This is because these two frequencies in the rod are actually
two different types of waves.
 The higher note is produced by longitudinal compression waves,
which means that an individual atom of aluminum is moving up
and down the length of the rod.  This type of wave is
analogous to sound waves in air, which are also longitudinal
compression waves.  The lower note comes from transverse
vibrations, like a vibrating guitar string.  In the
transverse vibrations, atoms are moving from side to side,
and the rod as a whole is bending.

If you listen carefully, you can tell that the transverse
vibration (the lower note) dies out quickly, but the
longitudinal mode keeps going for a long time.  That gives
you an easy way to isolate the longitudinal mode, which is
the one we're interested in; just wait for the transverse
wave to die out before you freeze the graph on the computer.

\mysubsubsection{Identifying the mode of vibration}
The rod is symmetric, so we expect its longitudinal wave patterns
to be symmetric, like those of the whistling tube. The rod is different,
however, because whereas we can excite a variety of wave patterns in the
tube by spinning it at different speeds, we find we only ever get one
frequency from the rod by tapping it at its end: it appears that there is
only one logitudinal wave pattern that can be excited strongly in the
rod by this method. The problem is that we then need to infer what the
pattern is.

Since you hold the rod at its center, friction should very
rapidly damp out any mode of vibration that has any motion at the center.
Therefore there must be a node at the center. We also know that at the ends,
the rod has nothing to interact with but the air, and therefore there is
essentially no way for any significant amount of wave energy to leak out;
we therefore expect that waves reaching the ends have 100\% of their energy
reflected. Since energy is proportional to the square of amplitude, this means
that a wave with unit amplitude can be reflected from the ends with an amplitude
of either $R=+1$ (100\% uninverted reflection) or $-1$ (100\% inverted). In the $R=-1$
case, the reflected wave would cancel out the incident wave at the end of the rod, and
we would have a node at the end, as in lab \ref{lab:standing-waves}. In the $R=+1$ case,
there would be an antinode. But when you tap the end of the rod on the floor, you are
evidently exciting wave motion by moving the end, and it would not be possible to excite
vibrations by this method if the vibrations had no motion at the end. We therefore conclude
that the rod's pattern of vibration must have a node at the center, and antinode at the ends.

There is an infinite number of possible wave patterns of this kind, but we will assume
that the pattern that is excited strongly is the one with the longest wavelength, i.e.,
the only node is at the center, and the only antinodes are those at the ends.

If you feel like it, there are a couple of possible tests you can try to do to check whether this
is the right interpretation. One is to see if you can detect any other frequencies of longitudinal
vibration that are
excited weakly. Another is to predict where the other nodes would be, if there were more than one,
and then see if the vibration is killed by touching the rod there with your other hand; if there is
a node there, touching it should have no effect.

\prelab

\prelabquestion   Find an equation to predict the frequencies of the
resonances in parts A and B. Note that they will not be
the same equations, since one tube is symmetric and the
other is asymmetric. For the symmetric one, you could simply staple your prelab from 
lab \ref{lab:standing-waves} onto the back. For the asymmetric one, go through the same
process that you were explicitly led through in the prelab for lab \ref{lab:standing-waves}.
This process starts with drawing the first few wave patterns. It's not necessary to define a
variable $N$ analogous to the one defined in lab \ref{lab:standing-waves}, and in fact it's
not obvious how one could define such a thing as the ``number of humps'' in the asymmetric case.
Instead of giving a single formula with an $N$ in it, it's fine to list expressions for the
first few frequencies, showing the pattern.

\selfcheck

Extract the speed of sound in air from either part A or part B,
without error analysis, and make sure you get something
reasonable. We don't necessarily expect it to be exactly the same as
a standard value, because it depends on conditions, especially the temperature.

\analysis

Make a graph of wavelength versus period for the resonances
of the graduated cylinder, check whether it looks like it
theoretically should, and if so, find the speed of sound
from its slope, with error bars, as discussed in appendix
\ref{appendix:graphing}.

Analyze part B either by using the same technique (if you
took data for multiple frequencies) or just by solving algebraically for
the speed of sound.

The effective length of the cylinder in part A should be
increased by 0.4 times its diameter to account for the small
amount of air beyond the end that also vibrates.  For part
B, where the whistling tube is open at both ends, you
should add 0.8 times its diameter.

When estimating error bars from part B, you may be tempted
to say that it must be perfectly accurate, since it's being
done by a computer.  Not so!  The graph only has a certain frequency
resolution, and in addition, the peak may be
a little ragged.

Extract the speed of sound in aluminum from your data in
part C, including error bars.
