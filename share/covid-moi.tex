\addtocounter{chapter}{-1}
\renewcommand\thechapter{c1.13a}
\lab{The moment of inertia}\label{lab:covid-moi}

\section*{About this lab}

\covid\ 
It is intended to be used around the 13th week of Physics 221.
It assumes knowledge of the moment of inertia. A few necessary facts about simple harmomic
motion are introduced as needed, so that topic does not need to have been covered already
in the lecture class.

\section*{Theory}
If you pick up a pen between your thumb and middle finger and wiggle it,
you are sensing the pen's moment of inertia. By definition, the moment of
inertia of an object is
\begin{equation}
  I = \int r^2 \der m,
\end{equation}
where $\der m$ is an infinitesimal element of mass, and $r$ is its distance
from the axis of rotation. Convince yourself that this has units of $\kgunit\cdot\munit^2$.

If we consider the pen as a rod with uniform mass per unit
length $\mu$, then we have $\der m=(\der m/\der x)\der x=\mu \der x$, so if the length of the pen is $a$, then this
integral becomes
\begin{equation}
  I = \int_{-a/2}^{a/2} x^2 \mu \der x = \frac{1}{12}\mu a^3,
\end{equation}
or, in terms of the total mass $M$,
\begin{equation}
  I = \frac{1}{12}Ma^2.
\end{equation}
Check that this has the right units, so that the only reason to do the integral was to find the unitless $1/12$ in front.

In this lab, we will be rotating an index card, which is a rectangular sheet of mass. Let its dimensions be $a$ and $b$.
Using iterated integrals, one can show that when such a sheet is rotated about a line passing through its center, the
moment of inertia becomes
\begin{equation}
  I = \frac{1}{12}M\left(a^2+b^2\right).
\end{equation}
Check that this makes sense when $b=0$. Use the Pythagorean theorem to give a simple interpretation
of the quantity $a^2+b^2$.

\emph{Parallel axis theorem:} Suppose that a body has mass $M$ and moment of inertia $I_0$ for rotation about a certain axis passing
through its center of mass. Now we define a second axis of rotation, parallel to the first, at a distance $d$ from it.
Then the moment of inertia for rotation about this new axis is $I=I_0+Md^2$.
Check that this equation has units that make sense.

Suppose that a pendulum consists of a rigid body that rotates about a certain axis, and the variables $I$, $M$, and $d$ are
as defined above. Then the period of the pendulum is
\begin{equation}
  T = 2\pi \sqrt{\frac{I}{Mgd}}.
\end{equation}
Check that the units of this equation make sense.

To get good precision, it's important to make sure that the card is accurately rectangular, so it's a good
idea to use a card that was accurately manufactured. To make the parallel axis theorem work, we need the
axis of rotation to be parallel to the original one, i.e., in this case, perpendicular to the card. Therefore,
you want to make sure that the card doesn't flutter as it swings.

\section*{Experiment}
Set up your index card as a pendulum, by sticking a pin through it at some off-center point and
suspending it from a vertical surface such as a board or a piece of cardboard. Measure the period
and the relevant dimensions.

You will want to get friction as small as possible, so that you can let it swing for many cycles
before stopping your stopwatch. For example, if you can get it to swing for 20 cycles before the
oscillations stop or become too small to see, then the stopwatch timing errors get divided by 20.

\section*{Analysis}
Compare with theory. Do an error analysis including a propagation of errors, and use your
error bars in comparing with theory, giving a probabilistic interpretation of the result.
For reference on error analysis, see appendices 2 and 3 of the lab manual at
\url{http://lightandmatter.com/lab_221.pdf}.

Note: In principle, there is a correction to the period due to damping. For damping in which
the frictional torque is proportional to the angular velocity, this correction is
a factor of $(1-1/4Q^2)^{-1/2}$, where $Q$ is the ``quality factor'' (see the Wikipedia
article ``Q factor.'') But friction is fairly weak in this experiment, so $Q$ is at least
$\sim 10$, and therefore this correction is negligible.
