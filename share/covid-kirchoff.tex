\renewcommand\thechapter{c2.10}
\lab{The loop and junction rules}\label{lab:kirchoff}

\covid\ 
It is intended to be used around the 10th week in Physics 222.

\apparatus
\equip{1.5 V battery}
\equip{battery holder}
\equip{multimeter}
\equip{\#0 or \#1 Phillips screwdriver (for replacing a fuse if necessary)}
\equip{replacement fuses for multimeter}
\equip{banana-plug cables (made by the student)}
\equip{2 resistors, 10-30 $\zu{k}\Omega$}
\equip{3 or 4 resistors, 100 $\Omega$ to 600 $\Omega$}

\goal{Test the loop and junction rules in two electrical circuits.}

\introduction

If you ask physicists what are the most fundamentally
important principles of their science, almost all of them
will start talking to you about conservation laws. A
conservation law is a statement that a certain measurable
quantity cannot be changed. A conservation law that is easy
to understand is the conservation of mass. No matter what
you do, you cannot create or destroy mass.

The two conservation laws with which we will be concerned in
this lab are conservation of energy and conservation of
charge. Energy is related to voltage, because voltage is
defined as $V=PE/q$. Charge is related to current, because
current is defined as $I=\Delta q/\Delta t$.

Conservation of charge has an important consequence for
electrical circuits:

When two or more wires come together at a point in a DC circuit,
the total current entering that point equals the total current
leaving it.

Such a coming-together of wires in a circuit is called a
junction. If the current leaving a junction was, say,
greater than the current entering, then the junction would
have to be creating electric charge out of nowhere. (Of
course, charge could have been stored up at that point and
released later, but then it wouldn't be a DC circuit --- the
flow of current would change over time as the stored
charge was used up.)

Conservation of energy can also be applied to an electrical
circuit. The charge carriers are typically electrons in
copper wires, and an electron has a potential energy equal
to $-eV$. Suppose the electron sets off on a journey through a
circuit made of resistors. Passing through the first
resistor, our subatomic protagonist passes through a voltage
difference of $\Delta V_1$, so its potential 
energy changes by $-e\Delta V_1$. To use
a human analogy, this would be like going up a hill of a
certain height and gaining some gravitational potential
energy. Continuing on, it passes through more voltage
differences, $-e\Delta V_2$, $-e\Delta V_3$, and so on. Finally, in a moment of
religious transcendence, the electron realizes that life is
one big circuit --- you always end up coming back where you
started from. If it passed through $N$ resistors before
getting back to its starting point, then the total change in
its potential energy was 
\begin{equation*}
	-e\left(\Delta V_1+\ldots+\Delta V_N\right) \qquad .
\end{equation*}
But just as there is no such
thing as a round-trip hike that is all downhill, it is not
possible for the electron to have any net change in
potential energy after passing through this loop --- if so,
we would have created some energy out of nothing. Since the
total change in the electron's potential energy must be
zero, it must be true that $\Delta V_1+\ldots+\Delta V_N=0$.
This is the loop rule:

The sum of the voltage differences around any closed loop
in a circuit must equal zero.

When you are hiking, there is an important distinction
between uphill and downhill, which depends entirely on which
direction you happen to be traveling on the trail.
Similarly, it is important when applying the loop rule to be
consistent about the signs you give to the voltage
differences, say positive if the electron sees an increase
in voltage and negative if it sees a decrease along its
direction of motion.

\observations

Because this lab uses a lot of different resistors, I found it helpful to
fold a piece of scotch tape around each one and write the resistance
value on it. The color codes are very small and hard to read.

\labpart{ The junction rule}

Construct a circuit like the one in the figure, using the battery
as your voltage source. To make things more
interesting, don't use equal resistors, but do use resistances that
are on the same order of magnitude as each other. Use 
resistors with values in the range of about 10 to 30 $\zu{k}\Omega$.
If they're much higher than that, the currents will be too low for the 
meter to measure accurately.
If they're much smaller than that, the
multimeter's internal resistance when used as an
ammeter becomes nonnegligible in comparison.

\fig{em-kir-circuit}

Insert your multimeter
in the circuit to measure all three currents that you need
in order to test the junction rule. Make sure that all three currents
can be measured on the same voltage scale, which probably means either
all of them on the 200 $\mu\zu{A}$ scale or all on the 2 mA scale.
Switching scales causes very large errors ($\sim10$\%!) because
the different scales are not well calibrated in any absolute sense.

If you get a zero current measurement when it doesn't seem like you
should, try the following debugging steps:

\begin{enumerate}
\item I find the WeePro's plugs confusing, because the mA plug is the same
as the V plug, but different from the A plug. Make sure you're in the right
plug for the current scale you've chosen.

\item Take the meter out of the circuit, change it to voltmeter mode, and
make sure there are voltage differences across each of the resistors. If not,
then you probably have a bad connection somewhere.

\item Check the fuse.
\end{enumerate}

\labpart{ The loop rule}

Now come up with a circuit to test the loop rule. Since the
loop rule is always supposed to be true, it's hard to go
wrong here! Make sure that (1) you have at least three resistors in a
loop, (2) the whole circuit is not just a single loop,
and (3) you hook in the power supply in a way
that creates non-zero voltage differences across all the
resistors. Measure the voltage differences you need to
measure to test the loop rule. Here it is best to use fairly
small resistances, so that the multimeter's large internal
resistance when used in parallel as a voltmeter will not
significantly reduce the resistance of the circuit. Do not
use resistances of less than about 100 $\Omega $, however,
or you may burn out the battery, burn up a resistor, or burn your fingers.

\section*{Preparation}

Draw a schematic showing where you will insert the
multimeter in the circuit to measure the currents in part A.

Invent a circuit for part B, and draw a schematic.

Pick a loop from your circuit, and
draw a schematic showing how you will attach the
multimeter in the circuit to measure the voltage differences in part B.

\analysis

Discuss whether you think your observations agree with
the loop and junction rules, taking into account systematic and random errors.

