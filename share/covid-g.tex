\addtocounter{chapter}{-1}
\renewcommand\thechapter{c1.2}
\lab{The earth's gravitational field}\label{lab:covid-g}

\section*{About this lab}

\covid\ 
It is intended to be used during the second week in Physics 205, 210, or 221.
It assumes knowledge of constant-acceleration motion and the relevant kinematical equations.

\observations

In this lab, you will measure the acceleration $g$ of a falling object in the earth's
gravitational field. The method is similar to the one used in lab \ref{lab:covid-measurement},
but this one will be for a free-falling object. Use an object such as a coin or a baseball
that is relatively dense and large, so that air resistance is negligible.

Because a free-falling object accelerates much more rapidly than an object on an inclined
plane, it's rather difficult to get a good measurement of $g$ using a stopwatch. If possible,
use slow-motion video on a smartphone, as described in the previous lab. I found that the
easiest way to determine a time was to play back the slow-motion video and time that with
a stopwatch. Since the video is playing at low speed, stopwatch timing errors are reduced
by that factor compared to a ``live'' stopwatch. Make the height as
large as possible, so that the time is long and can be measured with good relative precision.


\analysis

Determine $g$, with propagation of errors. Compare with the global average value of $9.8\ \munit/\sunit^2$,
using a statistical test as described in the example in appendix 2.

\prelab

\prelabquestion  
Find the equation you will use in order to determine $g$ from the raw data.



