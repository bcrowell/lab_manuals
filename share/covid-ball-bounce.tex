\renewcommand\thechapter{c1.8a}
\lab{Ball Bounce}\label{lab:ballbounce}

\apparatus
\equipn{Android or iOS smartphone, or a computer with a microphone}
\equipn{The WavePad Audio Editor software \url{https://www.nch.com.au/wavepad/index.html}, which runs on Windows, Mac, iOS, and Android.}
\equipn{Some bouncy balls, such as rubber balls, ping-pong balls, or tennis balls.}
\equipn{Access to some hard surface, such as linoleum, tile, concrete.  If this is not available (or if their computer can't be moved) a wooden table, or ceramic plate, or steel saute pan should work.}
\equipn{A ruler or meter stick}
\equipn{Digital scale}

\goal{This experiment uses sound recording software to record the impact time for a ball bouncing on a hard surface. Students can analyse the timing of the bounces to make quantitative comparisons of the elasticity of various balls bouncing on different surfaces}

\introduction

In this class you've learned that energy is a conserved quantity, meaning that it can never be created or destroyed, but only converted from one form to another. We will often assume that the mechanical energy of a system (kinetic plus potential) is conserved, but in real systems we do not always find that this is true. 

If you've ever played with a bouncing ball, you know that if you drop it and leave it alone it will eventually stop bouncing.  In this case you started with some gravitational potential energy, and end with less gravitational potential energy AND no kinetic energy! Clearly the mechanical energy of this system changed, so what happened to it, and how is this consistent with conservation of energy?

\observations

Find a sufficiently hard surface for your drop. Set your microphone or smartphone near by, and open up the WavePad Audio Editor. Determine the height you will drop your ball from and measure its value.  A value of 30-50 cm seems to work well. Measure the mass of the ball you will be dropping.

Press "record" and drop the ball. Record until the ball stops bouncing.  If your ball bounces way off to the side, you may need to repeat the process a few times to get a stable result. Label and save your audio file in case you need to reference it later (part of being a diligent scientist is never discarding your primary data).  

Scroll through the recorded audio waveform and record the times for each of the bounces. Carefully consider the uncertainty in the times you measure.  As you compare the audio waveforms, do you notice any patterns?

Repeat these steps for a few combinations of balls and impact surfaces, so you can compare results. Do at least four combinations (2 balls x 2 surfaces).

\selfcheck

Remember, ALL measured values should have uncertainties. Before you wrap up your data taking, make sure you have everything you need.

\prelab

\prelabquestion
You have a ball with a mass of $m=15.0$g which strikes the ground at $t_1=1.023$s and $t_2=1.456$s. What is the maximum height of the bounce?

\prelabquestion
What mathematical function has the property that its value changes by a multiplicative factor when x is iterated by a fixed amount, e.g., $f(2)=\frac{1}{2}f(1) , f(3)=\frac{1}{2}f(2)=\frac{1}{4}f(1), ...$ ?

\analysis

While the ball is in the air, we can assume that the only force acting on it is gravity, so it is in free-fall.  Use the timing information you measured to calculate the max height for each bounce, and therefore, the mechanical energy of the ball for each bounce.

Make a plot of the mechanical energy vs the number of bounces for each of the combinations you tested.  What sort of function does this plot look like? Use this data to make a \emph{quantitative} measurement of how quickly energy is lost for the different combinations of balls and surfaces you tested.
