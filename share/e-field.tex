\lab{Electric Fields}\label{lab:e-field}

\apparatus
\equip{board and U-shaped probe   ruler}
\equip{DC power supply (Thornton)}
\equip{multimeter}
\equip{scissors}
\equip{stencils for drawing electrode shapes on paper}

\begin{goals}

\item[] To be better able to visualize electric fields and
understand their meaning.

\item[] To examine the electric fields around certain charge distributions.
\end{goals}

\introduction

By definition, the electric field, $E$, at a particular
point equals the force on a test charge at that point
divided by the amount of charge, $E=F/q$. We can plot the electric
field around any charge distribution by placing a test
charge at different locations and making note of the
direction and magnitude of the force on it. The direction of
the electric field at any point P is the same as the
direction of the force on a positive test charge at P. The
result would be a page covered with arrows of various
lengths and directions, known as a ``sea of arrows'' diagram..

In practice, Radio Shack does not sell equipment for
preparing a known test charge and measuring the force on it,
so there is no easy way to measure electric fields. What
really is practical to measure at any given point is the
voltage, $V$, defined as the electrical  energy (potential energy) that a test
charge would have at that point, divided by the amount of
charge ($E/Q$). This quantity would have units of J/C (Joules per
Coulomb), but for convenience we normally abbreviate this
combination of units as volts. Just as many mechanical
phenomena can be described using either the language of
force or the language of energy, it may be equally useful to
describe electrical phenomena either by their electric
fields or by the voltages involved.

Since it is only ever the difference in potential energy (interaction energy)
between two points that can be defined unambiguously, the
same is true for voltages. Every voltmeter has two probes,
and the meter tells you the difference in voltage between
the two places at which you connect them. Two points have a
nonzero voltage difference between them if it takes work (either positive or negative) to
move a charge from one place to another.
If there is a voltage difference between two points in
a conducting substance, charges will move between them just like water will
flow if there is a difference in levels. The charge will
always flow in the direction of lower potential energy (just
like water flows downhill). 

All of this can be visualized most easily in terms of maps
of constant-voltage curves (also known as equipotentials); you may be familiar with topographical
maps, which are very similar. On a topographical map, curves
are drawn to connect points having the same height above sea
level. For instance, a cone-shaped volcano would be
represented by concentric circles. The outermost circle
might connect all the points at an altitude of 500 m, and
inside it you might have concentric circles showing higher
levels such as 600, 700, 800, and 900 m. Now imagine a
similar representation of the voltage surrounding an
isolated point charge. There is no ``sea level'' here, so we
might just imagine connecting one probe of the voltmeter to
a point within the region to be mapped, and the other probe
to a fixed reference point very far away. The outermost
circle on your map might connect all the points having a
voltage of 0.3 V relative to the distant reference point,
and within that would lie a 0.4-V circle, a 0.5-V circle,
and so on. These curves are referred to as constant-voltage curves,
because they connect points of equal voltage.
In this lab, you are going to map out
constant-voltage curves, but not just for an isolated point charge,
which is just a simple example like the idealized example
of a conical volcano.

You could move a charge along a constant-voltage curve in either
direction without doing any work, because you are not moving
it to a place of higher potential energy.  If you do not do
any work when moving along a constant-voltage curve, there must not
be a component of electric force along the surface (or you
would be doing work). A metal wire is a constant-voltage curve. We
know that electrons in a metal are free to move. If there
were a force along the wire, electrons would move because of
it. In fact the electrons would move until they were
distributed in such a way that there is no longer any force
on them. At that point they would all stay put and then
there would be no force along the wire and it would be a
constant-voltage curve. (More generally, any flat piece of conductor
or any three-dimensional volume consisting of conducting
material will be a constant-voltage  region.)

There are geometrical and numerical relationships between
the electric field and the voltage, so even though the
voltage is what you'll measure directly in this lab, you can
also relate your data to electric fields. Since there is not
any component of electric force parallel to a constant-voltage
curve, electric field lines always pass through constant-voltage
curves at right angles. (Analogously, a stream flowing
straight downhill will cross the lines on a topographical
map at right angles.) Also, if you divide the work equation $(\Delta\text{energy})=Fd$ by
$q$, you get $(\Delta\text{energy})/q=(F/q)d$, which translates into $\Delta V=-Ed$.
(The minus sign is because $V$ goes down when some other form of energy is released.)
This means that you
can find the electric field strength at a point P by
dividing the voltage difference between the two constant-voltage
curves on either side of P by the distance between them.
You can see that units of V/m can be used for the $E$ field
as an alternative to the units of N/C suggested by its
definition --- the units are completely equivalent.

\figcaption{em-fie-schematic}{A simplified schematic of the apparatus, being used with
pattern 1 on page \pageref{em-fie-patterns}.}

\figcaption{em-fie-photo}{A photo of the apparatus, being used with pattern 3 on page \pageref{em-fie-patterns}.}

\section*{Method}

The first figure shows a simplified schematic of the apparatus. The power supply
provides an 8 V voltage difference between the two metal electrodes, drawn in black.
A voltmeter measures the voltage difference between an arbitrary reference voltage and
a point of interest in the gray area around the electrodes. The result will be somewhere
between 0 and 8 V. A voltmeter won't actually work if it's not part of a complete circuit,
but the gray area is intentionally made from a material that isn't a very good insulator,
so enough current flows to allow the voltmeter to operate.

The photo shows the actual apparatus. The electrodes are painted with silver paint on a detachable
board, which goes underneath the big board. What you actually see on top is just a piece of
paper on which you'll trace the equipotentials with a pen. The voltmeter is connected to
a U-shaped probe with a metal contact that slides underneath the board, and a hole in the
top piece for your pen. 

Turn your large board upside down. Find the small detachable board
with the parallel-plate capacitor pattern (pattern 1 on page \pageref{em-fie-patterns}) on it, and screw
it to the underside of the equipotential board, with the
silver-painted side facing down toward the tabletop. Use the washers to
protect the silver paint so that it doesn't get scraped off when you tighten
the screws. Now connect the voltage source
(using the provided wires) to the two large screws on either
side of the board. Connect the multimeter so that you can measure
the voltage difference across the terminals of the voltage
source. Adjust the voltage source to give 8 volts.

\fig{em-fie-patterns}\label{em-fie-patterns}

If you press down on the board, you can slip the paper
between the board and the four buttons you see at the
corners of the board. Tape the paper to your board, because the buttons aren't
very dependable. There are plastic stencils in some of the envelopes,
and you can use these to draw the electrodes accurately onto your paper
so you know where they are. The photo, for example, shows pattern 3
traced onto the paper.

Now put the U-probe in place so that
the top is above the equipotential board and the bottom of
it is below the board. You will first be looking for places
on the pattern board where the voltage is one volt --- look
for places where the meter reads 1.0 and mark them through
the hole on the top of your U-probe with a pencil or pen.
You should find a whole bunch of places there the voltage
equals one volt, so that you can draw a nice constant-voltage
curve connecting them. (If the line goes very far or curves
strangely, you may have to do more.) You can then repeat the
procedure for 2 V, 3 V, and so on.  Label each
constant-voltage curve.  Once you've finished tracing the equipotentials,
everyone in your group will need one copy of each of the two
patterns you do, so you will need to photocopy them or
simply trace them by hand.

If you're using the PRO-100 meters, they will try to outsmart you by automatically
choosing a range. Most people find this annoying. To defeat this misfeature,
press the RANGE button, and you'll see the AUTO indicator on the screen turn off.

Repeat this procedure with another pattern. Groups 1 and 4 should do patterns
1 and 2; groups 2 and 5 patterns 1 and 3; groups 3, 6, and 7 patterns 1 and 4.

\prelab

\prelabquestion  Looking at a plot of constant-voltage curves, how could you
tell where the strongest electric fields would be? (Don't
just say that the field is strongest when you're close to
``the charge,'' because you may have a complex charge
distribution, and we don't have any way to see or measure
the charge distribution.)

\prelabquestion What would the constant-voltage curves look
like in a region of uniform electric field (i.e., one in which
the $\vc{E}$ vectors are all the same strength, and all in the
same direction)?

\selfcheck

Calculate at least one numerical electric field value to
make sure you understand how to do it.

You have probably found some constant-voltage curves that form closed
loops. Do the electric field patterns ever seem to close back on themselves?
Make sure you understand why or why not.

Make sure the people in your group all have a copy of each pattern.

\analysis

On each plot, find the strongest and weakest electric fields, and
calculate them.

On top of your plots, draw in electric field vectors. You will then
have two different representations of the field superimposed
on one another.

As always when
drawing vectors, the lengths of the arrows should represent
the magntitudes of the vectors, although you don't need to
calculate them all numerically or use an actual scale.
Remember that
electric field vectors are always perpendicular to constant-voltage
curves.
The electric field
lines point from high voltage to low voltage, just as the
force on a rolling ball points downhill.
