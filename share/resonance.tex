\lab{Resonance}\label{resonancelab}\label{lab:resonance}

\apparatus
\equipn{vibrator}{1/group}
\equipn{stopwatch}{1/group}
\equipn{multimeter}{1/group}
\equip{banana plug cables}
\equip{Leybold 521 545 17-volt DC power supplies}
\equip{24 V AC power supplies}
\equip{spray lubricant}

\begin{goals}

\item[] Observe the phenomenon of resonance.

\item[] Investigate how the width of a resonance depends on
the amount of damping.
\end{goals}

\introduction

To break a wine glass, an opera singer has to sing the right
note. To hear a radio signal, you have to be tuned to the
right frequency. These are examples of the phenomenon of
resonance: a vibrating system will respond most strongly to
a force that varies with a particular frequency.

\figcaption{vw-res-mechanical}{Simplified mechanical drawing of the
vibrator, front view.}
\figcaption{vw-res-electrical}{Electrical setup, top view.}

\apparatus
In this lab you will investigate the phenomenon of resonance
using the apparatus shown in the figure. If the motor is
stopped so that the arms are locked in place, the metal disk
can still swing clockwise and counterclockwise because it is
attached to the upright rod with a flexible spiral spring. A
push on the disk will result in vibrations that persist for
quite a while before the internal friction in the spring
reduces their amplitude to an imperceptible level. This
would be an example of a free vibration, in which energy is
steadily lost in the form of heat, but no external force
\equip{pumps in energy to replace it.}
Suppose instead that you initially stop the disk, but then
turn on the electric motor. There is no rigid mechanical
link to the disk, since the motor and disk are only
connected through the very flexible spiral spring. But the
motor will gently tighten and loosen the spring, resulting
in the gradual building up of a vibration in the disk.

\observations

\labpart{ Period of Free Vibrations}

Start without any of the electrical stuff hooked up. Twist
the disk to one side, release it, and determine its period
of vibration. (Both here and at points later in the lab, you
can improve your accuracy by timing ten periods and dividing
the result by ten.) This is the \emph{natural} period of the
vibrations, i.e., the period with which they occur in the
absence of any driving force.

\labpart{ Damping}

Note the coils of wire at the bottom of the disk. These are
electromagnets. Their purpose is not to attract the disk
magnetically (in fact the disk is made of a nonmagnetic
metal) but rather to increase the amount of damping in the
system. Whenever a metal is moved through a magnetic field,
the electrons in the metal are made to swirl around. As they
eddy like this, they undergo random collisions with atoms,
causing the atoms to vibrate. Vibration of atoms is heat, so
where did this heat energy come from ultimately? In our
system, the only source of energy is the energy of the
vibrating disk. The net effect is thus to suck energy out of
the vibration and convert it into heat. Although this
magnetic and electrical effect is entirely different from
mechanical friction, the result is the same. Creating
damping in this manner has the advantage that it can be made
stronger or weaker simply by increasing or decreasing the
strength of the magnetic field.

Turn off all the electrical equipment and leave it
unplugged from the wall. Connect the circuit shown in the top left of the
electrical diagram, consisting of a power supply to run the
electromagnet. You do not yet need the power
supply for driving the motor. The power supply has a built-in meter labeled ``A,'' for ``Amperes,''
the metric
unit of electrical current.
This will tell you how
much electrical current is flowing through the electromagnet,
giving you a numerical measure of how strong your
damping is. Although this does not directly
tell you the amount of damping force in units of newtons
(the force depends on velocity), the force is proportional to the current.

For this part of the lab, you do not yet need any of the electrical setup
for the motor (right-hand side of the figure).

Once you have everything hooked up, check with your
instructor before plugging things in and turning them on. If
you do the setup wrong, you could blow a fuse, which is no
big deal, but a more serious goof would be to put too much
current through the electromagnet, which could burn it up,
permanently ruining it.

The $Q$ of an oscillator is defined as the number of
oscillations required for damping to reduce the energy of
the vibrations by a factor of 535 (a definition originating
from the quantity $e^{2\pi}$). As planned in your prelab,
measure the $Q$ of the system with the electromagnet turned
off, then with a low current through the electromagnet,
and then a higher current. There are differences among the oscillators.
To compensate for this,
the currents you should use should be based on which oscillator you have. 
These are designed to give $Q$ values of about 8 and 15; if your actual
$Q$'s are much lower, use the spray lubricant to lubricate the bearing.

\begin{tabular}{llll}
      &              & low current & high current \\
group & oscillator   & (high $Q$)  &    (low $Q$)\\
1     &      1       &   140  mA   &    350 mA \\
2     &      2       &   150       &    490  \\
3     &      3B      &   360       &    530 \\
4     &      4       &   280       &    500  \\
5     &      5B      &   400       &    560 \\
6     &      6B      &   430       &    610  \\
7     &      7B      &   420       &    610  
\end{tabular}

You will be using these two current
values throughout the lab.
The Leybold power supply has separate knobs that set both a maximum electrical
current and a maximum voltage. The easiest way to get the desired current is to
\emph{first} turn the voltage knob to zero, then set the current (``A'') knob
to 100\%, and then turn up the voltage. The two currents listed are intended
to give $Q$ values of about 15 and 8. If you find that your $Q$'s are much
lower than this, ask your instructor for help lubricating the bearing.

For parts B and C, you do not yet need the multimeter used for measuring the speed of the motor.

\labpart{ Frequency of Driven Vibration}

Now connect the 24 V AC power supply to the motor.
The motor has coarse (0 to 100) and fine (-6 to +6) adjustment knobs.
These knobs are not
calibrated in Hz, and their readings don't even correlate linearly with
frequency, so to measure the frequency of the motor you need to use the stopwatch.

Set the damping current to the higher of the two values.
Turn on the motor and drive the system at a frequency very
different from the oscillator's natural frequency. You will notice that
it takes a certain amount of time, perhaps a minute or two,
for the system to settle into a steady pattern of vibration.
This is called the \emph{steady-state response}
 to the driving force of the motor.

Does the system respond by vibrating at its natural
frequency, at the same frequency as the motor, or at some
frequency in between?

\labpart{ Resonance}

For parts D and E, you will observe the response of the oscillator
as a function of driving frequency and construct a graph 
with the square of the
amplitude on the $y$ axis and the frequency on the $x$ axis.
The reason for using the square of the amplitude is that part E
is about the full width at half maximum (FWHM), and the
FWHM is
measured between the two points where the \emph{energy} of
the steady-state vibration equals half its maximum value.
Energy is proportional to the square of the amplitude.

The purpose of part D is to determine at what frequency you
obtain the strongest response.

We want to see how the results of parts D and E depend on damping,
so we want graphs for both $Q$ values.
To make this part less time-consuming, you will only do the low-$Q$
graph. A high-$Q$ graph is given below.

\figcaption{vw-res-high-q-graph}{High-Q graph, taken by B. Crowell with oscillator \#3,
natural frequency=$0.645$ Hz.}



When changing from one frequency to another, you have to allow time
for the vibrator to reach its steady state. To make it easier to
tell when the steady state is happening, it helps if you stop the vibration
by hand after changing the frequency; otherwise you can get complicated
patterns of motion in which the exponentially decaying motion left over
from the old frequency happens on top of the new driven motion.

In order to see what's going on, you should make the graph as
you go along, and let the spreadsheet program plot each point
as you type it in (see appendix \ref{appendix:graphing}).  You need to make the software
understand that you intend it to graph every row that you ever
enter into the spreadsheet, not just the ones that already exist when you first
make the graph. To do this, use the mouse to select the first hundred rows of the
two columns you're graphing, including the many blank cells below the actual data.
Then make the graph. Although normally you're expected to do your analysis in this
class totally independently, in this lab it's OK if you just print out multiple
copies of the graph for your group.

As you're making your graph, you will see that you have certain specific
places where you need to fill in data. It can be difficult to ``tune in''
the desired motor frequency based on the markings on the knobs.
For this reason, the motor has an electrical output labeled ``Ux,''
which gives a numerical indication of the frequency of the motor. Hook the multimeter up
to it now, and set the rotary knob to a DC voltage scale as suggested in the figure,
selecting the most precise scale that doesn't overload the meter. Although the readout
is not calibrated in Hz, it is highly reproducible. As you take each data point,
write down the reading on the meter. Then if, for example, you decide that you need
to go back and get a data-point at a frequency that lies between two frequencies
that you already have, you can dial in a reading on the meter that lies between the
two previous readings.

\labpart{ Width of the Resonance}

The goal of this part of the lab is to determine 
the FWHM m4_ifdef([:__sn:],[:, $\Delta f$, :])  of the resonance.
Once you've located the peak of the curve in part D, the parts of the
graph you need to fill in for part E are the sides.

\labpart{ Resonance Strength}

Set the motor to the resonant frequency, i.e., the frequency
at which you have found you obtain the strongest response.
Now measure the amplitude of the vibrations you obtain with
each of the two damping currents. How does the strength of
the resonance depend on damping?

m4_ifdef([:__sn:],[:%
\labpart{Phase (optional)}

Observe the phase response, $\delta$, by comparing the
motion of the disk with the motion of the pointer attached to
the driving arm. Does $\delta$ have the expected behavior
at $\omega \gg \omega_\zu{o}$ and for $\omega \ll \omega_\zu{o}$?
:])

\prelab

\prelabquestion  Plan how you will determine the $Q$ of your oscillator
in part B. [Hint: Note that the energy of a vibration is
proportional to the square of the amplitude.]

\analysis

Parts C, D, E, and F are all quantitative comparisons with theory.
In each case, if your abstract doesn't give a theoretical number and
an experimental number, you're doing something wrong.
In parts D and E, you should analyze both your own graph for the low $Q$ and
the one supplied in the lab manual for the high $Q$.
