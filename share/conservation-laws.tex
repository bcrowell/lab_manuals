\lab{Conservation Laws}\label{lab:conservation-laws}

\apparatus
\emph{Part A:}
\equipn{vacuum pump (Lapine)}{1}
\equipn{electronic balance (Ohaus Scout Pro)}{1}
\equipn{plastic-coated flask}{1/group}
\emph{Part B: }
\equipn{beaker}{1/group}
\equip{propyl alcohol   200 mL/group}
\equip{canola oil   200 mL/group}
\equipn{funnels}{2/group}
\equipn{100-mL volumetric flask}{1/group}
\equip{rubber stopper, fitting in}
\equipn{volumetric flask}{1/group}
\equipn{1-ml pipette and bulb}{1/group}
\equipn{magnetic stirrer}{1/group}
\equipn{triple-beam balance}{1/group}

\longgoal

People believe that objects cannot be made to disappear or
appear.  If you start with a certain amount of matter, there
is no way to increase or decrease that amount.  This type of
rule is called a conservation law in physics, and this
specific law states that the amount of matter is conserved,
i.e., must stay the same.  In order to make this law
scientifically useful, we must define more carefully how the
``amount'' of a substance is to be defined and measured
numerically.  Specifically, there are two issues that
scientifically untrained people would probably not agree on:

\begin{itemize}
\item[] Should air count as matter?  If it has weight, then it
probably should count.  In this lab, you will find out if
air has weight, and, if so, measure its density.

\item[] Should the amount of a substance be defined in terms of
volume, or is mass more appropriate?  In this lab, you will
determine whether mass and/or volume is conserved when water
and alcohol are mixed.
\end{itemize}

\introduction

Styles in physics come and go, and once-hallowed principles
get modified as more accurate data come along, but some of
the most durable features of the science are its conservation
laws.  A conservation law is a statement that something
always remains constant when you add it all up.  Most people
have a general intuitive idea that the amount of a substance
is conserved.  That objects do not simply appear or
disappear is a conceptual achievement of babies around the
age of 9-12 months.  Beginning at this age, they will try
to retrieve a toy that they have seen being
placed under a blanket, rather than just assuming that it no
longer exists. Conservation laws in physics have the
following general features:

\begin{itemize}
\item[] Physicists trying to find new conservation laws will try
to find a measurable, numerical quantity, so that they can
check quantitatively whether it is conserved.  One needs an
operational definition of the quantity, meaning a definition
that spells out the operations required to measure it.

\item[] Conservation laws are only true for closed systems.  For
instance, the amount of water in a bottle will remain
constant as long as no water is poured in or out.  But if
water can get in or out, we say that the bottle is not a
closed system, and conservation of matter cannot be applied to it.

\item[] The quantity should be additive.  For instance, the amount
of energy contained in two gallons of gasoline is twice as
much as the amount of energy contained in one gallon; energy
is additive.  An example of a non-additive quantity is
temperature.  Two cups of coffee do not have twice as high a
temperature as one cup.

\item[] Conservation laws always refer to the total amount of the
quantity when you add it all up.  If you add it all up at
one point in time, and then come back at a later point in
time and add it all up, it will be the same.
\end{itemize}

How can we pin down more accurately the concept of the
``amount of a substance''?  Should a gallon of shaving cream
be considered ``more substantial'' than a brick? At least
two possible quantities come to mind: mass and volume.  Is
either conserved?  Both?  Neither?  To find out, we will
have to make measurements.

We can measure mass by the ``see-saw method'' --- when two
children are sitting on the opposite sides of a see-saw, the
less massive one has to move farther out from the fulcrum to
make it balance.  If we enslave some particular child as our
permanent mass standard, then any other child's mass can
be measured by balancing her on the other side and
measuring her distance from the fulcrum.  A more practical
version of the same basic principle that does not involve
human rights violations is the familiar pan balance
with sliding weights.

Volume is not necessarily so easy to measure.  For instance,
shaving cream is mostly air, so should we find a way to
measure just the volume of the bubbly film itself?  Precise
measurements of volume can most easily be done with liquids
and gases, which conform to a vessel in which they are placed.

Should a gas, such as air, be counted as having any
substance at all?  Empedocles of Acragas (born ca. 492 BC)
was the originator of the doctrine that all material
substances are composed of mixtures of four elements: earth,
fire, water and air.  The idea seems amusingly naive now
that we know about the chemical elements and the periodic
table, but it was accepted in Europe for two thousand years,
and the inclusion of air as a material substance was
actually a nontrivial concept.  Air, after all, was
invisible, seemed weightless, and had no definite shape. 
Empedocles decided air was a form of matter based on
experimental evidence: air could be trapped under water in
an inverted cup, and bubbles would be released if the cup
was tilted.  In China around
300 BC, Zou Yan came up with a similar theory, and his five
elements did not include air.

Does air have weight?  Most people would probably say no,
since they do not feel any physical sensation of the
atmosphere pushing down on them.  A delicate house of cards
remains standing, and is not crushed to the floor by the
weight of the atmosphere.

But compare that to the experience of a dolphin.  A
dolphin might contemplate a tasty herring suspended in front
of it and conjecture that water had no weight, because the
herring did not involuntarily shoot down to the sea floor
because of the weight of the water overhead.  Water does
have weight, however, which a sufficiently skeptical dolphin
physicist might be able to prove with a simple experiment. 
One could weigh a 1-liter metal box full of water and then
replace the water with air and weigh it again.  The
difference in weight would be the difference in weight
between 1 liter of water and 1 liter of air.  Since air
is much less dense than water, this would approximately
equal the weight of 1 liter of water.

Our situation is similar to the dolphin's, as was first
appreciated by Torricelli, whose experiments led him to
conclude that ``we live immersed at the bottom of a sea
of...air.''  A human physicist, living her life immersed in
air, could do a similar experiment to find out whether air
has weight.  She could weigh a container full of air, then
pump all the air out and weigh it again.  When all the
matter in a container has been removed, including the air,
we say that there is a vacuum in the container.   In
reality, a perfect vacuum is very difficult to create.  A
small fraction of the air is likely to remain in the
container even after it has been pumped on with a vacuum
pump.  The amount of remaining air will depend on how good
the pump is and on the rate at which air leaks back in to
the container through holes or cracks.

\mysubsubsection{Cautions}

Please do not break the glassware!  The vacuum flasks and
volumetric flasks are expensive.

The alcohol you will be using in this lab is chemically
different from the alcohol in alcoholic drinks.  It is
poisonous, and can cause blindness or death if you drink it.
It is also flammable.

\observations

\labpart{ Density of air}

You can remove the air from the flask by attaching the
vacuum pump to the vacuum flask with the rubber and glass
tubing, then turning on the pump.  You can use the scale to
determine how much mass was lost when the air was evacuated.

Make any other observations you need in order to find out
the density of air and to estimate error bars for your result.

\labpart{ Is volume and/or mass conserved when two fluids are mixed?}

The idea here is to find out whether volume and/or mass is
conserved when water and alcohol are mixed.  The obvious way
to attempt this would be to measure the volume and mass of a
sample of water, the volume and mass of a sample of alcohol,
and their volume and mass when mixed.  There are two
problems with the obvious method: (1) when you pour one of
the liquids into the other, droplets of liquid will be left
inside the original vessel; and (2) the most accurate way to
measure the volume of a liquid is with a volumetric flask,
which only allows one specific, calibrated volume to be measured.

\fig{me-con-flask}

Here's a way to get around those problems.  Put the magnetic
stirrer inside the flask.  Pour water through a funnel into
a volumetric flask, filling it less than half-way.  (Do not
use the pipette to transfer the water.)  A common mistake is
to fill the flask more than half-way.  Now pour a thin layer
of cooking oil on top.  Cooking oil does not mix with water,
so it forms a layer on top of the water.  (Set aside one
funnel that you will use only for the oil, since the oil
tends to form a film on the sides.)  Finally, gently pour
the alcohol on top.  Alcohol does not mix with cooking oil
either, so it forms a third layer.  By making the alcohol
come exactly up to the mark on the calibrated flask, you can
make the total volume very accurately equal to 100 mL.  In
practice, it is hard to avoid putting in too much alcohol
through the funnel, so if necessary you can take some back
out with the pipette.

If you put the whole thing on the balance now, you know both
the volume (100 mL) and the mass of the whole thing when the
alcohol and water have been kept separate.  Now, mix
everything up with the magnetic stirrer.  The water and
alcohol form a mixture.  You can now test whether the volume
or mass has changed.

If the mixture does not turn out to have a volume that looks
like exactly 100 mL, you can use the following tricks to
measure accurately the excess or deficit with respect to 100
mL.  If it is less than 100 mL, weigh the flask, pipette in
enough water to bring it up to 100 mL, weigh it again, and
then figure out what mass and volume of water you added
based on the change in mass.  If it is more than 100 mL,
weigh the flask, pipette out enough of the mixture to bring
the volume down to 100 mL, weigh it again, and make a
similar calculation using the change in mass and the density
of the mixture.  If you need to pipette out some of the oily
mixture, make sure
to wash and rinse the pipette thoroughly afterwards.

m4_ifdef([:__sn:],[:%
\emph{Note for next week's lab:} You will need safety goggles for next week's lab.
The school considers this to be something students are responsible for buying.
You can buy them at the bookstore, Home Depot, etc. The physics department has accumulated a small
hoard of goggles, and some of you may already have goggles from a chem class. To save money, you may wish to pool your money and buy only the
additional number that are actually needed. If you would like to donate the goggles to us, we'd
appreciate it. You should also wear close-toed shoes to lab next week.
:])

\prelab

\prelabquestion  Suppose that the initial mass in part B is\linebreak[4]
$m_i=280.3\pm0.3$ g and the final mass (after mixing) is $m_f=281.8\pm0.3$ g,
with the error bars as determined in appendix \ref{appendix:basicerranal}.
Calculate the change in mass $\Delta m$, and use the technique described in 
appendix \ref{appendix:errpropagation} to find the error bars on this result.

\prelabquestion As in the examples in appendix \ref{appendix:errpropagation},
find out by how many standard deviations this result for $\Delta m$ differs
from zero, and give a probabilistic interpretation of whether or not this is consistent
with conservation of mass.

\selfcheck

Do a quick analysis of both parts without error analysis.
Plan how you will do your error analysis. 

\analysis

A. If your results show that air has weight, determine the
(nonzero) density of air, with an estimate of your random errors.

B. Decide whether volume and/or mass is conserved when
alcohol and water are mixed, taking into account your
estimates of random errors.
