\renewcommand\thechapter{c2.6}
\lab{Polarization}\label{lab:polarization}

\section*{About this lab}

\covid\ 
It is intended to be used around the 6th week in Physics 222.

\apparatus
\equip{polarizing films (3)}
\equip{web-connected device with LCD screen}
\equip{cardboard}
\equip{scissors}


\begin{goals}

\item[] Make qualitative observations about the polarization of light.

\item[] Test quantitatively the hypothesis that polarization
relates to the direction of the field vectors in an
electromagnetic wave.
\end{goals}

\introduction

 It's common knowledge that there's more to light than meets
the eye: everyone has heard of infrared and ultraviolet
light, which are visible to some other animals but not to
us. Another invisible feature of the wave nature of light is
far less well known. Electromagnetic waves are transverse,
i.e., the electric and magnetic field vectors vibrate in
directions perpendicular to the direction of motion of the
wave. Two electromagnetic waves with the same wavelength can
therefore be physically distinguishable, if their electric
and magnetic fields are twisted around in different
directions. Waves that differ in this way are said to have
different polarizations.

\figcaption{op-pol-emwave}{An electromagnetic wave has electric
and magnetic field vectors that vibrate in the directions
perpendicular to its direction of motion. The wave's direction
of polarization is defined as the line along which the electric
field lies.}

Maybe we polarization-blind humans are missing out on
something. Some fish, insects, and crustaceans can detect
polarization. Most sources of
visible light (such as the sun or a light bulb) are
unpolarized. An unpolarized beam of light contains a random
mixture of waves with many different directions of
polarization, all of them changing from moment to moment,
and from point to point within the beam.

\section{Initial observations}

\labpart{Polarized and unpolarized light in your environment}

Pick up one of the polarizing films. If you could see it under a powerful enough microscope,
you would see what are essentially microscopic wires forming parallel lines. These ``wires''
are typically long chains of molecules impregnated with iodine to make them conducting. When a beam
of light arrives at the film along the direction perpendicular to the plane of the film, the electric field
is in the plane of the film. If the electric field is aligned with the chains, it causes charges
to flow, doing work on them. By conservation of energy, this reduces the energy of the wave.
On the other hand, if the electric field is perfectly perpendicular to the chains, then
the distance it can push a charge is zero, so no work is done, and there is little or no absorption.

If you hold the film in front of one eye, you will mainly notice that the light is dimmer,
which is what makes the film look gray.
If the light coming to you is unpolarized, then some parts of it are blocked completely, some
are not blocked at all, and most are at angles in between so that they are partially blocked.

But if you try looking at different light sources, and rotate the film, you should be able to
find some light sources that are partially or completely polarized. For example, the blue light coming
from the sky is partially polarized along a line connecting that point to the sun.

\labpart{LCD screens}

Try looking at a black and white LCD screen, such as the one on a
calculator or the multimeter that came with the lab kit. You should find that you can almost perfectly
black it out by orienting the film correctly.
Internally, the way this device turns a pixel on or off is that it manipulates the polarization
so that the light will be either blocked or let through by a polarizing film. As a consequence,
the light coming from the screen is almost perfectly polarized. 

Color LCD screens can be a little more complicated. On my desktop computer's monitor, the red, green,
and blue pixels all emit light that is almost perfectly polarized in the same plane (E horizontal and B vertical). But on a tablet
computer I looked at, the polarization was consistent for all the colors, but was rather weak and
was at a 45-degree angle. On my wife's iPhone, the polarization was weak and had different strengths
for the different colors, so that a white region took on a color cast when the film was rotated.
I think this may be due to either a different LCD technology or the fact that these are touch-screen devices.

\section{Testing the model}

Above I've proposed a model of both light waves and the behavior of the polarizing film. Let's
test some predictions of this model.

\labpart{Light from a flame}

Consider a flame as a source of light. It emits light when the charged particles in it
(ions) are accelerated by collisions. Because these collisions are randomly oriented,
we have a prediction that by symmetry, the light from a flame should be unpolarized. Test this
prediction using a flame such as a birthday candle, cigarette lighter, a gas stove burner, or
a gas furnace's pilot light.

You will also probably find that unpolarized light is emitted by lightbulbs of all
the common types  (incandescent, fluorescent, or LED), although this is not directly
predicted by any symmetry principle, and it is possible for solids to emit polarized light.

\labpart{Partial polarization by reflection}

Find a nice bright unpolarized source of light, and then observe its reflection at a moderate grazing
angle by a glossy, insulating surface such as a dinner plate. You should find that the
reflected light is very strongly polarized

When initially unpolarized light is reflected like this,
the symmetry is broken because there is a plane of reflection that is different from the
other planes in three-dimensional space. Our model then allows there to be (but does not
obviously \emph{require}) a difference in the intensity of reflection for the different
polarizations. If you try observing reflection at different angles, you will probably
find that the polarizing effect is most pronounced when the light ray hits the plate at
a moderate angle such as 45 degrees.

Consider the case of reflection straight back along the line perpendicular to the reflecting
surface, such as when you stand directly below an overhead light fixture and hold the plate
flat. In this situation, the incident and reflected rays of light lie along the same
line (or nearly so), and therefore there is no well-defined plane of reflection. Therefore
you should see no polarization in this case.

You will
also probably not see any polarization effect with a polished piece of metal such as
a piece of silverware, or the reflective aluminum backing of a bathroom mirror.
Metals are excellent reflectors of light, so the reflection is almost
perfect at all angles, and there is no partial loss of energy by either polarization.

By the way, the polarizing films we're using are not labeled in any way to show which way
they're oriented --- I don't even know for sure if the manufacturer bothers to make them
all consistent. It's not critical for this lab, but if you're interested, you can use polarization by reflection to figure
out the orientation of your film. For mine, I found that when the plane of reflection
was vertical, the film cut the intensity most strongly when I held it in ``landscape''
orientation. In this situation, the reflected light is preferentially polarized with its
electric field in the vertical plane (in the plane of reflection) and its magnetic field
horizontal. Therefore when I hold my film in landscape orientation, its conducting chains
are oriented up and down.


\figcaption{op-pol-90degsuperpos}{If we know what happens when polarized light comes to the
film at the special angles of 0\degunit
and 90\degunit, then the principle of superposition allows us to predict what will happen
at an intermediate angle such as 45\degunit.}

\section{More than one film}


\labpart{ Two polarizing films}

By now you've seen that as you rotate the film through 90 degrees,
its effect interpolates between the two extremes.
It would not
make sense for the film simply to throw away any waves that
were not perfectly aligned with it, because, as shown in the figure above, a field oriented
on a slant can be analyzed into two vector components, at 0
and 90\degunit with respect to the film. Even if one
component is entirely absorbed, the other component should
still be transmitted.

\fig{op-pol-crossedpolaroids}

Based on these considerations, now think about what will
happen if you look through two polarizing films at an angle
to each other, as shown in the figure above.  What do you predict will
happen as you change the angle $\theta $? How about the special case of $\theta=90\degunit$?

Go ahead and try it. You don't
need any special light source, just the ambient light that
you're using to see in the room.

\labpart{ Three polarizing films}

\fig{op-pol-3polaroids}

Now suppose you start with two films at a 90\degunit angle
to each other, which as you've seen above eliminates all light.
But next we're going to sandwich a third film between them
at a 45\degunit angle, as shown in the figure.
Make a prediction about what will happen, and then try it.
If the result is not what you expected, see if you can figure
out why it happened. If you're stumped, you may want to talk to
your instructor.

\section{Quantitative Observations}

\labpart{ Intensity of light passing through two polarizing films}

In this part of the lab, you will make numerical measurements
of the transmission of completely polarized light when it passes
through a polarizing filter at an angle $\theta$ with respect to the
initial direction of polarization. You can set up the filtering
in one of two ways: (1) If you have a computer monitor that emits highly
polarized light (as I think most desktop monitors do), then you can hold
a filter in front of the monitor and rotate it into different orientations.
(2) If you're using a phone or a tablet, then you will probably need to
use two films, one to clean up the initial polarization and the other
at an angle $\theta$ with respect to the first one. In this situation,
the phone or tablet's partial polarization is probably at a 45-degree angle
relative to horizontal, so your first film will be at that angle.

The technique we will use is to exploit the fact that under controlled
psychological conditions, the human eye can be incredibly good at
judging whether two sources of light are equal in brightness. In the
early days of photography, inventors like Frederick Wratten exploited
this in order to produce surprisingly good calibrations of things like
the sensitivity of photographic emulsions and the opacity of smoked
glass. To set this up, we'll use a software tool that I wrote, which
you can access through a web browser at \url{lightandmatter.com/speckle}.

\fig{op-pol-speckle-setup}

The photo shows the software in use. I've made a cardboard mask and taped
it over part of the monitor. This turns out to be a crucial part of
the technique, for purely psychological reasons, which I'll discuss later. Through the two holes in
the mask, we see two samples of light from the computer screen. The brightnesses
of these two spots is controlled independently by two slider controls at the top
of the screen, which run from 0 to 255 on the monitor's poorly calibrated scale
of brightness. The left spot is currently set to 255 and the right one to 143.
The spot on the left is being seen through
a polarizing filter, held at an angle of 35\degunit, which can be checked against the
protractor scale below. (In real use, I would normally line up the edge of the filter
parallel to one of the lines on the scale, which are at multiples of 10\degunit.)

You should easily be able to tell that the left spot appears brighter than the right
one in the photo. We would then fix this either by adjusting one of the sliders or
by rotating the film. I found that it worked well if I always kept the left side
at 255, rotated the film to conveniently measurable angles, and then fiddled with
the brightness of the right spot to make it match the left one according to my eye.
By that method, my next step in this example would be to raise the right one from
143 to some brighter value, and fiddle with it to make it match.

There are two potential problems with this technique, which we will address.

(1) There is a powerful optical illusion that makes brightness comparisons
like this one amazingly inaccurate unless the stimulus to the eye-brain system
is set up correctly. This is the purpose of the cardboard mask. This is called
the contrast effect (there is a Wikipedia article with that title), discovered
by the French chemist Chevreul.  The figure below is taken from the Wikipedia
article and was crated by user Qef. Of the two inner gray rectangles, the one
on the right appears darker, even though it is actually exactly the same as
the one on the left. The mask eliminates this effect by surrounding each spot
with the same background.
Before I added the mask to my technique, I was getting
results that were off by a factor of 2 or more. 

\fig{op-pol-contrast-effect}

(2) The second issue is that a computer monitor's grayscale, which runs from 0 to 255,
is not a linear scale of brightness. For example, 200 on this scale is not twice as
much light as 100. These days the calibration of the scale is supposed to be standardized
according to a standard called sRGB, but monitors are not usually terribly well calibrated
to that standard, and the standard is not a linear one but rather something that has
been chosen so that photos look pleasant.

To deal with this issue, the software has a second mode, which can be selected by
clicking on the radio button that says ``Speckle.'' This mode is used without the
cardboard mask or the polarizer. Let's say that in the example above, we found that
a grayscale value of 166, without the filter, matched the maximum-brightness spot
as seen through the filter. We would then switch to speckle mode to convert this 166 to an absolute
brightness. In speckle mode, there is a gray square on the left, whose brightness we
would set to 166, while on the right is a square filled with a speckle pattern in which
every pixel is either white or black. 

\fig{op-pol-speckle-calibration}

The figure above shows the software being used in this mode. The square on the left
is set to 166 on the monitor's nonlinear scale of brightness. In the square on the right,
some pixels are randomly set to white and some to black. The fraction of white pixels
is 166/255. If the monitor's grayscale was a linear one, then these would appear
equal in brightness when viewed from far enough away that the individual pixels can't
be distinguished by the eye. In reality, on my monitor, if I step a few meters away from the computer and
take off my glasses, the speckle pattern appears considerably brighter. (Your mileage may
vary depending on how you're viewing the lab manual right now.) So I would walk back to the
computer, adjust the speckle pattern to a higher brightness, and try again until I was
satisfied that they matched as well as possible.

Note that the filter is not 100\% efficient at passing the light of the desired polarization
that it's supposed to let through. For this reason, you will want to divide each of your
brightness measurements $P(\theta)$ by $P(0)$ in order to get a number that can be
compared with the theoretical prediction of $P'/P$. This kind of calibration by
some overall scaling factor is called a normalization. Your normalization will guarantee
that theory and experiment agree perfectly at $\theta=0$, which is a cheat and doesn't
actually help to test the theory.

If possible, do these measurements in a dark room, or at least with dim lighting. Otherwise
you will be affected by the reflection of ambient light from the screen.
I found that with my monitor, in a room that I couldn't darken completely,
this didn't affect the results enough to matter very much, and I think the limitation
at low brightness levels had more to do with my vision.
Whatever the reason, I found that for $\theta\gtrsim 80\degunit$, I was no longer sure if I could
get the unfiltered dark square to be the same brightness as the filtered white square.


\section*{Preparation}

\prelabquestion Given the angle $\theta$ between the polarizing films, predict the 
ratio $|\vc{E}'|/|\vc{E}|$ of the transmitted electric field to the incident
electric field.

\prelabquestion Based on your answer to P1, predict the ratio $P'/P$ of the transmitted
power to the incident power.

\prelabquestion Sketch a graph of your answer to P2. Superimposed on the same graph, show
a qualitative prediction of how it would change if the polaroids were not
100\% perfect at filtering out one component of the field.

\analysis

Graph your results from the quantitative part of the lab, and superimpose a
theoretical curve for comparison. Discuss how your results
compare with theory.

