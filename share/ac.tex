\lab{AC Circuits}\label{lab:ac}

\apparatus
\equipn{Heath coils}{2/group}
\equipn{0.022 $\mu\zu{F}$ capacitor}{2/group}
\equip{0.05 $\mu\zu{F}$ capacitor}
\equip{470 $\Omega$ resistor}
\equip{Pasco PI-8127 function generator (in lab benches in 415)}
\equip{oscilloscope}
\equip{banana to BNC converters}
\equip{alligator clips}

\begin{goals}

\item[] Observe the resonant behavior of an LRC circuit.

\item[] Predict and observe the behavior of capacitances and inductances in parallel and series.

\item[] Observe phase relationships in capacitors and inductors.

\item[] Predict and observe the complex impedance of a capacitor.
\end{goals}

\section*{Preliminary}

For use later in the lab, summarize what you know about how resistances, capacitances, and inductances
combine in parallel and series. An easy way to do this is to use the fact that the corresponding
\emph{impedances} combine like resistances.

\begin{tabular}{lp{30mm}p{30mm}}
          & series & parallel \\
resistances & & \\
capacitances & & \\
inductances & & 
\end{tabular}

%\vfill\pagebreak

\newcommand{\predict}{\emph{Predict:\ }}
\newcommand{\predictnocolon}{\emph{Predict\ }}
\newcommand{\obs}{\emph{Observe:\ }}
\newcommand{\obsnocolon}{\emph{Observe\ }}
\newcommand{\explain}{\emph{Explain\ }}
\newcommand{\measure}{\emph{Measure\ }}

\labpart{Resonance}

\predict
The Heath coils are not intended to be used as inductors, and are not labeled with inductance values, but we
expect them to have $L\sim1\ \zu{H}$ (\emph{Fields and Circuits}, ch.~13, problem 10). Make a rough estimate of the resonant frequency of this series LRC circuit:

\figcaption{em-ac-resonance}{Observing the response of an LRC circuit to a driving voltage.}

\pagebreak

\obs Use the setup shown above to observe the current as a function of the driving frequency. Although the
oscilloscope is a voltmeter, not an ammeter, by using it to measure $V_R$, we are getting a measure of $I_R$
as well, by Ohm's law. Note the grounds, which have to coincide.

For safety, put the function generator's voltage at zero while setting the circuit up, then always use it
after that on the lowest practicable setting. The voltages $V_C$ and $V_L$ can be large, even when the function
generator's voltage is small. When operating near resonance, reduce the voltage as much as possible. It should be
possible to do the whole lab without ever exceeding $\sim 0.3\ \zu{V}$ on the function generator's output.

Determine the actual resonant frequency, and compare with your prediction. (Afterward, turn off the voltage
for safety when changing the circuit for the next part.)

%%%%%%%%%%%%%%%%%%%%%%%%%%%%%%%%%%%%%%

\labpart{Effect of $C$ on the resonant frequency}

\predict
Predict the new resonant frequency when $C$ is changed to $0.05\ \mu\zu{F}$. Use ratios, not a plug-in.

\obs
Check your prediction. (Afterward, turn off the voltage.)

%%%%%%%%%%%%%%%%%%%%%%%%%%%%%%%%%%%%%%

\labpart{Inductances in series and parallel}

Switch back to the default value of $C$ from part A.
Predict and observe the cases where the Heath coil is replaced by \emph{two} Heath coils (1) in series,
and (2) in parallel.
(Turn off the voltage when making changes to the circuit.)

%%%%%%%%%%%%%%%%%%%%%%%%%%%%%%%%%%%%%%

\pagebreak

\labpart{Capacitances in series}

\predict
Predict the resonant frequency when the capacitance in the original circuit is provided by two $0.022\ \mu\zu{F}$
capacitors in \emph{series}.

\obs
Check your prediction. (Afterward, turn off the voltage.)

\labpart{Phase relationship between voltage and current for an inductor}

At some point it becomes inconvenient to take detailed measurements on this circuit because of the grounding of
both the function generator's output and the scope's inputs. For this reason, let's drive it this way:

\figcaption{em-ac-isolated}{Driving the LRC circuit inductively, so that it is isolated from the function generator's ground.}

The varying magnetic field made by the first coil induces a curly electric field, which is felt as a voltage on the second
coil. This is a type of transformer, used in this case for electrical isolation. Now we can take any measurements we like
on the LRC circuit, as long as the grounded sides of the scope's two channels are connected to the same point.

But because of this constraint imposed by the grounds of the scope's inputs, we are forced to set things up in such
a way  that the signs in our measurements of $V_L$ and $V_R$ are inconsistent. For example, if the electric field is
to the right, then channel 1 will read positive, but channel 2 will read negative. To compensate for this, we will
tell the scope to negate channel 1. Press CH 1 MENU and do Invert On.

\obsnocolon the phase relationship between $V_L$ (channel 1) and $I_L$ (which is the same as $I_R$ and therefore has the
same phase as $V_R$, measured on channel 2).

\explain this phase relationship.

\pagebreak

\labpart{Measurement of a complex impedance}

\predictnocolon the complex impedance of the capacitor at this frequency, including both the magnitude and the argument.

For safety, turn the voltage on the function generator to zero before going on. Rearrange the connections to the scope
so you can measure $V_C$ and $V_R$. Think carefully about grounds. It may be necessary to rearrange the order of the
series circuit.

\newcommand{\mytilde}[1]{\widetilde{#1}}
\newcommand{\mytildewide}[1]{\stackrel{\sim}{\smash{#1}\rule{0pt}{1.1ex}}}
%      ... for use with wide characters like W
%      https://tex.stackexchange.com/questions/471312/large-tilde-over-math-symbol-with-appropriate-horizontal-positioning


\measure the amplitudes and phases of these voltages, and use them to find
\begin{equation*}
  Z_C = \frac{\mytilde{V}_C}{\mytilde{I}} = \frac{\mytilde{V}_C}{\mytilde{V}_R} \cdot R.
\end{equation*}

\subsection*{Troubleshooting}

In part E, touching the housings of the coils together may produce a zero signal. Leave some air.

If the function generator won't turn on, check whether the AC power cable is firmly inserted in
the connector for the DC power supply.

Some capacitors are labeled in unclear ways, or the ink has faded. If necessary, use a multimeter
to check their values.

With the Tektronix TDS1001B scopes, I have observed a problem
in which internal interference occurs in the scope when the
time base is set to 1 ms or shorter. This interference looks like
a periodic spike superimposed on the signal. It becomes a problem
if it makes triggering not work right. One possible solution is
to use the run/stop button on the scope to get a frozen image of a single trace,
so you don't need steady triggering.

We have sometimes observed signals that have strange waveforms rather than sine waves.
This problem was fixed by pressing the default setup on the scope. It must result from
some unfortunate interaction between the scope and the circuit.

\subsection*{Additional notes for the instructor}

An alternative technique for dealing with grounding in this lab is to connect all the grounded
inputs and outputs to one point in the circuit, and then use the arithmetic functions on the
scope to display the difference between inputs as necessary.

The Heath coil has a DC resistance of about 62 ohms, and the signal generator
may have an output impedance of several hundred ohms. All the parts of this lab
are constructed so as to be essentially insensitive to this (except that the $Q$ of
the circuit is affected).
