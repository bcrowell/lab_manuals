\renewcommand\thechapter{c2.5}
\lab{The dipole and superposition}\label{lab:dipole-and-superposition}

\section*{About this lab}

\covid\ 
It is intended to be used around the 5th week in Physics 222.
It is a modified version of Minilab 5 from \emph{Fields and Circuits}.

There is an introductory video for this lab at
\url{https://youtu.be/X5yhfo3QASs} .
In the video I refer to Minilab 5 in \emph{Fields and Circuits}, but
it applies equally well to this version of the lab.

\apparatus
\equip{banana plug cables (4, soldered by the student)}
\equip{piece of thread (not in kit)}
\equip{solenoid (made by the student, see below)}
\equip{ferrite disk magnet}
\equip{1.5 V battery and holder (optionally two of these)}
\equip{multimeter}
\equip{two small-value resistors, ca. 5 to 20 ohms}
\equip{alligator clips (6)}
\equip{timing device (stopwatch, computer, or cell phone app)}

\section*{Building a solenoid}

You will need to build a solenoid like the one shown in the
introductory video for this lab. This is the main purpose of the thin
wire provided in the kit. I made the one shown in the video out of 54
turns of 22 gauge wire wrapped around a two-inch part of a Starbucks
cup, which used about 10 m of wire.  For the kits, we will probably
send you both 7 m of 22 gauge wire and a 100 meter roll of
special-purpose 26 gauge magnet wire, which has red enamel insulation
on it. Below I'll describe how to make the solenoid out of either type
of wire. Whichever type of wire you use, carefully count the number of
turns, and write it down, because we'll be reusing this magnet for later
labs, and the number of turns will matter.

\widefigcaption{making-solenoid}{Making a solenoid. 1-2.~Making a cardboard form. 3.~The tail of the wire comes out
through a hole in the cardboard. 4.~200 turns of wire. 5.~Scraping the enamel insulation of the end of each lead.
6.~Soldering heavier pigtails on as strong leads. 7.~More duct tape as strain relief on the pigtails. Banana plug connectors
soldered on the ends.}

If using the 22 gauge wire, then
it turns out that the wire we bought for your kit comes wound on little plastic spools
and works quite well as a solenoid if you just leave it on the spool! All I had to do was strip the ends and wrap a piece of
scotch tape around it.

If using the 26 gauge magnet wire, then you will need to wind about 30 meters worth
of the wire onto your form, counting turns carefully as you go. More turns will make a stronger
magnet. Secure the wire carefully
at the ends so that it doesn't unspool itself, and leave long tails. At the end of each tail,
you will need to scrape off about an inch of the insulation. This can be done with a piece of
fine-grit sandpaper, an emery board, a razor blade or exacto knife, or a screwdriver.
Solder some pieces of heavier wire onto these ends to serve as connectors. Protect the
delicate ends of the wire from damage, e.g., by using many turns of tape to strap them
to the coil. Provide some strain relief on the heavier connector wires, so that if you pull
on them, they won't damage anything. For example, you could tie a knot and then thread the
wire through a hole in the cardboard. Use your multimeter to test that you have good connections.
To do this, connect to the COM and $\Omega$ plugs, turn the rotary knob to the 200 $\Omega$ scale,
and make sure the reading is only a few ohms of resistance. If you want this project to be
a nice reusable and rugged piece of equipment, you could go to the hardware store and buy some
epoxy to coat the whole thing in a tough shell and protect the delicate wires.

You will end up needing to build two solenoids for this lab course. The second one will have
slightly different specs, and is described in lab \ref{lab:covid-electromagnetism}.

\section*{Theory}

Although this lab uses magnetic fields, you do not actually have to know very much at all about magnetic fields.
Theoretically, we expect the following three closely related things to be true:

\begin{enumerate}
\item Magnetic fields from different sources \emph{superpose}, i.e., they add together at any given point in space.
\item The magnetic field is a vector.
\item The magnetic field made by an electromagnet is proportional to the electric
current $I$ passing through it, meaning a measure of the amount of charge per unit time that flows past a certain point.
\end{enumerate}

Really all three of these together are needed in order to make much of a meaningful statement of superposition.
Statement 1 by itself is not really enough. 
Superposition means addition, and statement 2 tells us that the kind of addition we're talking about is vector addition.
Statement 1 talks about fields due to different sources, but we need statement 3 to quantify what kind of sources
we're talking about: electric currents. (Even in a permanent magnet, the sources of the magnetic field are
currents. They are the currents of the atoms orbiting in the atoms.)

Facts 1 and 2 are the same as for electric or gravitational fields, so in this sense the lab is really more like a lab about
fields in general.

The purpose of the lab is to test whether superposition is true in the detailed sense described above.

\section*{Circuits}

The lab also uses a simple electric circuit, but you don't need to know much about circuits to do the lab.
The circuit contains a battery, which provides a voltage of 1.5 V. You can think of the battery as a pump, and the
voltage as a measure of how much pressure the pump provides. When the circuit is hooked up to form a complete loop,
a current flows, which is measured in amperes, A. An ampere is one coulomb per second of charge.

If we hook up
the battery directly to the magnet, a fairly large amount of current will flow. Ordinarily this would be considered
to be an example of a \emph{short circuit}. We would typically expect that a short circuit would rapidly kill the battery, and
would also release a lot of heat.
Because the magnet wire is skinny, and copper is not a perfect conductor, the amount of current is not great
enough to cause these bad things to happen, and it is possible to take data under these conditions, but don't
leave it hooked up for too long.

In order to probe the physical relationships in this lab with other (smaller) currents, we will add a resistor to your circuit.
The resistor causes less current to flow. By putting in various values for this resistance,
we will be able to vary the current without having a power supply with a knob on it. This is discussed in more detail
under Observations.

The smallest resistance
values in your kit are  a couple of resistors in the range of about 8 to 22 ohms.
The battery, solenoid, and resistor go in a series circuit, meaning that they form a single loop, like a necklace.
To turn off the circuit, you can break it at any point so that it no longer forms a complete loop. It's a good idea
not to leave it on all the time, because the resistor can get hot.

As demonstrated in the video, you will hang the ferrite disk magnet from a thread, with its axis in a horizontal plane
and coinciding with the axis of the solenoid. The magnet then has some period of oscillation $T$ as it twists.
The oscillation of the magnet is simple harmonic motion, so the period and
the magnetic field $B$ supplied by the solenoid are related by $T\propto |B|^{-1/2}$.
This is simply the same relationship as for a pendulum except that the gravitational field $g$ is replaced by the magnetic field $B$.
If superposition holds, then varying the current by some factor should change $B$ by the same factor, and therefore
$T$ should change by one over the square root of that factor.

An added complication is that there will be an ambient magnetic field, which consists of the earth's magnetic field
plus whatever field exists because of the building materials in the building you're in. For my setup, I found that
the field of the magnet was actually similar in size to the earth's field. To make it easy to take this into account,
you should align your solenoid and your disk magnet's axes with the earth's field. If you turn off the solenoid, this should
happen naturally as the disk magnet acts like a compass. You can then position the solenoid with its axis along the same line.
By doing this, you make the two magnetic field vectors lie along the same line, so that vector addition is just like addition of
real numbers, taking into account plus and minus signs for direction.
Theory then predicts that $T\propto |B_s+B_a|^{-1/2}$, where $B_s$ is the field of the solenoid and
$B_a$ is the ambient field. For our purposes this is more conveniently expressed as
$\pm T^{-2} \propto B_s+B_a$. Since theory predicts that $B_s\propto I$, we can test the prediction that
a graph of $\pm T^{-2}$ versus $I$ is linear.

\observations

To vary the current, you can use different values of the resistance (including zero additional resistance, meaning
there is only the resistance of the solenoid itself). Taking data with the solenoid turned off gives you a data
point with zero current. When you turn on the solenoid, with some zero or nonzero additional resistance,
you can get two data points with opposite signs for the current, by reversing the connection to the battery.

For each setup, you can measure the current using ammeter. The ammeter goes in series in the circuit. So including
the meter, the final circuit is one big series circuit (like a necklace) with the following components in it:
battery, solenoid, resistor, meter.

If you're using the sequence of topics in \emph{Fields and Circuits}, then you haven't yet learned formally how
to measure a current using an ammeter. In this situation, your instructor will help you with a warm-up exercise
in which you light up a lightbulb and measure the current.

If possible, try to make all your currents under 200 mA, because then you can do all your measurements on the 200 mA
scale of the meter, which gives you 3 or 4 sig figs. This is a huge win over the 10 A scale, which will only give two
sig figs. Don't do some of your measurements on one scale and some on the other, because there is probably quite a big
discrepancy between their calibrations.

Depending on what resistance values you have available, you may find it helpful to make a smaller resistance
by putting two resistors in parallel, as demonstrated in the video with two 10 ohm resistors.
Watch out because
quite a bit of heat will be generated in this double-current setup, as demonstrated in the video when I
burn my fingers. Turn the circuit on only for as long as necessary to get a measurement of the period,
and then turn it off and let the resistors cool.

\analysis


Make the graph and use its linearity to test whether superposition holds.

If you use the meter on its 200 mA scale, then  the 3-4 sig fig current measurements are
probably not a significant source of error. The main source of error would instead be the times.
Do multiple trials of each time measurement in order to estimate the error bars on the raw data.
Use this to estimate error bars on the $T^{-2}$ values (propagation of errors, appendix 3).

These error bars will probably be too small to see on the graph, as will the gaps between the points
and the line. Therefore, fit a line to the
graph (appendix 4), subtract out the linear fit from the data, and get the differences from the
line. Differences like these are called ``residuals.'' Graph the residuals as a function of the
current. See if the error bars on the $T^{-2}$	values
are big enough to explain the residuals. See the examples in appendix 4 of how to interpret graphs
and test a theoretical graph against experimental data points.

\section*{Preparation}

Make your solenoid.

I suggest that	you take all the resistors and small
capacitors out of the little bubble envelope and sort them out
in an egg carton or something similar. The resistors are labeled
with color codes similar to the 4-band color codes described in \emph{Fields and Circuits} ch. 8, figs j and k.
The ones we bought for fall 2020 actually had a slightly different 5-band color code instead.
Because the bands are so tiny and hard to see, it may be easier to determine the values simply
by hooking them up to your multimeter in resistance mode. To do this, connect the resistor
to the COM plug and the $\Omega$ plug. The symbol $\Omega$ stands for ohms. Put the rotary
dial on one of the ohm settings.

If you do want to use the color codes, then here is how the 5-band code on these resistors
seems to be defined. First note how four of the bands are closer together, and one is
slightly set apart from them, with a slightly bigger gap in between. Orient the resistor so that as you go from left to right,
the bands go color-color-color-color-gap-color. Then read the code as in the 4-band code illustrated
in the book, but with an extra digit.

Example: green blue black red (gap) brown\\
 = 5-6-0-2-\ldots = $560\times10^2\ \Omega = 56\ \zu{k}\Omega$.

(I think the rightmost band is always brown on these, and it indicates either a 1\% tolerance
or a 100 ppm temperature coefficient.)
For the power of 10, I think gold indicates $10^{-1}$, and silver is $10^{-2}$. (Gold and silver
do not represent tolerances as in the 4-band color codes.) The small-value resistors you want
for this lab are likely to be the ones with the gold and silver powers of 10.