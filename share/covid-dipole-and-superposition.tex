\renewcommand\thechapter{c2.6a}
\lab{The dipole and superposition}\label{lab:dipole-and-superposition}

\section*{About this lab}

\covid\ 
It is intended to be used during the sixth week in Physics 222.
It is a modified version of Minilab 5 from \emph{Fields and Circuits}.

\apparatus
\equip{banana plug cables (4, soldered by the student)}
\equip{piece of thread (not in kit)}
\equip{solenoid (made by the student, see below)}
\equip{ferrite disk magnet}
\equip{1.5 V battery and holder (optionally two of these)}
\equip{multimeter (optional)}
\equip{two 10-ohm resistors (brown-black-red-gold color code)}
\equip{alligator clips (6)}
\equip{timing device (stopwatch, computer, or cell phone app)}

\section*{Preparation}
You will need to have made at least four banana plug cables, using the method shown in the video that describes that.

You will also need to build a solenoid like the one shown in the
introductory video for this lab. This is the main purpose of the thin
wire provided in the kit. I made the one shown in the video out of 54
turns of 22 gauge wire wrapped around a two-inch part of a Starbucks
cup, which used about 10 m of wire.  For the kits, we will probably
send you both 7 m of 22 gauge wire and a 100 meter roll of
special-purpose 26 gauge magnet wire, which has red enamel insulation
on it. Below I'll describe how to make the solenoid out of either type
of wire. Whichever type of wire you use, carefully count the number of
turns, and write it down, because we'll be reusing this magnet for later
labs, and the number of turns will matter.

If using the 22 gauge wire, then
it turns out that the wire we bought for your kit comes wound on little plastic spools
and works quite well as a solenoid if you just leave it on the spool! All I had to do was strip the ends and wrap a piece of
scotch tape around it.

If using the 26 gauge magnet wire, then you will need to wind about 30 meters worth
of the wire onto your form, counting turns carefully as you go. More turns will make a stronger
magnet. Secure the wire carefully
at the ends so that it doesn't unspool itself, and leave long tails. At the end of each tail,
you will need to scrape off about an inch of the insulation. This can be done with a piece of
fine-grit sandpaper, an emery board, a razor blade or exacto knife, or a screwdriver.
Solder some pieces of heavier wire onto these ends to serve as connectors. Protect the
delicate ends of the wire from damage, e.g., by using many turns of tape to strap them
to the coil. Provide some strain relief on the heavier connector wires, so that if you pull
on them, they won't damage anything. For example, you could tie a knot and then thread the
wire through a hole in the cardboard. Use your multimeter to test that you have good connections.
To do this, connect to the COM and $\Omega$ plugs, turn the rotary knob to the 200 $\Omega$ scale,
and make sure the reading is only a few ohms of resistance. If you want this project to be
a nice reusable and rugged piece of equipment, you could go to the hardware store and buy some
epoxy to coat the whole thing in a tough shell and protect the delicate wires.

You will end up needing to build two solenoids for this lab course. The second one will have
slightly different specs, and is described in lab \ref{lab:covid-electromagnetism}.

\section*{Theory}

Although this lab uses magnetic fields, you do not actually have to know very much at all about magnetic fields except for
the following facts: (1) the magnetic field acts like a vector; (2) magnetic fields from different sources \emph{superpose}, i.e.,
they add like vectors at any given point in space; (3) the magnetic field made by an electromagnet is proportional to the electric
current passing through it, meaning a measure of the amount of charge per unit time that flows past a certain point.
Facts 1 and 2 are the same as for electric or gravitational fields, so in this sense the lab is really more like a lab about
fields in general.

The lab also uses a simple electric circuit, but you don't need to know much about circuits to do the lab.
The circuit contains a battery, which provides a voltage of 1.5 V. You can think of the battery as a pump, and the
voltage as a measure of how much pressure the pump provides. When the circuit is hooked up to form a complete loop,
a current flows, which is measured in amperes, A. An ampere is one coulomb per second of charge. If we hooked up
the battery directly to the magnet, a large amount of current would flow. This would rapidly kill the battery, and
would also release a lot of heat (possibly enough to damage the wire if you used the 26 gauge magnet wire).
To keep this from happening, we will add a resistor to your circuit. The smallest
value supplied in the assortment of resistors we got from our supplier was 10 ohms, which worked reasonably well for me.
The battery, solenoid, and resistor go in a series circuit, meaning that they form a single loop, like a necklace.
To turn off the circuit, you can break it at any point so that it no longer forms a complete loop. It's a good idea
not to leave it on all the time, because the resistor can get hot.

As demonstrated in the video, you will hang the ferrite disk magnet from a thread, with its axis in a horizontal plane
and coinciding with the axis of the solenoid. The magnet then has some period of oscillation $T$ as it twists.
The oscillation of the magnet is simple harmonic motion, so the period and
the magnetic field $B$ supplied by the solenoid are related by $T\propto B^{-1/2}$.
If superposition holds, then varying the current by some factor should change $B$ by the same factor, and therefore
$T$ should change by one over the square root of that factor.

An added complication is that there will be an ambient magnetic field, which consists of the earth's magnetic field
plus whatever field exists because of the building materials in the building you're in. For my setup, I found that
the field of the magnet was actually similar in size to the earth's field. To make it easy to take this into account,
you should align your solenoid and your disk magnet's axes with the earth's field. If you turn off the solenoid, this should
happen naturally as the disk magnet acts like a compass. You can then position the solenoid with its axis along the same line.
By doing this, you make the two magnetic field vectors lie along the same line, so that vector addition is just like addition of
real numbers, taking into account plus and minus signs for direction.


\observations

We are not providing you with a variable power supply, the issue arises of how to vary the current.
The simplest thing you can do is to create three different fields in the following way: (1) solenoid turned off;
(2) solenoid on and reinforcing the earth's field; and (3) solenoid on and opposing the earth's field.
If you do this, then you get currents $-I_0$, 0, and $I_0$. Then in its simplest incarnation,
the lab can be analyzed by graphing $\pm T^{-2}$ against the current, arbitrarily taking $I_0$ to be 1 unit. You can then test theory, which
predicts that this graph should be a line (with the x intercept being the current required in order to
cancel the earth's field).

The video demonstrates some optional, fancier things you can try.

If you include the meter, then you can measure
the current directly. This would let you know, for example, if the currents $\pm I_0$ were not quite the same
because of small changes in the resistance of the alligator clip connections when you make and break those connections.

You can double the current either by putting two resistors in parallel (as shown in the video) or by
using a second battery (if you have one handy, and some way to rig up connections). 
If you do this, then you can get currents $-2I_0$, $-I_0$, 0, $I_0$, and $2I_0$.
Watch out because
quite a bit of heat will be generated in the double-current setup, as demonstrated in the video when I
burn my fingers. Turn the circuit on only for as long as necessary to get a measurement of the period,
and then turn it off and let the resistor cool off.





