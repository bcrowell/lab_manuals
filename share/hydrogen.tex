\lab{Spectroscopy}\label{lab:hydrogen}

\apparatus
\equipn{H gas discharge tube}{3}
\equipn{He gas discharge tube}{3}
\equipn{Hg gas discharge tube (PASCO OS-9286)}{3}
\equipn{N2 gas discharge tube (in green carousel)}{3}
\equipn{spectrometer}{1/group}
\equipn{diffraction grating, 600 lines/mm}{1/group}
\equipn{small screwdriver}{1}
\equipn{black cloth}{1}
\equipn{piece of plywood}{1}
\equipn{block of wood}{1}
\equipn{penlight}{1/group}
\equip{light block}

\section*{Goals}

The lab has four parts. Each group will only do two parts.

\begin{description}

\item[A] Set up the optics of the spectrometer. (Done once by the instructor. Never needs to be redone unless
           something gets messed up.)

\item[B] Calibrate the spectrometer. (Done by all groups.)

\item[C] Observe the visible line spectrum of hydrogen, and determine the mass of the electron.
             (Done by groups 1-3.)

\item[D] Use an energy sum to test a hypothesis about the energy levels of the nitrogen molecule, $\zu{N}_2$.
             (Done by groups 4-6.)
\end{description}

\introduction

What's going on inside an atom? The question would have
seemed nonsensical to physicists before the 20th century ---
the word ``atom'' is Greek for ``unsplittable,'' and there
was no evidence for subatomic particles. Only after Thomson
and Rutherford had demonstrated the existence of electrons
and the nucleus did the atom begin to be imagined as a tiny
solar system, with the electrons moving in elliptical orbits
around the nucleus under the influence of its electric
field. The problem was that physicists knew very well that
accelerating charges emit electromagnetic radiation, as for
example in a radio antenna, so the acceleration of the
electrons should have caused them to emit light, steadily
lose energy, and spiral into the nucleus, all within a microsecond,.

Luckily for us, atoms do not spontaneously shrink down to
nothing, but there was indeed evidence that atoms could emit
light. The spectra emitted by very hot gases were observed
to consist of patterns of discrete lines, each with a
specific wavelength. The process of emitting light always
seemed to stop short of finally annihilating the atom ---
why? Also, why were only those specific wavelengths emitted?

The first step toward understanding the structure of the atom was
Einstein's theory that light consisted of particles
(photons), whose energy was related to their frequency by
the equation $E_{photon}=hf$, or substituting
$f=c/\lambda$, $E_{photon}=hc/\lambda$  .

According to this theory, the discrete wavelengths that had
been observed came from photons with specific energies. It
seemed that the atom could exist only in specific states of
specific energies. To get from an initial state with energy
$E_i$ to a final state with a lower energy $E_f$, conservation
of energy required the atom to release a photon with an
energy of $E_{photon}=E_i-E_f$. 

Not only could the discrete line spectra be explained, but
if the atom possessed a state of least energy (called a
``ground state''), then it would always end up in that
state, and it could not collapse entirely. Knowing the
differences between the energy levels of the atom, it was
then possible to work backwards and figure out the atomic energy levels
themselves. 

\section*{Method}

The apparatus you will use to observe the spectrum of
hydrogen or nitrogen is shown in the first figure below. For a given wavelength, the
grating produces diffracted light at many different angles:
a central zeroth-order line at $\theta=0$, first-order lines on
both the left and right, and so on through higher-order
lines at larger angles.  The line of order $m$ occurs at an
angle satisfying the equation $m\lambda=d\sin\theta$.

To measure a wavelength, you will move the telescope until
the diffracted first-order image of the slit is lined up
with the telescope's cross-hairs and then read off the
angle. Note that the angular scale on the table of the
spectroscope actually gives the angle labeled $\alpha$ in
the figure, not $\theta$.

\subsection*{Sources of systematic errors}

There are three sources of systematic error:

\begin{itemize}
\item[] \emph{angular scale out of alignment:\/} If the angular scale is out
of alignment, then all the angles will be off by a constant amount.

\item[] \emph{factory's calibration of $d\/$:\/} The factory that made the
grating labeled it with a certain spacing (in lines per
millimeter) which can be converted to $d$ (center-to-center
distance between lines). But their manufacturing process is
not all that accurate, so the actual spacing of the lines is
a little different from what the label says.

\item[] \emph{orientation of the grating:\/} Errors will be caused if the grating
is not perpendicular to the beam from the collimator, or if the lines on the
grating are not vertical (perpendicular to the plane of the circle).
\end{itemize}

\subsection*{Eliminating systematic errors}

A trick to eliminate the error due to 
misalignment of the angular scale is to observe the same line on both the
right and the left, and take $\theta$ to be half the difference
between the two angles, i.e., $\theta=(\alpha_R-\alpha_L)/2$.
Because you are subtracting two angles, any source of error
that adds a constant offset onto the angles is eliminated.
A few of the spectrometers have their angular scales out of alignment
with the collimators by as much as a full degree, but that's of
absolutely no consequence if this technique is used.

Regarding the calibration of $d$, 
the first person who ever did this type of experiment simply
had to make a diffraction grating whose $d$ was very precisely
constructed. But once someone
has accurately measured at least one wavelength of one
emission line of one element, one can simply
determine the spacing, $d$, of any grating using a line
whose wavelength is known.

You might think that these two tricks would be enough to get rid
of any error due to misorientation of the grating, but they're not.
They will get rid of any error of the form $\theta\rightarrow\theta+c$
or $\sin\theta\rightarrow c\sin\theta$, but misorientation of the grating
produces errors of the form $\sin\theta\rightarrow\sin\theta+c$.
Part A below describes some additional adjustments that help
to get rid of additional sources of error.

\observations

Turn on the mercury discharge tube right away, to let it get warmed up.

\setcounter{labpartctr}{0}


The second figure below shows the optics from the side, with the
telescope simply looking down the throat of the collimator
at $\theta=0$. You are actually using the optics to let you
see an image of the slit, not the tube itself. The point of
using a telescope is that it provides angular magnification,
so that a small change in angle can be seen visually.

A lens is used inside the collimator to make the light from
the slit into a parallel beam. This is important, because we
are using $m\lambda =d\sin\theta$ to determine the
wavelength, but this equation was derived under the
assumption that the light was coming in as a parallel beam.
To make a parallel beam, the slit must be located accurately
at the focal point of the lens. This adjustment should have
already been done, but you will check later and make sure. A
further advantage of using a lens in the collimator is that
a telescope only works for objects far away, not nearby
objects from which the reflected light is diverging
strongly.  The lens in the collimator forms a virtual image
at infinity, on which the telescope can work.

The objective lens of the telescope focuses the light,
forming a real image inside the tube. The eyepiece then acts
like a magnifying glass to let you see the image. In order
to see the cross-hairs and the image of the slit both in
focus at the same time, the cross-hairs must be located
accurately at the focal point of the objective, right
on top of the image.

\setup

\labpart{Adjusting the optics}

This part describes optical adjustments that
should already have been done for you. You will want to skim it
so that if someone has messed up your spectrometer, you will have
some idea of what could be wrong. If this happens, ask your instructor
for help in fixing the problem.

First you must check that the cross-hairs are at the
focal point of the objective. If they are, then the image
of the slits formed by the objective will be at the same point
in space as the crosshairs. You'll be able to focus your eye on both
simultaneously, and there will be no parallax
error depending on the exact position of your eye.
The easiest way to check this is to look through the telescope
at something far away ($\gtrsim$ 50 m), and move your head left
and right to see if the crosshairs move relative to the image.
Slide the eyepiece in and out to achieve a comfortable focus.
If this adjustment is not correct, you may
need to move the crosshairs in or out; this is done
by sliding the tube that is just outside the eyepiece tube.
(You need to use the small screwdriver to loosen the screw
on the side, which is recessed inside a hole. The hole may
have a dime-sized cover over it.)

The white plastic pedestal should have already been adjusted
properly to get the diffraction grating oriented correctly
in three dimensions, but you should check it carefully.
There are some clever features built into the apparatus to help in
accomplishing this. As shown in the third figure, there are three
axes about which the grating could be rotated. Rotation about axis 1 is
like opening a door, and this is accomplished by rotating the entire pedestal like a lazy Susan. Rotation about
axes 2 (like folding down a tailgate) and 3 are accomplished using the
tripod of screws underneath the pedestal. 
The eyepiece of the telescope is of a type called a Gauss eyepiece, with a diagonal piece of
glass in it.  When the grating is oriented correctly about axes 1 and 2 and the telescope
is at $\theta=0$, a beam
of light that enters through the side of the eyepiece is partially reflected to
the grating, and then reflected from the grating back to the eye. If these two axes
are correctly adjusted, the reflected image of the crosshairs is superimposed on the
crosshairs.

First get a rough initial adjustment of the pedestal by moving the telescope to 90 degrees
and sighting along it like a gun to line up the grating.
Now loosen the screw (not shown in the diagram) that frees the rotation of the pedestal.
Put a desk lamp behind the slits of the collimator, line up the telescope with
the $m=0$ image (which may not be exactly at $\alpha=180$ degrees), remove the desk
lamp, cover the whole apparatus with the black cloth, and position a penlight
so that it shines in through the hole in the side of the eyepiece. Adjust axes
1 and 2. If you're far out of adjustment, you may see part of a circle of light,
which is the reflection of the penlight; start by bringing the circle of light
into your field of view. When you're done, tighten the screw that keeps the pedestal
from rotating. The pedestal is locked down to the tripod screws by the tension in
a spring, which keeps the tips of two of the screws secure in dimples underneath the
platform. Don't lower the screws too much, or the pedestal will no longer
stay locked; make a habit of gently wiggling the pedestal after each adjustment
to make sure it's not floating loose. Two of the spectrometers have the diagonal
missing from their eyepieces, so if you have one of those, you'll have to borrow
an eyepiece from another group to do this adjustment.

For the adjustment of axis 3, place a piece of masking tape so that it covers exactly
half of the slits of the collimator. Put the Hg discharge tube behind the slits.
The crosshairs should be near the edge of the tape in the $m=0$ image. Move the telescope out to
a large angle where you see one of the high-$m$ Hg lines, and adjust the tripod screws
so that the crosshairs are at the same height relative to the edge of the tape.

\labpart{Calibration}

Groups doing part C (hydrogen) will use the blue line from mercury as a calibration.
Groups doing part D (nitrogen) will use the yellow line from helium.
In theory it shouldn't matter what known line we use for calibration, but in practice
there may be small aberrations in the spectrometer, and their effect is minimized by
using calibration lines of nearly the same wavelength as the unknown lines to be measured.

Put the mercury or helium tube behind the collimator. Make sure the
hottest part of the tube is directly in
front of the slits. You will need to use pieces of wood to get the height right.
You want the tube as close to the slits as possible, and
lined up with the slits as well as possible; you can adjust
this while looking through the telescope at an $m=1$ line,
so as to make the line as bright as possible.

If your optics are adjusted correctly,
you should be able to see the microscopic
bumps and scratches on the knife edges of the collimator,
and there should be no parallax of the crosshairs relative
to the image of the slits.

Here is a list of the wavelengths of the most prominent visible Hg and He lines, in nm, to high 
precision.\footnote{The table gives the wavelengths in vacuum. Although we're doing the lab in air, our goal is to find
what the hydrogen or nitrogen wavelengths would have been in vacuum; by calibrating using vacuum wavelengths
for mercury, we end up getting vacuum wavelengths for our unknowns as well.}

Mercury:\\
\noindent\begin{tabular}{llp{40mm}}
404.656  & violet & There is a dimmer violet line nearby at 407.781 nm.\\
435.833  & blue & \\
491.604  & blue-green & Dim. You may also see another blue-green line that is even dimmer. \\
546.074  & green & \\
         & yellow & This is actually a complex set of lines, so it's not useful for calibration.
\end{tabular}

Helium:\\
\noindent\begin{tabular}{llp{40mm}}
447.148 & bright blue-purple & \\
471.314 & dim blue & \\
492.193 & dim green & \\
501.567 & bright green & \\
587.562 & yellow & \\
667.815 & dim red & \\
706.5 & very dim red
\end{tabular}

Start by making sure
that you can find all of the lines lines in the correct sequence --- if
not, then you have probably found some first-order
lines and some second-order ones. If you can find some lines
but not others, use your head and search for them in the
right area based on where you found the lines you did see.
You may see various dim, fuzzy lights through the telescope
--- don't waste time chasing these, which could be coming
from other tubes or from reflections. The real lines will be
bright, clear and well-defined. By draping the black cloth over
the discharge tube and the collimator, you can get rid of stray
light that could cause problems for you or others. The discharge tubes also have
holes in the back; to block the stray
light from these holes, either put the two discharge tubes back
to back or use one of the small ``light blocks'' that slide over the hole.

We will use the wavelength $\lambda_c$ of the blue Hg or yellow He line as a calibration. Measure its two
angles $\alpha_L$ and $\alpha_R$, and check that the resulting value of $\theta_c$ is
close to the approximate ones predicted in prelab question P1. The nominal value of the
spacing of the grating given in that prelab question is not very accurate.
Having measured $\theta_c$,
then we can sidestep the determination of the grating's spacing entirely 
and determine an unknown wavelength $\lambda$
by using the relation
\begin{equation*}
  \lambda = \frac{\sin\theta}{\sin\theta_c} \lambda_c \qquad .
\end{equation*}
The angles are measured using a vernier scale, which
is similar to the one on the vernier calipers you have
already used in the first-semester lab course. Your final
reading for an angle will consist of degrees plus minutes.
(One minute of arc, abbreviated 1', is 1/60 of a degree.)
The main scale is marked every 30 minutes. Your initial,
rough reading is obtained by noting where the zero of the
vernier scale falls on the main scale, and is of the form
``xxx\degunit0' plus a little more'' or ``xxx\degunit30'
plus a little more.'' Next, you should note which line on
the vernier scale lines up most closely with one of the
lines on the main scale. The corresponding number on the
vernier scale tells you how many minutes of arc to add for
the ``plus a little more.''

As a check on your results,
everybody in your group should take independent readings of every
angle you measure in the lab, nudging the telescope to the side after
each reading. Once you have independent results for a particular angle,
compare them. If they're consistent to within one or two minutes of
arc, average them. If they're not consistent, figure out what went
wrong.

\labpart{Spectroscopy of Hydrogen}

If your group is doing this part of the lab, you will study the spectrum of light emitted by
the hydrogen atom, the simplest of all atoms, with just one
proton and one electron. In 1885, before electrons and
protons had even been imagined, a Swiss schoolteacher named
Johann Balmer discovered that the wavelengths emitted by
hydrogen were related by mysterious ratios of small
integers. For instance, the wavelengths of the red line and
the blue-green line form a ratio of exactly 20/27. Balmer
even found a mathematical rule that gave all the wavelengths
of the hydrogen spectrum (both the visible ones and the
invisible ones that lay in the infrared and ultraviolet).
The formula was completely empirical, with no theoretical
basis, but clearly there were patterns lurking in the
seemingly mysterious atomic spectra.

Niels Bohr showed that the energy levels of hydrogen obey a relatively simple equation,\label{bohr-equation}
\begin{equation*}
	E_n = -\frac{mk^2e^4}{2\hbar^2}\cdot\frac{1}{n^2}
\end{equation*}
where $n$ is an integer labeling the level, $k$ is the
Coulomb constant, $e$ is the fundamental unit of charge, $\hbar$
is Planck's constant over $2\pi$, and $m_e$ is the mass of the electron.
All the energies of the photons in the emission spectrum
could now be explained as differences in energy between
specific states of the atom. For instance the four visible
wavelengths observed by Balmer all came from cases where the
atom ended up in the $n=2$ state, dropping down from the
$n=3,$ 4, 5, and 6 states.

Although the equation's sheer size may appear for\-mid\-ab\-le,
keep in mind that the quantity $mk^2e^4/2\hbar^2$ in front is just a
numerical constant, and the variation of energy from one
level to the next is of the very simple mathematical form
$1/n^2$. It was because of this basic simplicity that the
wavelength ratios like 20/27 occurred. The minus sign occurs
because the equation includes both the electron's potential
energy and its kinetic energy, and the standard choice of a
reference-level for the potential energy results in negative values. 

Now try swapping in the hydrogen tube in place of the mercury tube, and go through
a similar process of acquainting yourself with the four lines in its visible spectrum, which are as
follows:

\begin{tabular}{lp{50mm}}
  violet & dim \\
  purple & \\
  blue-green & \\
  red
\end{tabular}

Again you'll again have to make
sure the hottest part of the tube is in front of the
collimator; this requires putting
books and/or blocks of wood under the discharge tube.

\labpart{Energy Sums}

The nitrogen discharge tube is housed in a green plastic
carousel. With the power off, rotate the carousel so that the nitrogen tube is
the one that is in the active position, and then turn on the power.
If you hold a diffraction grating up to your eye and look at the
tube, you will see a remarkable spectrum, unlike the visible light
spectrum of almost any other gas. This is because the $\zu{N}_2$ molecule has
an extremely strong bond, requiring twice the energy to break compared to
otherwise similar gases such as $\zu{H}_2$ or $\zu{O}_2$.
Whereas these other gases would break up into individual atoms under the extreme
conditions present in a discharge tube, the nitrogen molecule holds together, so
that you are seeing the spectrum of the molecule, not the atom.
For this reason, the spectrum of nitrogen contains a large number of lines.

\fig{mo-hel-nitrogen}

But these lines are not random. They occur in sets, each of which looks like
a comb with an approximately equal spacing between the ``teeth.'' The figure
shows a portion of the spectrum, including three sets of lines, which I have
labeled r (red), o (orange to green), and g (green).

I have spent some time
trying to interpret the origin of these lines, and I believe the interpretation
is something like this. Each of these lines is the emission of a photon as
the molecule goes from an initial state to a final state that has less energy.
The initial state has some energy because the electrons are in an excited state
(labeled B by spectroscopists) and also some energy because the molecule is vibrating,
like two masses connected by a spring. The final state has the electrons in
a lower-energy state (labeled A), and is also vibrating. The initial and final
electronic states B and A are the same in all cases, but the vibrational states
differ. An idealized quantum-mechanical vibrator turns out to have a series of
energy states like a ladder with nearly evenly spaced rungs. States higher on
the ``ladder'' are vibrating more violently --- classically, they vibrate with
greater amplitude. The rungs of the vibrational ladder are labeled $v=0,$ 1, 2, and so on.
(Because of the Heisenberg uncertainty principle, some vibrational energy is present
even in the $v=0$ state.) I think the states in the red set are from a state $v$ to
a state $v-3$, i.e., a change of 3 units in the vibrational quantum number. The o
set would be a change of 4, and g a change of 5.

This hypothesis can be tested as follows. If it is true, we could pick out a set of
energy levels like the following example:

\fig{mo-hel-sum}

The letters $e$, $f$, $g$, and $h$ are the energy differences that would be observed
as the energies of the photons. In a set like this, we would have
\begin{equation*}
  h-g = f-e,
\end{equation*}
since each side of the equation would be equal to the energy difference between the
$v=4$ and 5 states of ladder A. Try to find a set of lines that
would be consistent with this interpretation. This may require some trial and
error, but I think it may work if $e$ is one of the lines near the middle of the
r set, $g$ is the orange line that is fourth from the short-wavelength end
of the o set, $f$ is the fifth in that set, and $h$ is in the g set.

\prelab

The week before you are to do the lab, briefly familiarize
yourself visually with the apparatus.

\prelabquestion  
The nominal (and not very accurate) spacing of the grating is stated
as 600 lines per millimeter. From this information, find $d$, and
predict the angles $\alpha_L$ and $\alpha_R$ at which you will observe the blue mercury line.

\widefigcaption{mo-hel-spectrometer}{The spectrometer}
\widefigcaption{mo-hel-optics}{Optics.}
\widefigcaption{mo-hel-orient}{Orienting the grating.}
\widefigcaption{mo-hel-vernier}{Prelab question 2.}

\prelabquestion  Make sure you understand the first three vernier
readings in the fourth figure, and then interpret the fourth reading.

\prelabquestion  In what sequence do you expect to see the mercury or helium lines on
each side? Make a drawing showing the sequence of the angles
as you go out from $\theta $=0.

\prelabquestion
The visible lines of hydrogen come from the $3\rightarrow2$,
$4\rightarrow2$, $5\rightarrow2$, and $6\rightarrow2$
transitions. Based on $E=hf$, which of these should
correspond to which colors?

\prelabquestion

\selfcheck

If you're doing the hydrogen part, then the following is a useful self-check. In homework problem
m4_ifdef([:__sn:],[:%
13-41,%
:])%
m4_ifdef([:__lm:],[:%  
36-11,%
:])%
you calculated the ratio $\lambda_{blue-green}/\lambda_{purple}$.
Before leaving lab, make sure that your wavelengths are consistent with this prediction,
to a precision of no worse than about one part per thousand.

\analysis

Throughout your analysis, remember that this is a high-precision
experiment, so you don't want to round off to less than five
significant figures.

We assume
that the following constants are already known:
\begin{align*}
  e	&= 1.6022\times10^{-19}\ \zu{C}\\
  k	&= 8.9876\times10^9\ \zu{N}\unitdot\munit^2/\zu{C}^2 \\
  h	&= 6.6261\times10^{-34}\ \zu{J}\unitdot\sunit \\
  c     &= 2.9979\times10^8\ \munit/\sunit
\end{align*}

The energies of the four types of visible photons emitted by
a hydrogen atom equal $E_n-E_2$, where $n=3,$ 4, 5, and 6.
Using the Bohr equation, we have
\begin{equation*}
  E_{photon} = A\left(\frac{1}{4}-\frac{1}{n^2}\right) \qquad ,
\end{equation*}
where $A$ is the expression from the Bohr equation that depends
on the mass of the electron. From the two lines you've measured,
extract a value for $A$. If your data passed the self-check above, then
you should find that these values for $A$ agree
to no worse than a few parts per thousand at worst.
Compute an average value of $A$, and extract the mass of the electron,
with error bars.

Finally, there is a small correction that should be made to the result
for the mass of the electron because actually the proton isn't infinitely
massive compared to the electron; in terms of the quantity $m$ given by the
equation on page \pageref{bohr-equation}, the mass of the electron, $m_e$, would
actually be given by $m_e=m/(1-m/m_p)$, where $m_p$ is the mass of the proton,
$1.6726\times10^{-27}$ kg.
