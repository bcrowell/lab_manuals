\addtocounter{chapter}{-1}
\renewcommand\thechapter{\arabic{chapter}b}
\lab{The Mass of the Electron}\label{lab:hydrogen}
\renewcommand\thechapter{\arabic{chapter}}

\apparatus
\equipn{H gas discharge tube}{3}
\equipn{Hg gas discharge tube (PASCO OS-9286)}{3}
\equipn{spectrometer}{1/group}
\equipn{diffraction grating, 600 lines/mm}{1/group}
\equipn{small screwdriver}{1}
\equipn{black cloth}{1}
\equipn{piece of plywood}{1}
\equipn{block of wood}{1}
\equip{light block}

\section*{Goals}

The lab is split up into parts a, b, and c. Student lab groups will do
either part b or part c. This part, b, is a measurement of the mass of the electron.

\introduction

What's going on inside an atom? The question would have
seemed nonsensical to physicists before the 20th century ---
the word ``atom'' is Greek for ``unsplittable,'' and there
was no evidence for subatomic particles. Only after Thomson
and Rutherford had demonstrated the existence of electrons
and the nucleus did the atom begin to be imagined as a tiny
solar system, with the electrons moving in elliptical orbits
around the nucleus under the influence of its electric
field. The problem was that physicists knew very well that
accelerating charges emit electromagnetic radiation, as for
example in a radio antenna, so the acceleration of the
electrons should have caused them to emit light, steadily
lose energy, and spiral into the nucleus, all within a microsecond,.

Luckily for us, atoms do not spontaneously shrink down to
nothing, but there was indeed evidence that atoms could emit
light. The spectra emitted by very hot gases were observed
to consist of patterns of discrete lines, each with a
specific wavelength. The process of emitting light always
seemed to stop short of finally annihilating the atom ---
why? Also, why were only those specific wavelengths emitted?

The first step toward understanding the structure of the atom was
Einstein's theory that light consisted of particles
(photons), whose energy was related to their frequency by
the equation $E_{photon}=hf$, or substituting
$f=c/\lambda$, $E_{photon}=hc/\lambda$  .

According to this theory, the discrete wavelengths that had
been observed came from photons with specific energies. It
seemed that the atom could exist only in specific states of
specific energies. To get from an initial state with energy
$E_i$ to a final state with a lower energy $E_f$, conservation
of energy required the atom to release a photon with an
energy of $E_{photon}=E_i-E_f$. 

Not only could the discrete line spectra be explained, but
if the atom possessed a state of least energy (called a
``ground state''), then it would always end up in that
state, and it could not collapse entirely. Knowing the
differences between the energy levels of the atom, it was
then possible to work backwards and figure out the atomic energy levels
themselves. 

m4_include(basic-spectrometer.tex)

\observations

Turn on the mercury discharge tube right away, to let it get warmed up.

\setcounter{labpartctr}{0}


The second figure below shows the optics from the side, with the
telescope simply looking down the throat of the collimator
at $\theta=0$. You are actually using the optics to let you
see an image of the slit, not the tube itself. The point of
using a telescope is that it provides angular magnification,
so that a small change in angle can be seen visually.

A lens is used inside the collimator to make the light from
the slit into a parallel beam. This is important, because we
are using $m\lambda =d\sin\theta$ to determine the
wavelength, but this equation was derived under the
assumption that the light was coming in as a parallel beam.
To make a parallel beam, the slit must be located accurately
at the focal point of the lens. This adjustment should have
already been done, but you will check later and make sure. A
further advantage of using a lens in the collimator is that
a telescope only works for objects far away, not nearby
objects from which the reflected light is diverging
strongly.  The lens in the collimator forms a virtual image
at infinity, on which the telescope can work.

The objective lens of the telescope focuses the light,
forming a real image inside the tube. The eyepiece then acts
like a magnifying glass to let you see the image. In order
to see the cross-hairs and the image of the slit both in
focus at the same time, the cross-hairs must be located
accurately at the focal point of the objective, right
on top of the image.

\setup

Skim lab \ref{lab:spectrometer} so you have some idea of the way the apparatus
has been carefully aligned in advance by the instructor or lab technician.

\labpart{Calibration}

You will use the blue line from mercury as a calibration.
In theory it shouldn't matter what known line we use for calibration, but in practice
there may be small aberrations in the spectrometer, and their effect is minimized by
using calibration lines of nearly the same wavelength as the unknown lines to be measured.

Put the mercury tube behind the collimator. Make sure the
hottest part of the tube is directly in
front of the slits. You will need to use pieces of wood to get the height right.
You want the tube as close to the slits as possible, and
lined up with the slits as well as possible; you can adjust
this while looking through the telescope at an $m=1$ line,
so as to make the line as bright as possible.

If your optics are adjusted correctly,
you should be able to see the microscopic
bumps and scratches on the knife edges of the collimator,
and there should be no parallax of the crosshairs relative
to the image of the slits.

Here is a list of the wavelengths of the most prominent visible Hg lines, in nm, to high 
precision.\footnote{The table gives the wavelengths in vacuum. Although we're doing the lab in air, our goal is to find
what the hydrogen or nitrogen wavelengths would have been in vacuum; by calibrating using vacuum wavelengths
for mercury, we end up getting vacuum wavelengths for our unknowns as well.}

Mercury:\\
\noindent\begin{tabular}{llp{40mm}}
404.656  & violet & There is a dimmer violet line nearby at 407.781 nm.\\
435.833  & blue & \\
491.604  & blue-green & Dim. You may also see another blue-green line that is even dimmer. \\
546.074  & green & \\
         & yellow & This is actually a complex set of lines, so it's not useful for calibration.
\end{tabular}

Start by making sure
that you can find all of the lines lines in the correct sequence --- if
not, then you have probably found some first-order
lines and some second-order ones. If you can find some lines
but not others, use your head and search for them in the
right area based on where you found the lines you did see.
You may see various dim, fuzzy lights through the telescope
--- don't waste time chasing these, which could be coming
from other tubes or from reflections. The real lines will be
bright, clear and well-defined. By draping the black cloth over
the discharge tube and the collimator, you can get rid of stray
light that could cause problems for you or others. The discharge tubes also have
holes in the back; to block the stray
light from these holes, either put the two discharge tubes back
to back or use one of the small ``light blocks'' that slide over the hole.

We will use the wavelength $\lambda_c$ of the blue Hg line as a calibration. Measure its two
angles $\alpha_L$ and $\alpha_R$, and check that the resulting value of $\theta_c$ is
close to the approximate ones predicted in prelab question P1. The nominal value of the
spacing of the grating given in that prelab question is not very accurate.
Having measured $\theta_c$,
then we can sidestep the determination of the grating's spacing entirely 
and determine an unknown wavelength $\lambda$
by using the relation
\begin{equation*}
  \lambda = \frac{\sin\theta}{\sin\theta_c} \lambda_c \qquad .
\end{equation*}
The angles are measured using a vernier scale, which
is similar to the one on the vernier calipers you have
already used in the first-semester lab course. Your final
reading for an angle will consist of degrees plus minutes.
(One minute of arc, abbreviated 1', is 1/60 of a degree.)
The main scale is marked every 30 minutes. Your initial,
rough reading is obtained by noting where the zero of the
vernier scale falls on the main scale, and is of the form
``xxx\degunit0' plus a little more'' or ``xxx\degunit30'
plus a little more.'' Next, you should note which line on
the vernier scale lines up most closely with one of the
lines on the main scale. The corresponding number on the
vernier scale tells you how many minutes of arc to add for
the ``plus a little more.''

As a check on your results,
everybody in your group should take independent readings of every
angle you measure in the lab, nudging the telescope to the side after
each reading. Once you have independent results for a particular angle,
compare them. If they're consistent to within one or two minutes of
arc, average them. If they're not consistent, figure out what went
wrong.

\labpart{Spectroscopy of Hydrogen}

You will study the spectrum of light emitted by
the hydrogen atom, the simplest of all atoms, with just one
proton and one electron. In 1885, before electrons and
protons had even been imagined, a Swiss schoolteacher named
Johann Balmer discovered that the wavelengths emitted by
hydrogen were related by mysterious ratios of small
integers. For instance, the wavelengths of the red line and
the blue-green line form a ratio of exactly 20/27. Balmer
even found a mathematical rule that gave all the wavelengths
of the hydrogen spectrum (both the visible ones and the
invisible ones that lay in the infrared and ultraviolet).
The formula was completely empirical, with no theoretical
basis, but clearly there were patterns lurking in the
seemingly mysterious atomic spectra.

Niels Bohr showed that the energy levels of hydrogen obey a relatively simple equation,\label{bohr-equation}
\begin{equation*}
	E_n = -\frac{mk^2e^4}{2\hbar^2}\cdot\frac{1}{n^2}
\end{equation*}
where $n$ is an integer labeling the level, $k$ is the
Coulomb constant, $e$ is the fundamental unit of charge, $\hbar$
is Planck's constant over $2\pi$, and $m$ is the mass of the electron.
All the energies of the photons in the emission spectrum
could now be explained as differences in energy between
specific states of the atom. For instance the four visible
wavelengths observed by Balmer all came from cases where the
atom ended up in the $n=2$ state, dropping down from the
$n=3,$ 4, 5, and 6 states.

Although the equation's sheer size may appear for\-mid\-ab\-le,
keep in mind that the quantity $mk^2e^4/2\hbar^2$ in front is just a
numerical constant, and the variation of energy from one
level to the next is of the very simple mathematical form
$1/n^2$. It was because of this basic simplicity that the
wavelength ratios like 20/27 occurred. The minus sign occurs
because the equation includes both the electron's potential
energy and its kinetic energy, and the standard choice of a
reference-level for the potential energy results in negative values. 

Now try swapping in the hydrogen tube in place of the mercury tube, and go through
a similar process of acquainting yourself with the four lines in its visible spectrum, which are as
follows:

\begin{tabular}{lp{50mm}}
  violet & dim \\
  purple & \\
  blue-green & \\
  red
\end{tabular}

Again you'll again have to make
sure the hottest part of the tube is in front of the
collimator; this requires putting
books and/or blocks of wood under the discharge tube.

We will use the purple and blue-green lines to determine the mass of the electron.

\prelab

\prelabquestion  
The nominal (and not very accurate) spacing of the grating is stated
as 600 lines per millimeter. From this information, find $d$, and
predict the angles $\alpha_L$ and $\alpha_R$ at which you will observe the blue mercury line.

\widefigcaption{mo-hel-spectrometer}{The spectrometer}
\widefigcaption{mo-hel-optics}{Optics.}
\widefigcaption{mo-hel-vernier}{Prelab question 2.}

\prelabquestion  Make sure you understand the first three vernier
readings in the fourth figure, and then interpret the fourth reading.

\prelabquestion  For the calibration with mercury,
in what sequence do you expect to see the lines on
each side? Make a drawing showing the sequence of the angles
as you go out from $\theta $=0.

\prelabquestion
The visible lines of hydrogen come from the $3\rightarrow2$,
$4\rightarrow2$, $5\rightarrow2$, and $6\rightarrow2$
transitions. Based on $E=hf$, which of these should
correspond to which colors?

\selfcheck

In homework problem
m4_ifdef([:__mod:],[:%
16-9 in \emph{Modern Physics},%
:])%
m4_ifdef([:__lm:],[:%  
36-11 in \emph{Light and Matter},%
:])%
you calculated the ratio\\ $\lambda_{blue-green}/\lambda_{purple}$.
Before leaving lab, make sure that your wavelengths are consistent with this prediction,
to a precision of no worse than about one part per thousand.

\analysis

Throughout your analysis, remember that this is a high-precision
experiment, so you don't want to round off to less than five
significant figures.

We assume
that the following constants are already known:
\begin{align*}
  e	&= 1.6022\times10^{-19}\ \zu{C}\\
  k	&= 8.9876\times10^9\ \zu{N}\unitdot\munit^2/\zu{C}^2 \\
  h	&= 6.6261\times10^{-34}\ \zu{J}\unitdot\sunit \\
  c     &= 2.9979\times10^8\ \munit/\sunit
\end{align*}

The energies of the four types of visible photons emitted by
a hydrogen atom equal $E_n-E_2$, where $n=3,$ 4, 5, and 6.
Using the Bohr equation, we have
\begin{equation*}
  E_{photon} = A\left(\frac{1}{4}-\frac{1}{n^2}\right) \qquad ,
\end{equation*}
where $A$ is the expression from the Bohr equation that depends
on the mass of the electron. From the two lines you've measured,
extract a value for $A$. If your data passed the self-check above, then
you should find that these values for $A$ agree
to no worse than a few parts per thousand at worst.
Compute an average value of $A$, and extract the mass of the electron,
with error bars.

Finally, there is a small correction that should be made to the result
for the mass of the electron because actually the proton isn't infinitely
massive compared to the electron; in terms of the quantity $m$ given by the
equation on page \pageref{bohr-equation}, the mass of the electron, $m_e$, would
actually be given by $m_e=m/(1-m/m_p)$, where $m_p$ is the mass of the proton,
$1.6726\times10^{-27}$ kg.
