\lab{Magnetism}\label{lab:magnetism}

\widefigcaption{em-dip-fieldvsdistance}{Part B, measuring the variation of the
bar magnet's field with respect to distance}\label{fig:dipbofr}

\apparatus
\equip{bar magnet (stack of 6 Nd)}
\equip{compass}
\equip{Hall effect magnetic field probes}
\equip{LabPro interfaces, DC power supplies, and USB cables}
\equip{2-meter stick}

\goal{Find how the magnetic field of a bar magnet changes with
distance along one of the magnet's lines of symmetry.}

\introduction

\labpart{Qualitative Mapping of the Magnet's Field}
You can use a compass to map out part of the magnetic
field of a bar magnet. The compass is affected by
both the earth's field and the bar magnet's field, and
points in the direction of their vector sum, but if you put the
compass within a few cm of the bar magnet, you're seeing mostly
its field, not the earth's. Investigate the bar magnet's field,
and sketch it in your lab notebook.

\labpart{Variation of Field With Distance: Deflection of a Magnetic Compass}
You can infer the strength of the bar magnet's field at a
given point by putting the compass there and seeing how
much it is deflected from north.

The task can be simplified quite a bit if you restrict
yourself to measuring the magnetic field at points along one
of the magnet's two lines of symmetry, shown in the figure two pages after this one.

\fig{em-dip-symmetry}

If the magnet is flipped across the vertical axis, the north
and south poles remain just where they were, and the field
is unchanged. That means the entire magnetic field is also
unchanged, and the field at a point such as point b, along
the line of symmetry, must therefore point straight up.

If the magnet is flipped across the horizontal axis, then
the north and south poles are swapped, and the field
everywhere has to reverse its direction. Thus, the field at
points along this axis, e.g., point a, must point straight up or down.

Line up your magnet so it is pointing east-west. Choose one
of the two symmetry axes of your magnet, and measure the
deflection of the compass at two points along that axis,
as shown in the second figure, at the end of the lab.
As part of your prelab, you will use vector addition to find an equation
for $B_m/B_e$, the magnet's field in units of the Earth's, in terms
of the deflection angle $\theta$. For your first point, find the
distance $r$ at which the deflection is 70 degrees; this angle is chosen because
it's about as big as it can be without giving very poor relative precision 
in the determination of the magnetic field. For your second data-point,
use twice that distance. By what factor does the field decrease
when you double $r$?

The lab benches contain iron or steel parts that distort the magnetic field.
You can easily observe this simply by putting a compass on the top of the bench and sliding it around
to different places. To work around this problem, lay a 2-meter stick across the space between two
lab benches, and carry out the experiment along the line formed by the stick.
Even in the air between the lab benches, the magnetic field due to the building materials
in the building is significant, and this field varies from place to place.
Therefore you should move the magnet while keeping the compass in one place.
Then the field from the building becomes a fixed part of the background
experienced by the compass, just like the earth's field.

Note that the measurements are very sensitive to the
relative position and orientation of the bar magnet and
compass. 

Based on your two data-points, form a hypothesis about the variation
of the magnet's field with distance according to a power law $B\propto r^p$.

\labpart{Variation of Field With Distance: Hall Effect Magnetometer}

In this part of the lab, you will test your hypothesis about the power
law relationship $B\propto r^p$; you will find out whether the field really
does obey such a law, and if it does, you will determine $p$ accurately.

This part of the lab uses a device called a Hall effect magnetometer for
measuring magnetic fields.  It works by sending an electric current through a
substance, and measuring the force exerted on those moving charges by the surrounding
magnetic field. The probe only measures the component of the
magnetic field vector that is parallel to its own axis. Plug the probe into CH 1 of the
LabPro interface, connect the interface to the computer's USB port, and
plug the interface's DC power supply in to it. Start up version 3 of Logger Pro, and
it will automatically recognize the probe and start displaying magnetic fields
on the screen, in units of mT (millitesla). The probe has two ranges, one that can
read fields up to 0.3 mT, and one that goes up to 6.4 mT. You can select either one using the
switch on the probe. To test your hypothesis with good precision, you need to obtain
data over the widest possible range of fields. Always use the more sensitive 0.3 mT scale
whenever possible, because it will give better precision for low fields. Be careful,
however, because if you expose the probe to a field that's beyond its maximum range, it
will give incorrect readings. Although you have an expectation about the direction of the
field (based both on symmetry arguments and on your qualitative results from part A),
it's a good idea to try orienting the probe in different ways to see what happens.

Two extra complications are that the Earth's field is adding on to the magnet's field, and the
absolute calibration of the probe is very poor by default.  
You can make the computer take care of both of these issues automatically, by zeroing the
sensor (Experiment$>$Zero) when it is exposed only to the Earth's field. This causes the
computer to impose a calibration such that the Earth's field is considered to be exactly zero. You may need
to redo the calibration each time you switch scales. If you then carry out the whole measurement with the
probe and the magnet's field both aligned east-west, the Earth's field has no effect.
You may need to bend the tip of the probe 90 degrees.

\prelab

\prelabquestion  In part B, suppose that
when the compass is 11.0 cm from the magnet, it is
45 degrees away from north. What is
the strength of the bar magnet's field at this location in space,
in units of the Earth's field?

\prelabquestion Find $B_m/B_e$ in terms of the deflection angle $\theta$ measured in part B. As a special
case, you should be able to recover your answer to P1.

\analysis

Determine the magnetic
field of the bar magnet as a function of distance. No error analysis is required. 
Look for a power-law relationship using the log-log graphing technique described in 
appendix \ref{appendix:powerlaws}. Does the power law hold for
all the distances you investigated, or only at large distances?
