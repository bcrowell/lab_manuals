\lab{Magnetism}\label{lab:magnetism}

\widefigcaption{em-dip-fieldvsdistance}{Part B, measuring the variation of the
bar magnet's field with respect to distance.}\label{fig:dipbofr}

\widefigcaption{em-dip-new-method}{Part C, a different method of measuring
the variation of field with distance. The solenoids are shown in cross-section,
with empty space on their interiors and their axes running right-left.}\label{fig:dipnew}

\apparatus
\equip{bar magnet (stack of 6 Nd)}
\equip{compass}
\equip{Hall effect magnetic field probes}
\equip{LabPro interfaces, DC power supplies, and USB cables}
\equip{2-meter stick}
\equipn{Heath solenoids}{2/group}
\equipn{Mastech power supply}{1/group}
\equipn{wood blocks}{2/group}
\equipn{PRO-100 multimeter (in lab bench}{1/group}
\equipn{another multimeter}{1/group}
\equip{D-cell batteries and holders}
\equipn{Cenco decade resistor box}{1/group}

\goal{Find how the magnetic field of a magnet changes with
distance along one of the magnet's lines of symmetry.}

\introduction

\labpart{Variation of Field With Distance: Deflection of a Magnetic Compass}
You can infer the strength of the bar magnet's field at a
given point by putting the compass there and seeing how
much it is deflected from north.

The task can be simplified quite a bit if you restrict
yourself to measuring the magnetic field at points along one
of the magnet's two lines of symmetry, shown in the top figure on the
page three pages after this one.

\fig{em-dip-symmetry}

If the magnet is flipped across the vertical axis, the north
and south poles remain just where they were, and the field
is unchanged. That means the entire magnetic field is also
unchanged, and the field at a point such as point b, along
the line of symmetry, must therefore point straight up.

If the magnet is flipped across the horizontal axis, then
the north and south poles are swapped, and the field
everywhere has to reverse its direction. Thus, the field at
points along this axis, e.g., point a, must point straight up or down.

Line up your magnet so it is pointing east-west. Choose one
of the two symmetry axes of your magnet, and measure the
deflection of the compass at two points along that axis,
as shown in the figure at the end of the lab.
As part of your prelab, you will use vector addition to find an equation
for $B_m/B_e$, the magnet's field in units of the Earth's, in terms
of the deflection angle $\theta$. For your first point, find the
distance $r$ at which the deflection is 70 degrees; this angle is chosen because
it's about as big as it can be without giving very poor relative precision 
in the determination of the magnetic field. For your second data-point,
use twice that distance. By what factor does the field decrease
when you double $r$?

The lab benches contain iron or steel parts that distort the magnetic field.
You can easily observe this simply by putting a compass on the top of the bench and sliding it around
to different places. To work around this problem, lay a 2-meter stick across the space between two
lab benches, and carry out the experiment along the line formed by the stick.
Even in the air between the lab benches, the magnetic field due to the building materials
in the building is significant, and this field varies from place to place.
Therefore you should move the magnet while keeping the compass in one place.
Then the field from the building becomes a fixed part of the background
experienced by the compass, just like the earth's field.

Note that the measurements are very sensitive to the
relative position and orientation of the bar magnet and
compass. 

Based on your two data-points, form a hypothesis about the variation
of the magnet's field with distance according to a power law $B\propto r^p$.

\labpart{Variation of Field With Distance: Hall Effect Magnetometer}

In this part of the lab, you will test your hypothesis about the power
law relationship $B\propto r^p$; you will find out whether the field really
does obey such a law, and if it does, you will determine $p$ accurately.

This part of the lab uses a device called a Hall effect magnetometer for
measuring magnetic fields.  It works by sending an electric current through a
substance, and measuring the force exerted on those moving charges by the surrounding
magnetic field. The probe only measures the component of the
magnetic field vector that is parallel to its own axis. Plug the probe into CH 1 of the
LabPro interface, connect the interface to the computer's USB port, and
plug the interface's DC power supply in to it. Start up version 3 of Logger Pro, and
it will automatically recognize the probe and start displaying magnetic fields
on the screen, in units of mT (millitesla). The probe has two ranges, one that can
read fields up to 0.3 mT, and one that goes up to 6.4 mT.
Select the more sensitive 0.3 mT scale using the
switch on the probe. 

The technique is shown in the bottom figure on the last page of the lab. Identical solenoids
(cylindrical coils of wire) are positioned with their axes coinciding, by lining up
their edges with the edge of the lab bench. When an electrical current passes through
a coil, it creates a magnetic field. At distances that are large compared to the size
of the solenoid, we expect that this field will have the same universal pattern as with
any magnetic dipole. The sensor is positioned on the axis, with wood blocks (not shown)
to hold it up. One solenoid is fixed, while the other is moved to different positions
along the axis, including positions (more distant than the one shown) at which we expect
its contribution to the field at the sensor to be of the universal dipole form.

The key to the high precision of the measurement is that in this configuration, the
fields of the two solenoids can be made to cancel at the position of the probe. Because
of the solenoids' unequal distances from the probe, this requires unequal currents.
Because the fields cancel, the probe can be used on its most sensitive and accurate
scale; it can also be zeroed when the circuits are open, so that the effect of any
ambient field is removed. For example, suppose that at a certain distance $r_m$,
the current $I_m$ through the moving coil has to be five times greater than the current
$I_f$ through the fixed coil at the constant distance $r_f$. Then we have determined that
the field pattern of these coils is such that increasing the distance along the axis
from $r_f$ to $r_m$ causes the field to fall off by a factor of five.

It's a good idea to take data all the way down to $r_m=0$, since this makes it possible
to see on a graph where the field does and doesn't behave like a dipole. Note that the
distances $r_f$ and $r_m$ can't be measured directly with good precision.

The Mastech power supply is capable of delivering a large amount of current, so it can
be used to provide $I_m$, which needs to be high when $r_m$ is large.
At large values of $r_m$, it can be difficult to get a power supply
to give a \emph{small} enough $I_f$. Try using a battery, and further reducing
the current by placing another resistance in series with the coil. The Cenco
decade resistance boxes can be used for this purpose; they are variable resistors
whose resistance can be dialed up as desired using decimal knobs.

For every current measurement, make sure to use the most sensitive possible scale
on the meter to get as many sig figs as possible. This is why the ammeter built into
the Mastech power supply is not useful here.
I found it to be a hassle to measure $I_m$ with an ammeter, because the currents
required were often quite large, and I kept inadvertently blowing the fuse on
the milliamp scale. For this reason, you may actually want to measure $V_m$, the
voltage difference across the moving solenoid. Conceptually, magnetic fields
are caused by moving charges, current is a measure of moving charge, and therefore
current is what is relevant here. But if the DC resistance of the coil is
fixed, the current and voltage are proportional to one another, assuming that the
voltage is measured directly across the coil and the resistance of the banana-plug
connections is either negligible or constant.

\prelab

\prelabquestion  In part B, suppose that
when the compass is 11.0 cm from the magnet, it is
45 degrees away from north. What is
the strength of the bar magnet's field at this location in space,
in units of the Earth's field?

\prelabquestion Find $B_m/B_e$ in terms of the deflection angle $\theta$ measured in part B. As a special
case, you should be able to recover your answer to P1.

\analysis

Determine the magnetic
field of the bar magnet as a function of distance. No error analysis is required. 
Look for a power-law relationship using the log-log graphing technique described in 
appendix \ref{appendix:powerlaws}. Does the power law hold for
all the distances you investigated, or only at large distances?
