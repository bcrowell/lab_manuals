\section*{Method}

The apparatus you will use to observe the spectrum of
hydrogen or nitrogen is shown in the first figure below. For a given wavelength, the
grating produces diffracted light at many different angles:
a central zeroth-order line at $\theta=0$, first-order lines on
both the left and right, and so on through higher-order
lines at larger angles.  The line of order $m$ occurs at an
angle satisfying the equation $m\lambda=d\sin\theta$.

To measure a wavelength, you will move the telescope until
the diffracted first-order image of the slit is lined up
with the telescope's cross-hairs and then read off the
angle. Note that the angular scale on the table of the
spectroscope actually gives the angle labeled $\alpha$ in
the figure, not $\theta$.

\subsection*{Eliminating systematic errors}

A trick to eliminate the error due to 
misalignment of the angular scale is to observe the same line on both the
right and the left, and take $\theta$ to be half the difference
between the two angles, i.e., $\theta=(\alpha_R-\alpha_L)/2$.
Because you are subtracting two angles, any source of error
that adds a constant offset onto the angles is eliminated.
A few of the spectrometers have their angular scales out of alignment
with the collimators by as much as a full degree, but that's of
absolutely no consequence if this technique is used.

Regarding the calibration of $d$, 
the first person who ever did this type of experiment simply
had to make a diffraction grating whose $d$ was very precisely
constructed. But once someone
has accurately measured at least one wavelength of one
emission line of one element, one can simply
determine the spacing, $d$, of any grating using a line
whose wavelength is known.
