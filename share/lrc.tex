\lab{LRC Circuits}\label{lab:lrc}

\apparatus
\equipn{Heath coils}{1/group}
\equipn{0.05 $\mu\zu{F}$ capacitor}{2/group}
\equipn{Pasco PI-9587C generator (under lab benches in 416)}{1/group}
\equipn{oscilloscope}{1/group}

\begin{goals}

\item[] Observe the resonant behavior of an LRC circuit.

\item[] Observe how the properties of the resonance curve change
when the $L$, $R$, and $C$ values are changed.
\end{goals}

\introduction

Radio, TV, cellular phones --- it's mind-boggling to imagine
the maelstrom of electromagnetic waves that are constantly
pass through us and our surroundings. Perhaps equally
surprising is the fact that a radio can pick up a wave with
one particular frequency while rejecting all the others
nearly perfectly. No seasoned cocktail-party veteran could
ever be so successful at tuning out the signals that are not
of interest. What makes radio technology possible is the
phenomenon of resonance, the property of an electrical or
mechanical system that makes it respond far more strongly to
a driving force that varies at the same frequency as that at
which the device naturally vibrates. Just as an opera singer
can only break a wineglass by singing the right note, a
radio can be tuned to respond strongly to electrical forces
that oscillate at a particular frequency.

\figcaption{em-lrc-simplified}{A simplified version of the circuit.}

\mysubsubsection{Circuit}

As shown in the figure, the circuit consists of the
Heath coil, a 0.05 $\mu F$ capacitor,
a 47-ohm resistor, and the sine
wave generator to supply a driving voltage. You will study
the way the circuit resonates, i.e., responds most strongly
to a certain frequency.

Some added complications come from the fact that the function
generator, coil, and oscilloscope do not behave quite like their
idealized versions. The coil doesn't act like a pure inductor; it also
has a certain amount of DC resistance, simply because the wire has finite
resistance. So in addition to the 47-ohm resistor, you will have 62
ohms of resistance coming from the resistance of the wire in
the coil. There is also some internal resistance
from the function generator itself, amounting to 600 ohms when
you use the outputs marked ``high $\Omega$.'' The ``$R$'' of the
circuit is really the sum of these three series resistances.

You will also want to put the oscilloscope in AC coupling mode, which
filters out any DC component (additive constant) on the signal.
The scope accomplishes this filtering by adding in a very small
(20 pF) capacitor, which appears in parallel in the circuit because
an oscilloscope, being a voltmeter, is always used in parallel.
In reality, this tiny parallel capacitance is so small compared to
capacitance of the 0.05 $\mu F$ capacitor that the resulting correction
is negligible (and that's a good thing, because if it wasn't negligible,
the circuit wouldn't be a simple series LRC circuit, and its behavior
would be much more complicated).

\observations

\labpart{ Observation of Resonance}

By connecting the oscilloscope to measure the voltage across
the resistor, you can determine the amount of power,
$P=V^2/R$, being taken from the sine wavegenerator by the
circuit and then dissipated as heat in the resistor. Make
sure that your circuit is hooked up with the resistor
connected to the grounded output of the amplifier, and hook
up the oscilloscope so its grounded connection is on the
grounded side of the resistor. As you change the frequency
of the function generator, you should notice a very strong
response in the circuit centered around one particular
frequency, the resonant frequency $f_o$. (You could measure
the voltage drop across the capacitor or the inductor
instead, but all the pictures of resonance curves in your
textbook are graphs of the behavior of the resistor. The
response curve of a capacitor or inductor still has a peak
at the resonant frequency, but looks very different off to the sides.)

The inductance of your solenoid is roughly 1 H based on the approximation
that it's a long, skinny solenoid (which is not a great approximation here).
 Based on this,
estimate the resonant frequency of your circuit,
\begin{equation*}
	\omega_\zu{o}=\frac{1}{\sqrt{LC}}  \qquad   .
\end{equation*}
Locate  $\omega_\zu{o}$ accurately, and use it to determine the
inductance of the Heath coil accurately.

\labpart{ Effect of Changing C}

Change the capacitance value by putting two capacitors in parallel, and determine the new resonant
frequency. Check whether the resonant frequency changes as
predicted by theory. This is like tuning your radio to a
different frequency. For the rest of the lab, go back to
your original value of $C$.

\labpart{ The Width of the Resonance}

The width of a resonance is customarily expressed as the
full width at half maximum, $\Delta f$, defined as the difference
in frequency between the two points where the power
dissipation is half of its maximum value. Determine the FWHM
of your resonance. You are measuring voltage directly, not
power, so you need to find the points where the amplitude of
the voltage across the resistor drops below its peak
value by a factor of $\sqrt{2}$.

\fig{em-lrc-response}

\labpart{ Effect of Changing R}

Replace the resistor with a 2200-ohm resistor, and remeasure
the FWHM. You should find that the FWHM has increased in
proportion to the resistance. (Remember that your resistance
always includes the resistance of the coil and the output
side of the amplifier.)

\labpart{ Ringing}\label{ringing}

An LRC circuit will continue oscillating even when there is
no oscillating driving force present. This unforced behavior
is known as ``ringing.''Drive your circuit with a square wave.
You can
think of this as if you are giving the circuit repeated
``kicks,'' so that it will ring after each kick.

Choose a frequency many many times lower than the resonant
frequency, so that the circuit will have time to oscillate
many times in between ``kicks.'' You should observe an
exponentially decaying sine wave.

The rapidity of the exponential decay depends on how much
resistance is in the circuit, since the resistor is the only
component that gets rid of energy permanently. The rapidity
of the decay is customarily measured with the quantity $Q$
(for ``quality''), defined as the number of oscillations
required for the potential energy in the circuit to drop by
a factor of 535 (the obscure numerical factor being 
$e^{2\pi}$). For our purposes, it will be more convenient to extract
$Q$ from the equation
\begin{equation*}
	V_{peak,i} = V_{peak,0}\cdot\exp\left[-\frac{\pi t}{QT}\right]
\end{equation*}
where $T$ is the period of the sine wave, $V_{peak,0}$ is the
voltage across the resistor at the peak that we use to
define $t=0$, and $V_{peak,i}$ is the voltage of a later peak,
occurring at time $t$.

Collect the data you will need in order to determine the $Q$
of the circuit, and then do the same for the other resistance value.

\labpart{ The Resonance Curve}

Going back to your low-resistance setup, collect voltage
data over a wide range of frequencies, covering at least a factor
of 10 above and below the resonant frequency. You will want to take closely spaced
data near the resonance peak, where the voltage is changing
rapidly, and less closely spaced points elsewhere. Far above
and far below the resonance, it will be convenient just to
take data at frequencies that change by successive factors of two.

(At very high frequencies, above $10^4$  Hz or so, you may
find that rather than continuing to drop off, the response
curve comes back up again. I believe that this effect arises
from nonideal behavior of the coil at high frequencies: there is
stray capacitance between one loop and the next, and this
capacitance acts like it is in parallel with the coil.) 

In engineering work, it is useful to create a graph of the
resonance curve in which the $y$ axis is in decibels,

\begin{align*}
     \text{db}      &= 10\log_{10}\left(\frac{P}{P_{max}}\right)  \\
     			&= 20\log_{10}\left(\frac{V}{V_{max}}\right)  \qquad   ,  
\end{align*}
and the $x$ axis is a logarithmic frequency scale. (On this
graph, the FWHM is the width of the curve at 3 db below the
peak.) You will construct such a graph from your data.

\analysis

Check whether the resonant frequency changed by the correct
factor when you changed the capacitance.

For both versions of the circuit, compare the FWHM of the
resonance and the circuit's $Q$ to the theoretical equations
\begin{align*}
	\Delta \omega	&= \frac{R}{L} \\
\intertext{and}
	Q	&= \frac{\omega_\zu{o}}{\Delta\omega} \qquad   .
\end{align*}
Note that there are a total of three resistances in series:
the 62-ohm resistance of the coil, the 47-ohm resistor, and
the $\sim 50$-ohm resistance of the sine-wave generator's output.
No error analysis is required, since the main errors are
systematic ones introduced by the nonideal behavior of the
coil and the difficulty of determining an exact, fixed value
for the internal resistance of the output of the amplifier.

Graph the resonance curve  --- you can probably save
yourself a great deal of time by using a computer to do the
calculations and graphing. To do the calculations, you
can go to my web page, www.lightandmatter.com \quad .
Go to the lab manual's web page, and then click on ``data-analysis tool for the LRC
circuits lab''. Once your data are ready to graph, I suggest
using computer software to make
your graph (see Appendix \ref{appendix:graphing}). 

On the high-frequency end, the impedance is dominated by the
impedance of the inductor, which is proportional to
frequency. Doubling the frequency doubles the impedance,
thereby cutting the current by a factor of two and the power
dissipated in the resistor by a factor of 4, which is 6.02
db. Since a factor of 2 in frequency corresponds in musical
terms to one octave, this is referred to as a 6 db/octave
roll-off. Check this prediction against your data. You
should also find a 6 db/octave slope in the limit of low
frequencies --- here the impedance is dominated by the
capacitor, but the idea is similar. (More complex filtering
circuits can achieve roll-offs more drastic than 6 db/octave.)  

\prelab

\prelabquestion Using the rough value of $L$ given in the lab manual, compute
a preliminary estimate of the angular frequency $\omega_\zu{o}$, and find the corresponding
frequency $f_\zu{o}$.

\prelabquestion Express $Q$ in terms of $L$, $R$, and $C$. 

\prelabquestion Show that your answer to P2 has the right units.

\prelabquestion Using the
rough value of $L$ given in the lab manual, plug numbers into your answer to P2,
and make a preliminary estimate of the $Q$ that you expect when using the lower
of the two resistance values. Your result should come out to be 6 (to one sig fig
of precision).

\prelabquestion In part \ref{ringing}, you could measure $t$ and $T$
using the time scale on the scope. However, all we care about is their ratio $t/T$;
think of a technique for determining $t/T$ that is both more precise and easier
to carry out.
