%-------------------------------------------------------------
\lab{The Local Gravitational Field}\label{lab:g}

\emph{Note to the lab technician: The computers to use are any of the following:
Lassie, Dug, Buck, Ribsy, and Mudge.}

\apparatus
\equip{(two stations):}
\equip{vertical plank with electromagnets}
\equip{steel balls (2/station)}
\equip{Linux computers with Audacity installed}
\equip{spirit levels}

\goal{Make a high-precision measurement of the strength of the Earth's
gravitational field, $g$, in Fullerton.}

\introduction
When objects fall, and all forces other than gravity are negligible,
we observe that the acceleration is the same, regardless of the object's
mass, shape, density, or other properties. However, the acceleration does
depend a little bit where on the earth we do the experiment, and even bigger
variations in acceleration can be observed by, e.g., going to the moon.
Thus, this acceleration can be considered as a property of space itself,
and we can refer to it as the gravitational field in that region of space.
Just as you would use a magnetic compass to find out about the magnetic field
in the classroom, you can use dropping masses to find out about the gravitational
field.

In this experiment, you'll measure the gravitational field, $g$, in the
classroom to sufficiently high precision that, if everybody does a good job and
we pool and average everyone's data to reduce random errors, we should be able to
get a value that is measurably different from the generic world-average value you
would find in a textbook.
% /home/bcrowell/Documents/teaching/lab/data/g

\section*{Measuring $g$ precisely}
You will measure $g$, the
acceleration of an object in free fall, using electronic
timing techniques. The idea of the method is that you'll
have two steel balls hanging underneath electromagnets at
different heights. You'll simultaneously turn off the two
magnets by breaking the same electric circuit, causing the balls to drop at
the same moment. The ball dropped from the lower height
$(h_1)$ takes a smaller time $(\Delta t_1)$ to reach the floor, and
the ball released from the greater height $(h_2)$ takes a
longer time $(\Delta t_2)$. The time intervals involved are short
enough that due to the limitations of your reflexes it is
impossible to make good enough measurements with stopwatch. 
Instead, you will record the sounds of the two balls'
impacts on the floor using the computer.  The computer
shows a graph in which the $x$ axis is time and the $y$ axis
shows the vibration of the sound wave hitting the microphone.
You can measure the time between the two visible ``blips''
on the screen.  You will measure three things: $h_1$, $h_2$,
and the time interval $\Delta t_2-\Delta t_1$ between the impact of the
second ball and the first.  From these data, with a little
algebra, you can find $g$.

The experiment would have been easier to analyze if we could
simply drop a single ball and measure the time from when it
was released to when it hit the floor.  But since our timing
technique is based on sound, and no sound is produced when
the balls are released, we need to have two balls.  If
$h_1$, the height of the lower ball, could be made very
small, then it would hit the floor at essentially the same
moment the two balls were released ($\Delta t_1$ would equal 0),
and $\Delta t_2-\Delta t_1$ would be essentially the same as $\Delta t_2$.  But
we can't make $h_1$ too small or the sound would not be loud
enough to detect on the computer.

When measuring $h_1$ and $h_2$, use the spirit level to get the
two-meter stick accurately vertical.

\section*{Using the computer software}

Start up the sound recording program, called Audacity.
Set the record level on high, using the control marked $-\ldots +$ next to
the microphone icon. Try recording a sound by clicking on the red ``record'' button.

\figcaption{me-fre-sound}{Two thumps, as recorded on the
computer through the microphone.}

To measure the time interval between the balls' impacts, you can measure the time at which
each clap occurs, and then subtract. To measure a time, 
use the "I-beam" cursor and click on the feature you're interested in.
Then click on the magnifying glass icon with the plus sign in it to zoom in.
If you zoom in several times like this, you'll start to see that your positioning
of the cursor was a little off. Correct this by clicking closer to the right place,
and then zoom in some more. Continue this process of zooming in and correcting
until you have measured the time as well as possible.
Read the time from the scale at the top of the window.

Troubleshooting:
\begin{itemize}
\item[] You may get this error message: ``Error while opening sound device. Please check the input device settings and the project sample rate.''
Quit the program and restart it.

\item[] Some of the computers have very low or very high gain on their microphones. To work around
this, do the following after you've recorded a sound: type control-A to select the whole recording,
and then do Effect$>$Amplify; the default is to amplify the sound by the maximum amount, which is what you want.
If the gain is too high, it may be necessary to lower it using the operating system's gain control, but
every OS update seems to change the location of this control.

\item[] If sound input isn't working, it may be because the wrong sound input device is selected. Right-click on the volume icon
in the menu bar. Under Edit:Preferences, check Mic select, and under Options do Mic Select:Mic 0.
\end{itemize}

You should make a series of measurements, and
make sure they agree at the level of a few times $10^{-4}$ s; if they
don't, there's something wrong with your technique. Also, you
should check that your result for $g$ makes sense.

Here are some common problems that cause inconsistent or wrong results:
\begin{itemize}
\item[] The balls are brushing against the electrical wires as they fall.
\item[] You're misidentifying the thumps.
\item[] The surface the balls are dropping onto has dents in it. 
\item[] You're not positioning the balls on the same spot on the magnets every time.
\item[] To drop the balls, you should break the circuit by pulling one of the banana-plug connectors out of the plug on the front
panel of the power supply. Don't use the switch; if you use the switch, it takes some time for the magnetic field to decay, and the balls stick for a while before dropping. 
\item[] Audacity will let you keep on making new recordings, stacking the graphs
vertically. However, if you do this you will introduce significant timing
errors. The reason is probably that Audacity is designed for use in multitrack recording
of music, so it tries to play back the previously recorded tracks while recording the
new one, and on cheap sound hardware this causes little timing glitches.
\end{itemize}

\analysis

Extract a value of $g$ from your data.

Derive error bars on your result, using the techniques in appendices \ref{appendix:basicerranal}
and \ref{appendix:errpropagation}.

\selfcheck

Extract the value of $g$, with error bars.
Read Appendix \ref{appendix:errpropagation} for information on how to do error analysis
with propagation of errors; get help from your instructor if necessary.

\prelab

\prelabquestion  \emph{If your instructor has assigned homework problem 
m4_ifdef([:__sn:],[:%
26 from ch. 3 of Mechanics,%
:])%
m4_ifdef([:__lm:],[:%
27 from ch. 3 of Light and Matter,%
:])%
 don't bother turning in another copy of your work for this prelab
question.} Derive an equation for $g$ in terms of the
quantities you'll measure, which are $h_1$,
$h_2$, and the time interval $\Delta t_2-\Delta t_1$. The point of the lab
is to measure $g$, so don't just say ``well of course $g$ is
9.8 $\munit/\sunit^2$.'' (You should check your equation by using the answer
checker for the homework problem.)
