\lab{Torque in Three Dimensions}\label{lab:torque-in-three-dimensions}

\apparatus
\equipn{metal hoop (baby buggy wheel)}{1/group}
\equipn{posts with clamps}{3/group}
\equip{spring scales}
\equipn{vertical stand for use with plumb bob}{1/group}
\equipn{meter stick}{1/group}
\equipn{hooks}{3/group}
\equip{string}
\equipn{plumb bob}{1/group}
\equipn{butcher paper}{1/group}
\equip{protractors}
\equip{compass}
\equip{scissors}

\goal{Test the hypothesis that the total torque and total force on an object are zero, for a system that cannot be analyzed within a plane.}

\fig{me-t3d-setup}

\observations

The basic idea here is to reenact lab \ref{lab:torque} in three
dimensions, so that the $\vc{r}$ and $\vc{F}$ vectors are not confined
to a plane. To make things simple, you'll use a circular hoop as your
object. By taking its center as the axis, and suspending it from three
strings attached at its circumference, you can make the magnitudes of
the $\vc{r}$ vectors all the same.
It's not easy to level the hop with a spirit level, because placing the
spirit level on it tends to make it sag; instead, just eyeball it carefully.
If you do this well, all the $\vc{r}$ vectors lie in the
horizontal plane. (This requires adjusting the knots so that the three
forces are all applied at points in the mid-plane of the wheel.) The
three strings, however, should not be vertical; they should point up
and out at random angles. The whole arrangement should not have any
symmetry.  There will be three force vectors for the three strings,
plus a downward force vector due to gravity. 

The figure on the following page shows some knots that are handy. The overhand on a bight is a secure
knot that can be used to attach the string to a spring scale. The square knot
is a good one for tying on to the wheel, since it can be tightened in order
to put the knot in the mid-plane. Make sure you make a square rather than a granny,
which tends to slip.

\fig{me-t3d-knots}

You can use whatever measuring techniques you need in order to completely determine all the $\vc{r}$ and $\vc{F}$ vectors in three
dimensions, but the general approach that seemed to work well for me was to lay a large sheet of butcher paper on the tabletop underneath
the apparatus, and project points on the apparatus down onto the paper using the plumb bob.
For convenience of measurement, make sure each string is a single strand (not an oval loop), and that
it's long enough so that its projection down onto the horizontal plane is fairly long, and can have its direction measured accurately.
Note that the strings' horizontal projections will not necessarily pass through the center of the hoop; that's okay, and in fact
there's no easy way to avoid it.

To locate the projection of the ring, you can mark points on its circumference, then at the end of the lab cut a circle of the right
size out of a piece of paper, lay it on the butcher paper, and trace its outline. It's not possible to trace the hoop itself, because
it's a baby buggy wheel, and its axle prevents it from lying flat. The diameter is 22.1 cm, and you can construct the circle using
a compass.

\fig{me-t3d-spherical-coords}

I found it convenient to work in spherical coordinates. Spherical coordinates $(r,\theta,\phi)$, illustrated in the figure above,\footnote{Wikimedia Commons user Andeggs, public domain} % http://commons.wikimedia.org/wiki/File:3D_Spherical.svg
are a generalization of polar coordinates to three dimensions. The angle $\phi$ is like longitude on the earth's surface
(or minus the longitude, actually, since it is conventionally measured counterclockwise from the $x$ axis).
The angle $\theta$ correponds to 90 degrees minus the latitude; it equals zero for a point on the $z$ axis directly above the origin, 180 degrees
for one directly below.

The $\vc{r}$ vectors of the points
at which the strings apply their forces have $r$ equal to the radius of the hoop, and $\theta=90 \degunit$. For the force vectors,
you'll have $(F,\theta_F,\phi_F)$, where $F$ is the magnitude of the force in newtons, and, e.g., $\theta_F=180\degunit$ for the
force of gravity, and $\theta_F<90\degunit$ for the forces of the strings.

\selfcheck

Make sure that the total force comes out close to zero.

\analysis

To make the error analysis manageable, you'll want to set up your whole analysis as either a spreadsheet or
a computer program.

Convert all four force vectors and all four $\vc{r}$ vectors into Cartesian coordinates. Find the total force by vector addition.
Compute the four torque vectors using the vector cross product, and find the total torque by vector addition.
Turn in a copy of your python code and its output with your writeup. 

\section*{Using a computer program for analysis}

If you have coding skills, I think it's much easier to do the analysis for this
lab, and especially the error analysis, by writing a computer program. A possible
convenient way to do this is to write code in a web browser, in your favorite language,
at the web site ideone.com. 

If you haven't coded before but are interested in
learning some basic coding for this lab, you can try the tutorial in the back of
the textbook on the computer language Python.
The way I would then start would be to  write the following two python functions:

\verb@cross_x(ax,ay,az,bx,by,bz)@ --- 
take the components of vectors $A$ and $B$ as input and calculates the $x$ component of their cross product as output.

\verb@cart_x(r,theta,phi)@ ---
find the x component of a vector given in spherical coordinates.
To get the trig functions to work, (1) remember to put\\
 \verb@import math@ at the top, (2) remember to do \verb@math.sin@
and \verb@math.cos@ rather than just \verb@sin@ and \verb@cos@, and (3) include a conversion from degrees to radians, since
your raw data will all be in degrees, but the python functions are defined in radians. Test your function on the example
given in P2.

\prelab

\prelabquestion Given $r=2.0$ cm, $\theta=37$ degrees, and $\phi=16$ degrees in spherical coordinates, find $x$, $y$, and $z$
in Cartesian coordinates. Answer: $x=1.2$ cm, $y=0.3$ cm, $z=1.6$ cm.
