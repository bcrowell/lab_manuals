\renewcommand\thechapter{c1.10a}
\lab{The Ballistic Pendulum}\label{lab:BallisticPendulum}

\apparatus
\equipn{A sheet of 8.5x11 inch printer paper}
\equipn{Modeling Clay}
\equipn{Masking Tape}
\equipn{Scissors}
\equipn{2 toothpicks}
\equipn{An index card, business card, or similar}
\equipn{a 1/2 inch steel bearing ball}
\equipn{A protractor}
\equipn{A ruler}
\equipn{A document clip, or other weight which can be used as a plumb for the protractor}
\equipn{The ball launcher from the Projectile Motion Lab}
\equipn{The end of a table}

\goal{In this lab, students will use modeling clay to construct a target suspended from a pendulum.  By having a projectile undergo an inelastic collision with this target this apparatus, known as a ballistic pendulum, allows us to indirectly measure the speed of the projectile.}

\introduction

In the days before the advent of the high-speed video camera, people had to get creative about the ways in which they measured the velocities of high speed projectiles.  When a bullet or cannon ball is traveling at hundreds of meters per second, a meter stick and stop-watch will not suffice.  Fortunately, our understanding of momentum gives us a way in. By analyzing the state of a system after a collision, we can extract information about what was going on before the collision.  In this case, measuring a speeding bullet may be hard, while measuring what happens to a massive object that the bullet hits is relatively easy.  This idea is put to use in a device called a "ballistic pendulum", which you will be constructing and using.  

\observations

\labpart{Setup}

    \begin{enumerate}
        \item Roll a piece of paper into a tight tube along its long end. Wrap masking tape around each end, and place a strip along the seam to keep the tube closed.
        \item Take a pin or some other sharp object and pierce the tube close to one end, along a diameter.  Insert a toothpick through the holes to act as a pivot point.  Work the toothpick around in the hole for a bit to loosen it up. You want there to be as little friction as possible.
        \item Take about 30g of modeling clay and work it onto a ball.  Take your thumb and work an indent into the ball which is just big enough for your bearing ball to fit into.  You should angle the indent downward to make a trap for the ball, as shown in the diagram.

\fig{me-bpe-target}

        \item Take the paper tube and press it down into the top of the ball. Work the clay up and around the tube, pressing in to make a secure connection. Be sure to align it correctly with the sting of your pendulum.
        \item In the next few steps we will make a device to help us measure the maximum angle our pendulum swings. Take an index card and place a small hole along the center line, close to one of the narrow ends.  Attach the index card to the toothpick which serves as the pivot point for your pendulum to they are aligned. You want there to be enough friction on the index card that it does not pivot under its own weight.
        \item Line the index card and the pendulum tube up so they are parallel, and mark a point close to the bottom of the index card that lines up with the edge of the tube.  Make a hole in the index card at this point, and insert a second toothpick. This toothpick should just barely touch the tube while they are both hanging vertically.
        \item Take the toothpick which will act as the pivot, and tape it down to the edge of a table so that it just perpendicularly from the table. Make sure there is enough overhang to accommodate your pendulum tube.  You can place a heavy object on top of the toothpick to help keep it stable.

    \end{enumerate}

\widefigcaption{me-bal-pendulum-assembly-smaller}{Assembly and testing of the ballistic pendulum.}

\labpart{Preliminary Tests}

\begin{enumerate}
    \item Place your index card on the pivot, followed by your pendulum.  Adjust the index card so that it just touches the tube.  Give your pendulum a little push.  Your index card should get pushed along as the pendulum swings upward, and stay in place to mark the angle of the highest swing.
    \item Take your ball launcher, and fire the ball into the target horizontally.  It should lodge itself in the target and not fall out as the pendulum swings.  If you can not get this to work consistently, you may need to tweak the shape of your trap.
\end{enumerate}
    
\labpart{Perform the Measurements}

\begin{enumerate}
    \item Place the ball into the target and take a measurement of the combined mass.
    \item Measure vertical distance from the pivot to where the ball will impact the target.
    \item With the ball in the target, balance the pendulum on a pen, or other narrow object, and determine the position of the center of mass, as measured from the pivot.
    \item Reset your pendulum, and fire your ball horizontally into the target so that the ball remains trapped in the target.
    \item Carefully remove the pendulum and re-collect the ball.
    \item Use your protractor to measure the final angle of the index card, which should be the highest point of your swing.
    \item Repeat your measurement 4 more times for a total of 5 measurements.
    \item Take another 30g of clay and add it to your target. Test your setup to make sure everything still works.
    \item Repeat your measurements with this new configuration.
    \item Remove your target from the bottom of the tube and cut the bottom half off your tube.  Wrap some masking tape around the freshly cut end and reattach your target to the bottom of the pendulum.
    \item Test to make sure everything still works and then repeat your measurement with this new configuration.
    \item Remove about 30g of clay from your target (so it has a mass of about 30g).
    \item Do a quick check to make sure everything is working and repeat your measurement with this new configuration.
\end{enumerate}

\selfcheck

Remember, ALL measured values should have uncertainties. Before you wrap up your data taking, make sure you have everything you need.

\prelab

\prelabquestion
A projectile with mass $m$ and initial speed $v$ collides with a stationary target of mass $M$. The smaller projectile becomes embedded in the first and they two move along together after the collision. If there are no net external forces acting on this two particle system during the collision, then what is the final speed of the combined object after the collision?

\prelabquestion
A simple pendulum of of mass $M$ and length $R$ is traveling with an initial speed $v$ when it is at its lowest point.  What is the maximum height that it will attain before it stops and swings back?

\analysis

For each of your configurations find the mean value, with error bounds, for your change in height and the mechanical energy after the collision. Use this information to find the initial speed of the bearing ball, with errors, for each of your configurations.

Address the following questions in your conclusion:
\begin{enumerate}
    \item Are the initial velocities you calculated consistent within errors?
    \item Was kinetic energy conserved in the collision between the bearing ball and the pendulum? If not, which trials lost the most energy?
    \item As part of the projectile motion lab, you calculated the initial velocity of the bearing ball from your launcher.  Is that value consistent with the one you found in this lab? If not, speculate as to why the result might be different.
\end{enumerate}
