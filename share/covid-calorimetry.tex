\addtocounter{chapter}{-1}
\renewcommand\thechapter{c1.9b}
\lab{Calorimetry}\label{lab:calorimetry}

\section*{About this lab}

\covid\ 
It is intended to be used around the 9th week of Physics 205, 210, or 221.

\apparatus
\equip{2~4 styrofoam cups}
\equip{thermometer}
\equip{aluminum pellets (50 g)}
\equip{cup}
\equip{cooking pot (large enough to hold a cup)}
\equip{digital balance}
\equip{oven mitt}
\equip{ice cube}

\begin{goals}

\item[] Determine the specific heat capacity of an object and the latent heat of fusion of water.

\end{goals}

\introduction
Heat is a transfer of energy by thermal processes because of a temperature difference. If there is a temperature change due to heat transfer, the heat can be calculated as
\begin{equation*}
Q = m c \Delta T,
\end{equation*}
where $m$ is the mass of material, $\Delta T$ is the temperature
change, and $c$ is known as the specific heat capacity. The units of the specific
heat capacity are 
$\junit /\kgunit \degunit\zu{C}$ or $\zu{cal}/\gunit \degunit\zu{C}$.

If an object undergoes a phase change, a quantity of heat is given by
\begin{equation*}
Q = L m
\end{equation*}
where m is the mass of material, and L is called latent heat.  Latent heat is the amount of energy required for a phase transition per unit mass. L is the latent heat of fusion when material changes its state from the solid-state to the liquid-state. The units of latent heat are $\junit /\kgunit$ or cal/$\gunit $.


In this lab, we will build a calorimeter and determine a specific heat capacity of aluminum and latent heat of fusion of water. Ideally, a calorimeter is thermally isolated. Since the energy of the system is conserved, the total heat transfer inside of a calorimeter is zero.  
\begin{equation*}
\sum_i^N Q_i = 0
\end{equation*}



\observations 

\labpart{Preparation}
Any insulated container can be used as a calorimeter. In the picture below, a calorimeter is made of styrofoam cups.  However, you can build a calorimeter by using any material. Write down how you make a calorimeter in the lab report. Here are some tips for making a calorimeter. 

\begin{itemize}
\item The material should be a good thermal insulator.
\item You need to seal the calorimeter well to minimize heat leak. 
\item You need to come up with a way to measure the inside temperature while minimizing heat leak. 
\end{itemize}

\fig{me-cal-equipment}

\fig{me-cal-setup}

\labpart{Specific heat of aluminum}
Prepare room temperature water and put it in the calorimeter. Then, place 50 g aluminum pellets inside of a heating cup. Put the cup in hot water to raise the temperature of the aluminum to be higher than 70 $^\circ$C. \textit{While heating the cup, keep the aluminum dry.} If the aluminum pellets are wet, the result will be altered by the hot water droplets. 

Once the aluminum pellets are ready, quickly, dump the pellets to the calorimeter and seal the container. Measure and record the final temperature when the system reaches equilibrium. 

CAUTION: When you hold a hot heating cup, wear an oven mitt. Be careful with hot boiling water. 

\labpart{Latent heat of fusion}
Prepare warm water (about 40 $^\circ$C) and an ice cube. Measure the mass and temperature of the water and ice. Put the water and ice cube in the calorimeter and measure the equilibrium temperature after the ice is completely melted.  

\subsection*{Additional note}
You need to know the initial temperature of the ice cube you use. Think about a method to measure the temperature. Alternatively, you can also set the initial temperature of an ice cube to a particular value instead.  


\analysis
When you do the calculations, assume the heat capacity of the calorimeter is negligible. 
\subsection*{Specific heat calculation}
From the measurements, determine the specific heat capacity of aluminum.
\subsection*{Latent heat Calculation}
From the measurements, determine the latent heat of fusion of water. 
\subsection*{Error analysis}
Calculate errors in the specific heat and latent heat by propagating errors from the raw data. Compare your result with the correct values. Then write down your conclusion in the lab report. 


\prelab

\prelabquestion 200 g of water initially at 10 $^\circ$ C is inside of a calorimeter. You pour 100 g of water at 50 $^\circ$ C into the calorimeter. What is the equilibrium temperature? We assume the heat capacity of the calorimeter is negligible. 

