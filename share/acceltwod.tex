\lab{Acceleration In Two Dimensions}\label{lab:acceltwod}

\apparatus
\equip{air track (small)}
\equip{cart}
\equip{LabPro-compatible photogates (in lab benches in 415)}
\equip{computer}
\equip{air blowers (in cupboards under lab benches)}
\equip{vernier calipers (for measuring wood blocks)}
\equipn{wood blocks}{2-3/group}
\equip{angle brackets}

\goal{Test whether the acceleration of gravity acts like a vector.}

\introduction

Vectors rule the universe.  Entomologists say that God must
have had an inordinate fondness for beetles, because there
are so many species of them.  Well, God must also have had a
special place in her heart for vectors, because practically
every natural phenomenon she invented is a vector:
gravitational acceleration, electric fields, nuclear forces,
magnetic fields, all the things that tie our universe
together are vectors.

\fig{me-inc-setup}

\setup

The idea of the lab is that if acceleration really acts like
a vector, then the cart's acceleration should equal the
component of the earth's gravitational acceleration vector
that is parallel to the track, because the cart is only free
to accelerate in the direction along the track.  There is
almost no friction, since the cart rides on a cushion of air
coming through holes in the track.

The speed of the cart at any given point can be measured as
follows.  The photogate consists of a light and a sensor on
opposite sides of the track.  When the cart passes by, the
cylindrical vane on top blocks the light momentarily, keeping
light from getting to the sensor.  The computer detects the
electrical signal from the sensor, and records the amount of
time, $t_b$, for which the photogate was blocked.  Given
$t_b$, you can determine the approximate speed that cart had
when it passed through the photogate.  The computer software
for measuring the time is on the Linux computers in 415.
Plug the photogate into the DIG1 plug
on the interface box, and connect the interface box to the computer
using the USB cable.
Double-click on the icon that says ``photogate.''


\observations

The basic idea is to release the cart at a distance $x$ away
from the photogate.  The cart accelerates, and you can
determine its approximate speed, $v$, when it passes through
the photogate.  (See prelab question P1.)


From $v$ and $x$, you can find the
acceleration.  You will take data with the track tilted at
several different angles, to see whether the cart's
acceleration always equals the component of  $g$ parallel to the track.

You can level the track to start with by adjusting the
screws until the cart will sit on the track without
accelerating in either direction.

The distance $x$ can be measured from the starting position
of the cart to half-way between the point where it first
blocks the photogate and the point where it unblocks the
photogate.  You can determine where these positions are by
sliding the cart into the photogate and watching the red LED
on the top of the photogate, which lights up when it is blocked.

Hints:

\begin{itemize}
\item[] Keep in mind that if the cart rebounds at the bottom of
the track and comes back up through the gate, you will get a
second, bogus time reading.

\item[] Note that you have no way to measure accurately to the
total amount of time over which the cart picked up speed
(which would be several seconds) --- what you measure is the
very short time required for the cart to pass through the photogate.

\item[] Release the cart by hand after starting up the air pump.
If you leave the cart on the track and then turn on the pump, there
will be a period of time when the pump is first starting up, and
the cart will drag.

\item[] The variable $x$ actually changes a little when you change $\theta$,
so don't just assume it's always the same.

\item[] Once you have data at your first angle, check whether theory and experiment
agree reasonably well.
\end{itemize}

You'll use the photogates again in lab 10, so make sure you understand
the technique thoroughly, and take notes on it so you'll remember how it's
done.

We use wood blocks to raise the track to various angles. To measure the thickness
of the blocks, you will use special calipers that have a vernier
scale, which is a tricky device that allows you to measure distances
to a precision of about a tenth of a millimeter --- about ten times better than
eyeball precision with a ruler. Your instructor will teach you how to use
the vernier scale.

The accurate measurement of the width of the vane, $w$, is critical
to the whole experiment. This has already been done for you, using the technique
described at the end of the lab. The results, in cm, are as follows.

\noindent \begin{tabular}{llllll}
1A & 2.13 & 1B & 2.18 & 1C & 2.05 \\
2A & 2.11 & 2B & 2.13 & 2C & 2.14 \\
3A & 2.10 & 3B & 2.18 & 3C & 2.07 \\
4A & 2.10 & 4B & 2.12 & 4C & 2.14 \\
5A & 2.15 & 5B & 2.18 & 5C & 2.14 \\
6A & 2.11 & 6B & 2.15 & 6C & 2.19 \\
7A & 2.15 &    &      & 7C & 2.16 \\
\end{tabular}

\section*{Technique for measuring the width of the vane}

\emph{Students don't need to do this.}

Ideally we would like to simply close the vernier calipers snugly on the
cylindrical vane and take a reading of its diameter. This is the way the
calipers are intended to be used. However, the beam of the photogate
has a finite width, and the electronics in the photogate are only roughly
calibrated to turn on and off when the edge of the vane intersects the
center of the beam, blocking half its light. Therefore, I had one of my classes in 2011 use
an arrangement like the one below to determine the effective width of the
vane when it is used with your own photogate. The idea is to use the moving jaw of the calipers
to push the cart from the point where it enters your photogate to the point
where it exits, as determined by the LED on top. I have found that the
photogates differ from one another by about a millimeter, so that if you
don't do this calibration with your own photogate, your velocities will be
off by as much as 5\%. 

\fig{me-inc-vernier}

\selfcheck

Find the theoretical and experimental accelerations for one
of your angles, and see if they are roughly consistent.

\prelab

\prelabquestion   \emph{Skip this question if the corresponding homework problem
has already been assigned.} (a) If $w$ is the width of the vane, and $t_b$ is defined
as suggested above, what is the speed of the cart when it
passes through the photogate? 
(b) Based on $v$ and $x$, how can you find $a$?

\prelabquestion  It is not possible to measure $\theta $ accurately with
a protractor.  How can $\theta $ be determined based on the
distance between the feet of the air track and the
height of the wood block? To figure out how to actually apply the trig to the
lab, you will need to draw a side view of the track with enough detail to
show the track as a rectangle and the feet sticking out.

\analysis

Extract the acceleration for each angle at which you took
data.  Make a graph with $\theta $ on the $x$ axis and
acceleration on the $y$ axis.  Show your measured accelerations
as points, and the theoretically expected dependence of $a$
on $\theta $ as a smooth curve.

Error analysis is not required for this lab, because the
random errors are small compared to systematic errors such
as the imperfect leveling of the track, friction,
warping of the track, and the measurement of $w$.
Appendix \ref{appendix:graphing} shows some examples of how
to compare theory and experiment on a graph.
