\renewcommand\thechapter{c2.8}
\lab{Electromagnetism}\label{lab:covid-electromagnetism}

\section*{About this lab}

\covid\ 
It is intended to be used during the eighth week in Physics 222.
The use of the oscilloscope has been covered in lab \ref{lab:covid-oscilloscope}.

\apparatus
\equip{oscilloscope (Hantek 6022)}
\equip{battery}
\equip{desktop computer or Android phone}
\equip{stack of 5 ferrite disk magnets}
\equip{hand-made solenoid (reused, or a second newly built one, see below)}
\equip{tape (not included in kit)}

\begin{goals}

\item[] Observe electric fields induced by changing magnetic fields.

\item[] Discover Lenz's law.

\end{goals}

\section*{Preparation}

You will need a solenoid to act as a passive receiving coil in this lab (not as a powered electromagnet).
One option is to use the solenoid you built for lab \ref{lab:dipole-and-superposition}. I tested this
and it worked well enough. However, the signals I got from it were relatively small, and the lab would
be easier to do with a purpose-built coil having a lot of turns. You will also need two coils for
a later lab, so I would suggest building the second one now. Use the 26 gauge magnet wire, and follow
the directions in lab \ref{lab:dipole-and-superposition}. However, for this coil, it will be more
convenient to use a somewhat different design than the one described there. Try winding 200 turns of
wire onto a 2-inch cardboard core cut from of a toilet paper roll. Remember to count and record the
exact number of turns for later use. This should use up about 30 meters of wire, and when you're
done it should have a resistance of about 4 ohms.

\introduction

Physicists hate complication, and when physicist Mich\-ael
Faraday was first learning physics in the early 19th
century, an embarrassingly complex aspect of the science was
the multiplicity of types of forces. Friction, normal
forces, gravity, electric forces, magnetic forces, surface
tension --- the list went on and on. Today, 200 years later,
ask a physicist to enumerate the fundamental forces of
nature and the most likely response will be ``four: gravity,
electromagnetism, the strong nuclear force and the weak
nuclear force.'' Part of the simplification came from the
study of matter at the atomic level, which showed that
apparently unrelated forces such as friction, normal forces,
and surface tension were all manifestations of electrical
forces among atoms. The other big simplification came from
Faraday's experimental work showing that electric and
magnetic forces were intimately related in previously
unexpected ways, so intimately related in fact that we now
refer to the two sets of force-phenomena under a single
term, ``electromagnetism.''

Even before Faraday, Oersted had shown that there was at
least some relationship between electric and magnetic
forces. An electrical current creates a magnetic field, and
magnetic fields exert forces on an electrical current. In
other words, electric forces are forces of charges acting on
charges, and magnetic forces are forces of moving charges on
moving charges. (Even the magnetic field of a bar magnet is
due to currents, the currents created by the orbiting
electrons in its atoms.)

Faraday took Oersted's work a step further, and showed that
the relationship between electricity and magnetism was even
deeper. He showed that a changing electric field produces a
magnetic field, and a changing magnetic field produces an
electric field. Faraday's work forms the basis for such
technologies as the transformer, the electric guitar, the
transformer, and generator, and the electric motor. It also
led to the understanding of light as an electromagnetic wave.

\section{Qualitative Observations}

In this lab you will use a permanent magnet to produce
changing magnetic fields. This causes an electric field to
be induced, which you will detect using the solenoid connected to the oscilloscope. The electric field
drives electrons around the solenoid, producing a current
which is detected by the oscilloscope.

When using a standalone oscilloscope for this type of measurement, the
nice setup is to put the time base on something very long like one
second.  Triggering becomes unnecessary, and the scopes we use at
school are programmed so that when the time base is very long, they
simply continuously display traces.

The USB oscilloscopes we're using are somewhat less convenient for
this purpose, but still usable. They do not have a hardware trigger.
They just continuously read in data, and the software simply doesn't
display the data unless it satisfies the trigger condition.

In the OpenHantek6022
software, the longest time base that is enabled by default is 100 ms, which is
shorter than I would have used for this lab. I asked the author to allow
longer time bases, and he did that in the very latest version of the software.
Make sure the version you're using is 20200609 or later.
Long time bases are still not enabled by default. To enable them, go to Oscilloscope:Settings:Scope,
and choose the appropriate option under ``Set slowest possible timebase.''

I believe the Windows software that comes with the scope does allow longer
timebases. I don't know about HScope, the android app.

\labpart{ A changing magnetic field}

\emph{Do not use a battery as in the previous lab.} Simply hook up
the probes of the scope to the leads of the solenoid. Since the solenoid
is not connected to ground, you will need to use both the minigrabber and
the ground clip. If you removed the ground clip from the probe (as I suggested),
you will need to snap it back on. Put the probe's switch to x1.

Set the scope to the most sensitive voltage scale and a timebase of 100 ms.
Set the trigger to automatic.

Take a stack of 5 of the ferrite disk magnets and place them near one
mouth of the solenoid, oriented so that their axes coincide. Shake the
magnet back and forth in the axial direction as fast as possible.

You should see an oscillating signal on the screen. This is the main phenomenon
being explored in this lab.

You can experiment with longer timebases and see if they're more convenient with
your software setup. With the OpenHantek6022 on a 500 ms timebase, this will make the sofwtare take
several seconds before drawing a trace.

\labpart{ A constant magnetic field}

Oscillate the magnet 5-10 times and then stop with the magnet close to the solenoid.
You should see the signal stop and go to the same zero level it was at when the
magnet was far away. A \emph{constant} magnetic field, even a strong one, produces no
electric field.

\labpart{ Moving the solenoid}

What happens if you hold the magnet still and move the solenoid? 

The poles of the magnet are its flat faces. In later parts
of the lab you will need to know which is north. Determine
this now by hanging it from a string and seeing how it aligns itself with the Earth's field.
The pole that points north is called the north pole of the
magnet. The field pattern funnels into the body of the magnet
through its south pole, and reemerges at its north pole.

\labpart{ A generator}

Tape the magnet securely to the eraser end of a pencil so
that its flat face (one of its two poles) is like the head
of a hammer, and mark the north and south poles of the
magnet for later reference. The axis of the magnet should be
perpendicular to the pencil. Put the pencil between your palms and spin the pencil rapidly near the
solenoid and observe the induced signal. You have built a
generator. (I have unfortunately not had any luck lighting a
lightbulb with the setup, due to the relatively high
internal resistance of the solenoid.)

\section*{Trying Out Your Understanding}

\labpart{ Changing the speed of the generator}

If you change the speed at which you spin the pencil, you
will of course cause the induced signal to have a longer or
shorter period. Does it also have any effect on the
\emph{amplitude} of the wave?

\labpart{Dependence on distance}

How does the signal picked up by your generator change with distance?

Try to explain what you have observed, and discuss your
interpretations with your instructor.

\section*{Lenz's Law}

Lenz's law describes how the clockwise or counterclockwise
direction of the induced electric field's whirl\-pool pattern
relates to the changing magnetic field. The main result of
this lab is a determination of how Lenz's law works. To
focus your reasoning, here are four possible forms for Lenz's law:

1. The electric field forms a pattern that is clockwise when
viewed along the direction of the $B$ vector of the
changing magnetic field.

2. The electric field forms a pattern that is counterclockwise
when viewed along the direction of the $B$ vector of the
changing magnetic field.

3. The electric field forms a pattern that is clockwise when
viewed along the direction of the $\Delta B$ vector of the
changing magnetic field.

4. The electric field forms a pattern that is counterclockwise
when viewed along the direction of the $\Delta B$ vector of
the changing magnetic field.

Your job is to figure out which is correct.

The most direct way to figure out Lenz's law is to 
make a tomahawk-chopping motion that ends up with the magnet at the mouth of the solenoid,
observing whether the pulse induced is positive or negative.
The idea of the ``chop'' is to bring the magnet rapidly closer
to the solenoid, keeping their axes parallel.

This doesn't work well in automatic triggering mode. Change to normal
triggering and set the trigger level to just a lttle bit above or a little
bit below the zero-signal level. Now depending on the polarity of the
pulse you produce, you may or may not trigger the scope. For example,
if you don't see anything when the trigger is set on a small positive
value, it may be that you're producing negative pulses, so try changing
the trigger to a small negative value.

What happens when you reverse the
chopping motion, or when you reverse the north and south
poles of the magnet? Try all four possible combinations and
record your results.

\fig{em-len-covid-scopepolarity}

It can be tricky to make the connection between the polarity
of the signal on the screen of the oscilloscope and the
direction of the electric field pattern. The figure shows an
example of how to interpret a positive pulse: the current
must have flowed from the minigrabber connector, into the
scope, and then back out through the ground clip.




