\addtocounter{chapter}{-1}
\renewcommand\thechapter{c1.4a}
\lab{The coefficient of kinetic friction}\label{lab:covid-mu-k}

\section*{About this lab}

\covid\ 
It is intended to be used during the fourth week in Physics 205 or 210.
It assumes knowledge of Newton's laws and models of friction. For Physics
221, use lab \ref{lab:covid-capstan}.

\observations

In this lab, you will measure the coefficient of kinetic friction for an object
sliding along a horizontal surface. The method is similar to the one used in lab \ref{lab:covid-measurement}.

The photo shows my version of the experiment. I flicked a penny across my desk, parallel
to the ruler. I put my watch in the picture in stopwatch mode to provide an absolute time
calibration. I initially tried this using a webcam and clicking through the video frame by
frame, but that gave results that were usable but not great, for the reasons described in lab \ref{lab:covid-measurement}.
Since I don't own a smartphone, I got my daughter to take some video using her Google phone
in slow-motion mode, and that worked better. I measured time and distance as described in
\ref{lab:covid-measurement}. Since I was using a penny on my desk, I was measuring the coefficient
of friction for something like copper on wood. You don't have to use a penny on wood. Those were
just what I had handy.

\fig{co-mu-k}


\analysis

Determine the coefficient of kinetic friction for your surfaces.

\prelab

\prelabquestion  
Use Newton's laws and the standard model of kinetic friction to determine $\mu_k$ from
the time and distance data measured in this lab. You should find that the mass cancels
out.




