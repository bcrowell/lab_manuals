
\renewcommand\thechapter{c1.4c}
\lab{Newton's Second Law}\label{lab:covid-2nd-law}

\section*{About this lab}

\covid\ 
It is intended to be used around the fourth week of Physics 205, 210, or 221.

\apparatus
\equip{SEOH pulley, plastic triple tandem}
\equip{string}
\equip{ruler}
\equip{about 20  quarters}
\equip{about 10  pennies}
\equip{lead shots}
\equip{digital balance}
\equip{suction cups with a hook}

\begin{goals}

\item[] According to Newton's second law of motion, the acceleration is proportional to the net force and inversely proportional to the mass. 
In this lab, we will test Newton's second law by using a system of two masses and a pulley. 
\end{goals}

\introduction

From the net force equations of the masses, you can get an equation for the acceleration of the system(This is a prelab quiz question or a homework problem). 
\begin{equation*}
a = \dfrac{\left(m_1-m_2\right)}{\left(m_1+m_2\right)} g
\end{equation*}
where $m_1$ and $m_2$ are masses of the two weights and $g$ is gravitational acceleration. 

When you plot acceleration ($a$) vs. net force ($\left(m_1 - m_2\right)g$) graph, the slope is $1/\left(m_1+m_2\right)$
%\section*{Preparation}

In this lab, you will verify this relationship between force and acceleration. 

\observations 
The basic setup of this lab is the same as the physics 210 lab manual Lab 3: Newton's Second Law (\url{http://www.lightandmatter.com/area3lab.html}). Please read the manual before you start this lab. 

\labpart{Preparation}
In this lab, we will use coins as weights hanging from a pulley. Make two hangers by using a piece of paper or some other materials. Then, place the same amount of coins on both hangers.(You can use lead shots as well.) Connect two weights by a pulley like an example setup on the picture. 

\fig{me-2nd-setup-covid}

\labpart{Measurement}
\textit{Do not change the total mass of the weights. Instead, move pennies(or lead shots) from one side to the other side of the hanger.} Repeat the experiments until all pennies on one side move to the other side. Calculate experimental accelerations by measuring the falling times and distances 


\labpart{Further improvement(optional)}
The lab setup is not ideal. You can improve the first result by minimizing the errors. Think about various methods to reduce errors. Then, adjust the experimental setup and make the second measurements. Write changes and the result of the improvement in the lab report. 


\analysis
Plot an acceleration(y-axis) vs. Net force(x-axis) graph. Compare the slope of the graph with the theoretical prediction. What can you conclude from your experimental finding? 

To compare the experimental slope with the theoretical prediction, you need to calculate the uncertainty in the slope,
as described in appendix 4.

\prelab

\prelabquestion (Mechanics Ch 5 Problem 20) This problem is the same setup as our experiment. Solve the pulley system problem in the textbook. 

