\renewcommand\thechapter{c1.11a}
\lab{Buoyancy}\label{lab:buoyancy}

\apparatus
\equipn{Modeling clay}
\equipn{A few small items that will sink. Nuts and bolts, small rocks, rubber balls, etc.}
\equipn{A popsicle stick}
\equipn{About 50cm of string}
\equipn{Masking tape}
\equipn{A plastic or paper cup}
\equipn{A small dish or bowl which the cup can fit in}
\equipn{A knife}
\equipn{Paper towels, or a dish towel for drying up}
\equipn{Digital scale}

\goal{In this lab, students will measure the buoyant force of several objects by dunking them in water. This information can be used to calculate an object's density, even when it has an amorphous shape which makes calculating the volume difficult.}

\introduction

By considering the static equilibrium of a fluid one can arrive at Archimedes's Principle, which states that an object submerged in a fluid will experience an upward force equal to the weight of the fluid that it displaced. This force is known as the buoyant force, and is the reason why boats float on water and helium filled balloons float through the air.

There is a story about Archimedes and how he discovered his principle of displacement.  Story goes that King Hiero II of Syracuse suspected that a goldsmith had cheated him by using silver in a crown that was meant to be pure gold.  If you can measure density, you can easily tell the difference between pure gold and an alloy which contains lighter metals.  The problem was that the crown was a shape too elaborate to calculate its volume, and Archimedes couldn't just melt it into a cube. Eventually, Archimedes had his "Eureka!" moment and realized that by measuring the amount of water the crown displaced he could determine its density.  He found that the crown was, in fact, not pure gold, and the goldsmith was punished accordingly.

In this lab you will perform a very similar measurement to the one performed by Archimedes and determine the density of some objects, and test Archimedes's Principle of Buoyancy in the process.

\observations

\labpart{Setup}

First make a holder to dip your objects in the water by tying the string about both ends of your popsicle stick and securing with masking tape.  You can tie a little overhand not to make a loop at the bottom, as shown below.

\fig{me-buo-mass-holder}

Next, prepare a plastic or paper cup by \emph{carefully} cutting out a slice near the top of the cup and folding it out, as shown below.

\fig{me-buo-cup}

Take some modeling clay and fashion it into an animal of your choice. I had black clay, so I made a cat. Make sure that your animal is the right size so that it can be completely submerged in your cup, below the cutout, without touching any of the sides. Be sure to take a picture of your animal and include it in your lab report. Measure the mass of your animal using the scale.

\fig{me-buo-cat}

\labpart{Measure the Buoyant Force}

\begin{enumerate}
        \item In a sink, fill your cup up all the way with water and allow the water to drain out the notch until it is done.
        \item Carefully move the cup and place it in the dish.
        \item Attach your clay animal to your holder and suspend it from your digital scale. Record the mass reading on the scale.
        \item Slowly dip your animal into the water until it is completely submerged, but not touching the bottom or sides of the cup, and measure the reading on the scale.

    \end{enumerate}
    
\fig{me-buo-dip}
    
\labpart{Measuring Volume Through Displacement}
    \begin{enumerate}
        \item Set your scale down on a level surface and zero it out.
        \item Place the empty, and dried off dish on the scale and measure the reading on the scale.
        \item Fill your cup and let it drain out as you did before, then carefully move it onto the dish.
        \item Measure the reading on the scale for the dish and the full cup.
        \item Suspend your clay animal from your holder and submerge it completely in the water, being careful not to touch the sides.
        \item Keep your animal submerged until water finishes draining from the cup.
        \item Measure the reading on the scale while your clay animal is still completely submerged, and not touching the sides or bottom.
        \item Remove the cup with the remaining water and your clay animal, and measure the mass of the dish with the spilled water in it.
    \end{enumerate}


Now repeat your measurements for at least two additional objects.

\selfcheck

A major source of uncertainty here is in the amount of water that spills out of the cup. Find a way to determine the precision of this measurement.

\prelab

\prelabquestion
You measure the weight of an object by placing it on a scale and measure a value $W_1$.  Next, you submerge the same object in water, which has a density $\rho_w=1 \frac{g}{cm^3}$ and measure an effective weight of $W_2$.  Use this information to find the volume of the submerged object.

\prelabquestion
A cup which is partially filled with liquid is placed on a scale and a mass, which is suspended on a string, is submerged in the liquid.  Does the value on the scale change?  Draw free body diagrams to support your answer.

\fig{me-buo-archimedes-scale}

\analysis

Calculate the density of your submerged objects separately  using the displaced volume of liquid that you measured, as well as the buoyant force you measured. These are two independent measurements which can be compared to test Archimedes's Principle.

Find your experimental uncertainty in the volume of water that is displaced by your submerged objects, and use this to calculate the uncertainty in your calculated density.

Compare the densities that you found using the two separate methods and determine whether they are consistent within experimental errors.
