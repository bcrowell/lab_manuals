\renewcommand\thechapter{c2.13a}
\lab{RC circuits}\label{lab:rc-circuits}

\section*{About this lab}

\covid\ 
It is intended to be used around the 13th week in Physics 222.

\apparatus
\equip{banana plug cables (8, soldered by the student)}
\equip{1.5 V battery and holder}
\equip{multimeter}
\equip{330 $\mu\zu{F}$ bipolar electrolytic capacitor}
\equip{10 nF bipolar capacitor (or similar value)*}
\equip{100 $\zu{k}\Omega$ resistor (or similar value)}
\equip{51 $\zu{k}\Omega$ resistor (or similar value)}
\equip{alligator clips (6)}
\equip{timing device (stopwatch, computer, or cell phone app)}
\equip{oscilloscope}

*The capacitors in the microfarad range are about the size of
your thumb, while the ones in the nanofarad range are ceramic disk capacitors that are smaller
than a fingernail and yellow. All the capacitors in this kit are nonpolar.
If the small-value capacitors in your kit have three-digit labels,
then the digits ABC are to be interpreted as $\zu{AB}\times10^\zu{C}$
picofarads.

\begin{goals}

\item[] Observe the exponential curve of a discharging capacitor.

\item[] Predict and test how the $RC$ time constant changes when the resistance is changed.
\end{goals}

\section*{About the equipment}

The big capacitor used in this lab is an unusually large value compared to those usually found
in most consumer electronics. It costs about \$2. It is an electrolytic capacitor, meaning that
between the plates there is a substance whose electrical properties cause the capacitance to
be much bigger than it would be with a capacitor having a vacuum or air between the plates.
Most electrolytic capacitors are polar devices, and if you hook them up with the wrong polarity,
they explode. This one works with either polarity.

The purpose of using this unusual high-value capacitor is that we can make the RC time constant
very long, and therefore we can use a multimeter rather than an oscilloscope to see what's going
on.

\introduction

God bless the struggling high school math teacher, but some
of them seem to have a talent for making interesting and
useful ideas seem dull and useless. On certain topics such
as the exponential function, $e^x$, the percentage of students
who figure out from their teacher's explanation what it
really means and why they should care approaches zero.
That's a shame, because there are so many cases where it's
useful. The graphs show just a few of the important
situations in which this function shows up.

\fig{em-rcc-exponentials}

The credit card example is of the form
\begin{equation*}
    y=ae^{t/k}   \qquad   ,
\end{equation*}
while the Chernobyl graph is like
\begin{equation*}
    y=ae^{-t/k}    \qquad   ,
\end{equation*}
In both cases, $e$ is the constant $2.718\ldots$, and $k$ is a
positive constant with units of time, referred to as the
time constant. The first type of equation is referred to as
exponential growth, and the second as exponential decay. The
significance of $k$ is that it tells you how long it takes
for $y$ to change by a factor of $e$. For instance, an 18\%
interest rate on your credit card converts to $k=6.0$ years.
That means that if your credit card balance is \$1000 in
1996, by 2002 it will be \$2718, assuming you never really
start paying down the principal.

An important fact about the exponential function is that it
never actually becomes zero --- it only gets closer and
closer to zero. For instance, the radioactivity near
Chernobyl will never ever become exactly zero. After a while
it will just get too small to pose any health risk, and at
some later time it will get too small to measure with
practical measuring devices.

Why is the exponential function so ubiquitous? Because it
occurs whenever a variable's rate of change is proportional
to the variable itself. In the credit card and Chernobyl examples,
\begin{gather*}
    (\text{rate of increase of credit card debt}) \\
    \qquad	\propto(\text{present credit card debt})    \\
    (\text{rate of decrease of the number of radioactive atoms}) \\
    \qquad	\propto(\text{present number of radioactive atoms})    
\end{gather*}

For the credit card, the proportionality occurs because your
interest payment is proportional to how much you currently
owe. In the case of radioactive decay, there is a proportionality
because fewer remaining atoms means fewer atoms available to
decay and release radioactive particles. This line of thought
leads to an explanation of what's so special about the
constant $e$. If the rate of increase of a variable $y$ is
proportional to $y$, then the time constant $k$ equals one
over the proportionality constant, and this is true only if
the base of the exponential is $e$, not 10 or some other number.

Exponential growth or decay can occur in circuits containing
resistors and capacitors. Resistors and capacitors are the
most common, inexpensive, and simple electrical components.
If you open up a cell phone or a stereo, the vast majority
of the parts you see inside are resistors and capacitors.
Indeed, many useful circuits, known as RC circuits, can be
built out of nothing but resistors and capacitors. In this
lab, you will study the exponential decay of the simplest
possible RC circuit, shown below, consisting of one resistor
and one capacitor in series.

\fig{em-rcc-simplified}

Suppose we initially charge up the capacitor, making an
excess of positive charge on one plate and an excess of
negative on the other. Since a capacitor behaves like
$V=Q/C$, this creates a voltage difference across the
capacitor, and by Kirchoff's loop rule there must be a
voltage drop of equal magnitude across the resistor. By
Ohm's law, a current $I=V/R=Q/RC$ will flow through
the resistor, and we have therefore established a proportionality,
\begin{gather*}
    (\text{rate of decrease of charge on capacitor}) \\
    \qquad	\propto(\text{present charge on capacitor})  \qquad .  \\
\end{gather*}

It follows that the charge on the capacitor will decay
exponentially. Furthermore, since the proportionality
constant is $1/RC$, we find that the time constant of the
decay equals the product of $R$ and $C$. (It may not be
immediately obvious that ohms times farads equals seconds, but it does.)

Note that even if we put the charge on the capacitor very
suddenly, the discharging process still occurs at the same
rate, characterized by $RC$. Thus RC circuits can be used to
filter out rapidly varying electrical signals while
accepting more slowly varying ones. A classic example occurs
in stereo speakers. If you pull the front panel off of the
wooden box that we refer to as ``a speaker,'' you will find
that there are actually \emph{two} speakers inside, a small
one for reproducing high frequencies and a large one for the
low notes. The small one, called the tweeter, not only
cannot produce low frequencies but would actually be damaged
by attempting to accept them. It therefore has a capacitor
wired in series with its own resistance, forming an RC
circuit that filters out the low frequencies while
permitting the highs to go through. This is known as a
high-pass filter. A slightly different arrangement of
resistors and inductors is used to make a low-pass filter
to protect the other speaker, the woofer, from high frequencies. 

\observations

\labpart{Predicting and measuring an RC time constant}

For parts A through C, you will need a long RC time constant,
so use the high-value capacitor.

Predict the RC time constant of the capacitor when used with
the larger of the two resistance values referred to under ``Apparatus.'' Hook up the resistor
and capacitor as a single loop. Connect the voltmeter to measure
the voltage across the capacitor. Finally, connect the battery
across the capacitor in order to charge it up. The circuit should
now be four elements, all in parallel.

When I first did this experiment, I hadn't thought through what
should happen when I connected the battery. I thought the capacitor
might take a long time, like the time RC, to charge up. But in this
setup, the circuit acts like an RC circuit containing the battery's
internal resistance (which is much smaller than that of the resistor),
so the charging-up process is very fast.

Measure the voltage $V_0$ of the battery, and calculate $V_0/e$. When
the circuit is discharged, the time required for the voltage to go from
$V_0$ to $V_0/e$ should be $\tau=RC$, which you have already predicted.

Disconnect the battery, and simultaneously start your timer.
Watch the voltmeter until it reaches $V_0/e$, and stop your timer.
This is the actual time constant, which can be compared with your
prediction.

\labpart{Changing the resistance}

Predict the new RC time constant when the original resistance is
replaced by the other, smaller value. Test your prediction.

\labpart{The decay curve}

Go back to the original resistance. Given the experimental value of
the time constant $\tau$ that you previously measured, calculate
$2\tau$, $3\tau$, $4\tau$, and $5\tau$ in units of minutes and seconds.
Charge up the capacitor and then, as before, start the timer as you
disconnect the battery. Record the voltage at the predetermined times.

\labpart{Oscilloscope}

In lab \ref{lab:covid-oscilloscope}, we measured the internal resistance of
the oscilloscope's calibration output by building a voltage divider. In this
lab we will estimate that resistance by a different technique and see if the
two results agree. The idea here is to use the scope's calibration output to
drive a known capacitance $C=10\ \zu{nF}$ (or something similar), which means that
we are effectively driving an RC circuit in which the resistance is the scope's
internal resistance $R$. The setup is exactly the same as in lab \ref{lab:covid-oscilloscope},
except that instead of the external resistor we have an external capacitor.
The calibration output's square wave output charges
the capacitor up with one polarity, then charges it up with the other polarity.

Since you have a previous measurement of $R$, you're not going blind here.
Estimate the RC time constant, and set the calibration output so that its half-period
is many times longer than this. Analyze the data as in homework problem 13-7
in Fields and Circuits.



\analysis

Compare your predictions with experiment for parts A and B.

For part C, test the theoretical prediction that the decay is
an exponential one. The nicest visual way to do this is to take
logs in order to linearize a graph as a function of time. Your
eye is a good judge of whether a line is a line.

For part D, compare with the previous estimate of $R$ from lab \ref{lab:covid-oscilloscope}.
