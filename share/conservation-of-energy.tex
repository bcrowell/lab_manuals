\lab{Conservation of Energy}\label{lab:conservation-of-energy}


\apparatus
\equip{air track}
\equip{cart}
\equip{springs (steel, 1.5 cm diameter)}
\equip{photogate}
\equip{computer}
\equip{aluminum rods, $\sim 45$ cm}
\equip{spirit levels}
\equip{air blowers}
\equip{spring scales}
\equip{string}
\equip{balances}
\equip{scissors}

\goal{Test conservation of energy for an object oscillating around
an equilibrium position.}

\figcaption{me-ene-sun}{This could be a vibration of the sun,
a water balloon, or a nucleus.}

\introduction

One of the most impressive aspects of the physical world is
the apparent permanence of so many of its parts.  Objects
such as the sun or rocks on earth have remained unchanged
for billions of years, so it might seem that they are in
perfect equilibrium, with zero net force on each part of the
whole.  In reality, the atoms in a rock do not sit perfectly
still at an equilibrium point --- they are constantly in
vibration about their equilibrium positions.  The unchanging
oblate shape of the sun is also an illusion.  The sun is
continually vibrating like a bell or a jiggling water
balloon, as shown in the (exaggerated) figure.  The nuclei
of atoms also jiggle spontaneously like little water
balloons. The fact that these types of motion continue
indefinitely without dying out or building up relates to
conservation of energy, which forbids them to get bigger or
smaller without transferring energy in or out.



\fig{me-ene-setup}



Our model of this type of oscillation about equilibrium will
be the motion of a cart on an air track between two springs.
 The sum of the forces exerted by the two springs should at
least approximately obey Hooke's law,
\begin{equation*}
	F  = -kx   \qquad   ,
\end{equation*}
where the equilibrium point is at $x=0$.  The negative sign
means that if the object is displaced in the positive
direction, the force tends to bring it back in the negative
direction, towards equilibrium, and vice versa.  Of course,
there are no actual springs involved in the sun or between a
rock's atoms, but we can still learn about this type of
situation in a lab experiment with a mass attached to a
spring.  In this lab, you will study how the changing
velocity of the object, in this case a cart on an air track,
can be understood using conservation of energy.  Recall that
for a constant force, the potential energy is simply
$-Fx$, but for a force that is different at different
locations, the potential energy is minus the area under the
curve on a graph of $F$ vs. $x$.  In the present case, the
area formed is a triangle with $\text{base}=x$, $\text{height}=kx$, and
\begin{align*}
   \text{area}     &=  \frac{1}{2}\text{base}\cdot\text{height}  \\
       &=     -\frac{1}{2}kx^2
\end{align*}

(counted as negative area because it lies below the $x$
axis), so the potential energy is

\fig{me-ene-area}\label{spring-constant}

\begin{equation*}
      PE  =   \frac{1}{2}kx^2  \qquad   .  
\end{equation*}

Conservation of energy, $PE+KE=\text{constant}$, requires the constancy of
\begin{equation*}
      \frac{1}{2}kx^2+\frac{1}{2}mv^2 = E_{total}  \qquad   .  
\end{equation*}

\section{Preliminary Observations}

You should do both of the following methods of determining
the spring constant.

\mysubsubsection{Determining the spring constant: method 1}

Pull the cart to the side with a spring scale, and determine the spring constant,
which is defined as minus the slope of the $F$ versus $x$ graph.
To avoid pulling at the wrong angle, it helps if you connect the spring scale
to the cart with a piece of string.
Find the combined spring constant of the two springs, $k$.
Since you'll only use method 1 as a rough check against the
more precise method 2, it is a waste of time to take more than two data points,
and you can use these points to find the slope without actually having to make
a graph.

\mysubsubsection{Determining the spring constant: method 2}

The second technique for determining $k$ is to raise one end up
and observe how far the cart's equilibrium is displaced.
This method is more accurate than method 1, but your $k$ value from
method 1 is still useful as a check.

You can determine the angle $\theta$ to which the track has been raised by the same
trig you used in lab \ref{lab:acceltwod}. At this angle, the component $mg\sin\theta$
of the gravitational force that is parallel to the track cancels out the force
$kx$ from the springs. Since you know $m$, $g$, $\theta$, and $x$, you can determine
$k$.

The main limitation on the precision of this measurement is that the displacement $x$
is fairly small. To maximize your precision, raise the track to as high an angle as
is practical. A good method is to prop it up with the aluminum rod,
using the spirit level to make sure the rod is vertical.


\observations

The technique is essentially the same as in lab \ref{lab:acceltwod}, which you will want
to review.

Before you start taking actual data, check whether you have
excessive friction by letting the computer record data while
the cart vibrates back and forth a few times through the
photogate. If the air track is working right, all the time
measurements should be nearly the same, but if the data show
the cart slowing down a lot from one vibration to the next,
then you have a problem with friction. A drop in velocity of
about 1\% over a half-cycle is reasonable.

Measure the velocity of the cart for many different values
of $x$ by moving the photogate to various positions. Make
sure you always release the cart from rest at the same
point, and when you are initially choosing this release
point, make sure that it is not so far from the center that
the springs are completely bunched up or dragging on the
track. Don't forget that the $x$ you use in the potential
energy should be the distance from the equilibrium position
to the position where the vane is centered on the photogate
--- if you don't think about it carefully, it's easy to make
a mistake in $x$ equal to half the width of the vane.
See lab \ref{lab:acceltwod}.

\selfcheck

Calculate the energies at the extremes, where $PE=0$ and
$KE=0$, and see whether the energy is staying roughly constant.
You should do this self-check as early as possible in the lab, so that
you can make sure you're not spending lots of time collecting data
that turn out to be bogus.

\prelab

\prelabquestion Find the value of $x$ from the figure below. (I've made the
centimeter scale unrealistic for readability --- the real track is
more than a meter long, not 14 centimeters.)

\fig{me-ene-prelab}

\prelabquestion In order to do the self-check, where would you have
to put the photogate?

\prelabquestion Plan how you will estimate the random errors on your
raw data. (Random errors here would refer to something that would cause
independent errors that could be up or down for any individual data point ---
not things like the measurement of $k$, which you do once for the whole
experiment.)

\analysis

Graph $PE$, $KE$, and the total energy as functions of $x$,
with error bars (see appendices 2, 3, and 4), all overlaid
on the same plot. Make sure to include the points with $KE=0$ and
$PE=0$.

As a shortcut in your error analysis, it's okay if you
do the error analysis for your most typical data-point, in which
the energy is split roughly 50-50 between PE and KE, and then assume
that the same error bars on $PE$, $KE$, and total energy apply to
all the other points on the graph as well.

As sources of error, you have certain things, such as the spring constant $k$,
that you determine once for the entire lab. These are more like systematic
errors rather than random errors. The error bars you draw on the graph are
meant to represent random errors, i.e., errors that could cause random scatter
in the points. Therefore you should not include systematic errors in these
error bars.

Discuss whether you think conservation of
energy has been verified. See the examples in appendix 4 re how to statistically interpret
this type of comparison of data with a curve on a graph.

