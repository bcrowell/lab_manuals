\lab{Interactions}\label{lab:interactions}

\apparatus
\equipn{single neodymium magnet}{1/group}
\equipn{large neodymium magnet}{1/group}
\equip{compass}
\equipn{triple-arm balance}{2/group}
\equip{clamp and 50-cm vertical rod for holding balance up}
\equip{string}
\equip{tape}
\equip{scissors}
\equip{pencils}
\equip{spring scales}
\equip{rubber stoppers}

\goal{Form hypotheses about interactions and test them.}

\introduction

Why does a rock fall if you drop it?  The ancient Greek
philosopher Aristotle theorized that it was because the rock
was trying to get to its natural place, in contact with the
earth.  Why does a ball roll if you push it? Aristotle would
say that only living things have the ability to move of
their own volition, so the ball can only move if you give
motion to it.  Aristotle's explanations were accepted by
Arabs and Europeans for two thousand years, but beginning in
the Renaissance, his ideas began to be modified drastically.
 Today, Aristotelian physics is discussed mainly by physics
teachers, who often find that their students intuitively
believe the Aristotelian world-view and strongly resist the
completely different version of physics that is now
considered correct.  It is not uncommon for a student to
begin a physics exam and then pause to ask the instructor,
``Do you want us to answer these questions the way you told
us was true, or the way we really think it works?''  The
idea of this lab is to make observations of objects, mostly
magnets, pushing and pulling on each other, and to figure
out some of the corrections that need to be made to
Aristotelian physics.

Some people might say that it's just a matter of definitions
or semantics whether Aristotle is correct or not.  Is
Aristotle's theory even testable?  One testable feature of
the theory is its asymmetry.  The Aristotelian description
of the rock falling and the ball being pushed outlines two
relationships involving four objects:

\fig{me-int-aristotle}

According to Aristotle, there are asymmetries involved in both situations.

(1) The earth's role is not interchangeable with that of the
rock.  The earth functions only as a place where the rock
tends to go, while the rock is an object that moves from
one place to another.

(2) The hand's role is not analogous to the ball's.  The
hand is capable of motion all by itself, but the ball can't
move without receiving the ability to move from the hand.

If we do an experiment that shows these types of asymmetries,
then Aristotle's theory is supported.  If we find a more
symmetric situation, then there's something wrong with Aristotle's theory.

\observations
m4_include(first_lab.tex) 

\labpart{ Comparing magnets' strengths}

To make an interesting hypothesis about what will happen in
part C, the main event of the lab, you'll need to know how
the top (single) and bottom (large) magnets' strengths
compare. Since the large magnet is made out of six of the small
magnets stuck together in a stack, it would seem logical that the large magnet would
be six times stronger than the single, but in this part
of the lab you're going to find out for sure.

\figcaption{me-int-strength}{Orient your magnet this way, as if it's rolling toward the compass from the north. With no magnet nearby, the compass points to magnetic north (dashed arrow).  The magnet deflects the compass to a new direction. }

One way of measuring the strength of a magnet is to place
the magnet to the north or south of the compass and see how
much it deflects (twists) the needle of a compass. You need
to test the magnets at equal distances from the compass,
which will produce two different angles.\footnote{There are two
reasons why it wouldn't make sense to find different distances
that produced the same angle. First, you don't know how the
strengths of the effect falls off with distance; it's not necessarily
true, for instance, that the magnetic field is half as strong at
twice the distance. Second, the point of this is to help you
interpret part C, and in part C, the large magnet's distance
from the single magnet is the same as the single magnet's distance
from the first magnet.}
It's also important to get everything oriented
properly, as in the figure.\footnote{Laying the magnet flat on the table causes
the compass needle to try to tilt out of the horizontal plane, which it's
not designed to do. Turning it so that it faces the compass also doesn't
work, because it makes the magnet's magnetic field lie along the same
north-south line as the Earth's, rather than perpendicular to it.}

Make sure
to take your data with the magnets far enough from the
compass that the deflection angle is fairly small (say 5 to
30\degunit).  If the magnet is close enough to the compass
to deflect it by a large angle, then the ratio of the angles
does not accurately represent the ratio of the magnets'
strengths. After all, just about any magnet is capable of
deflecting the compass in any direction if you bring it
close enough, but that doesn't mean that all magnets are
equally strong.


\labpart{ Qualitative observations of the interaction of two magnets}

Play around with the two magnets and see how they interact
with each other. Can one attract the other?  Can one repel
the other?  Can they act on each other simultaneously? Do
they need to be touching in order to do anything to each
other?   Can A act on B while at the same time B does
not act on A at all?  Can A pull B toward itself at the
same time that B pushes A away?  When holding one of the
heavier magnets, it may be difficult to feel when there is
any push or pull on it; you may wish to have one person hold
the magnet with her eyes closed while the other person moves
the other magnet closer and farther.

\labpart{ Measurement of interactions between two magnets}

Once you have your data from parts A and B, you are
ready to form a hypothesis about the following situation. 
Suppose we set up two balances as shown in the figure.  The
magnets are not touching.  The top magnet is hanging from a
hook underneath the pan, giving the same result as if it was
on top of the pan.  Make sure it is hanging under the
\emph{center} of the pan. You will want to make sure the
magnets are pulling on each other, not pushing each other
away, so that the top magnet will stay in one place.

\fig{me-int-twoscales}

The balances will not show the magnets' true masses, because
the magnets are exerting forces on each other.  The top
balance will read a higher number than it would without any
magnetic forces, and the bottom balance will have a lower
than normal reading.  The difference between each magnet's
true mass and the reading on the balance gives a measure of
how strongly the magnet is being pushed or pulled by the other magnet.

How do you think the amount of pulling experienced
by the two magnets will compare?  In other words, which
reading will change more, or will they change by the same
amount?  Write down a hypothesis;  you'll test this
hypothesis in part C of the lab. If you think the forces
will be unequal predict their ratio.

Discuss with your instructor your results from parts A and
B, and your hypothesis about what will happen with the two
balances.

Now set up the experiment described above with two balances.
 Since we are interested in the changes in the scale
readings caused by the magnetic forces, you will need to
take a total of four scale readings: one pair with the
balances separated and one pair with the magnets close
together as shown in the figure above.

When the balances are together and the magnetic forces are
acting, it is not possible to get both balances to reach
equilibrium at the same time, because sliding the weights on
one balance can cause its magnet to move up or down, tipping
the other balance.  Therefore, while you take a reading from
one balance, you need to immobilize the other in the
horizontal position by taping its tip so it points
exactly at the zero mark.

You will also probably find that as you slide the weights,
the pointer swings suddenly to the opposite side, but you
can never get it to be stable in the middle (zero) position.
 Try bringing the pointer manually to the zero position and
then releasing it.  If it swings up, you're too low, and if
it swings down, you're too high.  Search for the dividing
line between the too-low region and the too-high region.

If the changes in the scale readings are very small (say a
few grams or less), you need to get the magnets closer
together.  It should be possible to get the scale readings
to change by large amounts (up to 10 or 20 g).

Part C is the only part of the experiment where you will
be required to analyze random errors using the techniques
outlined in Appendices 2 and 3 at the back of the lab
manual.  Think about how you can get an estimate of the
random errors in your measurements. Do you need to do
multiple measurements? Discuss this with your instructor
if you're uncertain.

\emph{Don't take apart your setup until lab is over, and you're
completely done with your analysis --- it's no fun to have
to rebuild it from scratch because you made a mistake!}

\labpart{ Measurement of interactions involving objects in contact}

You'll recall that Aristotle gave completely different
interpretations for situations where one object was in
contact with another, like the hand pushing the ball, and
situations involving objects not in contact with each other,
such as the rock falling down to the earth.  Your magnets
were not in contact with each other.  Now suppose we try the
situation shown below, with one person's hand exerting a
force on the other's.  All the forces involved are forces
between objects in contact, although the two people's hands
cannot be in direct contact because the spring scales have
to be inserted to measure how strongly each person is
pulling.  Suppose the two people do not make any special
arrangement in advance about how hard to pull.  How do you
think the readings on the two scales will compare?  Write
down a hypothesis, and discuss it with your instructor before continuing.

\fig{me-int-twohands}

Calibrate the spring scales by sliding the metal tabs.
Now carry out the measurement shown in the figure.

\selfcheck

Do all your analysis in lab, including error analysis for
part C.   Error analysis is discussed in appendices \ref{appendix:basicerranal}
and \ref{appendix:errpropagation}; get help from your instructor if necessary.

\analysis

In your writeup, present your results from all four parts of
the experiment, including error analysis for part C.
