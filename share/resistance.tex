\lab{Electrical Resistance}\label{lab:resistance}

\apparatus
\equipn{DC power supply (Thornton)}{1/group}
\equipn{digital multimeters (Fluke and HP)}{2/group}
\equip{resistors, various values}
\equip{unknown electrical components}
\equip{alligator clips}
\equip{spare fuses for multimeters --- Let students replace fuses themselves.}

\begin{goals}

\item[] Measure curves of voltage versus current for three
objects: your body and two unknown electrical components.

\item[] Determine whether they are ohmic, and if so, determine their resistances.
\end{goals}

\introduction

Your nervous system depends on electrical currents, and
every day you use many devices based on electrical currents
without even thinking about it. Despite its ordinariness,
the phenomenon of electric currents passing through liquids
(e.g., cellular fluids) and solids (e.g., copper wires) is a
subtle one. For example, we now know that atoms are composed
of smaller, subatomic particles called electrons and nuclei,
and that the electrons and nuclei are electrically charged,
i.e., matter is electrical. Thus, we now have a picture of
these electrically charged particles sitting around in
matter, ready to create an electric current by moving in
response to an externally applied voltage. Electricity had
been used for practical purposes for a hundred years,
however, before the electrical nature of matter was proven
at the turn of the 20th century.

Another subtle issue involves Ohm's law,
\begin{equation*}
       I = \frac{\Delta V}{R} \qquad ,
\end{equation*}
where $\Delta V$ is the voltage difference applied across an
object (e.g., a wire), and $I$ is the current that flows in
response. A piece of copper wire, for instance, has a
constant value of $R$  over a wide range of voltages. Such
materials are called ohmic. Materials with non-constant  are
called non-ohmic. The interesting question is why so many
materials are ohmic. Since we know that electrons and nuclei
are bound together to form atoms, it would be more
reasonable to expect that small voltages, creating small
electric fields, would be unable to break the electrons and
nuclei away from each other, and no current would flow at
all --- only with fairly large voltages should the atoms be
split up, allowing current to flow. Thus we would expect $R$
to be infinite for small voltages, and small for large
voltages, which would not be ohmic behavior. It is only
within the last 50 years that a good explanation has been
achieved for the strange observation that nearly all solids
and liquids are ohmic.

\section*{Terminology, Schematics, and Resistor Color Codes}

The word ``resistor'' usually implies a specific type of
electrical component, which is a piece of ohmic material
with its shape and composition chosen to give a desired
value of $R$. Any piece of an ohmic substance, however, has
a constant value of $R$, and therefore in some sense
constitutes a ``resistor.'' The wires in a circuit have
electrical resistance, but the resistance is usually
negligible (a small fraction of an Ohm for several
centimeters of wire). 

The usual symbol for a resistor in an electrical schematic
is this \figinsidetext{em-ohm-resistor}, but some recent schematics 
use this \figinsidetext{em-ohm-modresistor}. The symbol \figinsidetext{em-ohm-battery}
represents a fixed source of voltage such as a battery,
while   \figinsidetext{em-ohm-varvoltage} represents an adjustable voltage source, such as
the power supply you will use in this lab.

In a schematic, the lengths and shapes of the lines
representing wires are completely irrelevant, and are
usually unrelated to the physical lengths and shapes of the
wires. The physical behavior of the circuit does not depend
on the lengths of the wires (unless the length is so great
that the resistance of the wire becomes non-negligible), and
the schematic is not meant to give any information other
than that needed to understand the circuit's behavior. All
that really matters is what is connected to what.

\fig{em-ohm-seriesandparallel}

For instance, the schematics (a) and (b) above are
completely equivalent, but (c) is different. In the first
two circuits, current heading out from the battery can
``choose'' which resistor to enter. Later on, the two
currents join back up. Such an arrangement is called a
parallel circuit. In the bottom circuit, a series circuit,
the current has no ``choice'' --- it must first flow through
one resistor and then the other.

Resistors are usually too small to make it convenient to
print numerical resistance values on them, so they are
labeled with a color code, as shown in the table and example below.

\fig{em-ohm-colorcodetable}
\fig{em-ohm-colorcodeeg}

\setup

Obtain your two unknowns from your instructor. Group 1 will
use unknowns 1A and $1B$, group 2 will use 2A and $2B$, and so on.

Here is a simplified version of the basic circuit you will
use for your measurements of $I$ as a function of $\Delta
V$. Although I've used the symbol for a resistor, the
objects you are using are not necessarily resistors, or even ohmic.  

\fig{em-ohm-simple}

Here is the actual circuit, with the meters included. In
addition to the unknown resistance $R_U$, a known resistor
$R_K$ ($\sim 1\zu{k}\Omega $ is fine) is included to limit the
possible current that will flow and keep from blowing fuses
or burning out the unknown resistance with too much current.
This type of current-limiting application is one of the
main uses of resistors.

\fig{em-ohm-complicated}



\observations

\labpart{ Unknown component A}

Set up the circuit shown above with unknown component A.
Most of your equipment accepts the banana plugs that your
cables have on each end, but to connect to $R_U$ and $R_K$
you need to stick alligator clips on the banana plugs. See
Appendix \ref{appendix:meter} for information about how to set up and use the
two multimeters. Do not use the pointy probes that come with
the multimeters, because there is no convenient way to
attach them to the circuit --- just use the banana plug
cables. Note when you need three wires to come together at
one point, you can plug a banana plug into the back of
another banana plug.

Measure $I$ as a function of $\Delta V$. Make sure to take
measurements for both positive and negative voltages.

Often when we do this lab, it's the first time in several months
that the meters have been used. The small hand-held meters have
a battery, which may be dead. Check the battery icon on the LCD screen.

\labpart{ Unknown component B}

Repeat for unknown component B.

\prelab

\prelabquestion  Check that you understand the interpretations of the
following color-coded resistor labels:

\begin{tabular}{ll}
   blue   gray   orange   silver    &= 68 k$\Omega$  $\pm$ 10\%  \\
   blue   gray   orange   gold    &= 68 k$\Omega$  $\pm$ 5\%  \\
   blue   gray   red   silver    &= 6.8 k$\Omega$  $\pm$ 10\%  \\
   black   brown   blue   silver    &= 1 M$\Omega$  $\pm$ 10\%  
\end{tabular}

Now interpret the following color code:

\begin{tabular}{ll}
   green   orange   yellow   silver    &=    ?  
\end{tabular}

\prelabquestion  Fit a line to the following sample data and use the
slope to extract the resistance (see Appendix \ref{appendix:graphing}).

\fig{em-ohm-samplegraph}

Your result should be consistent with a resistor color code
of green-violet-yellow.

\prelabquestion  Plan how you will measure $I$ versus $\Delta V$ for both
positive and negative values of $\Delta V$ , since the power
supply only supplies positive voltages.

\prelabquestion  Would data like these indicate a negative resistance, or
did the experimenter just hook something up wrong? If the
latter, explain how to fix it.

\fig{em-ohm-negative}

\prelabquestion  Explain why the following statement about the resistor
$R_K$ is incorrect:
``You have to make $R_K$ small compared to $R_U$, so it
won't affect things too much.''


\analysis

Graph $I$ versus $\Delta V$  for all three unknowns. Decide
which ones are ohmic and which are non-ohmic. For the ones
that are ohmic, extract a value for the resistance (see
appendix \ref{appendix:graphing}). Don't
bother with analysis of random errors, because the main
source of error in this lab is the systematic error in the
calibration of the multimeters (and in part C the
systematic error from the subject's fidgeting).

\section*{Programmed Introduction to Practical Electrical Circuits}

Physics courses in general are compromises between the
fundamental and the practical, between exploring the basic
principles of the physical universe and developing certain
useful technical skills. Although the electricity and
magnetism labs in this manual are structured around the
sequence of abstract theoretical concepts that make up the
backbone of the lecture course, it's important that you
develop certain practical skills as you go along. Not only
will they come in handy in real life, but the later parts of
this lab manual are written with the assumption that you
will have developed them.

As you progress in the lab course, you will find that the
instructions on how to construct and use circuits become
less and less explicit. The goal is not to make you into an
electronics technician, but neither should you emerge from
this course able only to flip the switches and push the
buttons on prepackaged consumer electronics. To use a
mechanical analogy, the level of electrical sophistication
you're intended to reach is not like the ability to rebuild
a car engine but more like being able to check your own oil.

In addition to the physics-based goals stated at the
beginning of this section, you should also be developing the
following skills in lab this week:

(1) Be able to translate back and forth between schematics
and actual circuits.

(2) Use a multimeter (discussed in Appendix \ref{appendix:meter}), 
given an explicit schematic showing how to
connect it to a circuit.

Further practical skills will be developed in the following lab.
