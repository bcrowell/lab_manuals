\lab{Absolute Zero}\label{lab:absolute-zero}

Note to the students:
As (hopefully) announced previously, you need to wear close-toed
shoes, and it is up to you
to buy your own safety glasses, although the physics department
has a few. 

Note to the lab technician: 
Please put the alcohol and acetone in a freezer overnight, then
put them in the ice chest right before lab. Please also purchase
dry ice. When you put out the waste disposal container, please
don't leave the cap with it, because we don't want students to screw
on the cap; if they did, evaporating dry ice could make it explode like a bomb.

\apparatus
\equip{electric heating pad}
\equip{oven mitts}
\equip{latex tubing}
\equip{ice chest to keep liquids cool}
\equip{acetone (1.5 liter for the whole class)}
\equip{alcohol}
\equip{mineral oil}
\equip{waste disposal container}
\equip{dry ice (9 lb)}
\equip{tongs}
\equip{hammer}
\equip{funnels}
\equip{gas pressure sensor}
\equip{temperature probe}
\equip{125 ml Erlenmyer flask}
\equipn{600 ml beaker}{2/group}
\equip{safety goggles}
\equip{fold-top sandwich bags}

\introduction

If heat is a form of random molecular motion, then it makes
sense that there is some minimum temperature at which the
molecules aren't moving at all. With fancy equipment,
physicists have gotten samples of matter to within a
fraction of a degree above absolute zero, but they have
never actually reached absolute zero (and the laws of
thermodynamics actually imply that they never can).
Nevertheless, we can determine how cold absolute zero is
without even getting very close to it. Kinetic theory tells
us that heat is composed of random molecular motion, and
temperature is interpeted as a measure of the average
kinetic energy per molecule; the zero of the \emph{absolute}
temperature scale occurs when all molecular motion is eliminated.
Suppose we heat up a gas so that the typical speeds of
the atoms are doubled. The kinetic energy depends on $v^2$,
so the result is that the temperature is quadrupled:
\begin{equation*}
  T \rightarrow 4T
\end{equation*}
In this lab, we'll be heating and cooling air while it
is sealed inside a flask with a fixed volume. We won't
actually be quadrupling the absolute temperature in this
lab, but just to get the idea, let's pretend that we were.
The hot gas exerts more pressure on the inside of the flask,
for two reasons: (1) the molecules are moving twice as fast,
so when they hit the sides of the flask, each impact is
twice as hard; (2) because the moleculres are moving twice
as fast, they also take less time to cross from one side
of the flask to the other, so the collisions occur twice
as frequently. The result is that the pressure is quadrupled:
\begin{equation*}
  P \rightarrow 4P
\end{equation*}
Based on these arguments, we conclude that in general, the
pressure of a gas maintained at constant volume is proportional
to its absolute temperature:
\begin{equation*}
  P \propto T
\end{equation*}
In this lab, you'll measure
the volume of a sample of air at temperatures between about $-70$ and
150 degrees C, and determine where absolute zero lies by
extrapolating to the temperature at which it would
have had zero pressure.

Because absolute zero is very far below room temperature, this is a long extrapolation.
Extrapolating a long way like this tends to be inaccurate unless you can get data
covering a large range, so that the slope is well determined. For this reason,
we want to get a set of temperatures that goes as high and as low as possible.

\observations
m4_include(first_lab.tex)

\begin{itemize}
\item \emph{Hot mineral oil}
Thoroughly dry the beaker in which you'll heat the mineral oil; if there are drops of water
mixed into the oil, the oil will spatter. Measure and plan the volume of oil you will use.
If you use too little, it won't cover the whole Erlenmeyer flask that holds the sample of air. If you use too much, it will
overflow and make a nasty mess when you dunk the flask in it.
Start heating the mineral oil. Keep an eye on the temperature. You should heat it up to about
$150\degcunit$; above that, it starts to smoke.
\end{itemize}

While you're waiting for the mineral oil to heat up,
plug the temperature and pressure sensors into CH1 and CH2 on
the LabPro interface. Use the Logger Pro software to view the temperature readout.
Put the stopper in the Erlenmyer flask in order to make a
sealed sample of air. There is an extra port on the stopper
with a blue stopcock; make sure the stopcock is closed, so that
the sample is sealed. Connect the flask to the pressure
sensor using the latex tubing. The temperature probe goes in
the liquid, not the air.

The whole lab is predicated on the ability to maintain the same sample
of gas at a range of temperatures. Therefore if you have a leak, you have
to do the whole lab over.
Make sure the bayonet connector with
the stopcock is firmly shoved into the hole in the rubber stopper.
If you notice that the pressure doesn't change as the temperature changes,
it means you have a leak.

A practical difficulty in this lab is that if the flask
is initially sealed, and then heated to a higher temperature,
the stopper tends to pop out due to the higher pressure.
To keep this from happening, we want to start off with a
sample of air that is hot and at atmospheric pressure; then
all the other pressures will be at lower than atmospheric
pressure, which will tend to suck the stopper down into the
flask rather than popping it out. To accomplish this,
take the oil off the burner,
open the stopcock on the flask, immerse the flask in the oil,
wait a little bit for the air inside the flask to heat
up, and then close the stopcock again. The Erlenmyer flask wants to bob up out of the
oil, so use some tape to hold it down.
Take pressure and temperature data.

Although it's undesirable that the small amount of air in the tubing
won't be at exactly the same temperature as the rest of the air, we can't
avoid this, because the mineral oil is hot enough to melt the tubing.

When you're done with the mineral oil, wash the beaker with soap and water.

We'll next do a series of measurements at lower temperatures:

\begin{itemize}
\item \emph{Room-temperature tap water} Make sure that the pressure
drops by about a third when you come down to this temperature. If it
doesn't, you probably have an air leak.

\item \emph{Ice-cold alcohol}

\item \emph{Acetone/dry ice slurry:}
Use the hammer to knock off a piece of dry ice. Remove the piece
using the tongs, stick it in a baggie, and crush it up some more
with a hammer.
Add the dry ice to the acetone (nail polish remover) slowly; if you
do it rapidly, it can fizz violently.
Mix the dry ice and acetone to make a slush.
Acetone is flammable, so avoid creating any sparks or flames.
This mixture cannot be dumped down the drain when you're done;
keep it so that it can be disposed of properly. To reduce the amount
of waste disposal, you can reuse another group's slurry.
You should be able to get the temperature down to about $-60$ to
$-80$ celsius. If you only get to $-20$, you're doing something wrong.
\end{itemize}

\analysis

Graph the temperature and pressure against each other. Does
the graph appear to be linear? If so, extrapolate to find
the temperature at which the pressure would be zero. 

If your data are nice and linear, then your main source of
error will be random errors, and you should then determine
error bars for your value of absolute zero using the
techniques discussed in Appendix \ref{appendix:graphing}.
The appendix discusses finding the \emph{slope} of a line,
although in this lab it's actually the $x$- or $y$-intercept that you want;
the technique is analogous, however.
The easiest way to estimate the error bars on the points
is to use the typical amount of scatter of the points about
the best-fit line. For example, if the systematic trend of the
data is linear, but the points generally lie an average of
about $5\degcunit$ away from the line, then the error bars
are approximately $5\degcunit$.
