\lab{The Michelson Interferometer}\label{lab:michelson}

\apparatus
\equipn{Michelson interferometer}{1/group}
\equipn{Na and H gas discharge tubes}{1/group}
\equipn{tools inside drawer}{1 set/group}
\equipn{$2\times4$ piece of wood}{3/group}
\equip{colored filters (Cambosco and others)}

\begin{goals}

\item[] Determine the wavelength of a line of the emission spectrum of sodium or hydrogen.

\end{goals}

The Michelson interferometer is a device for measuring the wavelength of
light, used most famously in the Michelson-Morley experiment of 1887,
which was later interpreted as disproving the existence of the luminiferous
aether and supporting Einstein's theory of special relativity.

\fig{op-mic-schematic}

As shown in the figure, the idea is to take a beam of light from the source,
split it into two perpendicular beams, send it to two mirrors, and then recombine
the beams again. If the two light waves are in phase when recombined, they will
reinforce, but if they are out of phase, they will cancel. Since the two waves
originated from the splitting of a single wave, the only reason they would be out
of phase was if the lengths of the two arms of the apparatus were unequal. Mirror
A is movable, and the distance through which it moves can be controlled and measured
extremely accurately using a micrometer connected to the mirror via a lever.
If mirror A is moved by distance equal to a quarter of a wavelength of the light,
the total round-trip distance traveled by the wave is changed by half a wavelength,
which switches from constructive to destructive interference, or vice versa.
Thus if the mirror is moved by a distance $d$, and you see the light go through
$n$ complete cycles of appearance and disappearance, you can conclude that the
wavelength of the light was $\lambda=2d/n$.

To make small and accurate adjustments
of the mirror easier to do, the micrometer is connected to it through a lever that
reduces the amount of movement by a factor $k$, approximately equal to 5.23; the micrometer reads the
bigger distance $D=kd$ that it actually travels itself, so the wavelength is
$\lambda=2D/kn$.

Another trick to make the apparatus easier to use is that the mirrors A and B are
slightly curved. This means that instead of seeing a field of light that varies
uniformly between dark and bright as you turn the knob, instead you see a set of
concentric rings (called fringes), which expand or contract depending on which
direction you turn the knob.

Turn on the sodium discharge tube, and let it warm up until it's yellow.

Remove the drawer from the box, and take out the tool kit. Unscrew the screws
on the bottom of the box that lock the interferometer to the floor of the box,
and \emph{very carefully} take the instrument out of the box. Screw the two
aluminum legs into the bottom of the interferometer, and lay a piece of
wood flat under the third leg, which is a threaded rod; this makes the apparatus
level.

Place the discharge tube near the entrance window of the apparatus. If you look
through the viewing window, you will see the image of the tube itself, reflected
through the mirrors. To make this into a uniform circle of light, place the
ground glass screen (inside the bag of tools) in the bracket at the entrance
window.

Mirror B needs to be perfectly perpendicular to mirror A, and its vertical plane
needs to be matched to mirror A's. This is adjusted using the knobs on mirror B,
one for vertical adjustment and one for horizontal. A rough initial adjustment
can be done by aligning the two images of the circular entrance window. You can
then hang the metal pointer (from the bag of tools) on the top of the ground glass
screen, and do a better adjustment so that the two images of the pointer's
tip coincide. You should now see a set of very fine concentric circular interference
fringes, centered on a point outside of the field of view. The final, fine
adjustment is obtained by bringing the center of this pattern to the center of the
field of view.

The micrometer has a millimeter scale running from 0 to 25 mm, with half-millimeter
divisions on the bottom. To take a reading on it, first read the number of
millimeters and half-millimeters based on where the edge of the cylindrical
rotating part lies on this scale. Then add on the reading from the vernier
scale that runs around the circumference of the rotating part, which runs from
0.000 to 0.500 mm. You should be able to estimate to the nearest thousandth of
a millimeter (tenth of a vernier division).

While looking at the interference fringes, turn the knob on the micrometer.
You will see them either expand like smoke rings, or contract and disappear
into the center, depending on which way you turn the knob. Rotate the knob
while counting about 50 to 100 fringes, and record the two micrometer readings before and
after. The difference between these is $D$. It helps if you prop your head on the table, and move the micrometer knob
smoothly and continuously. Moving your head disturbs the pattern, and halting the
micrometer knob tends to cause backlash that confuses the count of fringes by
plus or minus one. 

It has been an ongoing project to get these spectrometers back in operation and fully calibrated for the first time in many years.
In spring 2006, my students in physics 223 gave them a thorough test drive. 
In spring 2007, we started taking data to determine $k$ accurately for each spectrometer, using
the known wavelength of the sodium emission line at 589 nm. Their data are on sheets inside each
spectrometer's box. That class also experimented with using the apparatus to measure the wavelengths
of some lines in the spectrum of hydrogen, which is of some fundamental interest because it is the
simplest of all atoms. Since hydrogen's spectrum, unlike sodium's, includes several different
visible lines of similar intensity, this required using colored filters to select the desired line.
They found that filter \#2 from the Cambosco box worked well for the red line, and \#8 for the blue-green line.
Lines with short wavelengths were more difficult to do.
For the next class that does the lab, my goal is to accumulate more calibration data, so we can start to
detect whether certain data points are off because of $\pm 1$ errors in counting the number of fringes.
I would also like to make more progress in measuring lines of the hydrogen spectrum accurately.
