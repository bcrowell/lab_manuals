\lab{The Michelson Interferometer}\label{lab:michelson}

\apparatus
\equipn{Michelson interferometer}{1/group}
\equipn{Na discharge tube}{1/group}
\equipn{tools inside drawer}{1 set/group}
\equipn{$2\times4$ piece of wood}{3/group}
\equipn{helium-neon laser}{3}
\equip{various diverging leses}

\begin{goals}

\item[] Observe the basic operation of a Michelson interferometer.

\item[] Investigate the feasibility of doing other measurements with the instrument.

\end{goals}

\introduction

The Michelson interferometer is a device for measuring the wavelength of
light, used most famously in the Michelson-Morley experiment of 1887,
which was later interpreted as disproving the existence of the luminiferous
aether and supporting Einstein's theory of special relativity.

\fig{op-mic-schematic}

As shown in the figure, the idea is to take a beam of light from the source,
split it into two perpendicular beams, send it to two mirrors, and then recombine
the beams again. If the two light waves are in phase when recombined, they will
reinforce, but if they are out of phase, they will cancel. Since the two waves
originated from the splitting of a single wave, the only reason they would be out
of phase was if the lengths of the two arms of the apparatus were unequal. Mirror
A is movable, and the distance through which it moves can be controlled and measured
extremely accurately using a micrometer connected to the mirror via a lever.
If mirror A is moved by distance equal to a quarter of a wavelength of the light,
the total round-trip distance traveled by the wave is changed by half a wavelength,
which switches from constructive to destructive interference, or vice versa.
Thus if the mirror is moved by a distance $d$, and you see the light go through
$n$ complete cycles of appearance and disappearance, you can conclude that the
wavelength of the light was $\lambda=2d/n$.

\apparatus

To make small and accurate adjustments
of the mirror easier to do, the micrometer is connected to it through a lever that
reduces the amount of movement by a factor $k$, which is supposed to be exactly equal to 5; the micrometer reads the
bigger distance $D=kd$ that it actually travels itself, so the wavelength is
$\lambda=2D/kn$.

When you look through the device, you see a certain field of view. Within that
field of view are mirror A and the image $\zu{B}'$ of mirror B, which is nearly superimposed on A.
The path-length difference between the two waves can be thought of as the distance between
A and $\zu{B}'$.
The field of view covers a certain small but finite range of angles, forming
a narrow cone with its vertex at your eye. For rays forming a small angle
$\alpha$ relative to the axis, the effect is to multiply the path-length distance
by a factor of $\cos\alpha$. If the two arms of the interferometer are exactly
equal, then the path-length distance between them is zero, and multiplying it
by $\cos\alpha$ has no effect. But when the arms are unequal, the condition for
constructive or destructive interference becomes dependent on $\alpha$, and the
result is that you see a bull's-eye pattern of concentric circular fringes.
When you turn the knob, these fringes expand or contract, appearing from or disappearing
into the center. The more equal the two arms, the smaller the number of fringes.

\setup

Turn on the sodium discharge tube, and let it warm up until it's yellow.

Remove the drawer from the box, and take out the tool kit. Unscrew the screws
on the bottom of the box that lock the interferometer to the floor of the box,
and \emph{very carefully} take the instrument out of the box. Screw the two
aluminum legs into the bottom of the interferometer, and lay a piece of
wood flat under the third leg, which is a threaded rod; this makes the apparatus
level.

Place the discharge tube near the entrance window of the apparatus. If you look
through the viewing window, you will see the image of the tube itself, reflected
through the mirrors. To make this into a uniform circle of light, place the
ground glass screen (inside the bag of tools) in the bracket at the entrance
window.

Mirror B needs to be perfectly perpendicular to mirror A, and its vertical plane
needs to be matched to mirror A's. This is adjusted using the knobs on mirror B,
one for vertical adjustment and one for horizontal. A rough initial adjustment
can be done by aligning the two images of the circular entrance window. You can
then hang the metal pointer (from the bag of tools) on the top of the ground glass
screen, and do a better adjustment so that the two images of the pointer's
tip coincide. You should now see a set of very fine concentric circular interference
fringes, centered on a point outside of the field of view. The final, fine
adjustment is obtained by bringing the center of this pattern to the center of the
field of view.

\labpart{Basic Operation}

To get an idea of the basic operation of the instrument, you'll perform a rough
measurement of the wavelength of the yellow line emitted by the sodium discharge tube.
This is not really a task for which the spectrometer is well suited for giving high-precision
results, but you should be able to approximately verify the known wavelength of
about 589 nm.

The micrometer has a millimeter scale running from 0 to 25 mm, with half-millimeter
divisions on the bottom. To take a reading on it, first read the number of
millimeters and half-millimeters based on where the edge of the cylindrical
rotating part lies on this scale. Then add on the reading from the vernier
scale that runs around the circumference of the rotating part, which runs from
0.000 to 0.500 mm. You should be able to estimate to the nearest thousandth of
a millimeter (tenth of a vernier division).

While looking at the interference fringes, turn the knob on the micrometer.
You will see them either expand like smoke rings, or contract and disappear
into the center, depending on which way you turn the knob. Rotate the knob
while counting at least about 20 fringes (optimally 50 or 100), and record the two micrometer readings before and
after. The difference between these is $D$. It helps if you prop your head on the table, and move the micrometer knob
smoothly and continuously. Moving your head disturbs the pattern, and halting the
micrometer knob tends to cause backlash that confuses the count of fringes by
plus or minus one. 

Check that you get approximately the right wavelength for the yellow sodium line.

\section*{Exploring Other Possibilities with the instrument}

Up until fall 2012, I was having my students work on improving the technique
used in part A to get high-precision data. This was time-consuming and fatiguing,
and we never had much luck getting great results, since it was difficult to
turn the micrometer knob smoothly enough to be sure whether we were miscounting
the number of fringes due to hand twitches or backlash in the gears. Then, in
a long-forgotten corner of the physics stockroom, guarded by jewel-eyed statues
of demons and curses written in Egyptian hieroglyphics, I came across a lab manual
from 1971 that discussed a variety of other things that could be done with these
interferometers that might be more interesting and successful. My goal for spring
2013 is to have each lab group try out one of the following possibilities and see
how it goes.

Groups 1, 4, and 7 should do part B, groups 2 and 5 part C, and groups 3 and 6 part D.

\labpart{Use of laser light}

It is supposed to be possible to use a laser as a source of light and project the fringes onto a screen
or the wall. This would be much more physically comfortable than hunching over and staring into the 
interferometer, and in fact for safety you should never look into the interferometer while doing
the laser version the experiment. I think the idea is to use the laser to produce a nearly
pointlike source of light that is much brighter than what can be provided by a discharge tube,
making the light bright enough to see when projected. By shining the laser beam onto a
diverging lens, you can produce a spreading cone of monochromatic, coherent light.
Adjust the cone so that it is about the right size to light up the whole diagonal mirror.
If this works, you could try to measure the wavelength of the laser light.

\labpart{Spacing of the sodium doublet}

Essentially the only way scientists have of getting detailed, high-precision information
about the structure of atoms and nuclei is by measuring the set of wavelengths of light
they emit. As you'll see later in the course, these wavelengths can be related to the sizes
of the ``quantum jumps'' between different energy levels of the atom. Each of these wavelengths
is referred to as a ``line,'' because in some types of spectrometers that's what they look like,
and the whole set of lines is called the spectrum of that atom.
An unfortunate fact of life for the spectroscopist is that the spectrum often contains ``doublets,''
meaning lines that are very close together and hard to distinguish, and in fact the yellow line
in the sodium spectrum that you measured in part A is really a doublet whose members differ
in wavelength by a fraction of a nanometer.

The Michelson interferometer
is particularly well adapted to a certain sneaky trick for measuring the difference in wavelength
between two lines in a doublet, even when the difference is extremely small, as it is for
sodium. The idea is that if the lines are roughly equal in intensity, then the bull's-eye pattern
you see is actually two bull's-eye patterns superimposed on top of one another. If you move
the micrometer to a random position, then it's a matter of chance how well these two patterns
agree. They could happen to agree perfectly, in which case the bull's eye would look just as distinct
as if there were only one line, but they could also disagree perfectly, so that there would no
contrast at all between light and dark. The ancient lab manual says: 
            ``Loosen the carriage lock screw and move the carriage by hand. Note that the fringes
            pass alternately from a condition of high contrast to one of almost complete
            disappearance. With the micrometer screw set near one of its extreme limits
            and the carriage at one of the conditions of almost complete disappearance,
            tighten the carriage lock screw." 
Then you're supposed to move to another condition of minimum
            contrast and take another micrometer reading.

The analysis then works like this. Let the wavelengths constituting the doublet be
be $\lambda_1$ and $\lambda_2$, differing by $\Delta \lambda = \lambda_2-\lambda_1$.
Let $d_1$ and $d_2$ be two successive path-length differences at which there is maximum
          contrast. Traveling a distance $\Delta d$
          between these two values of $d$, we pass through $N+1$ cycles worth of the shorter
          $\lambda_1$ and $N$ cycles  for the longer $\lambda_2$, so
          $2 \Delta d = (N+1) \lambda_1=N \lambda_2$. Algebra gives
          $\Delta \lambda=\lambda_1 \lambda_2/2 \Delta d$.
    The fractional precision with which we can measure $\Delta d$ equals the fractional
           precision with which you can measure $\Delta \lambda$, which is a huge win,
           because $\Delta \lambda$ is very small.

Attempt to measure $\Delta \lambda$ for the yellow sodium doublet.

\labpart{Unknown index of refraction}

The Michelson interferometer is unusual in its ability to work with white light.
This is because it's possible to get into the position where the lengths of the
two arms are equal, so that constructive interference occurs at the center of
the bull's eye for all wavelengths. At any other position, we get
a random mixture of constructive and destructive interference for all the different
wavelengths that are present. In fact, this allows the equal-arm condition
to be determined extremely accurately. It's hard to think of any other technique
that allows two large distances to be made equal to within micrometers or nanometers!

Using this technique, one can accurately determine the unknown index of refraction of an
thin piece of some material such as a glass microscope slide. The idea is to equalize
the arms, then insert the unknown and readjust the interferometer in order to
restore the interference pattern. In the new condition, it's not the lengths of the
two arms that are equal but their \emph{optical} lengths, i.e., the time it takes
light to travel along them.
