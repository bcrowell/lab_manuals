\addtocounter{chapter}{-1}
\renewcommand\thechapter{c1.12}
\lab{The mechanical advantage of a pulley system}\label{lab:covid-mechanical-advantage}

\section*{About this lab}

\covid\ 
This is a lab intended for students to do at home during the Covid-19 epidemic. 

\apparatus
\equipn{SEOH pulley, plastic triple tandem}{2}
\equip{string}
\equip{ruler}
\equip{a large supply of coins}
\equip{digital balance}
\equip{suction cup with hook}

\begin{goals}

\item[] Analyze the motion of systems of pulleys.

\item[] Measure the mechanical advantage of a pulley system and compare with theory.

\end{goals}

\introduction

A system of pulleys, such as the one in the figure, can change the direction and magnitude of an
applied force. The ratio of the output force to the input force is the
mechanical advantage of a pulley system, $MA =F_\text{out}/F_\text{in}$.

\fig{me-pul-example}

Generally, a fixed pulley changes the direction of the force, and a
moving pulley changes the mechanical advantage. If the pulleys are
ideal (mass and friction are negligible), the ideal mechanical
advantage (IMA) is equal to the number of supporting strings.

\observations 

\labpart{An advantage of 2}
By using two of the pulleys, set up a system with an ideal mechanical advantage of 2,
as shown in the photo. The pulleys come in two sets of three. For this part of the lab you will use only two, while
the other four pulley wheels have no string wound over them.

\fig{me-pul-advantage-2}

Hang some number of coins from the hook of the moving pulleys on the moving side. The weight of these coins sets
$F_\text{out}$. Next, hang coins from the $F_\text{in}$ side until $F_\text{in}$ balances $F_\text{out}$. 

Because theory only predicts the ratio of these two forces, we don't expect that the two weights will be
uniquely determined. We should be able to scale them both up or down by the same factor while maintaining
equilibrium.  To get the best possible resolution for your setup, use as large a number of coins on both sides as
is practical.

Take a picture once the system is in equilibrium. 

Because of friction, you will probably find that even if one of the forces is fixed, the other one can have a range
of values while still staying in equilibrium. Since $F_\text{out}$ is bigger, you can measure it with better resolution
while changing it one coin at a time. Keep $F_\text{in}$ fixed and determine the range of possible values
for $F_\text{out}$ that maintains equilibrium.

\labpart{An advantage of 3}

Do a similar measurement for the system shown in the second photo, which has an ideal mechanical advantage of 3.

\fig{me-pul-advantage-3}

\labpart{Higher mechanical advantages}

Continue for mechanical advantages of 3, 4, 5, and 6. The third photo shows the case of an advantage of 6

\fig{me-pul-advantage-6}

\labpart{Further improvements (optional)}
The lab setup is not ideal. You can improve the first result by minimizing the errors. Think about various methods to reduce errors. Then, adjust the experimental setup and make the second measurements. Write changes and the result of the improvement in the lab report. 

\analysis
Determine the actual mechanical advantages. Each value will be a range, both because of the
finite resolution of the measurements and because of friction. The range can be determined
by the usual technique of propagation of errors as described
in appendix \ref{appendix:errpropagation}.

If you do find that there is a range of values due to friction, then it would be of interest
to determine whether the ideal value falls within this range or outside it. (This is not
obvious and may depend on whether the friction is kinetic or, as in this lab, static.)

If a person exerts a force, $F_\text{in}$, over a distance, $d$, what
is the work done by the person? What is the work done on the object
hanging on the pulley system? Does this violate conservation of
energy?

\section*{Preparation}


\prelabquestion (Mechanics Ch 5 problem 36)
A person can pull with a maximum force F. What is the maximum mass that the person can lift with the pulley setup shown in the figure in the introduction?
