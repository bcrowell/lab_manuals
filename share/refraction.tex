\lab{Refraction and Images}\label{lab:refraction}

\apparatus
\equip{plastic box}
\equip{propanol (1 liter/group, to be reused)}
\equip{laser}
\equip{spiral plastic tube and fiber optic cable for demonstrating total internal reflection}
\equip{ruler}
\equip{protractor}
\equip{butcher paper}
\equip{funnel}

\begin{goals}

\item[] Test whether the index of refraction of a liquid is proportional to its density.

\item[] Observe the phenomena of refraction and total internal reflection.

\item[] Locate a virtual image in a plastic block by ray tracing,
and compare with the theoretically predicted position of the image.
\end{goals}

\introduction

Without the phenomenon of refraction, the lens of your eye
could not focus light on your retina, and you would not be
able to see. Refraction is the bending of rays of light that
occurs when they pass through the boundary between two media
in which the speed of light is different. 

Refraction occurs for the following
reason. Imagine, for example, a beam of light entering a
swimming pool at an angle. Because of the angle, one side of
the beam hits the water first, and is slowed down. The other
side of the beam, however, gets to travel in air, at its
faster speed, for longer, because it enters the water later
--- by the time it enters the water, the other side of the
beam has been limping along through the water for a little
while, and has not gotten as far. The wavefront is therefore
twisted around a little, in the same way that a marching band
turns by having the people on one side take smaller steps.

\fig{op-ref-snell}

Quantitatively, the amount of bending is given by Snell's law:
\begin{equation*}
    n_i\sin \theta_i = n_t \sin \theta_t    ,
\end{equation*}
where the subscript $i$ refers to the incident light and
incident medium, and $t$ refers to the transmitted light and
the transmitting medium. This relation can be taken as defining
the quantities $n_i$ and $n_t$, which are known as the indices of
refraction of the two media. Note that the angles are defined
with respect to the normal, i.e., the imaginary line
perpendicular to the boundary. 

Also, not all of the light is transmitted. Some is reflected
--- the amount depends on the angles. In fact, for certain
values of $n_i$, $n_t$, and $\theta_i$, there is no value of
$\theta_t$ that will obey Snell's law ($\sin\theta_t$ would
have to be greater than one). In such a situation, 100\% of
the light must be reflected. This phenomenon is known as
total internal reflection. The word internal is used because
the phenomenon only occurs for $n_i>n_t$. If one medium is
air and the other is plastic or glass, then this can only
happen when the incident light is in the plastic or glass,
i.e., the light is trying to escape but can't. Total internal
reflection is used to good advantage in fiber-optic cables
used to transmit long-distance phone calls or data on the
internet --- light traveling down the cable cannot leak out,
assuming it is initially aimed at an angle close enough to
the axis of the cable.

Although most of the practical applications of the
phenomenon of refraction involve lenses, which have curved
shapes, in this lab you will be dealing almost exclusively with flat surfaces.

\mysubsubsection{Preliminaries}

Check whether your laser's beam seems to be roughly parallel.

\observations

\labpart{ Index of refraction of alcohol}

The index of refraction is sometimes referred to as the
optical density. This usage makes sense, because when a substance
is compressed, its index of refraction goes up. In this part
of the lab, you will test whether the indices of refraction of
different liquids are proportional to their mass densities.
Water has a density of $1.00\ \zu{g}/\zu{cm}^3$ and an index
of refraction of 1.33. Propanol has a density of $0.79\ \zu{g}/\zu{cm}^3$.
You will find out whether its index of refraction is lower than water's in the same proportion.
The idea is to pour some alcohol into a transparent plastic box and measure
the amount of refraction at the interface between air and alcohol.

Make the measurements you have planned in order to determine
the index of refraction of the alcohol. The laser and the box can
simply be laid flat on the table. Make sure that the laser
is pointing towards the wall.

\labpart{ Total internal reflection}


Try shining the laser into one end of the spiral-shaped
plastic rod. If you aim it nearly along the axis of the
cable, none will leak out, and if you put your hand in front
of the other end of the rod, you will see the light coming
out the other end. (It will not be a well-collimated beam
any more because the beam is spread out and distorted when
it undergoes the many reflections on the rough and
curved inside the rod.)

There's no data
to take.
The point of having this as part of the lab is simply that
it's hard to demonstrate to a whole class all at once.

\labpart{ A virtual image}

Pour the alcohol back into the container for reuse, and pour water
into the box to replace it.

Pick up the block, and have your partner look sideways
through it at your finger, touching the surface of the
block. Have your partner hold her own finger next to the
block, and move it around until it appears to be as far away
as your own finger. Her brain achieves a perception of
depth by subconsciously comparing the images it receives
from her two eyes. Your partner doesn't actually need to be
able to see her own finger, because her brain knows how to
position her arm at a certain point in space.
 Measure the distance $d_i$, which is
the depth of the image of your finger relative to the front of the block.

\fig{op-ref-image}

\fig{op-ref-laser}

Next we will use the laser to simulate the rays coming from the finger,
as shown in the figure.
Shine the laser at the point
where your finger was originally touching the block, observe
the refracted beam, and draw it in. Repeat this whole
procedure several times, with the laser at a variety of
angles. Finally, extrapolate the rays leaving the block back
into the block. They should all appear to have come from the
same point, where you saw the virtual image. You'll need to
photocopy the tracing so that each person can turn in a copy
with his or her writeup.

\prelab

m4_ifdef([:__sn:],[:%
You should have already done the laser safety checklist for the relativity lab.
:],[:
Print out, complete, and turn in the laser safety checklist, appendix \ref{appendix:lasersafety}.
:])

\prelabquestion  Laser beams are supposed to be very nearly parallel (not
spreading out or contracting to a focal point). Think of a
way to test, roughly, whether this is true for your laser.

\prelabquestion  Plan how you will determine the index of refraction in part A.

\prelabquestion
m4_ifdef([:__sn:],[:%
\emph{Skip this question if you were assigned problem 12-63 in Simple Nature.}
:],[:%
:])%
 You have complete freedom to choose any incident angle you
like in part A. Discuss what choice would give the highest possible precision
for the measurement of the index of refraction.

\analysis

Using your data for part A, extract the index of refraction of propanol, with error bars.
Test the hypothesis that the index of refraction is proportional to the density
in the case of water and propanol.

Using trigonometry and Snell's law, make a theoretical
calculation of $d_i$. You'll need to use the small-angle
approximation $\sin  \theta \approx\tan \theta\approx\theta$, for $\theta $
measured in units of radians. (For large angles, i.e.
viewing the finger from way off to one side, the rays will
not converge very closely to form a clear virtual image.)

Explain your results in part C and their meaning.

Compare your three values for $d_i:$ the experimental value
based on depth perception, the experimental value found by
ray-tracing with the laser, and the theoretical value
found by trigonometry.
