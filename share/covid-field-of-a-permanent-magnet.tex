\renewcommand\thechapter{c2.1a}
\lab{Field of a permanent magnet}\label{lab:field-of-a-permanent-magnet}

\section*{About this lab}

\covid\ 
It is intended to be used around the first week in Physics 222.

\apparatus
\equip{permanent magnet}
\equip{compass}
\equip{ruler}

\begin{goals}

\item[] Find how the magnetic field of a magnet changes with
distance along one of the magnet's lines of symmetry.

\end{goals}

You can infer the strength of the bar magnet's field at a
given point by putting the compass there and seeing how
much it is deflected from the direction of the ambient
field due to the earth and magnetic materials in the building.

The task can be simplified quite a bit if you pick one of the
magnet's lines of symmetry and measure the field at points
along that line. It should have
an axis of symmetry that coincides with its center-line
in the long direction, and another such axis for the short direction.
The field at points on the first axis should be parallel to the axis,
while the field on the second axis should be perpendicular to that axis.

1. Line up your magnet so it is pointing perpendicular to the ambient field,
(nominally east-west). Choose one
of the two symmetry axes, and measure the
deflection of the compass at two points along that axis.
For your first point, find the
distance $r$ at which the deflection is 70 degrees; this angle is chosen because
it's about as big as it can be without giving very poor relative precision 
in the determination of the magnetic field. For your second data-point,
use twice that distance. Using the result of homework problem 
1-2 in Fields and Circuits, by what factor does the field decrease
when you double $r$?

You will probably find that the ambient field in the room is strongly
influenced by the magnetic field of the building and possibly the
furniture.  For example, in the lab room where I usually do this
exercise, the lab benches contain iron or steel parts that distort the
magnetic field, as my students can easily observe putting a compass on
the top of the bench and sliding it around to different places. 
A good way to work around this problem is to position a ruler a few 
inches above the top of your table, and carry out the experiment along the line
formed by the stick.

It is also common to find that the
magnetic field due to the building materials in the building is
significant, and that this field varies from place to place.  Therefore you
should move the magnet while keeping the compass in one place.  Then
the field from the building becomes a fixed part of the background
experienced by the compass, just like the earth's field. 

Note that the measurements are very sensitive to the
relative position and orientation of the bar magnet and
compass. 

2. Based on your two data-points, form a hypothesis about the variation
of the magnet's field with distance according to a power law $B\propto
r^p$. 

3. Take additional data at a range of distances, including the smallest
and largest distances that it is practical to do. Graph the data on a log-log
plot (i.e., with the log of $B$ on one axis and the log of $r$ on the other),
and test whether your hypothesis actually holds.

\section*{Preparation}

If the bar magnet's field follows the power law $B\propto r^p$, for some constant $p$,
predict how the log-log plot should look.


