\renewcommand\thechapter{c2.1}
\lab{Field of a permanent magnet}\label{lab:field-of-a-permanent-magnet}

\section*{About this lab}

\covid\ 
It is intended to be used around the first week in Physics 222.

\apparatus
\equip{permanent magnet (ferrite disk)}
\equip{compass}
\equip{ruler}

\begin{goals}

\item[] Find how the magnetic field of a magnet changes with
distance along one of the magnet's lines of symmetry.

\end{goals}


\section*{Preliminary testing}

We got these magnets and compasses from very low-cost suppliers who don't do much
quality control. For this reason I bought extra ones and did some basic testing
before mailing them out, getting rid of the compasses that didn't work.
However, it will be a good idea for you to reproduce my
testing, partly just to get a better idea of how you expect these things to behave.

In the first photo, I'm holding the compass in my hand in the air above my desk, so
that it will not interact with any magnetic materials in the desk. Its red needle is
pointing in the direction that I know is approximately magnetic north here in
Southern California (roughly toward a spot in north-central Canada, about 13 degrees
east of north). If it's the white needle on yours that points north, that's not a
problem for this lab, although you would want to know that before depending on
the compass when you went out hiking.

\fig{em-fpe-preliminary-test}

If you vigorously spin the compass in order
to change its orientation, you should find that it responds by returning to the
correct orientation. If it just sticks in whatever orientation you put it in, then that's
a problem. It could just be that the needle is dragging against the bottom plate, which
isn't a problem if you can just tilt the compass a little bit in order to keep it from
dragging.

If the compass doesn't point toward magnetic north, but does always return to a consistent
direction, it could be that you just have building materials in your house that are
overwhelming the earth's field. If so, then that's not a problem for this experiment,
but it would be good to understand. All you need is some ambient field. It doesn't matter
what direction it's in. You could check by taking the compass outside.

In the second photo, I place the magnet close enough to the compass so that its field
is much stronger than the ambient field, and in an orientation that deflects the compass
by about 90 degrees. In the third photo I reverse the orientation of the magnet, and the
compass responds by reversing itself. This is the behavior we expect if the compass really is magnetized
and the magnet really is magnetized as well.

If one or of these two tools is just
an unmagnetized piece of magnetic material, and only the other one is magnetized, then they'll
interact just like a magnet sticking on your fridge, but their interaction won't demonstrate any
polarity. To see what this would be like, you could try testing each tool against
some object made of a magnetic material like steel or iron, e.g., a pair of scissors.
The scissors are nonpolar, so they will attract either tip of the compass needle,
and will attract the disk magnet regardless of the disk magnet's orientation.
(Be aware that while doing this kind of testing, you could also inadvertently magnetize
your scissors. To keep from getting confused by this, don't bring the scissors within
about an inch of the disk magnet.)

\section*{Observations}

You can infer the strength of the bar magnet's field at a
given point by putting the compass there and seeing how
much it is deflected from the direction of the ambient
field due to the earth and magnetic materials in the building.

The task can be simplified quite a bit if you pick one of the
magnet's lines of symmetry and measure the field at points
along that line. It should have
an axis of symmetry that coincides with its center-line
in the long direction, and another such axis for the short direction.
The field at points on the first axis should be parallel to the axis,
while the field on the second axis should be perpendicular to that axis.

1. Line up your magnet so it is pointing perpendicular to the ambient field,
(nominally east-west). Choose one
of the two symmetry axes, and measure the
deflection of the compass at two points along that axis.
For your first point, find the
distance $r$ at which the deflection is 70 degrees; this angle is chosen because
it's about as big as it can be without giving very poor relative precision 
in the determination of the magnetic field. For your second data-point,
use twice that distance. Using the result of homework problem 
1-2 in Fields and Circuits, by what factor does the field decrease
when you double $r$?

You will probably find that the ambient field in the room is strongly
influenced by the magnetic field of the building and possibly the
furniture.  For example, in the lab room where I usually do this
exercise, the lab benches contain iron or steel parts that distort the
magnetic field, as my students can easily observe putting a compass on
the top of the bench and sliding it around to different places. 
A good way to work around this problem is to position a ruler a few 
inches above the top of your table, and carry out the experiment along the line
formed by the stick.

It is also common to find that the
magnetic field due to the building materials in the building is
significant, and that this field varies from place to place.  Therefore you
should move the magnet while keeping the compass in one place.  Then
the field from the building becomes a fixed part of the background
experienced by the compass, just like the earth's field. 

Note that the measurements are very sensitive to the
relative position and orientation of the bar magnet and
compass. 

2. Based on your two data-points, form a hypothesis about the variation
of the magnet's field with distance according to a power law $B\propto
r^p$. 

3. Take additional data at a range of distances, including the smallest
and largest distances that it is practical to do. Graph the data on a log-log
plot (i.e., with the log of $B$ on one axis and the log of $r$ on the other),
and test whether your hypothesis actually holds.

\section*{Preparation}

If the bar magnet's field follows the power law $B\propto r^p$, for some constant $p$,
predict how the log-log plot should look.


