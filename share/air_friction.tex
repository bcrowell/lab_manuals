\lab{Air Friction}\label{lab:air-friction}

\apparatus
\equipn{coffee filters}{10/group}
\equipn{stopwatch}{1/group}
\equip{computer with sonar sensor}
\equip{wire cages to protect sensors}

\goal

Determine how the force of air friction depends on the
velocity of a moving object.

As an alternative, you may create your own technique for
doing the same sort of experiment for friction between two
surfaces wetted with a liquid such as water, vegetable oil,
or machine oil --- the result might be more interesting,
since it is not to be found in textbooks. If you are
interested in doing this, discuss it in advance with your instructor.

\introduction

Friction between solid objects occurs all the time in our
daily lives. The frictional force exerted by the air on a
solid object is not as often evident, but it is responsible
for the wind blowing our hair, for the slow dropping of a
feather, and for our cars' poorer gas mileage at freeway
speeds compared to more moderate speeds.

The latter effect suggests that air friction might increase
with speed, unlike solid-solid friction, which is nearly
independent of speed. By Newton's first law, a car or a jet
plane cruising at constant speed must have zero total force
on it, so if the air friction force gets stronger with
speed, that would explain why a greater forward-pushing
force would be needed to travel at high speeds. For
instance, a car traveling at low speed might have a -10 kN
air friction force pushing backward on it, so in order to
have zero total force on it the road must be making a
forward force of +10 kN. At a higher speed, air friction
might increase to -30 kN, so the road would need to make a
forward force of +30 kN. The car convinces the road to make
the stronger force by pushing  backward on the road more
strongly: by Newton's third law, the car's force on the road
and the road's force on the car must be equal in magnitude
and opposite in direction. The car burns more gas because it
must push harder against the road.

Your goal in this lab is to find a proportionality relating
the force of air friction to the velocity at which the air
rushes over the object. For instance, you may find the rule
\begin{equation*}
	F\propto v   \qquad ,
\end{equation*}
which is a shorthand for
\begin{equation*}
	F  = (\text{some number})(v)   \qquad   .
\end{equation*}

The numerical value of ``some number'' is not very
interesting, because we would expect it to be different for
different objects, which is why you would write your result
as $F\propto v$. This proportionality would tell you for
instance that anytime the speed was doubled, the result
would be twice as much air friction.

Suppose instead you find that doubling the speed makes the
force eight times greater, multiplying the speed by 10 makes
the force 1000 times greater, and so on. In each case, the
force is being multiplied by the third power of the increase
in the speed, i.e., $F\propto v^3$.

\observations

There are two possible methods for measuring the terminal velocity
of the filter.

The first method is shown in the figure below. We use coffee
filters because they don't tumble or sway very much as they
fall, and because they allow us to easily change the mass of
our falling object by nesting more coffee filters inside the
bottom one, without changing its aerodynamic properties.
The filters will start speeding up when you release them
near the ceiling, but as they speed up, the upward force of
air friction on them increases, until they reach a speed at
which the total force on them is zero. Once at this speed,
they obey Newton's first law and continue at constant speed.
If the number of coffee filters is small,
they will have reached their maximum speed within the first
half a meter or so. By the time they are even with the edge
of the lab bench, they are moving at essentially their full
speed. You can then use the stopwatch to determine how long
it takes them to cover the distance to the floor, which will
allow you to find their speed. During this final part of the
fall, you know the upward force of air friction must be as
great as the downward force of gravity, so you can
determine what it was.

\fig{me-air-howto}

A different technique is to drop the filters onto a sonar
sensor of the type used in lab \ref{lab:kinematics}. You can
put the sensor on the floor facing up, and put the wire cage
over it to keep it from getting damaged by being stepped on
inadvertently.

For a long time, I had my students do the lab using only the
first method, but now I'm experimenting with the second method.
A couple of advantages of the second method are that (1) it doesn't
depend on human reflexes, and (2) it gives you a real-time picture
of the motion, so it's easier to tell whether the filters are
actually reaching terminal velocity. The latter is an important
issue, because it gives you a better chance of being able to take
data over a wide range of values for $F$ and $v$, but without
running into problems with cases where the filters don't really
reach terminal veloicity. On the other hand, the computer method
has some practical problems, such as the tendency of the filters
to drift sideways instead of heading straight down onto the sensor.
This is an opportunity for you to do something like what real
scientists do: use your ingenuity and try different things to see
what works best!

Note that if the coffee filters get too flattened out,
they'll flutter, giving lousy results.

Take data with stacks of various numbers of coffee filters.
You will get the most clearcut determination of the power
law relationship if your data cover the largest possible
range of values. It's a good idea to take some data with a
large number of filters, dropping them from the balcony
outside so they have time to get up to their final speed.
This is also the only way you can tell for sure whether you're
taking data at terminal velocity: the results at the two different
heights (inside and outside) should be consistent.

\prelab

\prelabquestion Suppose you tried to do this lab with stacks of coins
instead of coffee filters. Assuming you had a sufficiently
accurate timing device, would it work?

\prelabquestion Criticize the following statement:

``We found that bigger velocities gave bigger air drag
forces, which demonstrates the proportionality $F\propto v$.''

\prelabquestion Criticize the following statement:

``We found $F\propto v^7$, which shows that you need more
force to make things go faster.''

\analysis

Use your raw data to compile a list of $F$ and $v$ values.
Use the methods explained in Appendix \ref{appendix:powerlaws} to see if you can
find a power-law relationship between $F$ and $v$. This will
require fitting a line to a set of data, as explained in
appendix \ref{appendix:graphing}. Both fitting a line to data and finding power
laws are techniques you will use several more times in this
course, so it is worth your while to get help now if
necessary in order to get confident with them.
