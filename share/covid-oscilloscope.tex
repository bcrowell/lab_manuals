\renewcommand\thechapter{c2.7}
\lab{The oscilloscope}\label{lab:covid-oscilloscope}

\section*{About this lab}

\covid\ 
It is intended to be used during the seventh week in Physics 222.

\apparatus
\equip{oscilloscope (Hantek 6022)}
\equip{battery}
\equip{desktop computer or Android phone}

\begin{goals}

\item[] Learn to use an oscilloscope.

\item[] Observe 60 Hz electric fields from household wiring, and verify the period.

\end{goals}

\introduction

One of the main differences you will notice between your
second semester of physics and the first is that many of the
phenomena you will learn about are not directly accessible
to your senses. For example, electric fields, the flow of
electrons in wires, and the inner workings of the atom are
all invisible.  The oscilloscope is a versatile laboratory
instrument that can indirectly help you to see what's going on. 

\mysubsubsection{The Oscilloscope}

An oscilloscope graphs an electrical signal that varies as
a function of time.
The graph is drawn from left to right across the screen,
being painted in real time as the input signal varies.

The input signal is
supplied in the form of a voltage, so an oscilloscope is
basically a voltmeter that measures a voltage as a function of
time.

Traditionally, oscilloscopes were stand-alone benchtop instruments
with their own built-in screens to display the graph of the signal.
In this lab, we will use an inexpensive (\$60) oscilloscope that
hooks up to a phone or computer through a USB cable.
Don't try to connect through a USB hub, which won't work with this scope.

\fig{em-osc-waveform-hantek}

\mysubsubsection{The Time Base and Triggering}

Since the $X$ axis represents time, there also has to be a way
to control the time scale, i.e., how fast the imaginary
``penpoint'' sweeps across the screen. For instance, setting
the knob on 10 ms causes it to sweep across one square
in 10 ms. This is known as the time base.

In the figure, suppose the time base is 10 ms.
The scope has 10
divisions, so the total time required for the beam to sweep
from left to right would be 100 ms. This is far too short a
time to allow the user to examine the graph.
The oscilloscope has a built-in method of overcoming this
problem, which works well for periodic (repeating)
signals. The amount of
time required for a periodic signal to perform its pattern
once is called the period. With a periodic signal, all you
really care about seeing is what one period or a few periods in
a row look like --- once you've seen one, you've seen them
all. The scope displays one screenful of the
signal, and then keeps on overlaying more and more copies of
the wave on top of the original one. Each
trace is erased when the next one starts, but is being overwritten
continually by later, identical copies of the wave form. You
simply see one persistent trace.

How does the scope know when to start a new trace? If the
time for one sweep across the screen just happened to be
exactly equal to, say, four periods of the signal, there
would be no problem. But this is unlikely to happen in real
life --- normally the second trace would start from a
different point in the waveform, producing an offset copy of
the wave. Thousands of traces per second would be superimposed
on the screen, each shifted horizontally by a different
amount, and you would only see a blurry band of light.

To make sure that each trace starts from the same point in
the waveform, the scope has a trigger. You use a
control in the software to set a certain voltage level, the trigger level, at
which you want to start each trace. In the figure, this is the pointer
on the right-hand side. The scope waits for the
input to move across the trigger level, and then begins a
trace. Once that trace is complete, it pauses until the
input crosses the trigger level again. To make extra sure
that it is really starting over again from the same point in
the waveform, you can also specify whether you want to start
on an increasing voltage or a decreasing voltage ---
otherwise there would always be at least two points in a
period where the voltage crossed your trigger level.

On old-fashioned analog oscilloscopes, the trigger would start
the sweep from the left edge of the screen, which is simpler
conceptually and is what was described above. Modern oscilloscopes
are usually ``storage'' scopes, which digitize the signal,
store it in memory in a rolling buffer, and then display some
segment of that memory when triggered. A storage scope is
like a time machine that can go back in time and show you
what happened before the trigger as well as after it. On
the Hantek scope, with the software we'll be using, the default is
that the time of the trigger is $t=0$ on the time axis, i.e., at
the center of the screen. In the figure, you should be able to
verify that the signal rises and crosses the trigger level at $t=0$.
This is indicated by the pointer above the Y axis.

\section*{Software}

Because this is a USB oscilloscope rather than a standalone device,
you need to run software in order to see the results. The manufacturer
supplies Windows software on a CD, but since I don't own a Windows machine,
I haven't tried it. The software I've been using is an open-source application
called OpenHantek6022, which is available at \url{github.com/OpenHantek/OpenHantek6022}.
It's multiplatform and runs on Windows, Mac, and Linux. It's worked fine for me
on Linux, and I didn't find that I needed to install any drivers. The OpenHantek6022
software is what I'll describe in this lab manual, but it seems to be a pretty close
copy of the manufacturer's Windows software. If you have an Android phone, there is
free software called Hscope that you can use. I've heard it's nice, but haven't tried it.
As far as I know there is no iPhone software for this device.

\section*{Observations}

\labpart{Demo mode (optional)}

If you're using the OpenHantek6022, there is a demo mode that lets you play with
a fake signal. There doesn't seem to be any way to start up demo mode except from
the command line. To do this, use the command \texttt{OpenHantek --demoMode}.
To understand what you're seeing in demo mode, go to the Help menu and select
User Manual, which pops up a PDF file.

Try changing the timebase to 1 ms. Turn off channel 2 (blue) by unchecking the
box. Set the voltage scale (next to CH 1) to something that looks appropriate.

Try playing with the trigger using the three pop-up menus under Trigger.
Set Source to CH 1, set the mode to normal, and, if necessary, adjust the
pointer at the right edge of the graph so that you do get triggering and
see a signal. You can use the pointer at the top of the graph to move the
trace left and right. Under Slope, pick a downward slope to trigger on, so
that it will trigger on the downward sloping piece of the sawtooth wave.
Try moving the pointer on the right up and down, and make sure you understand
the effect you're seeing.

\labpart{Measuring DC voltages}

If you did the optional part using demo mode, quit the software and restart it not
in demo mode.
Change the scope from normal triggering to automatic triggering (Trigger:Mode:Automatic).
The scope comes with a cable and probe. At one end of the cable is a round BNC
connector. Push this onto the yellow CH1 connector on the scope, and rotate it
clockwise so that it clicks on. 
At this point you should see a flat trace on the oscilloscope, probably with some
noticeable noise superimposed.
Although this is an open circuit, which is not the normal way of using a voltmeter,
the voltage you are measuring should be zero, at least on the average. If it's
not displayed as zero, click on the yellow CH1 text on the left side of the graph,
and move the trace vertically so that it lies on the axis.

At the other end of the cable are two
connectors, because the oscilloscope is a voltmeter, and a voltmeter measures
a voltage difference.

One of these connectors is of a type called a minigrabber.  It
contains a conducting hook, which is retracted behind a syringe-style
insulating sheath. Pull back on the sheath, against the resistance of
the spring inside, and you'll see the hook.  This acts like the V plug
on a voltmeter.

The other connector an alligator clip on its own removable little separate cable. The is called a ground
clip. If you're using a desktop machine, the ground clip
is connected to the household ground. If you're using a phone or laptop, it's not.
This acts like the COM plug on a voltmeter, so that the scope measures the voltage
difference between the minigrabber and the ground clip.

The probe probably has a switch on it with two positions, labeled x1 and x10.
Switch it to x1.

Touch the minigrabber to the ground clip, and you should see the noise go
away, because the voltage difference between the two terminals of the probe is now exactly zero.

Set CH 1 to an appropriate voltage scale so that you can measure the voltage across
a battery, and do that by connecting the ground clip and minigrabber to the terminals. Try flipping the battery around and verify that the sign flips.

\labpart{60 Hz fields in your home}

If you saw noise before when the scope's probe was an open circuit, that was probably
mostly from electric fields originating from the power lines in your house, which oscillate
at 60 Hz. We're now going to intentionally observe those signals.

For this part of the lab, you don't need to connect the ground clip to anything.
We can take all of our voltage measurements for this part of the lab
with respect to your phone or computer's ground. 
Use the ground clip if and only if you're taking measurements on a device such as the battery that's
``floating'' rather than connected to ground.

Put CH 1 on its most sensitive voltage scale. Pull back on the sheath of the minigrabber
to expose the hook, and touch the hook with a finger. You should see a big, strong
oscillating signal on the scope. What is happening here is that your body is acting
as one plate of a capacitor, and the AC wiring in the walls is acting as the other
plate.
To get your right hand free, place a  paperclip or piece of in the minigrabber, so that you
only have to use your left hand to touch the paperclip.

The idea now is to verify that the signal has a frequency of 60 Hz. Fiddle with the time
base to get something appropriate. At the upper left corner of the screen is a stop icon.
Click it, and the oscilloscope trace will freeze so you can measure it easily. (On a storage
scope, this is a useful trick that can be used when a signal has noise and isn't clean enough
to trigger on very steadily.) Measure the period from the screen.

\labpart{The scope's calibration output}

You can also measure the scope's calibration output, which is a little
contact labeled with a square wave symbol and ``1kHz 2Vp-p'' (p-p
meaning peak-to-peak).  The frequency is actually selectable in the
software, but defaults to 1 kHz.  Clip the hook of the minigrabber to
the calibration output.

You don't need to use the ground clip for this part of the lab, because the
scope is grounded rather than ``floating.''

There is another little connector next to it that
is a ground, which you will never need to use in this course.
The oscilloscope is a voltmeter, so it would be
natural for you to think that you should connect its probes to \emph{two} points and measure a voltage
difference. This is unnecessary, because the probe's ground is actually connected to USB ground anyway.

If you now adjust the scales appropriately, you should see a nice clean square wave that you can trigger
on very solidly. You can try triggering on rising and falling slopes and verify that you understand what
happens. You can try using the menu to change the frequency of the square wave.

\labpart{The internal resistance of the scope's calibration output}

When I first saw the design of this scope's panel, I was very worried that I or
my students would inadvertently destroy our scopes. The calibration output's contact is only
a few millimeters away from the ground contact. I imagined that if we were to inadvertently
short between these two contacts with a screwdriver or something, we would get lots of
current flowing through USB power to USB ground, potentially destroying both the
scope and the computer's USB bus. Actually, it turns out that the scope has an internal
resistor hidden behind the calibration output. The value of this resistor turns out to
be important for some of our later labs, but there is no documentation on its existence
or what it is. Determining this unknown resistance is a nice little exercise in circuits.

Set things up as in the earlier part of the lab so that you can observe the
scope's calibration output. Very little current is flowing in this setup, because
the probe has an internal resistance of 1 $\zu{M}\Omega$. This causes the scope to act,
to a good approximation, as an ideal voltmeter that has an infinite resistance and
doesn't disturb the circuit that it's being used to look at.

Now find a resistor with a value of a few $\zu{k}\Omega$ and put it in parallel with
the probe, by connecting from the calibration output to the ground clip. The calibration
output is now driving perhaps a milliamp of current through the resistor. This is a little
bit of a goofy thing to do, because the ground clip is not meant to be a current-carrying
connection, but it's harmless because the current is small.

\fig{em-osc-hantek-internal-r}

The circuit in the diagram is not pure series or pure parallel, it's a mixture of both.
But because the probe and input have such a high resistance, they don't interfere with
the operation of the circuit at all. Therefore you can put your hand over the box in
the diagram that says ``probe \& input,'' and think about the circuit as just the
part that you haven't covered up. That means that the circuit can be approximated as acting like a purely
\emph{series} circuit, consisting of the voltage source $V$ in series with the resistances
$R_0$ and $R$.

When you added the external resistor $R$, you should have seen the voltage measured on the scope go
down a little bit. This happens with any real-world voltage source: when you load it,
its voltage gets dragged down. Your external resistor $R$ is now forming a voltage divider
with the resistor $R_0$ inside the scope. Take measurements of the voltage $V_R$ when the external
resistor is this value, infinity, and one or two other values going down to a few hundred
ohms.

\analysis

The only complicated analysis is for part E. Depending on the schedule of your class,
you may actually start this lab before you know enough to analyze part E. For example, if
your text is \emph{Fields and Circuits}, then you won't know how to analyze it until
you've read ch.~9.
Analyze the voltage divider purely symbolically, finding
$R_0$ in terms of $R$, $V_R$, and $V$. (Note that when the resistor $R$ is not present,
you effectively have $R=\infty$, which is a free data point. We could notate the
voltage $V_R$ in this condition as $V_\infty$.)
Then, by plugging in to your equation, extract the internal resistor's value from each
measurement of $V_R$. If it's roughly the same in all cases, then that supports the hypothesis that it
really is a resistor, rather than some other current-limiting safety feature.
