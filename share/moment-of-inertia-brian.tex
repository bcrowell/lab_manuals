\lab{Moment of Inertia and Angular Momentum}\label{lab:moment-of-inertia-brian}

\apparatus
\equipn{rotating platform w/ heavy disc and ring}{1/group}
\equipn{50g weight with hook}{1/group}
\equipn{rotary motion sensor}{1/group}
\equipn{calipers}{1/group}
\equipn{digital balance, 4kg max}{1/class}
\equipn{digital balance, 6kg max}{1/class}
\equip{right-angle clamps and rods}

\goal{Measure moments of inertia using kinematics; predicting / confirming conservation of angular momentum}

\introduction

The figure shows a method for determining an unknown moment of inertia.
A rotating platform of radius $R$ has
a string wrapped around it. The string is threaded over a pulley and
down to a hanging weight of mass $m$. The mass is released from rest,
and its acceleration $a$ is measured. From these data, the total moment
of inertia of the platform plus the terrier can be determined (prelab question
P1 or a homework problem).

\fig{me-moi-terrier}

Since we were unable to obtain a set of standard terriers from our
suppliers, we will use a disk and a ring as unknowns. The moments of
inertia of each unknown can be found by determining the total moment
of inertia and then subtracting the moment of inertia of the bare platform.

For part C of the lab, you will be using the rotary motion sensor to
measure $\omega_i$ and $\omega_f$ for an inelastic collision (dropping
one heavy object onto another that is initially rotating). Since there
are no external torques on this system, angular momentum is conserved,
and we have

\begin{equation*}
I_i \omega_i = I_f \omega_f.
\end{equation*}

You check this equation after measuring moments of
inertia of objects that will appear $I_i$ and $I_f$

\setup

You will use the rotary motion sensor pulley for all parts of the
experiment despite the fact that you won't need it connected to the
computer until part C. 

Place the rotating platform on the table. Have the string from the
rotating platform pass over the pulley from the rotary motion sensor
when connecting the hanging weight. Make sure the height of the top of
the pulley is about level with the middle of the rotating platform
drum, or else the string will tend to move off of the drum. 

We will be able to ignore the mass / moment of inertia of the rotary
motion sensor for this entire lab because it is so small compared to
the moment of inertia of the rotating platform. 

Weigh your hanging mass ($\pm0.1$g), heavy solid disc, and heavy ring
using the digital balances. The heavy items are about 5 kg each, and
so if you don't have a digital balance that can weigh such massive
objects, you might have to get creative. One suggestion is to weigh a
heavy item on two scales (at the same time) and add the readings from
the two scales. 

\observations

\labpart{Measuring the moment of inertia of the platform}

By measuring the time it takes for the mass to fall, calculate the
moment of inertia of the rotating platform in kg$\,\cdot\,$m$^2$.
Include error bars in your calculation, which are probably dominated
by your uncertainty in the time it takes the mass to fall. 

\labpart{Measuring the moments of inertia of the disk and ring}

Using the mass and radii of the two heavy objects, calculate their
moments of inertia. The ring has a finite thickness, but it's a good
approximation to take the average of the inner radius and the outer
radius. There is no need to calculate an error for these, since the
error is dominated by systematic errors (the difference in these
objects' shapes relative to the ideal case). 

Your calculated values are predictions that you will test by measuring
the moments of inertia using the expression for $I$ in the
Introduction. Repeat part A, but this time placing the heavy ring and
the heavy disc separately on the rotating platform. Measure each
moment of inertia with error. Keep in mind that these values are
$I_{RP} + I_\text{disc}$ and $I_{RP} + I_\text{ring}$, so you'll have
to subtract off $I_{RP}$ to find the moment of inertia of each heavy
object by itself. 

Do your results agree with your calculated values?

\labpart{Predicting and Measuring $\omega_f / \omega_i$}

Connect the rotary motion sensor to the computer (Dig/Sonic port 1),
and start LoggerPro using pendulum.cmbl which you can download from
the instructor's website:\\
staffwww.fullcoll.edu/bshotwell/pendulum.cmbl.zip\\ 
You will ignore the ``force'' output in LoggerPro. 

Place the disc on the platform. Get it rotating, and then drop the
ring on top of the rotating disc. Predict what you will measure for
$\omega_f / \omega_i$ by using your measured moments of inertia found
in the previous parts. 

Using the output from the rotary motion sensor, find $\omega_f /
\omega_i$. Note that the graphs are not exactly linear, since there is
a small torque on the rotating platform system from the dangling mass.
However, so long as you look at the slopes \emph{just before} and
\emph{just after} the sticky collision, then you can ignore the
angular impulse from the small external torque since the collision
happens so quickly. 

Does your measured value for $\omega_f / \omega_i$ agree with your prediction? Discuss.

\prelab

\prelabquestion For the technique described for parts A and B, 
find the total moment of inertia
$I$ of the platform plus the object sitting on top of it, in terms
of the acceleration $a$ of the hanging mass, its mass $m$, and the radius
$r$ of the cylinder about which the string is wound. If you were assigned
this calculation as a homework problem, skip this prelab question.

\prelabquestion You will not actually measure the acceleration $a$
of the falling weight directly, but rather the distance $h$ it travels
in a time $t$. Find the acceleration in terms of the raw data.

\prelabquestion In part C, we will be dropping the ring onto the solid
disc, meaning that $I_i = I_{RP}+I_\text{disc}$ and $I_f =
I_{RP}+I_\text{disc}+I_\text{ring}$. Assuming $I_{RP}$ is negligible
compared to the other two, and assuming that the solid disc and the
ring are the same mass and radius, what would you expect to be the
ratio of $\omega_f / \omega_i$? 

\analysis

Give the moments of inertia of all three objects (with error bars),
and compare the heavy objects to your calculated values. Discuss all
of the predictions you made and whether they were confirmed or
contradicted in the lab. If your predictions were not accurate, give
possible reasons for why. What errors did you include in your error
bars, and what effects were not included? 
