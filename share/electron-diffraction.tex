\lab{Electron Diffraction}\label{lab:electron-diffraction}

\apparatus

\equip{cathode ray tube (Leybold 555 626)}
\equip{high-voltage power supply (new Leybold)}
\equip{100-k$\Omega$ resistor with banana-plug connectors}
\equip{Vernier calipers}


\begin{goals}

\item[] Observe wave interference patterns (diffraction patterns)
of electrons, demonstrating that electrons exhibit wave
behavior as well as particle behavior.

\item[] Learn what it is that determines the wavelength of an electron.
\end{goals}

\introduction


The most momentous discovery of 20th-century physics has
been that light and matter are not simply made of waves or
particles --- the basic building blocks of light and matter
are strange entities which display both wave and particle
properties at the same time. In our course, we have already
learned about the experimental evidence from the photoelectric
effect showing that light is made of units called photons,
which are both particles and waves. That probably disturbed
you less than it might have, since you most likely had no
preconceived ideas about whether light was a particle or a
wave. In this lab, however, you will see direct evidence
that electrons, which you had been completely convinced were
particles, also display the wave-like property of interference.
Your schooling had probably ingrained the particle
interpretation of electrons in you so strongly that you used
particle concepts without realizing it. When you wrote
symbols for chemical ions such as $\zu{Cl}^{-}$ and $\zu{Ca}^{2+}$, you
understood them to mean a chlorine atom with one excess
electron and a calcium atom with two electrons stripped off.
By teaching you to count electrons, your teachers were
luring you into the assumption that electrons were
particles. If this lab's evidence for the wave properties of
electrons disturbs you, then you are on your way to a deeper
understanding of what an electron really is --- both a
particle and a wave.

\figcaption{mo-edf-tube}{The electron diffraction tube. The distance
labeled as 13.5 cm in the figure actually varies from about 12.8 cm to
13.8 cm, even for tubes that otherwise appear identical. This doesn't
affect your results, since you're only searching for a proportionality.}

\section*{Method}

What you are working with is basically the same kind of
vacuum tube as the picture tube in your television. As in a
TV, electrons are accelerated through a voltage and shot in
a beam to the front (big end) of the tube, where they hit a
phosphorescent coating and produce a glow. You cannot see
the electron beam itself. There is a very thin carbon foil
(it looks like a tiny piece of soap bubble) near where the
neck joins the spherical part of the tube, and the electrons
must pass through the foil before crossing over to the
phosphorescent screen.

The purpose of the carbon foil is to provide an ultra-fine
diffraction grating --- the ``grating'' consists of the
crystal lattice of the carbon atoms themselves! As you will
see in this lab, the wavelengths of the electrons are very
short (a fraction of a nanometer), which makes a conventional
ruled diffraction grating useless --- the closest spacing
that can be achieved on a conventional grating is on the
order of one micrometer. The carbon atoms in graphite are
arranged in sheets, each of which consists of a hexagonal
pattern of atoms like chicken wire. That means they are not
lined up in straight rows, so the diffraction pattern is
slightly different from the pattern produced by a ruled grating.

Also, the carbon foil consists of many tiny graphite
crystals, each with a random orientation of its crystal
lattice. The net result is that you will see a bright spot
surrounded by two faint circles. The two circles represent
cones of electrons that intersect the phosphor. Each cone
makes an angle $\theta $ with respect to the central axis of
the tube, and just as with a ruled grating, the angle is given by
\begin{equation*}
         \sin\theta   =  \lambda /d  
\end{equation*}

where $\lambda $ is the wavelength of the wave. For a ruled
grating, $d$ would be the spacing between the lines. In this
case, we will have two different cones with two different
$\theta $'s, $\theta_1$ and $\theta_2$, corresponding to two
different $d's$, $d_1$ and $d_2$. Their geometrical
meaning is shown below.\footnote{See \url{http://bit.ly/XxoEYr} for more information.}

\figcaption{mo-edf-graphite}{The carbon atoms in the graphite crystal are
arranged hexagonally.}

\mysubsubsection{Safety}

This lab involves the use of voltages of up to 6000 V. 
Do not be afraid of the equipment, however;
there is a fuse in the high-voltage supply that limits the
amount of current that it can produce, so it is not
particularly dangerous.  Read the safety checklist on high
voltage in Appendix \ref{appendix:hvsafety}. Before beginning the lab, make sure
you understand the safety rules, initial them, and show your
safety checklist to your instructor. If you don't understand
something, ask your instructor for clarification.

In addition to the high-voltage safety precautions, please
observe the following rules to avoid damaging the apparatus:

\_\_\_\_\_\_ The tubes cost \$1000. Please treat them with
respect! Don't drop them! Dropping them would also be a safety
hazard, since they're vacuum tubes, so they'll implode violently
if they break.

\_\_\_\_\_\_ Do not turn on anything until your instructor
has checked your circuit.

\_\_\_\_\_\_ Don't operate the tube continuously at the highest
voltage values (5000-6000 V). It produces x-rays when used at these
voltages, and the strong beam also decreases the life of the tube.
You can use the circuit on the right side of
the HV supply's panel, which limits its own voltage to 5000 V.
Don't leave the tube's heater on when you're not actually taking
data, because it will decrease the life of the tube.

\setup

You setup will consist
of two circuits, a heater circuit and the high-voltage circuit. 

The heater circuit is to heat the cathode, increasing the
velocity with which the electrons move in the metal and
making it easier for some of them to escape from the
cathode. This will produce the friendly and nostalgia-producing
yellow glow which is characteristic of all vacuum-tube
equipment. The heater is simply a thin piece of wire, which
acts as a resistor when a small voltage is placed across it,
producing heat.
Connect the heater connections, labeled F1 and F2, to the
6-V AC outlet at the back of the HV supply. 

The high-voltage circuit's job is to accelerate the electrons
up to the desired speed. An electron that happens to jump out
of the cathode will head ``downhill'' to the anode. (The anode
is at a \emph{higher} voltage than the cathode, which would make
it seem like it would be uphill from the cathode to the anode.
However, electrons have negative charge, so they're like
negative-mass water that flows uphill.) The high voltage
power supply is actually two different power supplies in one
housing, with a left-hand panel for one and a right-hand panel
for the other.
Connect the
anode (A) and cathode (C) to the right-hand panel of the
HV supply, and switch the switch on the HV supply to the
right, so it knows you're using the right-hand panel.

The following connections are specified in the documentation,
although I don't entirely understand what they're for.
First, connect the electrode X to the same plug as the
cathode.\footnote{If you look inside the tube, you can see
that X is an extra electrode sandwiched in between the anode
and the cathode. I think it's meant to help produce a focused
beam.}
Also, connect F1 to C with the wire that has the 100-k$\Omega$
resistor spliced into it. The circuit diagram on page \pageref{fig:mo-edf-circuit} summarizes
all this.

Check your circuit with your instructor before turning it on!

\widefigcaption{mo-edf-circuit}{The circuit for the new setup.\label{fig:mo-edf-circuit}}

\observations

You are now ready to see for yourself the evidence of the
wave nature of electrons, observe the diffraction pattern
for various values of the high voltage, and figure out what
determines the wavelength of the electrons. You will need to
do your measurements in the dark.

You will measure the $\theta $'s, and thus determine the
wavelength, $\lambda $, for several different voltages. Each
voltage will produce electrons with a different velocity,
momentum, and energy.  

Hints:

\begin{itemize}
\item[] While measuring the diffraction pattern, don't touch the
vacuum tube --- the static electric fields of one's body
seem to be able to perturb the pattern.

\item[] It is easiest to take measurements at the highest
voltages, where the electrons pack a wallop and make nice
bright rings on the phosphor. Start with the highest
voltages and take data at lower and lower voltages until you
can't see the rings well enough to take precise data. To get
unambiguous results, you'll need to take data with the
widest possible range of voltages.

\item[] In order to reach a definite conclusion about what
$\lambda $ is proportional to, you will need accurate data.
Do your best to get good measurements. Pay attention to
possible problems incurred by viewing the diffraction
patterns from different angles on different occasions. Try
repeating a measurement more than once, and seeing how big
your random errors are.

\item[] You need to get data down to about 2 or 3 kV in order to
get conclusive results from this experiment. The tubes are not
quite identical, and were not designed to operate at such low
voltages, so they haven't been tested under those conditions.
Experience has shown that some of the tubes work at lower voltages
than others. The group that has the tube that works the best at
low voltages can share their low-voltage data with the other groups.
\end{itemize}

\analysis

Once you have your data, the idea is to plot $\lambda$ as a function
of quantities such as $KE$, $p$, $1/KE$, or $1/p$.
If the graph is a straight line through the origin, then the experiment
supports the hypothesis that the wavelength is proportional to that
quantity. You can simplify your
analysis by leaving out constant factors, and P5 asks you to consider
how you can rule out some of these possibilities without having
to make all the graphs.

What does $\lambda $ seem to be proportional to? Your data
may cover a small enough range of voltage that more than one
graph may look linear. However, only one will be consistent
with a line that passes \emph{through the origin}, as
it must for a proportionality. This is why it
is important to have your graph include the origin.

\prelab

The week before you are to do the lab, briefly familiarize
yourself visually with the apparatus.

Read the high voltage safety checklist.

\prelabquestion  
The figure  shows the vacuum tube as having
a particular shape, which is a sphere with the foil and phosphor
at opposite ends of a diamater. In reality, the tubes we're using now are not quite that
shape. To me, they look like they may have been shaped so that the phosphor
surface is a piece of a sphere centered on the foil. Therefore the arc lengths across
the phosphor can be connected to diffraction angles very simply via the
definition of radian measure. Plan how you will do this.

\prelabquestion  If the voltage difference across which the electrons are
accelerated is $V$, and the known mass and charge of the
electron are $m$ and $e$, what are the electrons' kinetic
energy and momentum, in terms of $V,m$, and $e?$ (As a
numerical check on your results, you should find that
$V=5700$ V gives $KE=9.1\times10^{-16}$ J and
$p=4.1\times10^{-23}\ \kgunit\unitdot\munit/\sunit$.)

\prelabquestion All you're trying to do based on your graphs is judge which
one could be a graph of a proportionality, i.e., a line passing through the
origin. Because of this, you can omit any constant factors from the equations
you found in P2. When you do this, what do your expressions turn out to be?

\prelabquestion  Why is it not logically possible for the wavelength to
be proportional to both $p$ and $KE$? To both
$1/p$ and $1/KE$?

\prelabquestion  I have suggested plotting $\lambda $ as a function of
$p$, $KE$, $1/p$ and $1/KE$ to see if $\lambda $
is directly proportional to any of them. Once you have your
raw data, how can you immediately rule out two of these four
possibilities and avoid drawing the graphs?

\prelabquestion  On each graph, you will have two data-points for each
voltage, corresponding to two different measurements of the
same wavelength. The two wavelengths will be almost the
same, but not exactly the same because of random errors in
measuring the rings. Should you get the wavelengths by
combining the smaller angle with $d_1$ and the larger angle
with $d_2$, or vice versa?
