\lab{Two-Source Interference}\label{lab:two-source-interference}

\apparatus
\equipn{ripple tank}{1/group}
\equipn{yellow foam pads}{4/group}
\equip{lamp and unfrosted straight-filament bulb}{1/group}
\equipn{wave generator}{1/group}
\equip{big metal L-shaped arms for hanging}
\equipn{the wave generator}{1/group}
\equip{little metal L-shaped arms with yellow}
\equipn{plastic balls}{2/group}
\equipn{rubber bands}{2/group}
\equipn{Thornton DC voltage source}{1/group}
\equipn{small rubber stopper}{1/group}
\equipn{power strip}{1/group}
\equipn{bucket}{1/group}
\equipn{mop}{1}
\equipn{flathead screwdriver}{1}
\equip{rulers and protractors}
\equip{kimwipes and alcohol for cleaning}
\equip{butcher paper}

\begin{goals}

\item[] Observe how a 2-source interference
pattern of water waves depends on the distance between the sources.


\end{goals}

\observations

m4_ifdef([:__lm:],[:%
Light is really made of waves, not rays, so when we treated it as
rays, we were making an approximation. You might think that when
the time came to treat light as a wave, things would get very difficult,
and it would be hard to predict or understand anything without doing
complicated calculations.
:],[:%
A car or a pool ball travels along a well-defined path, but a wave doesn't.
Not only that, but waves don't simply collide and bounce like pool balls.
They overlap and add on to one another, so that they can reinforce or cancel.
This seems like it would be extremely complicated to calculate.
:])

\fig{op-int-photo}

Life isn't that bad. It turns out that all of the most important
ideas about light as a wave can be seen in one simple experiment, shown in the
first figure.\footnote{The photo is from the textbook PSSC Physics, which has
a blanket permission for free use after 1970.} A wave comes up from the bottom
of the page, and encounters a wall with two slits chopped out of it. The result
is a fan pattern, with strong wave motion coming out along directions like X and
Z, but no vibration of the water at all along lines like Y. The reason for this
pattern is shown in the second figure. The two parts of the wave that get through
the slits create an overlapping pattern of ripples. To get to a point on line X, both
waves have to go the same distance, so they're in step with each other, and
reinforce. But at a point on line Y, due to the unequal distances involved,
one wave is going up while the other wave is going down, so there is cancellation.
The angular spacing of the fan pattern depends on both the wavelength of the waves,
$\lambda$, and the distance between the slits, $d$.

\fig{op-int-diagram}

The ripple tank is tank that sits about 30 cm above the
floor. You put a little water in the tank, and produce
waves. There is a lamp above it that makes a point-like
source of light, and the waves cast patterns of light on a
screen placed on the floor. The patterns of light on the
screen are easier to see and measure than the ripples themselves.

In reality, it's not very convenient to produce a double-slit
diffraction pattern exactly as depicted in the first figure,
because the waves beyond the slits are so weak that they are
difficult to observe clearly. Instead, you'll simply produce
synchronized circular ripples from two sources driven by a motor.

Put the tank on the floor. Plug the hole in the side of the
tank with the black rubber stopper. If the plastic is dirty, clean
it off with alcohol and kimwipes. Wet the four yellow
foam pads, and place them around the sides of the tank. Pour in water to
a depth of about 5-7 mm. Adjust the metal feet to level the
tank, so that the water is of equal depth throughout the
tank. (Do not rotate the wooden legs themselves, just the feet.)
If too many bubbles form on the plastic, wipe them off with a ruler.

Make sure the straight-filament bulb in the light source is
rotated so that when you look in through the hole, you are
looking along the length of the filament. This way the lamp
acts like a point source of light above the tank. To test that
it's oriented correctly, check that you can cast a perfectly
sharp image of the tip of a pen.

The light source is intended to be clamped to the wooden post, but
I've found that that works very poorly, since the clamp doesn't
hold it firmly enough. Instead, clamp the light source to the
lab bench's lip or its leg.
Turn it on. Put the butcher paper
on the floor under the tank. If you make
ripples in the water, you should be able to see the wave
pattern on the screen.

The wave generator consists of a piece of wood that hangs by
rubber bands from the two L-shaped metal hangers. There is a
DC motor attached, which spins an intentionally unbalanced
wheel, resulting in vibration of the wood. The wood itself
can be used to make straight waves directly in the water,
but in this experiment you'll be using the two little
L-shaped pieces of metal with the yellow balls on the end to
make two sources of circular ripples. The DC motor runs off
of the DC voltage source, and the more voltage you supply,
the faster the motor runs.

Start just by sticking one little L-shaped arm in the piece
of wood, and observing the circular wave pattern it makes.
Now try two sources at once, in neighboring holes. Pick a
speed (frequency) for the motor that you'll use throughout
the experiment --- a fairly low speed works well. Measure
the angular spacing of the resulting diffraction pattern for
several values of the spacing, $d$, between the two sources of ripples.

Use  the methods explained in Appendix \ref{appendix:powerlaws}
and look for any kind of a power law relationship
for the dependence of the angular spacing on $d$.
