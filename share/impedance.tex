
\lab{Impedance}\label{lab:impedance}

 
\begin{itemize}
\item[] Observe how the impedances of capacitors and inductors
change with frequency.

\item[] Observe how impedances combine according to the arithmetic of complex numbers.
\end{itemize}

\setup

We'll start by observing the impedance of a capacitor.
Ideally, what we want is this:

\fig{em-imp-simplified}

However, we want to know not just the amplitude of the
voltage and current sine-waves but the phase relationship
between them as well, which we can't get from a regular
meter. We need to use an oscilloscope, and oscilloscopes only
measure voltage, not current. This leads us to something
like the following setup:

\fig{em-imp-groundloop}

Here ch. 2 tells us the voltage across the resistor, which
is related to the current in the resistor according to Ohm's
law. By the junction rule, the current in the resistor is
the same as the current through the capacitor.

But even now, we're not out of the woods. In this setup, the
ground of ch. 2 is connected to the same wire as the active
(+) connection to ch. 1, which would cause ch. 1 to read
zero, and would short across the capacitor as well.
Instead, we need this:

\fig{em-imp-actual}

Now both GND connections are going to the same point in the
circuit. Because we've swapped the connections to ch. 1, its
trace will be upside-down, and inconsistent with ch. 2.
There is a special control on the scope for inverting  ch.
2, which makes the two channels consistent again.

\observations

\labpart{ Impedance of the capacitor}

Hook up the circuit as shown, using a 1 $k\Omega $
resistance and a  0.2 $\mu F$ capacitance. The HP signal
generator has a ground strap connecting one of its output
terminals to ground. Disconnect this ground strap, since
grounding either side of the signal generator would mean
that either the resistor or the capacitor would be connected
to ground on both sides. Try a frequency of 100 Hz.

Observe the phase relationship between $V_C$, on ch. 1, and
the signal on ch. 2, which essentially tells us the current
$I_C$ except for a factor of 1/R. Sketch this phase
relationship in your raw data. Because $V_C=q/C$ and
$I=dq/dt$, the current through the capacitor should be
proportional to dV/dt. Based on the phase relationship you
observed, does this seem to be true?

Measure the phase angle numerically from the oscilloscope.
Is it what you expect?

Determine the magnitude of the capacitor's impedance.

Suppose you represent the signal that is ahead in phase
using a point that is more counterclockwise in the complex
plane. Sketch the locations of the voltage and current in
the complex plane. (You can arbitrarily choose one of them
to be along the real axis if you like.) Where would the
impedance then lie in the plane?

Now change the frequency to 1000 Hz, and see what changes.
Sketch your new impedance in the complex plane. Do you find
the expected relationship between impedance and frequency?

\labpart{ Inductance of the Heath coil}

Make the measurements you need in order to calculate the
theoretical inductance of the inductor, using the equation
derived in the prelab. The approximation may be off by as
much as a factor of two, since the solenoid isn't long and
skinny, but it's useful so you have some idea of what to expect.

\labpart{ Impedance of the inductor}

Now repeat all the above steps using the Heath coil as an inductor. 

\labpart{ Impedances in series}

Put the capacitor and inductor in series, and collect the
data you'll need in order to determine their combined
impedance at several frequencies ranging from 100 to 1000 Hz. 

\analysis

Use your data from part C to determine an experimental
value of the coil's inductance, and compare with the
theoretical result based on your measurements in part B.

Graph the theoretical and experimental impedance of the
series combination in part D, overlaying them on the same
graph. Show theory as a curve and experiment as discrete
data-points. Do the same kind of graph for the parallel combination.
