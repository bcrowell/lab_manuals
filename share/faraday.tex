\lab{Faraday's Law}\label{lab:faraday}

\apparatus
\equip{Pasco PI-8127 function generator (in lab benches in 415; see note below)}
\equip{solenoid (Heath)   1/group plus a few more}
\equipn{oscilloscope}{1/group}
\equipn{100 $\Omega$ resistor}{1/group}
\equip{secondary coils (see note below)}
\equip{palm-sized pieces of iron or steel}
\equip{masking tape}
\equip{rulers}

Notes: It is also probably possible to do this lab using the Pasco PI-9587C function
generators, but I haven't tested it.

We have a variety of coils that can be used as secondary coils. The text below describes
using a loop made out of a 4-meter piece of wire. We have a bunch of these made up
using white wire with banana plugs on the ends. Although these can be made to work,
the signal is rather weak because of the small number of turns. We have a set of
small rectangular coils with various numbers of turns, PASCO SF-8617. I've successfully
used the 3200-turn coil, which produces a big signal. The others have are also probably
usable. I have also made a couple of hand-wrapped coils for use in this lab.

\begin{goals}

\item[] Observe electric fields induced by changing magnetic fields.

\item[] Test Faraday's law.
\end{goals}

\introduction

Physicists hate complication, and when physicist Mi\-ch\-a\-el
Faraday was first learning physics in the early 19th
century, an embarrassingly complex aspect of the science was
the multiplicity of types of forces. Friction, normal
forces, gravity, electric forces, magnetic forces, surface
tension --- the list went on and on. Today, 200 years later,
ask a physicist to enumerate the fundamental forces of
nature and the most likely response will be ``four: gravity,
electromagnetism, the strong nuclear force and the weak
nuclear force.'' Part of the simplification came from the
study of matter at the atomic level, which showed that
apparently unrelated forces such as friction, normal forces,
and surface tension were all manifestations of electrical
forces among atoms. The other big simplification came from
Faraday's experimental work showing that electric and
magnetic forces were intimately related in previously
unexpected ways, so intimately related in fact that we now
refer to the two sets of force-phenomena under a single
term, ``electromagnetism.''

Even before Faraday, Oersted had shown that there was at
least some relationship between electric and magnetic
forces. An electrical current creates a magnetic field, and
magnetic fields exert forces on an electrical current. In
other words, electric forces are forces of charges acting on
charges, and magnetic forces are forces of moving charges on
moving charges. (Even the magnetic field of a bar magnet is
due to currents, the currents created by the orbiting
electrons in its atoms.)

Faraday took Oersted's work a step further, and showed that
the relationship between electricity and magnetism was even
deeper. He showed that a changing electric field produces a
magnetic field, and a changing magnetic field produces an
electric field. Faraday's law,
\begin{equation*}
      \int \vc{E}\cdot\bell  =  -\der\Phi_B/\der t  
\end{equation*}
relates the integral of the electric field around a
closed loop to the rate of change of the magnetic flux
through the loop. It forms the basis for such technologies
as the transformer, the electric guitar, the amplifier, and
generator, and the electric motor.

\observations

\labpart{ Qualitative Observations}

To observe Faraday's law in action you will first need to
produce a varying magnetic field. You can do this by using a
function generator to produce a current in a solenoid that
that varies like a sine wave as a function of time. The
solenoid's magnetic field will thus also vary sinusoidally.

The emf  in Faraday's law can be observed around a loop of
wire positioned inside or close to the solenoid. To make the
emf larger and easier to see on an oscilloscope, you will
use 5-10 loops, which multiplies the flux by that number of
loops.

The only remaining complication is that the rate of change
of the magnetic flux, $\der \Phi_B/\der t$,
is determined by the rate of change
of the magnetic field, which relates to the rate of change
of the current through the solenoid, $\der I/\der t$. The oscilloscope,
however, measures voltage, not current. You might think that
you could simply observe the voltage being supplied to the
solenoid and divide by the solenoid's 62-ohm resistance to
find the current through the solenoid. This will not work,
however, because Faraday's law produces not only an emf in
the loops of wire but also an emf in the solenoid that
produced the magnetic field in the first place. The current
in the solenoid is being driven not just by the emf from the
function generator but also by this ``self-induced'' emf.
Even though the solenoid is just a long piece of wire, it
does not obey Ohm's law under these conditions. To get
around this difficulty, you can insert the 100 $\Omega$
resistor in the circuit in series with the function
generator and the solenoid. The resistor does obey Ohm's law, so by
using the scope to observe the voltage drop across it you
can infer the current flowing through it, which is the same
as the current flowing through the solenoid.

Create the solenoid circuit, and hook up one channel of the
scope to observe the voltage drop across the  resistor.
A sine wave with a frequency on the order of 1 kHz will work.

Wind the 2-m wire into circular loops small enough to fit
inside the solenoid, and hook it up to the other channel of the scope.

\fig{em-far-position}

As always, you need to watch out for ground loops. The output
of the function generator has one of its terminals grounded,
so that ground and the grounded side of the scope's input have
to be at the same place in the circuit.

The signals tend to be fairly noisy. You can clean them up
a little by having the scope average over a series of traces.
To turn on averaging, do\\
 Acquire$>$Average$>$128. To turn it back
off, press Sample.

First try putting the loops at the mouth of the solenoid,
and observe the emf induced in them. Observe what happens
when you flip the loops over. You will observe that the two
sine waves on the scope are out of phase with each other.
Sketch the phase relationship in your notebook, and make
sure you understand in terms of Faraday's law why it is the
way it is, i.e., why the induced emf has the greatest value
at a certain point, why it is zero at a certain point, etc.

Observe the induced emf at with the loops at several other
positions such as those shown in the figure. Make sure you
understand in the resulting variations of the strength of
the emf in terms of Faraday's law.

\labpart{ A Metal Detector}

Obtain one of the spare solenoids so that you have two of
them. Substitute it for the loops of wire, so that you can
observe the emf induced in the second solenoid by the first
solenoid. If you put the two solenoids close together with
their mouths a few cm apart and then insert a piece of iron
or steel between them, you should be able to see a small
increase in the induced emf. The iron distorts the magnetic
field pattern produced by the first solenoid, channeling
more of the field lines through the second solenoid.

\labpart{ Quantitative Observations}

This part of the lab is a quantitative test of Faraday's
law. Going back to the setup for part A, measure the
amplitude (peak-to-peak height) of the voltage across the
resistor. Check against your prediction from prelab question 1.

There is a feature of the scope that seems to come on by
default and changes your voltages by a factor of 10. Since this
occurs for both channels, it ends up not affecting your results
if you express them as a ratio between the amplitudes measured
on the two channels. However, if you want to turn this off, you can.
Press CH 1 Menu, and you will see something that
says Probe 10X Voltage. This means that all your voltage measurements
will be off by this factor. Push the button and then set this to
Attenuation 1X rather than 10X. Do this for channel 2 as well.

Another thing that can cause some confusion is that the function generator
will probably say ``0.00A rms,'' which will make you think that there is
a blown fuse or an open circuit. Actually, the current that flows in the
solenoid circuit is simply so small that it rounds off to 0.00 on this
readout.

Choose a position for the loops of wire that
you think will make it as easy as possible to calculate $\der \Phi_B/\der t$
accurately based on knowledge of the variation of the
current in the solenoid as a function of time. Put the loops
in that position, and measure the amplitude of the induced
emf. Repeat these measurements with a frequency that
is different by a factor of two.

\selfcheck

Before leaving, analyze your results from part C and make
sure you get reasonable agreement with Faraday's law.

\analysis

Describe your observations in parts A and B and interpret
them in terms of Faraday's law.

Compare your observations in part C quantitatively with Faraday's law.
The solenoid isn't very long, so the approximate expression for the
interior field of a long solenoid isn't very accurate here. To correct
for that, multiply the expression for the field by the correction
factor $\zeta = (\cos\theta_1-\cos\theta_2)/2$,
(\emph{Fields and Circuits}, ch.~11, problem 13), where
$\theta_1$ and $\theta_2$ are angles between the axis and the lines connecting
the point of interest to the edges of the solenoid's mouths.

This analysis is horrible to do and to read if you do it all numerically.
Let $V$ be the voltage across the resistor and $\mathcal{E}$ the emf measured
on the secondary coil. It's nice to work with the unitless ratio $\mathcal{E}/V$.
For your theoretical result, you should be
able to express this ratio symbolically in terms of the following
symbols:
\begin{align*}
 k  &= \text{Columb constant} \\
 c  &= \text{speed of light} \\
 N_1  &= \text{number of turns in the primary coil} \\
 N_2  &= \text{number of turns in the secondary coil} \\
 f  &= \text{frequency} \\
 A  &= \text{area of the secondary coil} \\
 \ell  &= \text{length of the primary coil} \\
 R  &= \text{resistance of the resistor} \\
 \zeta  &= \text{correction factor for the magnetic field}.
\end{align*}
Check that the units work. Only plug in numbers at the end.

\prelab

\prelabquestion Find the theoretical equation for $\mathcal{E}/V$.
