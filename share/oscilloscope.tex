\lab{The Oscilloscope}\label{lab:oscilloscope}

\apparatus
\equipn{oscilloscope (Tektronix TDS 1001B)}{1/group}
\equipn{microphone (RS 33-1067)}{for 6 groups}
\equipn{microphone (Shure C606)}{for 1 group}
\equipn{PI-9587C sine wave generator}{1/group}
\equip{various tuning forks, mounted on wooden boxes}

If there's an equipment conflict with respect to the sine wave
generators, the HP200CD sine wave generators can be used instead.

\begin{goals}

\item[] Learn to use an oscilloscope.

\item[] Observe sound waves on an oscilloscope.
\end{goals}

\introduction

One of the main differences you will notice between your
second semester of physics and the first is that many of the
phenomena you will learn about are not directly accessible
to your senses. For example, electric fields, the flow of
electrons in wires, and the inner workings of the atom are
all invisible.  The oscilloscope is a versatile laboratory
instrument that can indirectly help you to see what's going on. 

\mysubsubsection{The Oscilloscope}

An oscilloscope graphs an electrical signal that varies as
a function of time.
The graph is drawn from left to right across the screen,
being painted in real time as the input signal varies.
In this lab, you
will be using the signal from a microphone as an input,
allowing you to see sound waves. 

The input signal is
supplied in the form of a voltage.
You are already familiar with the term ``voltage'' from
common speech, but you may not have learned the formal
definition yet in the lecture course. Voltage, measured in
metric units of volts (V), is defined as the electrical
potential energy per unit charge. For instance if 2 nC of
charge flows from one terminal of a 9-volt battery to the
other terminal, the potential energy consumed equals 18 nJ.
To use a mechanical analogy, when you blow air out between
your lips, the flowing air is like an electrical current,
and the difference in pressure between your mouth and the
room is like the difference in voltage. For the purposes of
this lab, it is not really necessary for you to work with
the fundamental definition of voltage.

The input connector on the front of the oscilloscope
accepts a type of cable
known as a BNC cable.
A BNC cable is a specific example of coaxial cable
(``coax''), which is also used in cable TV, radio, and
computer networks. The electric current flows in one direction
through the central conductor, and returns in the opposite
direction through the outside conductor, completing the
circuit. The outside conductor is normally kept at ground,
and also serves as shielding against radio interference. 
The advantage of coaxial cable is that it is capable of
transmitting rapidly varying signals without distortion.

\fig{em-osc-coax}

Most of the voltages we wish to measure are not big enough
to use directly for the vertical deflection voltage, so the
oscilloscope actually amplifies the input voltage, i.e., the
small input voltage is used to control a much larger voltage
generated internally. The amount of amplification is
controlled with a knob on the front of the scope. For
instance, setting the knob on 1 mV selects an amplification
such that 1 mV at the input deflects the electron beam by
one square of the 1-cm grid. Each 1-cm division is referred
to as a ``division.''

\fig{em-osc-waveform} % \label{fig:em-osc-waveform} is much lower down, to force ref to be on the page where the fig actually lands

\mysubsubsection{The Time Base and Triggering}

Since the $X$ axis represents time, there also has to be a way
to control the time scale, i.e., how fast the imaginary
``penpoint'' sweeps across the screen. For instance, setting
the knob on 10 ms causes it to sweep across one square
in 10 ms. This is known as the time base.

In the figure, suppose the time base is 10 ms.
The scope has 10
divisions, so the total time required for the beam to sweep
from left to right would be 100 ms. This is far too short a
time to allow the user to examine the graph.
The oscilloscope has a built-in method of overcoming this
problem, which works well for periodic (repeating)
signals. The amount of
time required for a periodic signal to perform its pattern
once is called the period. With a periodic signal, all you
really care about seeing what one period or a few periods in
a row look like --- once you've seen one, you've seen them
all. The scope displays one screenful of the
signal, and then keeps on overlaying more and more copies of
the wave on top of the original one. Each
trace is erased when the next one starts, but is being overwritten
continually by later, identical copies of the wave form. You
simply see one persistent trace.\label{fig:em-osc-waveform} % label is much lower down than fig, to force ref to be on the page where the fig actually lands

How does the scope know when to start a new trace? If the
time for one sweep across the screen just happened to be
exactly equal to, say, four periods of the signal, there
would be no problem. But this is unlikely to happen in real
life --- normally the second trace would start from a
different point in the waveform, producing an offset copy of
the wave. Thousands of traces per second would be superimposed
on the screen, each shifted horizontally by a different
amount, and you would only see a blurry band of light.

To make sure that each trace starts from the same point in
the waveform, the scope has a triggering circuit. You use a
knob to set a certain voltage level, the trigger level, at
which you want to start each trace. The scope waits for the
input to move across the trigger level, and then begins a
trace. Once that trace is complete, it pauses until the
input crosses the trigger level again. To make extra sure
that it is really starting over again from the same point in
the waveform, you can also specify whether you want to start
on an increasing voltage or a decreasing voltage ---
otherwise there would always be at least two points in a
period where the voltage crossed your trigger level.

\section*{Setup}


To start with, we'll use a sine wave generator, which makes
a voltage that varies sinusoidally with time. This gives you
a convenient signal to work with while you get the scope working.
Use the black and white outputs on the PI-9587C.

The figure on the last page is a simplified
drawing of the front panel of a digital oscilloscope, showing
only the most important controls you'll need for this lab.
When you turn on the oscilloscope, it will take a while to start
up.

\mysubsubsection{Preliminaries:}

\begin{itemize}
\item[] Press DEFAULT SETUP.

\item[] Use the SEC/DIV knob to
put the time base on something reasonable compared to the
period of the signal you're looking at. The time base is
displayed on the screen, e.g., 10 ms/div, or 1 s/div.

\item[] Use the VOLTS/DIV knob to
put the voltage scale (Y axis) on a reasonable scale
compared to the amplitude of the signal you're looking at.

\item[] The scope has two channels, i.e., it can accept input
through two BNC connectors and display both or either. You'll
only be using channel 1, which is the only one represented in
the simplified drawing. By default, the oscilloscope draws
graphs of both channels' inputs; to get rid of ch. 2, hold
down the CH 2 MENU button (not shown in the diagram) for a couple
of seconds. You also want to make sure that the scope is triggering
on CH 1, rather than CH 2. To do that, press the TRIG MENU
button, and use an option button to select CH 1 as the source.
Set the triggering mode to normal, which is the mode in which
the triggering works as I've described above.
If the trigger level is set to a level that the signal never
actually reaches, you can play with the knob that sets the trigger
level until you get something. A quick and easy way to do this
without trial and error is to use the SET TO 50\%
button, which
automatically sets the trigger level to midway between the top
and bottom peaks of the signal.

\item[] You want to select AC, not DC or GND, on the channel you're using. You
are looking at a voltage that is alternating, creating an
alternating current, ``AC.'' The ``DC'' setting is only
necessary when dealing with constant or very slowly varying
voltages. The ``GND'' simply draws a graph using $y=0$,
which is only useful in certain situations, such as when you
can't find the trace. To select AC, press the CH 1 MENU
button, and select AC coupling.

\end{itemize}

Observe the effect of changing the voltage scale and time
base on the scope. Try changing the frequency and amplitude
on the sine wave generator.

You can freeze the display by pressing RUN/STOP, and then
unfreeze it by pressing the button again.

\section{Preliminary Observations}

Now try observing signals from the microphone.

Notes for the group that uses the Shure mic: As with the Radio Shack
mics, polarity matters. The tip of the phono plug connector is the
live connection, and the part farther back from the tip is the
grounded part. You can connect onto the phono plug with alligator
clips.

Once you have your setup working, try measuring the period
and frequency of the sound from a tuning fork, and make sure
your result for the frequency is the same as what's
written on the tuning fork.

\observations

\labpart{ Periodic and nonperiodic speech sounds}

Try making various speech sounds that you can sustain
continuously: vowels or certain consonants such as ``sh,''
``r,'' ``f'' and so on. Which are periodic and which are not?

Note that the names we give to the letters of the alphabet
in English are not the same as the speech sounds represented
by the letter. For instance, the English name for ``f'' is
``ef,'' which contains a vowel, ``e,'' and a consonant,
``f.'' We are interested in the basic speech sounds, not the
names of the letters. Also, a single letter is often used in
the English writing system to represent two sounds. For
example, the word ``I'' really has two vowels in it,
``aaah'' plus ``eee.''

\labpart{ Loud and soft}

What differentiates a loud ``aaah'' sound from a soft one?

\labpart{ High and low pitch}

Try singing a vowel, and then singing a higher note with the
same vowel. What changes?

\labpart{ Differences among vowel sounds}

What differentiates the different vowel sounds?

\labpart{ Lowest and highest notes you can sing}

What is the lowest frequency you can sing, and what is the highest?

\prelab

\prelabquestion  In the sample oscilloscope trace shown on
page \pageref{fig:em-osc-waveform}, what is
the period of the waveform? What is its frequency? The time base is
10 ms.

\prelabquestion  In the same example, again assume the time base is 10
ms/division. The voltage scale is 2 mV/div\-ision. Assume
the zero voltage level is at the middle of the vertical
scale. (The whole graph can actually be shifted up and down
using a knob called ``position.'')  What is the trigger
level currently set to? If the trigger level was changed to
2 mV, what would happen to the trace?

\prelabquestion  Referring to the chapter of your textbook on sound,
which of the following would be a reasonable time base to
use for an audio-frequency signal? 10 ns, 1$\mu$ s, 1 ms, 1 s

\prelabquestion  Does the oscilloscope show you the signal's period,
or its wavelength? Explain.

\analysis

The format of the lab writeup can be informal. Just describe
clearly what you observed and concluded.

\widefigcaption{em-osc-panel}{A simplified diagram of the controls on a digital oscilloscope.}
