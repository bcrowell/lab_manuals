\renewcommand\thechapter{\arabic{chapter}a}
\lab{Setup of the Spectrometer}\label{lab:spectrometer}
\renewcommand\thechapter{\arabic{chapter}}

\apparatus
\equipn{Hg gas discharge tube (PASCO OS-9286)}{3}
\equipn{spectrometer}{1/group}
\equipn{diffraction grating, 600 lines/mm}{1/group}
\equipn{small screwdriver}{1}
\equipn{black cloth}{1}
\equipn{piece of plywood}{1}
\equipn{block of wood}{1}
\equipn{penlight}{1/group}
\equip{light block}

\section*{Goals}

The lab has three parts. This one, part a, is about setting
up the optics of the spectrometer. 
This is to be done once by the instructor or lab technician. 
It never needs to be redone unless something gets messed up.

\introduction


\section*{Method}

The apparatus  is shown in the first figure below. For a given wavelength, the
grating produces diffracted light at many different angles:
a central zeroth-order line at $\theta=0$, first-order lines on
both the left and right, and so on through higher-order
lines at larger angles.  The line of order $m$ occurs at an
angle satisfying the equation $m\lambda=d\sin\theta$.

To measure a wavelength, students will move the telescope until
the diffracted first-order image of the slit is lined up
with the telescope's cross-hairs and then read off the
angle. Note that the angular scale on the table of the
spectroscope actually gives the angle labeled $\alpha$ in
the figure, not $\theta$.

\subsection*{Sources of systematic errors}

There are three sources of systematic error:

\begin{itemize}
\item[] \emph{angular scale out of alignment:\/} If the angular scale is out
of alignment, then all the angles will be off by a constant amount.

\item[] \emph{factory's calibration of $d\/$:\/} The factory that made the
grating labeled it with a certain spacing (in lines per
millimeter) which can be converted to $d$ (center-to-center
distance between lines). But their manufacturing process is
not all that accurate, so the actual spacing of the lines is
a little different from what the label says.

\item[] \emph{orientation of the grating:\/} Errors will be caused if the grating
is not perpendicular to the beam from the collimator, or if the lines on the
grating are not vertical (perpendicular to the plane of the circle).
\end{itemize}

\subsection*{Eliminating systematic errors}

A trick to eliminate the error due to 
misalignment of the angular scale is to observe the same line on both the
right and the left, and take $\theta$ to be half the difference
between the two angles, i.e., $\theta=(\alpha_R-\alpha_L)/2$.
Because you are subtracting two angles, any source of error
that adds a constant offset onto the angles is eliminated.
A few of the spectrometers have their angular scales out of alignment
with the collimators by as much as a full degree, but that's of
absolutely no consequence if this technique is used.

Regarding the calibration of $d$, 
the first person who ever did this type of experiment simply
had to make a diffraction grating whose $d$ was very precisely
constructed. But once someone
has accurately measured at least one wavelength of one
emission line of one element, one can simply
determine the spacing, $d$, of any grating using a line
whose wavelength is known.

You might think that these two tricks would be enough to get rid
of any error due to misorientation of the grating, but they're not.
They will get rid of any error of the form $\theta\rightarrow\theta+c$
or $\sin\theta\rightarrow c\sin\theta$, but misorientation of the grating
produces errors of the form $\sin\theta\rightarrow\sin\theta+c$.
Part A below describes some additional adjustments that help
to get rid of additional sources of error.

\section*{Theory of Operation}

\setcounter{labpartctr}{0}


The second figure below shows the optics from the side, with the
telescope simply looking down the throat of the collimator
at $\theta=0$. You are actually using the optics to let you
see an image of the slit, not the tube itself. The point of
using a telescope is that it provides angular magnification,
so that a small change in angle can be seen visually.

A lens is used inside the collimator to make the light from
the slit into a parallel beam. This is important, because we
are using $m\lambda =d\sin\theta$ to determine the
wavelength, but this equation was derived under the
assumption that the light was coming in as a parallel beam.
To make a parallel beam, the slit must be located accurately
at the focal point of the lens. This adjustment should have
already been done, but you will check later and make sure. A
further advantage of using a lens in the collimator is that
a telescope only works for objects far away, not nearby
objects from which the reflected light is diverging
strongly.  The lens in the collimator forms a virtual image
at infinity, on which the telescope can work.

The objective lens of the telescope focuses the light,
forming a real image inside the tube. The eyepiece then acts
like a magnifying glass to let you see the image. In order
to see the cross-hairs and the image of the slit both in
focus at the same time, the cross-hairs must be located
accurately at the focal point of the objective, right
on top of the image.

\section*{Adjustments}

First you must check that the cross-hairs are at the
focal point of the objective. If they are, then the image
of the slits formed by the objective will be at the same point
in space as the crosshairs. You'll be able to focus your eye on both
simultaneously, and there will be no parallax
error depending on the exact position of your eye.
The easiest way to check this is to look through the telescope
at something far away ($\gtrsim$ 50 m), and move your head left
and right to see if the crosshairs move relative to the image.
Slide the eyepiece in and out to achieve a comfortable focus.
If this adjustment is not correct, you may
need to move the crosshairs in or out; this is done
by sliding the tube that is just outside the eyepiece tube.
(You need to use the small screwdriver to loosen the screw
on the side, which is recessed inside a hole. The hole may
have a dime-sized cover over it.)

The white plastic pedestal should have already been adjusted
properly to get the diffraction grating oriented correctly
in three dimensions, but you should check it carefully.
There are some clever features built into the apparatus to help in
accomplishing this. As shown in the third figure, there are three
axes about which the grating could be rotated. Rotation about axis 1 is
like opening a door, and this is accomplished by rotating the entire pedestal like a lazy Susan. Rotation about
axes 2 (like folding down a tailgate) and 3 are accomplished using the
tripod of screws underneath the pedestal. 
The eyepiece of the telescope is of a type called a Gauss eyepiece, with a diagonal piece of
glass in it.  When the grating is oriented correctly about axes 1 and 2 and the telescope
is at $\theta=0$, a beam
of light that enters through the side of the eyepiece is partially reflected to
the grating, and then reflected from the grating back to the eye. If these two axes
are correctly adjusted, the reflected image of the crosshairs is superimposed on the
crosshairs.

First get a rough initial adjustment of the pedestal by moving the telescope to 90 degrees
and sighting along it like a gun to line up the grating.
Now loosen the screw (not shown in the diagram) that frees the rotation of the pedestal.
Put a desk lamp behind the slits of the collimator, line up the telescope with
the $m=0$ image (which may not be exactly at $\alpha=180$ degrees), remove the desk
lamp, cover the whole apparatus with the black cloth, and position a penlight
so that it shines in through the hole in the side of the eyepiece. Adjust axes
1 and 2. If you're far out of adjustment, you may see part of a circle of light,
which is the reflection of the penlight; start by bringing the circle of light
into your field of view. When you're done, tighten the screw that keeps the pedestal
from rotating. The pedestal is locked down to the tripod screws by the tension in
a spring, which keeps the tips of two of the screws secure in dimples underneath the
platform. Don't lower the screws too much, or the pedestal will no longer
stay locked; make a habit of gently wiggling the pedestal after each adjustment
to make sure it's not floating loose. Two of the spectrometers have the diagonal
missing from their eyepieces, so if you have one of those, you'll have to borrow
an eyepiece from another group to do this adjustment.

For the adjustment of axis 3, place a piece of masking tape so that it covers exactly
half of the slits of the collimator. Put the Hg discharge tube behind the slits.
The crosshairs should be near the edge of the tape in the $m=0$ image. Move the telescope out to
a large angle where you see one of the high-$m$ Hg lines, and adjust the tripod screws
so that the crosshairs are at the same height relative to the edge of the tape.

\widefigcaption{mo-hel-spectrometer}{The spectrometer}
\widefigcaption{mo-hel-optics}{Optics.}
\widefigcaption{mo-hel-orient}{Orienting the grating.}

