\lab{Resistivity}\label{lab:resistivity}

\apparatus
\equipn{power supply (Thornton), in lab benches}{1/group}
\equipn{multimeter (PRO-100), in lab benches}{1/group}
\equipn{digital multimeters (FlukeHP)}{1/group}
\equip{resistors of various values}
\equip{alligator clips}
\equip{spare fuses for multimeters --- Let students replace fuses themselves.}
\equip{Play-Doh (Crayloa super soft dough, \$10/3 lb at school supply store on Commonwealth)}
\equip{1 g aluminum slotted weights for use as electrical contacts}

\begin{goals}

\item[] Measure how the electrical resistance of a cylinder depends on its dimensions.

\item[] Invent a constant of proportionality incorporated into this relationship, giving a measure of the intrinsic electrical properties of a material
called its resistivity.
\end{goals}

\introduction

Your nervous system depends on electrical currents, and
every day you use many devices based on electrical currents
without even thinking about it. Despite its ordinariness,
the phenomenon of electric currents passing through liquids
(e.g., cellular fluids) and solids (e.g., copper wires) is a
subtle one. For example, we now know that atoms are composed
of smaller, subatomic particles called electrons and nuclei,
and that the electrons and nuclei are electrically charged,
i.e., matter is electrical. Thus, we now have a picture of
these electrically charged particles sitting around in
matter, ready to create an electric current by moving in
response to an externally applied voltage. Electricity had
been used for practical purposes for a hundred years,
however, before the electrical nature of matter was proved
at the turn of the 20th century.

We observe that some materials, such as metals, conduct electricity
more easily than others, such as wood. This suggests that measurements
of the resistance of an object made out of a certain substance can be
used to learn things about the atomic-level structure of the substance
(e.g., how many free charge carriers it has per unit volume), or
as a test to identify an unknown substance.

\fig{em-ohm-colorcodetable}
\fig{em-ohm-colorcodeeg}

\observations

\labpart{A known resistor}

In lab \ref{lab:electricity}, you found the resistance of a lightbulb.
To keep things simple, that setup used only a single multimeter, but
that required rearranging how everything was hooked up every time you
wanted to switch between measuring voltage and measuring current.
In today's lab, you will use a similar setup, but with two meters simultaneously connected, one for each purpose.
Build your setup, and test it by measuring data on a known resistor.
To avoid creating large currents or making the resistor so hot that it
will burn your fingers, use one with a fairly high value, at least in
the kiloohm range.

Resistors are usually too small to make it convenient to
print numerical resistance values on them, so they are
labeled with a color code, as shown in the table and example above.


As a test of whether your resistor is actually ohmic, take data
at two different voltages and determine whether the current varies
in proportion.

\labpart{Dependence of resistance on dimensions}

For a fixed substance such as sand or plywood, the resistance is not
a fixed number of ohms. The resistance of a sample depends on its size,
its shape, and the way in which the electrical contacts are made to it.
In this part of the lab you will take measurements with cylindrical samples of play-doh,
with the electrical contacts made using the thin disk-shaped 1 g aluminum weights.

Two different numbers are required in order to define the dimensions of a cylinder.
(There are various ways to define these; you can make your own choice.)
Measure the resistance for different values of these two variables.
Practice control of variables by keeping one constant while varying the other.
You will learn the most by taking extreme values of the variables.

Leaving the power supply on for a long time causes corrosion to the electrical contacts
and chemical changes in the play-doh, so don't do that.

To get good results:
\begin{itemize}
\item[] Quite a bit of the resistance comes from the surface of contact between the aluminum disk
and the play-dough. What seems to
matter the most is that you need a very smooth surface, because on a rough surface, most of the
metal isn't even making contact.
\item[] It seems to help if you cut into the play-dough and then press the play-dough against the
contact with your fingers to make better contact.
\item[] I got better results by subtracting the resistance found by placing the two contacts
very close, where the resistance was almost entirely due to the contacts.
\end{itemize}


\analysis

For each variable in part B,
use the graphing technique given in appendix \ref{appendix:powerlaws} to see if you can
find a power law that relates the resistance to that variable. If the results are at least
approximately consistent with power laws, then you can combine them to create a relationship
of the form $R=\rho(\ldots)^p(\ldots)^q$, where $(\ldots)$ represents the two variables,
$p$ and $q$ are constants found from your graph, and the constant of proportionality $\rho$
(Greek letter rho) is a property of the substance, called its resistivity. If you find that
$p$ and $q$ have values that are close to nice simple numbers, carry out your analysis under
the assumption that those simple numbers are right. What units does $\rho$ have?
Find the resistivity of play-doh.

\prelab

\prelabquestion  Check that you understand the interpretations of the
following color-coded resistor labels:

\begin{tabular}{ll}
   blue   gray   orange   silver    &= 68 k$\Omega$  $\pm$ 10\%  \\
   blue   gray   orange   gold    &= 68 k$\Omega$  $\pm$ 5\%  \\
   blue   gray   red   silver    &= 6.8 k$\Omega$  $\pm$ 10\%  \\
   black   brown   blue   silver    &= 1 M$\Omega$  $\pm$ 10\%  
\end{tabular}

Now interpret the following color code:

\begin{tabular}{ll}
   green   orange   yellow   silver    &=    ?  
\end{tabular}
