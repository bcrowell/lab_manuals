\renewcommand\thechapter{c1.13b}
\lab{The moment of inertia of a pulley}\label{lab:covid-pulley}

\section*{About this lab}

\covid\ 
This is a lab intended for students to do at home during the Covid-19 epidemic. 

\apparatus
\equip{SEOH pulley, plastic triple tandem}
\equip{string}
\equip{ruler}
\equip{about 3 $\times$ pennies}
\equip{leg shots}
\equip{digital balance}
\equip{suction cup with hook}

\begin{goals}

\item[] In the previous Newton's second law of motion lab, we assumed the pulley is ideal. In this lab, we will not ignore the motion of a pulley in the same experiment. Determine the moments of inertia of three different pulleys.  
\end{goals}

\introduction

When a pulley is not ideal, we need to include the pulley's motion to the system. The pulley is rotating, so we can use a net torque equation. 
\begin{equation*}
\sum^N_i \tau_i = I \alpha
\end{equation*}
where $\tau_i$ is a torque, $I$ is a moment of inertia, and $\alpha$ is angular acceleration. 

The kinetic energy of the pulley is,
\begin{equation*}
KE = \dfrac{1}{2} I \omega^2
\end{equation*}
where $\omega$ is angular velocity of a pulley. 

In this lab, you will determine the moments of inertia of three different pulleys. 

\observations 
The basic setup of this lab is the same as the physics 210 lab manual Lab 3: Newton's Second Law (\url{http://www.lightandmatter.com/area3lab.html}). Please read the manual before you start this lab. 

\labpart{Preparation}
First, measure radii and masses of three pulleys. (There are two types of pulley systems. One kind of pulley cannot be disassembled. If you cannot measure the masses of pulleys, use the following values: Large=4.3 g, Medium = 3.7 g, and Small= 1.4 g. )  Then, prepare the apparatus like Newton's second law lab. However, in this lab, we will use weights comparable to the masses of pulleys. The masses of pulleys are similar to the mass of a penny. We will use pennies or lead shots as weights on hangers.  (e.g., 1 penny on one hanger and 2 pennies on the other hanger. ) We will not change the masses of the weights throughout this lab. 

\fig{me-2nd-setup-covid}

\labpart{Measurement}

Measure the falling times and distances of the weight for three different pulleys. It would be best if you made multiple measurements to estimate the standard deviations of the time and distance measurements. 

\labpart{Further improvement(optional)}
The lab setup is not ideal. You can improve the first result by minimizing the errors. Think about various methods to reduce errors. Then, adjust the experimental setup and make the second measurements. Write changes and the result of the improvement in the lab report. 

\analysis

\subsection*{Theoretical prediction}
Calculate the theoretical moments of inertia of pulleys by assuming the pulleys are solid disks. Since a pulley has a groove, the radius is not a constant. You can get the upper and the lower limits of the moment of inertia of a pulley by measuring the outer rim radius and the groove radius. 

\subsection*{data analysis}
Based on the time and distance measurements, determine moments of inertia of the three pulleys experimentally. You can calculate the moment of inertia from the equations of motion of the weights and a pulley or the conservation of energy equation. 

\subsection*{error analysis}
Propagate the errors in the experimental moments of inertia. Then compare the experimental moments of inertia with the theoretical predictions. 
Does your experimental finding verify the theory? Explain what the sources of errors are. 

\prelab 
