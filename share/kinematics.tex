\lab{Kinematics}\label{lab:kinematics}

\apparatus
\equipn{computer }{1/group}
\equipn{track  }{1/group}
\equipn{dynamics cart  }{1/group}
\equipn{fan  }{1/group}
\equipn{AA batteries  }{4/group}
\equipn{aluminum slugs  }{2/group}
\equipn{motion detector (in lab bench)}{1/group}
\equipn{interface box (in lab bench)}{1/group}
\equip{WD-40}

\goal{ Learn how to relate the motion of an object to its
position-versus-time graph.}

\introduction

Analyzing motion is the most fundamental thing we do in
physics. The most versatile way of representing motion is
with a graph that has the object's position on the upright
axis and time on the horizontal axis. It takes some practice
to be able to sketch and interpret these graphs, but once
you get used to them, they become very intuitive.

\apparatus
The object whose motion you'll study is a cart that rolls on
a track. You can either push the cart by hand, start it
moving with a shove, or clamp a fan on top of it to make it
speed up or slow down steadily. To measure the cart's
motion, you'll use a little sonar gun that sends out clicks.
When it hears the echo from the cart, it figures out how far
away the cart was based on the time delay and the known
speed of sound. The sonar gun is connected to a computer,
which produces a position-versus-time graph.

\fig{me-kin-setup}

\setup

Check that all four wheels on the cart will spin for about 20-30 seconds
if you flick them hard. If they only spin for a few seconds, see if
you can fix the problem by spraying WD-40 on the bearings.

Set the cart on the track without the fan. Prop the motion
detector (sonar gun) at one end of the track. Make sure the white
switch underneath the speaker is set to the logo of a cart.

The first few parts of the lab involve moving the cart by hand, and
don't require the fan. However, you should click the fan attachment
onto the cart anyway, because it makes a bigger target for the sonar.

Because you're going to need to print a graph, make sure the interface
box is plugged into one of the Windows computers, not one of the Linux
computers, which aren't networked to the printer.
With the computer turned off, plug the motion detector into
the DIG/SONIC plug on the interface box.

Start up the computer. For compactness, I'll use notation
like this to describe the computer commands:

Start$>$All Programs$>$Logger Pro$>$Logger Pro 3

This is the command to start the computer software running.
``Start'' means to click on the start menu at the bottom
left corner of the screen, ``Programs'' means to select that
from the menu, and so on. There are two different versions of the
software installed; use version 3.

You'll get
two graphs on the screen, but you only want one, the $x-t$
graph. To remove the $v-t$ graph, right-click on it and do ``delete.''

If you now click the button to tell it to collect data, the
motion detector should start clicking rapidly, and it you
move the cart back and forth you should see a graph of its
motion. Make sure it is able to sense the cart's motion
correctly for distances from 15 cm to the full length of the
track. If it doesn't work when the cart is at the far end of
the track, try the following: (1) Make small adjustments to the the vertical and horizontal
orientation of motion detector. (2) Make sure the fan attachment is in place.
(3) Remove objects such as computer monitors or backpacks from the cone of
the sonar beam.

\observations

%References don't work correctly:
%\ref{part:kinematics:fastandslow}
%\ref{part:kinematics:slowandrapidaccel}
In parts A through E, you don't need to take
detailed numerical data --- just sketch the graphs in your lab
notebook. Each graph will have garbage data at the beginning
and the end, and you need to make sure you understand what's what.

\labpart{Fast and slow motion}\label{part:kinematics:fastandslow}

Moving the cart by hand, make a graph for slow motion and
another for fast motion. Make sure the motion is steady, and
don't get confused by the parts of the graph that come
before and after your period of steady pushing. Sketch the
graphs and make sure you understand them.

\labpart{Motion in two different directions}

Now try comparing the graphs you get for the two different
directions of motion. Again, record what they look like and
figure out what you're seeing.

\labpart{ Reproducing a graph}

Now see if you can produce a graph that looks like this:

\fig{me-kin-graph}

\labpart{ Accelerating away from the sensor}

Suppose the fan is mounted on the cart as shown in the
figure, so that if the cart is released from a position
close to the motion detector, it will begin moving away from
it. Predict what you think the cart's position-time graph
will look like, and show your prediction to your instructor
before getting a fan.

Before putting the batteries in the fan, make sure the fan's
switch is off (to the right). Put the batteries in and clamp
the fan on the cart.

Set up the situation described above, and compare the
results with what you predicted.

\labpart{ Slow or Rapid Acceleration}\label{part:kinematics:slowandrapidaccel}

The aluminum slugs can be used to replace two of the
batteries so that the fan will exert about half as much
force. Discuss with your partners what you think will happen
if you repeat your previous run with a weakened fan. Now try it.

\labpart{ Changing the direction of motion}

Change the fan back to full strength.

Now suppose instead of releasing the cart from rest close to
the motion detector, you started it moving with a push
toward the motion sensor, from the far end of the track. It
will of course slow down and eventually come back. Discuss
with your partners what the position-time graph would
look like. Now try it.

\labpart{Rate of changing speed}

The goal of this part of the lab is to determine whether the speed
of the cart in part F was changing at a constant rate, i.e., by the
same amount every second. This is the only part of the lab that you
need to discuss in your writeup.

Zoom in on the relevant part of your graph from part F.
To zoom in,  draw a box with the mouse and click on the magnifying glass icon.
Print out a \emph{big} copy; choose landscape mode in the print dialog box.
\footnote{If the printer isn't working, here's what you need to do instead.
Do File$>$Export Data, and select ``.txt'' for the
type of the file. Use a text editor such as WordPad to
delete the header from the
file. Save it in your FC student directory. If you want to work
on it at home, you can also email it to yourself or save it on a flash drive.
Get into OpenOffice or Excel, and open the file.
Appendix \ref{appendix:graphing} describes how to use
OpenOffice.
Whatever method you use, make sure the whole group
will end up with copies.}
(Note that if you take
different data later, you may need to fiddle with this again
because you'll be zoomed in on the wrong part of the new graph.)

\prelab

\prelabquestion  Draw predictions of the four graphs you'll obtain in parts A
and B.

\selfcheck

Do the analysis in lab.

\analysis

At one-second intervals, draw nice long tangent lines on the
curve from part G and determine their slope. Some slopes
will be negative, and some positive.  Summarize this series
of changing speeds in a table. Did the velocity increase by
about the same amount with every second?

Rather than trying to read distances from your graph's vertical axis in units
of meters, and times from its horizontal axis in units of seconds,
the simplest thing to do is simply to use a ruler to
measure vertical and horizontal distances on the graph, and determine the slopes
from these; although the resulting slopes won't be in any standard
units, that won't affect your conclusion. For the best possible precision,
you could make the ruler lines as long as possible, i.e., extend them all
the way to the edge of the paper.

You don't need to discuss part A-F in your writeup.
