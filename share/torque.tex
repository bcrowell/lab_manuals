\lab{Torque}\label{lab:torque}

\apparatus
\equipn{meter stick with holes drilled in it}{1/group}
\equip{spring scales, calibrated in newtons}
\equip{slotted weights  (in lab benches in 415)}
\equip{holder for slotted weights}
\equip{string}
\equip{protractors}
\equip{Ohaus balances}

\goal{Test whether the total force and torque on an object at
rest both equal zero.}

\introduction

It is not enough for a boat not to sink.  It also must not
capsize.  This is an example of a general fact about
physics, which is also well known to people who overindulge
in alcohol: if an object is to be in a stable equilibrium at
rest, it must not only have zero net force on it, to keep
from picking up momentum, but also zero net torque, to keep
from acquiring angular momentum.

\fig{me-tor-setup}

\observations

Weigh your meter stick before you do anything else; they don't all
weigh the same amount.

Construct a setup like the one shown above.  Avoid any
symmetry in your arrangement.  There are four forces
acting on the meter stick:
\begin{align*}
       \vc{F}_H &=    \text{the weight hanging underneath}  \\
       \vc{F}_M &=    \text{Earth's gravity on the meter stick itself}  \\
       \vc{F}_L &=    \text{tension in the string on the left}  \\
       \vc{F}_R &=    \text{tension in the string on the right} 
\end{align*}

Each of these forces also produces a torque.

In order to determine whether the total force is zero, you
will need enough raw data so that for each torque you can
extract (1) the magnitude of the force vector, and (2) the
direction of the force vector.  In order to add up all the
torques, you will have to choose an axis of rotation, and
collect enough raw data to be able to determine for each
force (3) the distance from the axis to the point at which
the force is applied to the ruler, and (4) the angle between
the force vector and the line connecting the axis with the
point where the force is applied.  Note that the meter
stick's own weight can be thought of as being applied
at its center of mass.

The meter stick has holes drilled in it that you can use to
attach the strings. You can make your analysis simpler by
tying the knots as shown below, so that all the forces act
at points along the center-line of the stick.

\fig{me-tor-knot}

You have a selection of spring scales, so use the
right one for the job --- don't use a 20 N scale to
measure 0.8 newtons, because it will not be possible to read it accurately.
Optimize your precision by choosing conditions that come as close as possible
to maxing out the scales.
For each spring scale, hang a known weight from it, and adjust the
calibration tab so that the scale gives the correct result.
If you need to swap in a new spring scale, don't forget to calibrate it.

\prelab

\prelabquestion  You have complete freedom in defining what point is to be considered the axis of
rotation ---  if one choice of axis causes the total torque
to be zero, then any other choice of axis will also cause
the total torque to be zero.  It is possible to simplify the
analysis by choosing the axis so that one of the four
torques is zero.  Plan how you will do this.

\prelabquestion  All the torques will be tending to cause rotation in the
same plane.  You can therefore use plus and minus signs to
represent clockwise and counterclockwise torques.  Choose
which one you'll call positive.  Using your choice of axis,
which of the four torques, $\tau_H,\tau_M,\tau_L$, and
$\tau_R$, will be negative, which will be positive,
and which will be zero?

\analysis

Determine the total force and total torque on the meter
stick.  For the forces, I think a graphical calculation will
be easier than an analytic one.

Finally, repeat your calculation of the total torque using a
different point as your axis. Although you're normally expected to
do your analysis completely independently, for this lab it's okay
if you find the total torque for one choice of axis, and your lab
partners do the calculation for their own choices. Present both
results in your own abstract.

m4_ifdef([:__sn:],[:%
Do error analysis for the total torque computed using one choice of axis.
Since this is a moderately complex calculation, you should set up all the number-crunching in
a python program, so that it becomes trivial to change
one of the inputs and recompute the output.
:],[:%
Error analysis is not required. For extra credit, you can do
error analysis for your total torque.
:])
