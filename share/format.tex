\myappendix{format}{Format of Lab Writeups}

	Lab reports must be three pages or less, not counting your
raw data. The format should be as follows: 
\label{format-really-here}

\textbf{Title}

\textbf{Raw data} --- \emph{Keep actual observations separate from what you
later did with them.}\\
These are the results of the measurements you take down
during the lab, hence they come first. Write your raw data directly in your
lab book; don't write them on scratch paper and recopy them
later. Don't use pencil. The point is to separate facts from
opinions, observations from inferences.

\textbf{Procedure} --- \emph{Did you have to create your own methods for
getting some of the raw data?}\\
Do not copy down the procedure from the manual. In this
section, you only need to explain any methods you had to
come up with on your own, or cases where the methods
suggested in the handout didn't work and you had to do
something different. Do not discuss how you did your
calculations here, just how you got your raw data.

\textbf{Abstract} --- \emph{What did you find out? Why is it important?}\\
The ``abstract'' of a scientific paper is a \emph{short} paragraph
at the top that summarizes the experiment's results in a few
sentences. 

Many of our labs are comparisons of theory and experiment. The abstract
for such a lab needs to say whether you think the experiment was consistent
with theory, or not consistent with theory.
If your results deviated from the ideal equations, don't be afraid
to say so.
After all, this is real life, and many of the equations we
learn are only approximations, or are only valid in certain
circumstances. However, (1) if you simply mess up, it is
your responsibility to realize it in lab and do it again,
right; (2) you will never get exact agreement with theory,
because measurements are not perfectly exact --- the
important issue is whether your results agree with theory to
roughly within the error bars.

The abstract is not a statement of what you hoped to find out. It's
a statement of what you \emph{did} find out. It's like the brief statement
at the beginning of a debate: ``The U.S. should have free trade with China.''
It's not this: ``In this debate, we will discuss whether the U.S. should have
free trade with China.''

If this is a lab that has just one important numerical result (or
maybe two or three of them), put them
in your abstract, with error
bars where appropriate. There should normally be no more
than two to four numbers here. Do not recapitulate your raw
data here --- this is for your final results.

If you're presenting a final result with error bars, make sure that
the number of significant figures is consistent with your error bars.
For example, if you write a result as $323.54\pm6$ m/s, that's wrong.
Your error bars say that you could be off by 6 in the ones' place, so the
5 in the tenths' place and the four in the hundredths' place are
completely meaningless.

If you're presenting a number in scientific notation, with error bars,
don't do it like this
\begin{equation*}
  1.234 \times 10^{-89}\ \munit/\sunit \pm 3 \times 10^{-92}\ \munit/\sunit \qquad ,
\end{equation*}
do it like this
\begin{equation*}
  (1.234 \pm 0.003)\times 10^{-89}\ \munit/\sunit \qquad ,
\end{equation*}
so that we can see easily which digit of the result the error bars apply
to.

\textbf{Calculations and Reasoning}  --- \emph{Convince me of
what you claimed in your abstract.}\\
Often this section consists of nothing more than the calculations
that you started during lab. If those calculations are clear enough
to understand, and there is nothing else of interest to explain,
then it is not necessary to write up a separate
narrative of your analysis here.
If you have a long series of similar calculations,
you may just show one as a sample. If your prelab involved
deriving equations that you will need, repeat them here
without the derivation. 

In some labs, you will need to go into some detail here by giving
logical arguments to convince me that the statements you made
in the abstract follow logically from
your data. Continuing the debate meta\-phor, if your abstract
said the U.S. should have free trade with China, this is the rest
of the debate, where you convince me, based on data and logic,
that we should have free trade.
