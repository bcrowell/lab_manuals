\lab{Electricity}\label{lab:electricity}

\apparatus
\equip{scotch tape}
\equip{rubber rod}
\equip{heat lamp}
\equip{fur}
\equip{bits of paper}
\equip{rods and strips of various materials}
\equip{30-50 cm rods, and angle brackets, for hanging charged rods}
\equipn{power supply (Thornton), in lab benches}{1/group}
\equipn{multimeter (PRO-100), in lab benches}{1/group}
\equip{alligator clips}
\equip{flashlight bulbs}
\equip{spare fuses for multimeters --- Let students replace fuses themselves.}

\enlargethispage{-\baselineskip}

\begin{goals}

\item[] Determine the qualitative rules governing electrical charge and forces.

\item[] Light up a lightbulb, and measure the current through it and the voltage difference across it.
\end{goals}

\vspace{10mm}

\introduction

Newton's law of gravity gave a mathematical formula for the
gravitational force, but his theory also made several
important non-mathematical statements about gravity:

\begin{itemize}
\item[] Every mass in the universe attracts every other mass in the universe.

\item[] Gravity works the same for earthly objects as for heavenly bodies.

\item[] The force acts at a distance, without any need for physical contact.

\item[] Mass is always positive, and gravity is always attractive, not repulsive.
\end{itemize}

The last statement is interesting, especially because it
would be fun and useful to have access to some negative
mass, which would fall up instead of down (like the
``upsydaisium'' of Rocky and Bullwinkle fame).

Although it has never been found, there is no theoretical
reason why a second, negative type of mass can't exist. 
Indeed, it is believed that the nuclear force, which holds
quarks together to form protons and neutrons, involves three
qualities analogous to mass. These are facetiously referred
to as ``red,'' ``green,'' and ``blue,'' although they have
nothing to do with the actual colors. The force between two
of the same ``colors'' is repulsive: red repels red, green
repels green, and blue repels blue. The force between two
different ``colors'' is attractive: red and green attract
each other, as do green and blue, and red and blue.

\vfill

When your freshly laundered socks cling together, that is an
example of an electrical force. If the gravitational force
involves one type of mass, and the nuclear force involves
three colors, how many types of electrical ``stuff'' are
there? In the days of Benjamin Franklin, some scientists
thought there were two types of electrical ``charge'' or
``fluid,'' while others thought there was only a single
type. In the first part of this lab, you will try to find out experimentally
how many types of electrical charge there are.

\vfill

\enlargethispage{-1\baselineskip}

The unit of charge is the coulomb, C; one coulomb is defined as the
amount of charge such that if two objects, each with a charge of
one coulomb, are one meter apart, the magnitude of the electrical
force between them is $9\times10^9\ \nunit$.
Practical applications of electricity usually involve an electric
circuit, in which charge is sent around and around in a circle and
recycled. Electric current, $I$, measures how many coulombs per second flow
past a given point; a shorthand for units of C/s is the ampere, A.
Voltage, $V$, measures the electrical potential energy per unit charge;
its units of J/C can be abbreviated as volts, V. Making the analogy between
electrical interactions and gravitational ones, voltage is like height.
Just as water loses gravitational potential energy by going over a waterfall,
electrically charged particles lose electrical potential energy as they flow
through a circuit. The second part of this lab involves building an
electric circuit to light up a lightbulb, and measuring both the current that
flows through the bulb and the voltage difference across it.

\observations

m4_include(first_lab.tex)

\labpart{Inferring the rules of electrical repulsion and attraction}

Stick a piece of scotch tape on a table, and then lay
another piece on top of it. Pull both pieces off the table,
and then separate them. If you now bring them close
together, you will observe them exerting a force on each
other. Electrical effects can also be created by rubbing the
fur against the rubber rod.

Your job in this lab is to use these techniques to test
various hypotheses about electric charge. The most common
difficulty students encounter is that the charge tends to
leak off, especially if the weather is humid. If you have
charged an object up, you should not wait any longer than
necessary before making your measurements. It helps if you
keep your hands dry.

To keep this lab from being too long, the class will pool its data
for part A. Your instructor will organize the results on the whiteboard.

\vfill

\emph{i. Repulsion and/or attraction}

Test the following hypotheses. Note that they are
mutually exclusive, i.e., only one of them can be true.

A) Electrical forces are always attractive.

R) Electrical forces are always repulsive.

AR) Electrical forces are sometimes attractive and
sometimes repulsive.

Interpretation: Once the class has tested these
hypotheses thoroughly, we will discuss what
this implies about how many different types of charge there might be.

\vfill

\emph{ii. Are there forces on objects that have not been specially prepared?}

So far, special preparations have been necessary in order to
get objects to exhibit electrical forces. These preparations
involved either rubbing objects against each other (against
resistance from friction) or pulling objects apart (e.g.
overcoming the sticky force that holds the tape together).
In everyday life, we do not seem to notice electrical forces
in objects that have not been prepared this way.

Now try to test the following hypotheses. Bits of paper are
a good thing to use as unprepared objects, since they are
light and therefore would be easily moved by any force.
\emph{Do not} use tape as an uncharged object, since it can
become charged a little bit just by pulling it off the roll.

U0) Objects that have not been specially prepared are immune
to electrical forces.

UA) Unprepared objects can participate in electrical forces
with prepared objects, and the forces involved are always attractive.

UR) Unprepared objects can participate in electrical forces
with prepared objects, and the forces involved are always repulsive.

UAR) Unprepared objects can participate in electrical forces
with prepared objects, and the forces involved can be either
repulsive or attractive.

These four hypotheses are mutually exclusive.

Once the class has tested these hypotheses thoroughly,
we will discuss what practical implications this has for
planning the observations for part iii.

\emph{iii. Rules of repulsion and/or attraction and the
number of types of charge}

Test the following mutually exclusive hypotheses:

1A) There is only one type of electric charge, and the force
is always attractive.

1R) There is only one type of electric charge, and the force
is always repulsive.

2LR) There are two types of electric charge, call them X
and Y. Like charges repel (X repels X and Y repels Y)
and opposite charges attract (X and Y attract each other).

2LA) There are two types of electric charge. Like charges
attract and opposite charges repel.

3LR) There are three types of electric charge, X, Y and
Z. Like charges repel and unlike charges attract.

On the whiteboard, we will
make a square table, in which the rows and columns correspond
to the different objects you're testing against each other
for attraction and repulsion. To test hypotheses 1A through 3LR, you'll
need to see if you can successfully explain your whole
table by labeling the objects with only one label, X,
or whether you need two or three.

Some of the equipment may look identical, but not be identical.
In particular, some of the clear rods have higher density than
others, which may be because they're made different types of plastic,
or glass. This could affect your conclusions, so you may want to
check, for example, whether two rods with the same diameter,
that you think are made of
the same material, actually weigh the same.

In general, you will find that some materials, and some combinations
of materials, are more easily charg\-ed than others. For example, if
you find that the mahogony rod rubbed with the weasel fur doesn't
charge well, then don't keep using it! The white plastic strips
tend to work well, so don't neglect them.

Once we have enough data in the table to reach a definite conclusion,
we will summarize the results from part A and then
discuss the following examples of incorrect reasoning about this
lab. 

(1) ``The first piece of tape exerted a force on the second,
but the second didn't exert one on the first.''\\
(2) ``The first piece of tape repelled the second, and the
second attracted the first.''\\
(3) ``We observed three types of charge: two that exert
forces, and a third, neutral type.''\\
(4) ``The piece of tape that came from the top was positive,
and the bottom was negative.''\\
(5) ``One piece of tape had electrons on it, and the other
had protons on it.''\\
(6) ``We know there were two types of charge, not three,
because we observed two types of interactions, attraction
and repulsion.''

\labpart{Measuring current and voltage}

As shown in the figure, measuring current and voltage requires
hooking the meter into the circuit in two completely different
ways.

\fig{em-ele-use-of-meters}

The arrangement for the ammeter is called a series circuit,
because every charged particle that travels the circuit has
to go through each component in a row, one after another.
The series circuit is arranged like beads on a necklace.

The setup for the voltmeter is an example of a parallel
circuit. A charged particle flowing, say, clockwise
around the circuit passes through the power supply and
then reaches a fork in the road, where it has a choice
of which way to go. Some particles will pass through the
bulb, others (not as many) through the meter; all of them
are reunited when they reach the junction on the right.

Students tend to have a mental block against setting up the
ammeter correctly in series, because it involves breaking the circuit
apart in order to insert the meter. To drive home this point,
we will act out the process using students to represent
the circuit components. If you hook up the ammeter incorrectly,
in parallel rather than in series, the meter provides an easy path for the flow
of current, so a large amount of current will flow. To protect the meter
from this surge, there is a fuse inside, which will blow, and the meter
will stop working. This is not a huge tragedy;
just ask your instructor for a replacement fuse and open up the
meter to replace it.

Unscrew your lightbulb from its holder and look closely at it. Note
that it has two separate electrical contacts: one at its tip and
one at the metal screw threads.

Turn the power supply's off-on switch to the off position, and
turn its (uncalibrated) knob to zero.
Set up the basic lightbulb circuit without any meter in it.
There is a rack of cables in the back of the room with banana-plug
connectors on the end, and
most of your equipment accepts these plugs.
To connect to the two brass screws on the lightbulb's base, you'll
need to stick alligator clips on the banana plugs. 

Check your basic circuit with your instructor, then turn on the power
switch and \emph{slowly} turn up the knob until the bulb lights.
The knob is uncalibrated and highly nonlinear; as you turn it up,
the voltage it produces goes zerozerozerozerozero\emph{six!}
To light the bulb without burning it out, you will need to
find a position for the knob in the narrow range where it
rapidly ramps up from 0 to 6 V.

Once you have your bulb lit, do not mess with the knob on the
power supply anymore. You do not even need to switch the power
supply off while rearranging the circuit for the two measurements
with the meter; the voltage that lights the bulb is only about
a volt or a volt and a half (similar to a battery), so it can't
hurt you.

We have a single meter that plays both the role of the voltmeter
and the role of the ammeter in this lab. Because it can do both
these things, it is referred to as a multimeter. Multimeters are
highly standardized, and the following instructions are generic
ones that will work with whatever meters you happen to be using
in this lab.

\emph{Voltage difference}

Two wires connect the meter to the circuit. 
At the places where three wires come together at
one point, you can plug a banana plug into the back of
another banana plug. At the meter, make one connection
at the ``common'' socket (``COM'') and the other at the
socket labeled ``V'' for volts. The common plug is called
that because it is used for every measurement, not just for
voltage.

Many multimeters have more than one scale for
measuring a given thing. For instance, a meter may have a
millivolt scale and a volt scale. One is used for measuring
small voltage differences and the other for large ones. You may not
be sure in advance what scale is appropriate, but that's not
big problem --- once everything is hooked up, you can try
different scales and see what's appropriate. Use the switch
or buttons on the front to select one of the voltage scales.
By trial and error, find the most precise scale that doesn't
cause the meter to display an error message about being overloaded.

Write down your measurement, with the units of volts, and stop
for a moment to think about what it is that you've measured.
Imagine holding your breath and trying to make your eyeballs
pop out with the pressure. 
Intuitively, the voltage difference is like the pressure difference
between the inside and outside of your body.

What do you think will happen if you unscrew the bulb, leaving
an air gap, while the
power supply and the voltmeter are still going? Try it.
Interpret your observation in terms of the breath-holding
metaphor.

\emph{Current}

The procedure for measuring the current differs only because
you have to hook the meter up in series and because you have
to use the ``A'' (amps) plug on the meter and select a current
scale.

In the breath-holding metaphor, the number you're measuring
now is like the rate at which air flows 
through your lips as you let it hiss out. Based on this
metaphor, what do you think will happen to the reading when
you unscrew the bulb? Try it.

Discuss with your group and check with your instructor:\\
(1) What \emph{goes through} the wires? Current? Voltage? Both?\\
(2) Using the breath-holding metaphor, explain why the voltmeter
needs \emph{two} connections to the circuit, not just one.
What about the ammeter?\\
While waiting for your instructor to come around and discuss
these questions with you, you can go on to the next part of
the lab.

\emph{Resistance}

The ratio of voltage difference to current is called the resistance
of the bulb, $R=\Delta V/I$. Its units of volts per amp can be
abbreviated as ohms, $\Omega$ (capital Greek letter omega).

Calculate the resistance of your lightbulb. Resistance is the
electrical equivalent of kinetic friction. Just as rubbing
your hands together heats them up, objects that have
electrical resistance produce heat when a current is passed
through them. This is why the bulb's filament gets hot enough
to heat up.

When you unscrew the bulb, leaving an air gap, what is the
resistance of the air?

Ohm's law is a generalization about the electrical properties
of a variety of materials. It states that the resistance is
constant, i.e., that when you increase the voltage difference,
the flow of current increases exactly in proportion.
If you have time, test whether Ohm's law holds for your
lightbulb, by cutting the voltage to half of what you had
before and checking whether the current drops by the same factor.
(In this condition, the bulb's filament doesn't get
hot enough to create enough visible light for your eye to see,
but it does emit infrared light.)
