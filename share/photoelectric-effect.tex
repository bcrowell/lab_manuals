\lab{The Photoelectric Effect}\label{lab:photoelectric-effect}

\apparatus
\equip{hand-held diffraction gratings}
\equip{Hg gas discharge tube (Pasco SE-5509)}
\equip{photodiode (Pasco SE-5509)}
\equip{power supply for discharge tube (Pasco BEM-5007)}
\equip{power supply (Pasco BEM-5001)}
\equip{high-sensitivity ammeter (Pasco BEM-5004)}

\begin{goals}

\item[] Use the photoelectric effect to test predictions of the wave
and wave-particle models of light.

\end{goals}

\introduction

The photoelectric effect, a phenomenon in which light shakes
an electron loose from an object, provided the first
evidence for wave-particle duality: the idea that the basic
building blocks of light and matter show a strange mixture
of particle and wave behaviors. At the turn of the twentieth
century, physicists assumed that particle and wave phenomena
were completely distinct. Young had shown that light could
undergo interference effects such as diffraction, so it had
to be a wave. Since light was a wave composed of oscillating
electric and magnetic fields, it made sense that when light
encountered matter, it would tend to shake the electrons. It
was only to be expected that something like the photoelectric
effect could happen, with the light shaking the electrons
vigorously enough to knock them out of the atom.

But once the effect was observed, physicists began running into
trouble interpreting how it behaved. There were various variables
they could adjust, such as:
\begin{itemize}
 \item the light's frequency (color), and
 \item the light's intensity (brightness).
\end{itemize}
Given these input conditions, there were outputs
they could look at, including
\begin{itemize}
  \item any time delay before electrons began to pop out,
  \item the rate at which the electrons then flowed (measured as a current on an ammeter), and
  \item the amount of kinetic energy they had.
\end{itemize}

At the time this was considered an obscure technical topic, but experimentalists
began generating data, which theorists then had zero success in interpreting.
Albert Einstein,
better known today for the theory of relativity, was the
first to come up with the radical, and correct, explanation,
which involved a radical rewriting of the laws of physics.

\setup

The Hg gas discharge tube emits
light with several different wavelengths.
Turn on the discharge tube immediately, because it takes a long time to warm up.
It has an external power supply, which is a black box.

The photodiode is a vacuum tube housed inside another box, with a small hole
to allow light to come in and hit one of the electrodes (the cathode) inside the vacuum tube.
On the front of the box, covering the hole, are two rotating wheels.  The wheel
that you can see has five colored filters. Each of these filters lets through only
one of the five wavelengths of light emitted by the Hg tube, so that you can control
the frequency.

The hole through which the light enters is actually
a hole in a second wheel, located behind the filter wheel. By clicking that wheel into different position we can
select holes of different sizes, which lets in different amounts of light.
Although this is convenient, it doesn't actually control the intensity of the light (watts per square meter)
but only the total amount (watts). Therefore we'll just leave this set for the 2 mm hole.

The setup does have a provision for controlling the actual intensity of the beam, which is that both
the Hg tube and the photodiode are placed on a track, facing one another.  The light from the Hg tube
spreads out in a cone, so that as you vary the distance to the photodiode, the intensity falls off as
the inverse square of the distance.

The photodiode has an output that can be connected to a extremely sensitive ammeter to measure the
rate at which electrons are ejected from the cathode and absorbed at the anode. Before connecting the
ammeter (labeled ``DC Current Amplifier'' on the front panel), set it to its most sensitive scale
($10^{-13}$ A) and depress button labeled ``Calibration.'' Since the meter is not connected to anything,
the current is truly zero. Use the knob to force the reading to zero, and then let the calibration button out.
Now use a BNC cable to connect the photodiode to the ammeter.

Once the discharge tube has warmed up, arrange things so that you can see a spot of its light, e.g., by
letting it fall on a white piece of paper. Hold the diffraction grating up to your eye and look at the
light. In the first-order ($m=\pm1$) fringes, you should be able to see that the light contains a
mixture of four discrete wavelengths of visible light. There is also an ultraviolet wavelength,
which you may be able to see as well if the paper fluoresces.

Now suppose we were considering the following possible models of light: (1) pure particle model,
(2) pure wave model, (3) a hybrid model in which light has both particle and
wave properties. You have just observed diffraction of the light
from your source. Which of the models are consistent with this observation, and which are immediately
ruled out?

The wavelengths are as follows:\label{hg-wavelengths}

\begin{tabular}{ll}
color   &wavelength (nm)\\
ultraviolet &  365\\
violet   &405\\
blue   &436\\
green  & 546\\
yellow  & 578
\end{tabular}

\figcaption{mo-pho-circuit-new}{Circuit.}

\mysubsubsection{Circuit}

The figure above shows a circuit diagram of the setup. Light comes in and knocks
electrons out of the curved cathode. If the voltage is
turned off, there is no electric field, so the electrons
travel in straight lines; some will hit the anode, creating
a current. If the voltage is
turned on, the electric field repels the electrons from the
wire electrode, and the current is reduced or perhaps even completely eliminated.

\observations

Although the photodiode box has filters on the front, no filter is perfect, and therefore these will
all let in some stray light of wavelengths other than the intended one. Therefore
the room should be very dark when you do your measurements.

\labpart{Time delay}

As explored in the prelab, there may be some
delay between the time when the light
is allowed to hit the cathode and the time when electrons begin to be ejected.
If so, then we lack even a rough a priori estimate of this time.
Investigate this. If the time seems to be extremely short, do what you planned in the
prelab to try to make it long enough to detect. If the time is much too long, do the
opposite so that you can actually observe the photoelectric effect. If you're able to get the time
delay into a range where it's measurable using eyeballs and a clock, do so. If not,
then try to set an upper or lower limit on it.

\labpart{Energy of the electrons}

Until you do the lab, it's not obvious how much energy the electrons would have
as they pop out of the cathode. It could be some fixed number, or it could depend on the
conditions you choose, and it could also have multiple values for the electrons produced
under a given set of conditions. In fact, we may expect a range of values for two reasons.

(1) As explored in a prelab question, the electrons will have random kinetic energies to start
with, due to their thermal motion

(2) The light penetrates to some depth in the cathode,
and an electron that starts at some depth will lose some amount of energy as it then comes out
to the surface. The electron's direction of motion of random. If it happens to be toward the
surface, then the thickness of material that it traverses will depend both on its initial
depth and on its random direction of motion.

This range of energies may have an upper limit for a given set of experimental conditions.
If so, then by applying a high enough voltage you should be able to eliminate the current
completely. The minimum voltage required to do this would be called the stopping voltage.

Before you can apply a voltage, you need to  set up the DC power supply as follows. There is a button between the
two LED readouts. Let this button out in order to select a range of voltages from 0 to -4.5 V.
On the bottom right side of the power supply are two banana plugs. Connect these to the
two plugs near the bottom of the photodiode. To get the polarity right, connect the 
positive (red) output of the power supply to the anode (marked A).

Find out whether there is a stopping voltage, and if so, measure it for the conditions you've chosen.
If there is never a sharp cutoff, you should still be able to determine some quantitative measure
of the voltage that corresponds to a \emph{typical} energy for the electrons, e.g., the voltage at
which some fixed fraction of the current is eliminated. From now on in the lab manual, I'll
just refer to this as ``the voltage'' for a given set of experimental conditions.

Determine the voltage, and compare with the estimate in the prelab of what voltage would
correspond to the thermal motion of the electrons. Based on this comparison, is the amount of
energy involved in the photoelectric effect much less than, much more than, or on the same
order of magnitude as the thermal energy?

\labpart{Dependence of voltage on intensity}

Vary the intensity of the light and determine whether and how the voltage depends on intensity.

We have observed strange results sometimes, which seem to occur because when the discharge tube is
very close to the photodiode, light gets in and hits the \emph{anode} (not just the cathode) and
causes the photoelectric effect in the wrong direction. This seems to occur with the shorter wavelengths
of light. You should be able to tell if this is happening because when you turn up the voltage high
enough, you get a current in the opposite direction. Check for this behavior and avoid taking data
under conditions that produce it.

\labpart{Dependence of voltage on frequency}

Do the same thing for the frequency.

\analysis

The point of the analysis is to try to compare the observed results with our expectations based
on two models of light:
the pure wave model, and a model in which light is both a particle and a wave --- we've seen
above that a pure particle model is untenable. For conceptual simplicity, however, we may find
it helpful to visualize the wave-particle model as if the light was purely particle-like, so
that a beam of light would be like a stream of machine-gun bullets. This is essentially what
Einstein does in his 1905 paper. He admits that this is not literally possible, but doesn't attempt
a more detailed reconciliation of the particle and wave pictures, which he doesn't know how to
achieve. For this reason, the title of the paper refers to the particle picture as a ``heuristic,''
which means a kind of non-rigorous way of getting an answer without using correct fundamental principles.

Time delay: You investigated the possible time delay in the wave model in some detail in the
prelab and/or homework. In a particle model, what would be your expectations about a time
delay? Compare with experiment.

Energy: Based on the particle model, we would expect one ``bullet'' of light to give its fixed
amount of energy to one electron, so that under a given set of experimental conditions, there
would be a maximum possible kinetic energy for the electrons, achieved when the electron originated
very close to the surface of the cathode. In the wave model it is more difficult to make a definite
prediction. Did you observe that there was a definite stopping voltage, or that there was no definite
cut-off in the current? Does this allow you to test either model?

Dependence of energy on intensity: In the particle model, a more intense beam of light would be
one containing a larger number of particles (per unit of cross-sectional area). In the wave model,
a more intense beam would be a higher-amplitude wave. Does either model lead you to predict anything
specific about the dependence of voltage on intensity? Test against experiment. 

Dependence of energy on frequency: Does the (typical or maximum) energy of the electrons $eV$
depend on the light's frequency $f$? If so, then in the particle model this probably means that
we're observing a change in the amount of energy per particle of light. We can't just equate the electron energy $eV$
to the energy of the particles of light, because the electrons lose a fixed amount of energy (called the
work function) as they emerge from the surface of the metal. We can get around the issue of this constant
offset by finding the slope of $eV$ versus $f$; a constant offset on the $y$ axis doesn't affect the slope
of a graph. Let's call this constant $h$. Estimate its numerical value. What units does it have?
There is no such universal constant in any of the classical laws of physics, so if it pops up here,
it indicates that some entirely new physical theory is being probed.

\prelab

\prelabquestion Under the hypothesis that light is purely a wave, the energy of a beam of
light would be smoothly and continuously accumulated by whatever it hit. Therefore it should
take some amount of time before an electron could accumulate enough energy to pop out of the metal.
You may have estimated this time scale in a homework problem
(%
m4_ifdef([:__lm:],[:\emph{Light and Matter} problem 34-11:],[:\emph{Simple Nature} problem 13-18:])%
),
 but that calculation depends on many crude assumptions and rough
estimates, so it's hard to know whether to trust it even as an order of magnitude estimate.
Therefore if we want to try to observe this time delay in this experiment, it's impossible to
know in advance what to look for, and it may be either too short (in which case we can't measure
it) or too long (in which case it will look like the apparatus simply isn't working). Suppose
that the time delay is too short to detect in the conditions you initially pick. How could
you change the conditions in order to make it longer, and possibly detectable?

\prelabquestion m4_ifdef([:__lm:],[:
Heat is the random motion of the microscopic particles of which matter is composed. Depending on the
level of detail in the treatment of thermodynamics in you first-semester course, you may know that the
typical energy per particle can be calculated as $kT$, where $T$ is the absolute temperature and $k$ is
a constant called Boltzmann's constant (not to be confused with the Coulomb constant). The result at
room temperature is on the order of $10^{-20}$ joules per particle.
:],[:
Suppose that the cathode is at temperature $T$. Then the electrons inside the cathode
already have some amount of kinetic energy, due to their thermal motion. Estimate the typical amount
of kinetic energy.
:])%
In the photoelectric effect, an electron will absorb some additional amount of
energy from the light, which is enough to pop it out of the cathode. Until doing the experiment,
we do not know how much this additional amount is, but during the lab you will be able to probe this
question by using a voltage $V$ to try to stop the electrons from making it across the gap. If
this additional energy was on the \emph{same} order of magnitude as the thermal energy (which it may not
be), estimate the voltage required.

\prelabquestion In this experiment, the light comes out of the discharge tube in a spreading cone.
Geometrically, how should the intensity $I$ of the light depend on the distance $r$? State a proportionality.

\prelabquestion Look up the wavelength of visible light and the typical distance between atoms in a solid.
How do they compare? In the wave model, should we expect a particular wave-train to hit one atom, or many?
