\renewcommand\thechapter{c1.14}
\lab{Torque}\label{lab:covid-torque}
\covid\ 

\apparatus
\equipn{plastic ruler}{1}
\equip{spring scales, calibrated in newtons}
\equip{lead shot}
\equip{modeling clay}
\equip{string}
\equip{protractors}
\equip{postal scale}
\equipn{Suction cups with a hook}{3}

\goal{Test whether the total force and torque on an object at
rest both equal zero.}

\introduction

It is not enough for a boat not to sink.  It also must not
capsize.  This is an example of a general fact about
physics, which is also well known to people who overindulge
in alcohol: if an object is to be in a stable equilibrium at
rest, it must not only have zero net force on it, to keep
from picking up momentum, but also zero net torque, to keep
from acquiring angular momentum.

\fig{me-tor-setup}

\observations

Prepare a 30 cm - 1 m stick and weigh your stick. You can also use the 30 cm ruler in the lab kit. 

In this lab, your goal is to show that the net force and net torque acting on the stick are both zero in equilibrium. 
Tie three strings at three different locations of the stick, and pull it by three spring scales similar to the setup shown above. 
Be creative and make your own setup. 
The figures in this lab manual are just possible examples. 

You can use suction cups with a hook to hold spring scales. 
You can use hang a weight to the stick. 

Make all three angles between the stick and the strings different. Avoid any symmetry in your arrangement. 

You can hang the stick vertically or place the stick on a table. If you hang the stick, you need to include the weight of the stick in your analysis. If you put the stick on a table, make sure the tensions are large enough so that the static friction is negligible compare to the spring scale forces.


There are four forces acting on the meter stick:
\begin{align*}
       \vc{F}_1 &=    \text{tension in the string 1}  \\
       \vc{F}_2 &=    \text{tension in the string 2}  \\
       \vc{F}_3 &=    \text{tension in the string 3}  \\
       \vc{F}_{M} &=    \text{weight of the stick itself} 
\end{align*}

Try at least two different configurations and measure forces, position of forces, and angle between forces and the stick. 

Each of these forces also produces a torque.

In order to determine whether the total force is zero, you
will need enough raw data so that for each torque you can
extract (1) the magnitude of the force vector, and (2) the
direction of the force vector.  In order to add up all the
torques, you will have to choose an axis of rotation, and
collect enough raw data to be able to determine for each
force (3) the distance from the axis to the point at which
the force is applied to the stick, and (4) the angle between
the force vector and the line connecting the axis with the
point where the force is applied.  Note that the stick's own weight can be thought of as being applied
at its center of mass.

When you measure distance, make sure the reading is the distance from the axis of rotation to the point at which the force is applied to the stick. 
You can also use lever arms to calculate torques. 

%\fig{me-tor-knot}

For each spring scale, hang a known weight from it, and adjust the
calibration tab so that the scale gives the correct result.


\prelab

\prelabquestion  You have complete freedom in defining what point is to be considered the axis of
rotation ---  if one choice of axis causes the total torque
to be zero, then any other choice of axis will also cause
the total torque to be zero.  It is possible to simplify the
analysis by choosing the axis so that one of the four
torques is zero.  Plan how you will do this.

\prelabquestion  All the torques will be tending to cause rotation in the
same plane.  You can therefore use plus and minus signs to
represent clockwise and counterclockwise torques.  Choose
which one you'll call positive.  Using your choice of axis,
which of the four torques, $\tau_H,\tau_M,\tau_L$, and
$\tau_R$, will be negative, which will be positive,
and which will be zero?

\analysis

Determine the total force and total torque on the meter
stick.  For the forces, I think a graphical calculation will
be easier than an analytic one.

Finally, repeat your calculation of the total torque using a
different point as your axis. Although you're normally expected to
do your analysis completely independently, for this lab it's okay
if you find the total torque for one choice of axis, and your lab
partners do the calculation for their own choices. Present both
results in your own abstract.

Error analysis is not required. For extra credit, you can do
error analysis for your total torque.
