\addtocounter{chapter}{-1}
\renewcommand\thechapter{\arabic{chapter}c}
\lab{The Nitrogen Molecule}\label{lab:nitrogen}
\renewcommand\thechapter{\arabic{chapter}}

\apparatus
\equipn{He gas discharge tube}{3}
\equipn{N2 gas discharge tube (in green carousel)}{3}
\equipn{spectrometer}{1/group}
\equipn{diffraction grating, 600 lines/mm}{1/group}
\equipn{small screwdriver}{1}
\equipn{black cloth}{1}
\equipn{piece of plywood}{1}
\equipn{block of wood}{1}
\equipn{penlight}{1/group}
\equip{light block}

\section*{Goals}

The lab has three parts. Each group will only do two parts.
In this lab, part c, you will use an energy sum to test a hypothesis about the energy levels of the nitrogen molecule, $\zu{N}_2$.

m4_include(basic-spectrometer.tex)
\labpart{Energy Sums}

The nitrogen discharge tube is housed in a green plastic
carousel. With the power off, rotate the carousel so that the nitrogen tube is
the one that is in the active position, and then turn on the power.
If you hold a diffraction grating up to your eye and look at the
tube, you will see a remarkable spectrum, unlike the visible light
spectrum of almost any other gas. This is because the $\zu{N}_2$ molecule has
an extremely strong bond, requiring twice the energy to break compared to
otherwise similar gases such as $\zu{H}_2$ or $\zu{O}_2$.
Whereas these other gases would break up into individual atoms under the extreme
conditions present in a discharge tube, the nitrogen molecule holds together, so
that you are seeing the spectrum of the molecule, not the atom.
For this reason, the spectrum of nitrogen contains a large number of lines.

\fig{mo-hel-nitrogen}

But these lines are not random. They occur in sets, each of which looks like
a comb with an approximately equal spacing between the ``teeth.'' The figure
shows a portion of the spectrum, including three sets of lines, which I have
labeled r (red), o (orange to green), and g (green).

I have spent some time
trying to interpret the origin of these lines, and I believe the interpretation
is something like this. Each of these lines is the emission of a photon as
the molecule goes from an initial state to a final state that has less energy.
The initial state has some energy because the electrons are in an excited state
(labeled B by spectroscopists) and also some energy because the molecule is vibrating,
like two masses connected by a spring. The final state has the electrons in
a lower-energy state (labeled A), and is also vibrating. The initial and final
electronic states B and A are the same in all cases, but the vibrational states
differ. An idealized quantum-mechanical vibrator turns out to have a series of
energy states like a ladder with nearly evenly spaced rungs. States higher on
the ``ladder'' are vibrating more violently --- classically, they vibrate with
greater amplitude. The rungs of the vibrational ladder are labeled $v=0,$ 1, 2, and so on.
(Because of the Heisenberg uncertainty principle, some vibrational energy is present
even in the $v=0$ state.) I think the states in the red set are from a state $v$ to
a state $v-3$, i.e., a change of 3 units in the vibrational quantum number. The o
set would be a change of 4, and g a change of 5.

This hypothesis can be tested as follows. If it is true, we could pick out a set of
energy levels like the following example:

\fig{mo-hel-sum}

The letters $e$, $f$, $g$, and $h$ are the energy differences that would be observed
as the energies of the photons. In a set like this, we would have
\begin{equation*}
  h-g = f-e,
\end{equation*}
since each side of the equation would be equal to the energy difference between the
$v=4$ and 5 states of ladder A. Try to find a set of lines that
would be consistent with this interpretation. This may require some trial and
error, but I think it may work if $e$ is one of the lines near the middle of the
r set, $g$ is the orange line that is fourth from the short-wavelength end
of the o set, $f$ is the fifth in that set, and $h$ is in the g set.

\labpart{Calibration}

You will use the yellow line from helium as a calibration.
In theory it shouldn't matter what known line we use for calibration, but in practice
there may be small aberrations in the spectrometer, and their effect is minimized by
using calibration lines of nearly the same wavelength as the unknown lines to be measured.

Put the helium tube behind the collimator. Make sure the
hottest part of the tube is directly in
front of the slits. You will need to use pieces of wood to get the height right.
You want the tube as close to the slits as possible, and
lined up with the slits as well as possible; you can adjust
this while looking through the telescope at an $m=1$ line,
so as to make the line as bright as possible.

If your optics are adjusted correctly,
you should be able to see the microscopic
bumps and scratches on the knife edges of the collimator,
and there should be no parallax of the crosshairs relative
to the image of the slits.

Here is a list of the wavelengths of the most prominent visible He lines, in nm, to high 
precision.\footnote{The table gives the wavelengths in vacuum. Although we're doing the lab in air, our goal is to find
what the nitrogen wavelengths would have been in vacuum; by calibrating using vacuum wavelengths
for mercury, we end up getting vacuum wavelengths for our unknowns as well.}

Helium:\\
\noindent\begin{tabular}{llp{40mm}}
447.148 & bright blue-purple & \\
471.314 & dim blue & \\
492.193 & dim green & \\
501.567 & bright green & \\
587.562 & yellow & \\
667.815 & dim red & \\
706.5 & very dim red
\end{tabular}

Start by making sure
that you can find all of the lines in the correct sequence --- if
not, then you have probably found some first-order
lines and some second-order ones. If you can find some lines
but not others, use your head and search for them in the
right area based on where you found the lines you did see.
You may see various dim, fuzzy lights through the telescope
--- don't waste time chasing these, which could be coming
from other tubes or from reflections. The real lines will be
bright, clear and well-defined. By draping the black cloth over
the discharge tube and the collimator, you can get rid of stray
light that could cause problems for you or others. The discharge tubes also have
holes in the back; to block the stray
light from these holes, either put the two discharge tubes back
to back or use one of the small ``light blocks'' that slide over the hole.

We will use the wavelength $\lambda_c$ of the hellow He line as a calibration. Measure its two
angles $\alpha_L$ and $\alpha_R$, and check that the resulting value of $\theta_c$ is
close to the approximate ones predicted in prelab question P1. The nominal value of the
spacing of the grating given in that prelab question is not very accurate.
Having measured $\theta_c$,
then we can sidestep the determination of the grating's spacing entirely 
and determine an unknown wavelength $\lambda$
by using the relation
\begin{equation*}
  \lambda = \frac{\sin\theta}{\sin\theta_c} \lambda_c \qquad .
\end{equation*}
The angles are measured using a vernier scale, which
is similar to the one on the vernier calipers you have
already used in the first-semester lab course. Your final
reading for an angle will consist of degrees plus minutes.
(One minute of arc, abbreviated 1', is 1/60 of a degree.)
The main scale is marked every 30 minutes. Your initial,
rough reading is obtained by noting where the zero of the
vernier scale falls on the main scale, and is of the form
``xxx\degunit0' plus a little more'' or ``xxx\degunit30'
plus a little more.'' Next, you should note which line on
the vernier scale lines up most closely with one of the
lines on the main scale. The corresponding number on the
vernier scale tells you how many minutes of arc to add for
the ``plus a little more.''

As a check on your results,
everybody in your group should take independent readings of every
angle you measure in the lab, nudging the telescope to the side after
each reading. Once you have independent results for a particular angle,
compare them. If they're consistent to within one or two minutes of
arc, average them. If they're not consistent, figure out what went
wrong.

\section*{Status as of August 2018}

In spring of 2018, my students and I worked on measuring and
interpreting this spectrum. 
A summary of our results is in a Google
Docs spreadsheet at \url{goo.gl/akrbcY}.
There is a basic explanation of
the physics in Simple Nature section 14.2.  More detailed information about my
interpretation of the lines is at\\
\url{physics.stackexchange.com/a/334451/4552}.

In the notation used in the material on stackexchange, the states are
labeled with a quantum number $v$.  The energy sum based on our data
come out quite nice for $v=9$ and 8 going to 5 and 4, accurate to
about 0.3\%, which is reasonable for this technique.  The energy sum
for $v=10$ and 9 going to 6 and 5 also works well, but with slightly
higher error, maybe partly because 10-5 is a doublet.  There are two
dim green lines that in my labeling system would be g0 and g-1.
These may be the
ones we would need in order to get a couple more energy sums.

I experimented with doing the measurements photographically.  A
student took a photo of a diffraction pattern using a cell phone.  I
first did a rough calibration against student data, then improved this
calibration by doing a linear fit to my own spectrometer data.
This worked fairly well, but doesn't work without actual spectrometer
data, which are needed for calibration.

We have digital spectrometers which may be helpful here and are worth
trying. Their resolution is supposed to be about 1 or 1.5 nm, which is
an order of magnitude worse than the analog spectrometers, but may be
adequate for this purpose, and they can display a spectrum as a graph,
which may be help enough to make up for the lower resolution.

I couldn't resolve the green band. I asked a couple of students the
next day, and they seemed to think that it was doable to resolve these
lines. Possibly my tube was behaving differently than theirs, but I'm
not sure I believe their data.  The red and orange bands came out
nice, all wavelengths being within a few tenths of a nm of Lofthus's
values.

\prelab

The week before you are to do the lab, briefly familiarize
yourself visually with the apparatus.

\prelabquestion  
The nominal (and not very accurate) spacing of the grating is stated
as 600 lines per millimeter. From this information, find $d$, and
predict the angles $\alpha_L$ and $\alpha_R$ at which you will observe the yellow helium line.

\widefigcaption{mo-hel-spectrometer}{The spectrometer}
\widefigcaption{mo-hel-optics}{Optics.}
\widefigcaption{mo-hel-orient}{Orienting the grating.}
\widefigcaption{mo-hel-vernier}{Prelab question 2.}

\prelabquestion  Make sure you understand the first three vernier
readings in the fourth figure, and then interpret the fourth reading.

\prelabquestion  For the calibration with helium,
in what sequence do you expect to see the lines on
each side? Make a drawing showing the sequence of the angles
as you go out from $\theta $=0.
