\lab{Newton's Second Law}\label{lab:atwood}

\apparatus
\equip{pulley}
\equip{spirit level}
\equip{string}
\equip{weight holders, not tied to string}
\equip{two-meter stick}
\equip{slotted weights (in lab benches in 415)}
\equipn{digital balance (Ohaus SP4001)}{3}
\equip{stopwatch}
\equip{foam rubber cushions}
\equip{graphite lubricant}

\goal{Find the acceleration of unequal weights hanging from a pulley.}

\observations

\fig{me-atw-pulley}

Set up unequal masses on the two sides of the pulley, and
determine the resulting acceleration by measuring how long
it takes for the masses to move a certain distance. 
Use the spirit level to make the plane of the pulley vertical; otherwise you
get extra friction.
Use relatively large
masses (typically half a kg or a kg each side) so that friction
is not such a big force in comparison to the other forces, and
the inertia of the pulley is negligible compared to the inertia
of the hanging masses.
Do several different combinations of masses, but \emph{keep
the total amount of mass constant} and just divide it
differently between the two holders.  Remember to take the
masses of the holders themselves into account. Make sure to
perform your measurements with the longest possible distance
of travel, because you cannot use a stopwatch to get an
accurate measurement of very short time intervals. The best
results are obtained with combinations of weights that give
times of about 4 to 20 seconds.

The brass weights were manufactured by the friendly crack-smoking
hillbillies at Glakad Science in North Carolina. They have a tolerance (not
disclosed in the distributor's catalog) of $\pm 2$\%, meaning
that a 500 g weight could be off by as much as ten grams!
Because of this, you will need to weigh your stacks of weights on
a digital balance to find out what they really are.

Count your weights before you start and make sure you have the full
set listed on the box. During the lab, keep the small ones in the
plastic cup. At the end of lab, count your weights again.

Your stopwatch timing errors are determined by your reflexes, which
are presumably the same for all mass combinations. You'll need to take a large
number of trials at some mass combination in order to find this error accurately.
It is pointless, however, to do multiple trials for every mass combination.

\selfcheck

Compare theoretical and experimental values of
acceleration for one of your mass combinations.
Check whether they come out fairly consistent.

\analysis

Use your measured times and distances to find the actual
acceleration, and make a graph of this versus $M-m$. 
See appendix 4 re graphing.
Show
these experimentally determined accelerations as dots.
Overlaid on the same graph, show the theoretical
equation %
m4_ifdef([:__sn:],[:(problem 5-20):],[:%:])
as a line or curve, as in the examples in \ref{appendix:graphing}.
Use propagation of errors (appendix \ref{appendix:errpropagation}) to determine error bars 
for your accelerations, and show them on your graph.
Even though all the times have the same error bars, the accelerations will not.
As in the examples in \ref{appendix:graphing}, compare theory and
experiment: did they agree
to within your error bars?

\prelab

\prelabquestion Criticize the following reasoning: 
The weight fell 1.0
m in 1 s, so $v=1$ m/s, and $a=v/t=1\ \munit/\sunit^2$.

\prelabquestion Since that won't work, plan how you really will
determine your experimental accelerations based on your
measured distance and times.
