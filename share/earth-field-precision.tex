\lab{The Earth's Magnetic Field (Physics 222)}\label{earth-field-precision}

\apparatus
\equip{digital multimeter}
\equip{neodymium magnet (6 discs stuck together)}
\equip{magnetic compass}
\equip{resistors}
\equip{decade resistor boxes}
\equip{rulers}
\equip{thread}
\equip{1-m aluminum rod}
\equip{stopwatch}
\equip{photogate}
\equip{laser}
\equip{aluminum rods, and clamps}
\equipn{D cell batteries and holders}{2/group}
\equip{Helmholtz coils (e/m apparatus)}
\equip{high-precision Helmholtz coil (one set)}
\equip{Hall effect magnetic field probes}
\equip{LabPro interfaces, DC power supplies, and USB cables}

\goal{Determine the horizontal  component
of the Earth's magnetic field in Fullerton, to high precision.}

\observations

Since you've already used the Hall effect magnetic field probes in
lab \ref{dipolefieldlab},
you might think that it would be relatively trivial to measure the Earth's magnetic field
precisely. However, the calibration of those probes is quite poor, so it's not possible
to get results with error bars smaller than about 10-20\%.

\figcaption{em-ear-helmholtz-coil}{The geometry of a Helmholtz coil.}

The basic idea of the more precise technique used in this lab is to hang a permanent magnet
from a thread, and observe the period of its oscillations
in the Earth's magnetic field. The idea is that if the Earth's
field is stronger, there is a stronger torque trying to align the magnet north-south,
and the frequency of the oscillations will therefore be higher. By measuring the
frequency of the oscillations, we can work backward and infer the strength of the
horizontal component of the Earth's field.

\figcaption{em-ear-helmholtz-contours}{A contour map of the field of a Helmholtz coil (top view
of the horizontal plane cutting through the center).}

One reason the technique isn't quite that simple is that the frequency of
the oscillations also depends on other quantities,
including the magnet's dipole moment and moment of
inertia, that are very difficult
to measure with better than about 10\% precision. A trick for
getting around this problem is to superimpose a known southward magnetic field
on the Earth's northward one, and adjust the known field so as to cancel the Earth's.
Reducing the field increases the period of the oscillations, and
if we could exactly cancel the horizontal component of the Earth's field,
then the period would be infinite. The known field is supplied by a type
of electromagnet called a Helmholtz coil, shown in the first figure. It consists of
two circular coils of wire, with their axes coinciding. In the classic design
(which is what's really properly called a Helmholtz coil), the separation $h$
between the planes of the two coils is equal to their radius, $b$. Having $h=b$
turns out to produce the most uniform possible field near the center of the whole
arrangement, in the sense that all the field's derivatives up to the fourth
derivative equal zero. The second figure (from the Wikipedia article, copyleft
licensed by Wikipedia) is a contour map showing how little
the field actually varies over a fairly large volume in the center. The ``octopus'' in
the middle is the region in which the field is between 99\% and 101\% of its value at the
center.

Even this version of the experiment turns out to need some further tweaking. It is difficult
to align the axis of the coils with the Earth's field, so we typically end up with a misalignment,
$\phi$, which is a few degrees. Therefore, the fields do not really cancel, and as the current through the coils
is tuned through the optimal value, the horizontal field becomes small, but not zero, and
swings around gradually from north to south. It becomes difficult to pick off the current that produces the
maximum period, partly because the period of the oscillations is not quite independent of
amplitude, and it becomes difficult to control the amplitude of the oscillations properly
when the equilibrium orientation is constantly changing. Even if we could precisely recognize
the current that gave the maximum period, that would be the current that canceled out the
component of the Earth's field along the coils' axis, i.e., we would be taking the vector $(B_x,B_y)$,
and changing it to $(B_x,0)$, where y is the axis of the coils. Thus we would really be measuring,
$B_y=B\cos\phi$, rather than $|\vc{B}|$. To get around this problem, you can use the following iterative
method: (1) Align the coils' axis approximately with the earth's field by eyeballing the alignment against
a magnetic compass. (2) Tune the current in the coil to the point where the magnet's equilibrium orientation
is perpendicular to the earth's field. This is pretty close to the current that would have canceled the
earth's field, if the alignment had been perfect. In this state, the magnet will point either to the east
or to the west, depending on the direction of the error in alignment. (3) Carefully, slowly rotate
the apparatus until the magnet's equilibrium orientation shifts to the north-south line. This is a state
in which the coil's field is exaclty on the same line as the Earth's, but their magnitudes are slightly
different. (4) Tune the current again to maximize the period.
In this final step, it becomes important to control the amplitude of the oscillations.
As shown in the figure, the error in the period is less than 0.1\% for amplitudes of less than about
10 degrees.

\figcaption{em-ear-period-amplitude}{The dependence of the period on amplitude. For angles less than
20 degrees, the motion is nearly simple harmonic,
and the period is independent of amplitude to within about 1\%. Higher amplitudes can be used, but it becomes
much more important to control the initial amplitude.}
% maxima:
% f(x) := (2/%pi)*elliptic_kc(sin(x/2)^2);
% plot2d(f(x*%pi/180)-1,[x,0.001,179.999],[y,.0001,100],[gnuplot_term,svg],[gnuplot_out_file,"foo.svg"],[xlabel,"amplitude (degrees)"],[ylabel,"correction to period"],[gnuplot_preamble,"set size square 0.5,0.5"],[logy]);
% inkscape won't accept svg output; see unix notes on how to fix this in an editor

The problem now boils down to the accurate determination of the field at the center of the Helmholtz coils
for a given amount of current, i.e., the ratio $B/I$. You'll derive the relevant expression as one of your prelab questions.
It depends on the accurate measurement of the dimensions $b$ and $h$. In general it's fairly difficult to
construct magnet coils so that their dimensions are accurately determinable, and the coils you'll use are no
exception. They consist of somewhat irregular bundles of wire tied together with cable ties, and they
aren't even circular; their vertical diameter is significantly different from their horizontal diameter.
As closely as I've been able to determine, they have $h=14.7\pm0.3$ cm, and an average $b$ of about $15.1\pm0.3$ cm,
but these error bars are uncomfortably large. They have $N=130$ turns of wire on each coil, i.e., 260 turns
on each complete set of Helmholtz coils.

Because of these problems, I've constructed a Helm\-holtz coil that has a much more precisely
measurable geometry. You can calculate $B/I$ for the precise coils, whose dimensions are
carefully constructed and easy to measure: $h=b=11.15\pm.05\ \zu{cm}$. They have $N=5$ turns of wire
in each coil.
Although there is only one copy of the precise Helmholtz coil, and it wouldn't
be convenient to use for this lab anyway (they produce weak fields, and their interior is not very
accessible), we can calibrate your coils against them. I'm planning to do this as a student lab for
the first time in spring 2009, and we'll use the data from that semester as a calibration for the coils,
by comparing $B_{earth}/I$ for them with  $B_{earth}/I$ for the precise coils.

The Helmholtz coils we're using are actually meant for lab \ref{charge-to-mass}, and they have a big,
extremely expensive vacuum tube stuck inside them for that purpose. With your instructor's help,
very carefully detach the base from the tube. Then unscrew the yokes that hold the tube in place, and
put the tube out of the way in the stockroom, with cusioning to make sure it doesn't get broken.

You need precisely controlled, steady currents for this lab, and DC power supplies aren't stable enough,
so you'll use batteries instead. To control the current precisely, you'll use the decade resistor
boxes, which are variable resistors that let you dial up any decimal number of ohms that you want.

We want to keep all magnetic materials far away from the magnet.
Clamp the 1-m aluminum rod to the vertical steel post, and hang the magnet from it, far from the post.

There are several possible methods for measuring the period of the oscillations, and one of my goals for
spring 2009 is to have my students test drive them. One is to use
a stopwatch to time, say, 20 oscillations.  A second method would be to use the magnetic field probe and graph the field as a function
of time.
A third method would be to use a photogate, in pendulum mode as described in appendix \ref{appendix:photogate}.
The photogates have steel screws in them, so you can't use them in the ordinary way, with the magnet swinging
through the infrared beam that goes across the center of the gate. Instead, you can open the shutter on
the inside of the photogate to change it into a mode where it senses light from the beam of an external
laser. The photogate can then be physically far away from the magnet so that the screws don't affect
the measurement. A possible problem with the photogate method is that it requires the amplitude of the oscillations to
be big enough so that the magnet blocks and unblocks the photogate, but with oscillations that big, the
dependence of the period on amplitude could be a significant source of error unless the amplitude was very
accurately controlled. This problem could possibly be solved by attaching a cardboard vane to the magnet, and
that would also get rid of the safety problem caused by reflecting the laser beam from the shiny magnet.
Of these methods, it's possible that one might be the most convenient for rough initial measurements, while
another would work best for the final, accurate measurement.

When you're done with all this, what you've actually measured is the magnetic field inside the
building. Many buildings have magnetic building materials, so the fields inside them are different
from the Earth's field. To correct for this, measure the period of the magnet's oscillation inside
and outside. If they are significantly different, correct according to $B_1/B_2=(T_2/T_1)^2$; this
follows from adapting the equation $\omega=\sqrt{k/m}$ for simple harmonic motion to the case of
rotation, with the torque $\btau=\vc{m}\times\vc{B}$ playing the role of the restoring force.

(Note to myself: Some of my own further notes about the lab are embedded in comments in the LaTeX source code
for the lab manual.)

\prelab

\prelabquestion  For an electromagnet consisting of a single circular
loop of wire of radius $b$, the field at a point on its
axis, at a distance $z$ from the plane of the loop, is given by
\begin{equation*}
      B = \frac{2\pi kIb^2}{ c^2( b^2+ z^2)^{3/2}} \qquad .
\end{equation*}
Starting from this equation, derive an equation for the
magnetic field at the center of a pair of Helmholtz coils, in terms of $h$, $b$, and $N$.
Find $B/I$ for both the high-precision coils and the low-precision ones, based on the
given values of $h$, $b$, and $N$. (The $B/I$ for the low-precision ones is useful as a check,
but has poor precision, which is why you'll calibrate against the high-precision ones.)

\prelabquestion Estimate the current that will be required in the low-precision coils in order
to cancel the Earth's field, about $2\times10^{-5}\ \zu{T}$.

\analysis

Find the earth's magnetic field, with error bars.


% One possible way to calibrate coils would be to find someone local who has an earth's field
% NMR setup such as the teachspin.com one, and get them to let me calibrate against that.
% A problem in all this is the absolute calibration of the ammeter. The calibration of the
% coils can never be better than the calibration of the ammeter I use during the calibration.
% It would be cool to be able to measure the periods quickly and accurately using the Hall
% probes. Called Vernier tech support, and the possibilities we came up with were (a) saving
% to a csv file, or (b) writing my own software. They only support linux interfacing via
% com (serial) port (ttys0(?)), not usb; there are code examples on their site. 
% Simplest thing to do would be to take a 5-second rolling time window, and do an fft,
% picking off the peak, and then keep on updating that peak's frequency as the time window
% rolls forward. There may be strong harmonics because the field is a strongly nonlinear
% function of the orientation of the magnet.
